% !TEX root = ../fields.tex
%%%%%%%%%%%%%%%%%%%%%%%%%%%%%%%%%%%%%%%%%%%%%%%%%%%%%%%%%%%%%%%%%%%%%%%%%%%%%%%%%%%%%%%%%%
\section{Exercises}
\begin{ExerciseList}
  %---------------------------------------------------------------------------------------
  \Exercise Define spherical polar coordinates by:
  \(
  \v{r}
  \,=\,
  r\sin(\theta)\cos(\phi)\,\eone
  \,+\,
  r\sin(\theta)\sin(\phi)\,\etwo
  \,+\,
  r\cos(\theta)\,\ethree
  \)

  \Question Show that
  \(
  \;
  h_r
  \,\equiv\,
  | \parpar{\vr}{r}|
  = 1
  \,,\;
  h_\theta
  \,\equiv\,
  | \parpar{\vr}{\theta}|
  = r
  \,,\;
  h_\phi
  \,\equiv\,
  | \parpar{\vr}{\phi} |
  =
  r\sin(\theta)  \,.
  \)

  The quantities $h_r$, $h_\theta$ and $h_\phi$ are the \emph{scale factors} for spherical
  polars.

  \Question Hence, show that
  \begin{eqnarray*}
    \er
    &\equiv&
    \frac{1}{h_r} \: \parpar{\vr}{r}
    \,=\,
    \sin(\theta)\cos(\phi) \, \eone +
    \sin(\theta)\sin(\phi) \, \etwo +
    \cos(\theta) \, \ethree \\[0.5ex]
    %
    \etheta
    &\equiv&
    \frac{1}{h_\theta} \: \parpar{\vr}{\theta}
    \,=\,
    \cos(\theta)\cos(\phi) \, \eone +
    \cos(\theta)\sin(\phi) \, \etwo -
    \sin(\theta) \, \ethree \\[0.5ex]
    %
    \ephi
    &\equiv&
    \frac{1}{h_\phi}\parpar{\vr}{\phi}
    \,=\,
    -\sin(\phi) \, \eone + \cos(\phi)\etwo
  \end{eqnarray*}

  \Question Verify that $\er$, $\etheta$, and $\ephi$ form an
  \emph{orthonormal} basis:
  \[
    \er \cdot \,\er = \etheta \cdot \,\etheta = \ephi \cdot \,\ephi = 1
    \;\mbox{and}\;
    \v{e}_{\,r} \cdot \,\etheta = \v{e}_{\,\theta} \cdot \,\ephi
    = \v{e}_{\,\phi} \cdot \,\er = 0 \,.
  \]

  \Question Show that:
  \(
  \quad
  \er \times \etheta = \ephi \,,
  \quad
  \etheta \times \ephi = \er \,,
  \quad
  \ephi \times \er = \etheta \,.
  \)

  \Question Hence, show that the vector element of area on a sphere of radius
  $r$ is
  \[
    {\rm d}\v{S}_r
    \,=\,
    h_\theta \, h_\phi \, \er \, {\rm d}\theta \, {\rm d}\phi
    \,=\,
    r^2\sin(\theta) \: \er \, {\rm d}\theta \, {\rm d}\phi \,.
  \]

  \Question Find expressions for the vector elements of area on the
  cone $\theta=\theta_0$ where $\theta_0$ is constant, and on the plane
  $\phi=\phi_0$, where $\phi_0$ is constant. Illustrate your results
  with sketches.

  \Question Show that the volume element is
  \[
    {\rm d}V
    \,=\,
    h_r \, h_\theta \, h_\phi \,{\rm d}r \, {\rm d}\theta \, {\rm d}\phi
    \,=\,
    r^2 \sin(\theta) \, {\rm d}r \, {\rm d}\theta \, {\rm d}\phi \,.
  \]

  %---------------------------------------------------------------------------------------
  \Exercise Show that the scale factors for cylindrical coordinates ($\rho$, $\phi$, $z$) are
  \(
  h_\rho {=} 1 \,,
  %\qquad 
  h_\phi {=} \rho \,,
  %\qquad
  h_z {=} 1 \,. \)

  Write down expressions for $\grad f\,,\,$ $\Div \v{a}$, $\curl\v{a}$ and $\nabla^2f$ in
  cylindrical coordinates. Evaluate the curl of each the following vector fields
  \[
    \mbox{(i)}
    \qquad   \v{a} = \ephi
    \qquad\quad
    \mbox{(ii)}
    \qquad    \v{a} = \rho \; \ephi
    \qquad\quad
    \mbox{(iii)}
    \qquad   \v{a} = {\ds \frac{1}{\rho}} \; \ephi
  \]
  Note that each of the fields $\v{a}$ has a non-zero circulation around the $z$ axis, but
  this does not imply that $\curl\v{a}\neq0$ in every case.

  %---------------------------------------------------------------------------------------
  \Exercise In cylindrical coordinates ($\rho$, $\phi$, $z$), a vector
  field $\vA$ has components
  \[
    A_\rho \,=\, A_z \,=\, 0, \quad A_\phi \,=\, \frac{a}{\rho}
  \]
  where $a$ is a constant. Therefore, $\vA$ is defined in an annular region outside an
  infinitesimal cylinder around the $z$ axis (i.e.~for $\rho>0$). Verify that $\curl\vA=0$.
  By considering a path along the arc of a circle of radius $\rho$ parallel to the $x{-}y$
  plane and centred on the z axis, show that for two points $P_1$ and $P_2$ of angular
  separation $\alpha$
  \[
    \intC \vA \,.\, \dd\v{r}
    \,=\,
    \left(2\pi n + \alpha\right) a
  \]
  where $C$ is a path joining $P_1$ to $P_2$ and $n$ is the number of times this path
  encircles the cylinder in the direction of increasing $\phi$. Why is this line integral
  not independent of the path between $P_1$ and $P_2$? How would you show that this result
  holds for \emph{any} path which wraps $n$ times around the cylinder?

  %---------------------------------------------------------------------------------------
  \Exercise Evaluate $\curl\v{a}$,
  where the vector field $\v{a}(\v{r})$ is given in Cartesians by
  \[
    \v{a}
    =
    \left(-x^2y - y^3\right) \, \v{e}_{\,x}
    +
    \left(x^3  + xy^2\right)  \, \v{e}_{\,y}
    +
    z^2  \, \v{e}_{\,z}
  \]

  \Question Show that in cylindrical coordinates
  $\v{a} = \rho^3 \, \ephi + z^2 \, \ez$.

  \Question Evaluate $\curl \v{a}$ in cylindrical coordinates, checking that your
  answer agrees with the result in Cartesian coordinates.

  \Question Show that
  \[
    \int_S \left( \curl \v{a} \right) \cdot \mbox{d}\v{S}
    ~=~
    2\pi \left( \beta^4 - \alpha^4 \right)
  \]
  where $S$ is the planar annular surface $\alpha \le \rho \le \beta$, $0 \le \phi < 2\pi$,
  $z = \gamma$, where $\alpha$, $\beta$ and $\gamma$ are positive constants.

  \Question  Use Stokes' theorem to check your result for (iii).

  \emph{Hint: The surface $S$ is bounded by the circles with radius
    $\rho=\alpha$ and $\rho=\beta$. In which direction does the line
    integral around the inner circle go?}
  %---------------------------------------------------------------------------------------
  \Exercise Parabolic cylindrical coordinates $(u, v, w)$ are defined by
  \[
    x = uv \,, \qquad
    y = \frac{1}{2}(v^{2}-u^{2}) \,, \qquad
    z = w \,
  \]
  where $(x, y, z)$ are Cartesian coordinates. Find the scale factors $h_u$, $h_v$ and
  $h_w\:\!$. Hence, find expressions for the curvilinear basis vectors $\{\v{e}_{\,u}, \,
    \v{e}_{\,v}, \, \v{e}_{\,w}\}\,$ in terms of $\ex$, $\ey$ and $\ez$, and show that they
  are mutually orthogonal.

  The vector field $\v{a}(\v{r})$ has components \( (a_{u}=v, \, a_{v}=-u, \, a_{w}=0) \)
  in the basis $\{\v{e}_{\,u}, \, \v{e}_{\,v}, \, \v{e}_{\,w}\}$,
  % \emph{i.e.},
  %so% that
  \[
    \v{a}(\v{r})
    =
    v  \, \v{e}_{\,u} - u \, \v{e}_{\,v}
  \]
  Show that $\Div \v{a} = 0$

  \item  Recall that the Laplacian $\nabla^2 f$ in spherical polars is given by
  \[
    \nabla^2 f
    ~ = ~
    \frac{1}{r^2}\,\frac{\partial}{\partial r}
    \left(r^2\, \frac{\partial f}{\partial r}\right)
    \,+\,
    \frac{1}{r^2\sin(\theta)} \:
    \frac{\partial}{\partial\theta}
    \left( \sin(\theta) \: \frac{\partial f}{\partial\theta}\right)
    \,+\,
    \frac{1}{r^2\sin^2\theta} \;
    \frac{\partial^2f}{\partial\phi^2} \;.
  \]

  In (super)natural units, the electrostatic potential $\Phi(\v{r})$ due to an electric
  dipole at the origin is given in spherical polars by
  \[
    \Phi(\v{r})
    ~=~
    \frac{\cos(\theta)}{r^2} \, ,
  \]
  where $r$ is the length of the position vector $\v{r}$. Show that, for $r\ne 0$,
  \[
    \nabla^2 \Phi
    \,=~
    0 \, .
  \]

  The wave function $\psi(\v{r})$ of a three-dimensional quantum harmonic oscillator in its
  ground state is given in spherical polars by
  \[
    \psi(\v{r})
    ~=~
    \exp \left( - r^2/2 \right) \,.
  \]
  Show that $\psi$ satisfies Schr\"odinger's equation (dimensions have been ignored for
  simplicity)
  \[
    -\nabla^2 \psi + r^2 \psi
    ~=~
    2E_0 \, \psi \,,
  \]
  and thus determine the value of the ground-state energy $E_0$.
  %---------------------------------------------------------------------------------------
  \Exercise A vector field $\v{A}$
  has components in spherical polar coordinates
  \begin{align*}
    A_{r}      & = r^{2}\cos(\theta)^2\sin(\theta)\cos(\phi) \,,  \\
    A_{\theta} & = -r^{2}\cos(\theta)\sin(\theta)^2\cos(\phi) \,, \\
    A_{\phi}   & = 0 \, .
  \end{align*}

  \Question Show that
  \subQuestion $\Div\v{A}=r\sin(\theta)\cos(\phi)$

  \subQuestion $\vB=\curl\v{A}$ has
  components
  \begin{align*}
    B_{r}      & =-r\sin(\theta)\cos(\theta)\sin(\phi) \,, \\
    B_{\theta} & =-r\cos(\theta)^2\sin(\phi) \,,           \\
    B_{\phi}   & = -r\cos(\theta)\cos(\phi) \,.
  \end{align*}
  \subQuestion $\nabla^{2}\v{A} = 0$
  \Question Find the components of $\v{A}$ in Cartesian coordinates, and comment on
  the suitability of spherical polars for this problem.

  \vfill
\end{ExerciseList}