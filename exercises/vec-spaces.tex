% !TEX root = ../fields.tex
%%%%%%%%%%%%%%%%%%%%%%%%%%%%%%%%%%%%%%%%%%%%%%%%%%%%%%%%%%%%%%%%%%%%%%%%%%%%%%%%%%%%%%%%%%
\section{Exercises}
\begin{ExerciseList}
  %---------------------------------------------------------------------------------------
  \Exercise A (plane) triangle is formed by the three vectors $\v{a}$,
  $\v{b}$ and $\v{c}$, such that
  \[
    \v{a} + \v{b} + \v{c} = 0 \; ,
  \]
  and the internal angles at the vertices are $\alpha$, $\beta$ and $\gamma$, so that the
  angles between the \emph{vectors} are $\pi - \alpha$ (between vectors $\v{b}$ and
  $\v{c}$), \ $\pi -\beta$ (between vectors $\v{c}$ and $\v{a}$) and $\pi -\gamma$ (between
  vectors $\v{a}$ and $\v{b}$). Show that $\v{a}\times\v{b} = \v{b}\times\v{c} =
    \v{c}\times\v{a}$, and hence establish the \emph{sine rule}:
  \[
    \frac{\sin\alpha}{|\v{a}|}
    \ = \
    \frac{\sin\beta}{|\v{b}|}
    \ = \
    \frac{\sin\gamma}{|\v{c}|}\;.
  \]

  Establish the \emph{cosine rule}:
  \[
    a^2 \ = \ b^2 + c^2 - 2bc\cos\alpha\;,
  \]
  where $a = |\v{a}|$, etc.

  \textit{Hint: Put $\v{a}=-\v{b}-\v{c}$ and evaluate $\v{a}\cdot\v{a}$.}
  %---------------------------------------------------------------------------------------
  \Exercise[label=bac-cab] Complete the proof, sketched in lectures, of the  identity
  \[
    \v{a}\times(\v{b}\times\v{c})
    \ = \
    \v{b} (\v{a}\cdot\v{c}) - \v{c} (\v{a}\cdot\v{b})
  \]
  Use this to deduce the identities

  \Question $(\v{a}\times\v{b})\times\v{c} \ = \
    (\v{a}\cdot\v{c})\v{b} - (\v{b}\cdot\v{c})\v{a}$;
  \Question
  $\v{a}\times(\v{b}\times\v{c}) + \v{b}\times(\v{c}\times\v{a}) +
    \v{c}\times(\v{a}\times\v{b}) \ = \ 0$;
  \Question $[\v{a}\times\v{b},\ \v{b}\times\v{c},\
        \v{c}\times\v{a}] \ = \ \stpr{a}{b}{c}^2$;
  %---------------------------------------------------------------------------------------
  \Exercise Establish the identities:

  \Question   $(\v{a}\times\v{b})\cdot(\v{c}\times\v{a}) \ =
    \ (\v{a}\cdot\v{c})(\v{b}\cdot\v{a}) - (\v{a}\cdot\v{a})(\v{b}\cdot\v{c})$~;
  \Question   $(\v{a}\times\v{b})\times(\v{c}\times\v{a}) \ =
    \ -\stpr{a}{b}{c}\,\v{a}$.
  %---------------------------------------------------------------------------------------
  \Exercise The velocity of a point particle, rotating with angular velocity
  vector $\v{\omega}$ through the origin, is $\v{v} = \v{\omega}\times
    \v{r}$.
  \Question
  The angular momentum vector of such a particle is defined as
  $\v{L} = m \v{r}\times \v{v}$ where $m$ is the mass.
  Show that $l$, the component of the angular momentum vector along the axis of rotation,
  is given by
  \[
    l = m \omega \left[ r^2 - | \v{r}\cdot \v{\hat{\omega}}|^2 \right]
  \]
  What is the geometrical meaning of the term in the square bracket?

  \Question The kinetic
  energy of such a particle is given by $K = \frac{m}{2} \v{v}^2$.
  Show that
  \[
    K
    =
    \frac{m w^2}{2} \left[ r^2 - | \v{r}\cdot \v{\hat{\omega}}|^2 \right]
  \]
  %---------------------------------------------------------------------------------------
  \Exercise Solve the following sets of equations for $\v{x}$, and give
  a geometrical interpretation to your result:

  \Question   $\v{a}\cdot\v{x} \ = \ l$\quad;
  \quad$\v{b}\times\v{x} \ = \ \v{c}$,\ \ where $\v{a}\cdot\v{b} \neq 0$~;
  \Question  $\v{a}\cdot\v{x} \ = \ l$\quad;
  \quad$\v{b}\cdot\v{x} \ = \ m$\quad;\quad
  $\v{c}\cdot\v{x} \ = \ n$, \ where $\stpr{a}{b}{c}\ \neq \ 0$.
  %---------------------------------------------------------------------------------------
  \Exercise Show that the solution of the vector equation
  \[k\v{x} + (\v{x}\times\v{a}) \ = \ \v{b}\;,\]
  where $k \neq 0$, is
  \[\v{x}(k^2 + |\v{a}|^2) \ = \ (\v{a}\times\v{b})
    + k\v{b} + \frac{(\v{a}\cdot\v{b})\v{a}}{k}\;.\]
  \textit{Hint: Take vector and scalar products of original equation with $\v{a}$.}
  %---------------------------------------------------------------------------------------
  \Exercise Show that the minimal distance of the point $\v{d}$ from the plane
  $\v{x}\cdot\v{a} = b$ is $|b - \v{d}\cdot\v{a}|/|\v{a}|$.
  %---------------------------------------------------------------------------------------
  \Exercise Show that the vectors $\v{A} = (1,2,3)$, $\v{B} = (1,-1,1)$ and
  $\v{C} = (0,1,1)$ (where the basis is right-handed orthonormal)
  are linearly independent.
  Express the vector $\v{D} = (2,1,1)$ in the form
  \[
    \v{D}
    \,=\,
    \alpha \, \v{A} \,+\, \beta \, \v{B} \,+\, \gamma \, \v{C}\,.
  \]%---------------------------------------------------------------------------------------
  \Exercise Find the general vector which is orthogonal to $\va$ and
  coplanar with $\va$ and $\vb$ where $\va= (8,2,-1)$ and $\vb=
    (3,-1,-3)$ in an orthonormal basis.
  %---------------------------------------------------------------------------------------
  \Exercise Verify that the components in the $\eone$ direction (of a right
  hand orthonormal basis) of the vectors $\v{A}\times(\v{B}\times\v{C})$
  and $\left[ \left(\v{A}\cdot\v{C}\right)\v{B} - \left(\v{A}\cdot\v{B}\right)\v{C} \right]$
  are equal for any $\v A, \v B, \v C$.
  %---------------------------------------------------------------------------------------
  \Exercise The vectors $\v{A}$, $\v{B}$, $\v{C}$ are non-coplanar.  Consider
  the reciprocal basis vectors defined as
  \[
    \v{A}^* \,=\, \frac{\v{B} \times \v{C}}{ \stpr{A}{B}{C}},\qquad
    \v{B}^* \,=\, \frac{\v{C} \times \v{A}}{ \stpr{A}{B}{C}},\qquad
    \v{C}^* \,=\,  \frac{\v{A} \times \v{B}}{\stpr{A}{B}{C}}\;.
  \]
  Show that

  \Question $\ \v{A}\cdot\v{A}^* \,=\, \v{B}\cdot\v{B}^* \,=\, \v{C}\cdot\v{C}^*
    \,=\, 1$

  \Question $\ \v{A}\cdot\v{B}^* \,=\, \v{A}\cdot\v{C}^* \,=\, \v{B}\cdot\v{A}^*
    \,=\, \cdots \,=\, 0$

  \Question $\left[\v{A}^*,\, \v{B}^*,\, \v{C}^*\right] \,=\,
    1/\stpr{A}{B}{C}$

  \Question $\v{A} \,=\, \left(\v{B}^*\times
    \v{C}^*\right) /\left[\v{A}^*,\, \v{B}^*,\, \v{C}^*\right]$
  %---------------------------------------------------------------------------------------
  \Exercise Let $\eone$, $\etwo$, $\ethree$ be an orthonormal basis

  \Question Show that the three vectors $\v{A} = \frac{1}{2}(\etwo+\ethree),\,
    \v{B}= \frac{1}{2}(\ethree+\eone),\, \v{C}= \frac{1}{2}(\eone+\etwo),$ form a basis (not orthonormal).

  Construct the reciprocal basis vector $\v{A}^{\ds\ast}$ defined by
  \[
    \v{A}^{\ds\ast}
    \,=\,
    \frac{ (\v{B} \times \v{C})}{\stpr{A}{B}{C}}
  \]
  and find its direction in the $\ei$ basis. Similarly, construct $\v{B}^{\ds\ast}$ and
  $\v{C}^{\ds\ast}$.

  \Question Sketch the original and reciprocal bases.
  %---------------------------------------------------------------------------------------
  \Exercise Consider three vectors $\v A = \sum_{i=1}^{3} A_i \,
    \v{w}_{\,i}$, \ $\v B = \sum_{i=1}^{3} B_i \, \v{w}_{\,i}$, \ $\v C = \sum_{i=1}^{3}
    C_i \, \v{w}_{\,i}$, where $\v{w}_{\,1}$, $\v{w}_{\,2}$, $\v{w}_{\,3}$ form a basis, which is not
  necessarily orthonormal.
  \Question Show  that the scalar triple product of the three
  vectors is given by
  \[
    \stpr{A}{B}{C}
    \,=\,
    \left|
    \begin{array}{ccc}
      A_1 & A_2 & A_3 \\
      B_1 & B_2 & B_3 \\
      C_1 & C_2 & C_3
    \end{array}
    \right| \,
    [\v{w}_{\,1},\v{w}_{\,2},\v{w}_{\,3}]
  \]
  \Question What does this formula reduce to $(a)$ in a right-handed
  orthonormal basis, and $(b)$ in a left-handed orthonormal basis?
  %---------------------------------------------------------------------------------------
  \Exercise Simplify the following expressions, where $\delt{ij}$ is the
  Kronecker delta symbol.
  \begin{tabbing}
    (iv)$^*$ $\sum_{j=1}^3\sum_{k=1}^3\delt{ij}\,\delt{jk}\,\delt{kl}$ \hspace{10mm}
    \= (ii) $\sum_{i=1}^3\sum_{j=1}^3A_i\,B_j\,\delt{ij}$ \hspace{10mm} \=
    (iii)   $\sum_{j=1}^3\delt{ij}\,\delt{jk}$  \kill
    (i)     $\sum_{i=1}^3A_i\,\delt{ij}$ \>
    (ii)    $\sum_{i=1}^3\sum_{j=1}^3A_i\,B_j\,\delt{ij}$  \>
    (iii)   $\sum_{j=1}^3\delt{ij}\,\delt{jk}$  \\
    \\
    (iv)  $\sum_{j=1}^3\sum_{k=1}^3\delt{ij}\,\delt{jk}\,\delt{kl}$
    \>
    (v)   $\sum_{i=1}^3 \sum_{j=1}^3 \sum_{k=1}^3 \sum_{l=1}^3 A_i \, B_j \, C_k \, D_l \, \delt{ij} \, \delt{kl}$
  \end{tabbing}
  %---------------------------------------------------------------------------------------
  \Exercise Repeat the previous exercise using the Einstein summation
  convention, \emph{i.e.} simplify
  \[
    \begin{array}{lllll}
      \mbox{(i) }	  A_{i}\,\delt{ij} \;                  &
      \mbox{(ii) }	  A_{i}\,B_{j}\,\delt{ij} \;          &
      \mbox{(iii) }   \delt{ij}\,\delt{jk} \;            &
      \mbox{(iv) } 	  \delt{ij}\,\delt{jk}\,\delt{kl} \; &
      \mbox{(v) }	  A_{i}\,B_{j}\,C_{k}\,D_{l}\,\delt{ij}\,\delt{kl}
    \end{array}
  \]
  where there is an implicit sum over each pair of dummy indices.
  %---------------------------------------------------------------------------------------
  \Exercise
  If a matrix $A$ has components $a_{ij}$ given by
  \[
    \ds
    \left(
    \begin{array}{rrr}
        1  & -1 & 0 \\
        -2 & 3  & 1 \\
        2  & 0  & 4
      \end{array}
    \right)
  \]
  show, assuming the summation convention, that
  \[
    \begin{array}{ccc}
      a_{ii} = 8;           & a_{1i} \, a_{i2} =-4; & a_{i2} \, a_{i3} =3;   \\[1ex]
      a_{1i} \, a_{2i} =-5; & a_{i2}  a_{3i} =-2;   & a_{ij} \, a_{ji} = 30.
    \end{array}
  \]

  Write down the components of the matrix given by $a_{ik} \, \delt{kj}$. (Hint: this is
  easy.)
  % and does not require any arithmetic)
  %---------------------------------------------------------------------------------------
  \Exercise Assuming the summation convention, evaluate
  $\eps{ijk}\,\eps{pqk}$, and compare with the value of
  $(\delt{ip}\,\delt{jq} - \delt{iq}\,\delt{jp})$ in the following cases
  \Question  $i=1$ ; $j=2$ ; $p=1$ ; $q=2$ ;
  \Question   $i=1$ ; $j=2$ ; $p=2$ ; $q=1$ ;
  \Question   $i=1$ ; $j=1$ ; $p$ and $q$ have any values (1, 2 or 3).
  %---------------------------------------------------------------------------------------
  \Exercise Given the identity
  \[
    \eps{ijk}\,\eps{pqk}
    \,=\,
    \delt{ip}\,\delt{jq} - \delt{iq}\,\delt{jp}\;,
  \]
  where the summation convention is implied, show
  % (by putting $q=j$ and summing $j$ from 1 to 3, {\it etc.\/})
  that \Question $\eps{ijk}\,\eps{pjk} \,=\, 2\delt{ip}$ ; \Question $\eps{ijk}\,\eps{ijk}
    \,=\, 6$ ; \Question $\eps{piq}\,\eps{qjr}\,\eps{rkp} \,=\, -\eps{ijk}$ ;

  %\parthh        $\eps{piq}\,\eps{qjr}\,\eps{rks}\,\eps{slp}
  %\ = \ \delt{ij}\,\delt{kl} + \delt{il}\,\delt{jk}$ .
  %---------------------------------------------------------------------------------------
  \Exercise Using the suffix notation and summation convention, show that
  \Question  $(\v{a}\times \v{b})\cdot (\v{a}\times \v{b}) \, \, =a^2 b^2-(\v{a}\cdot\v{b})^2$ ;
  \Question  $(\v{a}\times \v{b})\times (\v{c}\times \v{d}) \,=\, \stpr{a}{b}{d} \, \v{c}-\stpr{a}{b}{c} \, \v{d}$.
  %---------------------------------------------------------------------------------------
  \Exercise Construct the coefficients $\ell_{ij}$ of the transformation
  matrix $L$, and calculate its determinant for
  \Question a reflection of basis in the $\eone {-} \etwo$ plane;
  \Question a rotation of basis about the $\etwo$ axis where a positive
  rotation is taken to be in right-handed screw direction.
  %---------------------------------------------------------------------------------------
  \Exercise\hfill\\[-3ex]
  \Question Construct a \emph{right-handed orthonormal} basis in which the first vector
  $\eone'$ is parallel to $\v{a} = (1,1,1)$, and the second vector $\etwo'$ is
  parallel to $\v{b} = (1,-1,0)$;

  \Question Construct a \emph{left-handed orthonormal} basis in which the first vector $\eone''$ is parallel
  to $\v{c} = (4,-3,12)$, and the third vector $\ethree''$ is parallel
  to $\v{d} = (-3,12,4)$. \{Note that $3^2+4^2+12^2=13^2$.\}

  \Question Calculate the coefficients $\ell_{ij}$ of the transformation matrix
  from the $\ei'$~basis to the $\ei''$~basis
  (using $\ell_{ij} = \ei'' \cdot \ej'$).
  Verify that its determinant is $-1$.

  \Exercise By constructing the relevant transformation matrices, and
  checking that they satisfy the relation $L L^T=I$, verify that the
  following triads of vectors form {\it orthonormal} systems, given that
  the $\ei$~basis is orthonormal.\hfill\\
  \begin{minipage}[b]{17em}
    \begin{eqnarray*}
      \eone'   & = & \sfrac{1}{\sqrt{3}}\left(\eone + \etwo + \ethree\right)\\
      \etwo'   & = & \sfrac{1}{\sqrt{2}}\left(\eone - \etwo\right)\\
      \ethree' & = & \sfrac{1}{\sqrt{6}}\left(\eone + \etwo - 2\ethree\right)
    \end{eqnarray*}
  \end{minipage}\qquad
  \begin{minipage}[b]{17em}
    \begin{eqnarray*}
      \eone''   & = & \sfrac{1}{5\sqrt{3}}\left(\eone +  7\etwo + 5\ethree\right)\\
      \etwo''   & = & \sfrac{1}{5\sqrt{2}}\left(7\eone - \etwo\right)\\
      \ethree'' & = & \sfrac{1}{5\sqrt{6}}\left(\eone + 7\etwo - 10\ethree\right)
    \end{eqnarray*}
  \end{minipage}\hfill\\[3ex]
  Find the transformation matrix $N$ from the $\ei'$~basis to the $\ei''$~basis, where ${\ei'' = n_{ij} \, \ej'}$.
  Is this transformation proper or improper?
  %---------------------------------------------------------------------------------------
  \Exercise Let $\v{a} = \alpha (\eone +\etwo)$ and $\v{b}=\beta (\etwo-\eone)$
  and let the basis $\{\ei\!'\}$ be related to the basis $\{\ei\}$ through
  \begin{eqnarray*}
    \eone\!'   & = & \sfrac{1}{3}\left(2\eone + \etwo  -2\ethree \right)\\
    \etwo\!'   & = & \sfrac{1}{3}\left(2\eone - 2\etwo +\ethree\right)\\
    \ethree\!' & = & \sfrac{1}{3}\left(\eone + 2\etwo + 2\ethree\right)
  \end{eqnarray*}

  \Question Construct the transformation matrix. \\Is the transformation proper or improper?

  \Question Calculate the transformed components of $\v{a}, \v{b}$.

  \Question Calculate the components of $\v{a} \times \v{b}$: \emph{first} by using the transformed
  components of $\v{a}, \v{b}$ to calculate the vector product; \emph{secondly} by
  calculating the components of $\v{a} \times \v{b}$ in the old basis then transforming.
  Comment on your results.
  %---------------------------------------------------------------------------------------
  \Exercise The two sets of basis vectors $\{\ei\}$ and $\{\ei\!'\}$ are
  both right-handed orthonormal triads such that $\eone\!'$ is in the
  direction of $(\eone+\ethree)$ and $\etwo\!'$ is in the direction of
  $(\eone+\etwo-\ethree)$.

  \Question Construct the correctly normalised basis vectors $\eone\!', \, \etwo\!', \, \ethree\!'$.

  \Question Write down the transformation matrix $L$ from the basis $\{\ei\}$ to the basis
  $\{\ei\!'\}$ and verify that $\det L = +1$.

  \Question Express the two vectors $\v{a}=2\eone+\etwo$ and $\v{b}=\eone-\etwo-\ethree$ in the basis
  $\{\ei\!'\}$.

  \Question Verify that the scalar product of these two vectors is an invariant of the
  transformation.

  \Question Verify that the components of the vector product, evaluated in the two bases, are related
  as follows
  \[
    (\v{a}\times\v{b})_{i}'
    \,=\,
    (\det L) \,\ell_{ij}\,(\v{a}\times\v{b})_{j}.
  \]

  %---------------------------------------------------------------------------------------
  \Exercise By considering two successive rotations through angles $\theta$,
  $\phi$,  about, for example, the $\etwo$ axis, establish
  the identities
  \begin{eqnarray*}
    \cos(\theta+\phi) = \cos\theta \cos\phi - \sin\theta \sin\phi,\\
    \sin(\theta+\phi) = \sin\theta\cos\phi +\cos\theta\sin\phi.
  \end{eqnarray*}

  \Exercise
  Consider the following rotations and find
  \Question the transformation matrix for a rotation through angle $\theta$ about the $x$-axis;
  \Question the transformation matrix for a rotation through angle $\phi$ about the $y$-axis;
  \Question the combined transformation matrix of first a rotation through angle $\theta$ about the
  $x$-axis followed by a rotation through angle $\phi$ about the $y$-axis. Verify that its
  determinant is $+1$.
  %---------------------------------------------------------------------------------------
  \Exercise
  Using the definition of the determinant
  \[
    \det A
    \,=\,
    \epsilon_{ijk} \, a_{1i} \, a_{2j} \, a_{3k}
    \,=\,
    \epsilon_{ijk} \, a_{i1} \, a_{j2} \, a_{k3},
  \]
  show that
  \[
    \epsilon_{ijk} \det A
    \,=\,
    \epsilon_{lmn} \, a_{li} \, a_{mj} \, a_{nk}
    \,=\,
    \epsilon_{lmn} \, a_{il} \, a_{jm} \, a_{kn}.
  \]

  Hence, show, using the definition of the determinant and this latter result, that
  \[
    \det{AB} \,=\, \det{A}\,\det{B}.
  \]
  \vfill
\end{ExerciseList}
