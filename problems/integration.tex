% !TEX root = ../fields.tex
%%%%%%%%%%%%%%%%%%%%%%%%%%%%%%%%%%%%%%%%%%%%%%%%%%%%%%%%%%%%%%%%%%%%%%%%%%%%%%%%%%%%%%%%%%
\begin{Exercise}[title={Divergence theorem on a cone},name={Problem},label=divcone]
  The surface $S_C$ of a cone can be parametrised by
  spherical polar coordinates $(r,\theta,\phi)$, where
  \(
  0 \le r \le b \,,
  \;
  0 \le \phi < 2\pi \,,
  \;
  \theta = \theta_0 = \mbox{constant} < \pi/2\, .
  \)

  \Question Sketch the surface $S_C$.

  \Question Show that the vector element of area on $S_C$ is given by
  \[
    \diff \v{S}
    \,=\,
    r \sin(\theta_0) \, \diff r \, \diff \phi \, \v{e}_{\,\theta} \,,
  \]
  where \( \v{e}_{\,\theta} \,=\, \cos(\theta_0) \cos(\phi) \: \ex + \cos(\theta_0)
  \sin(\phi) \: \ey - \sin(\theta_0) \: \ez \, . \)

  \Question The vector field
  $\v{a}(\v{r})$ is given by \( \v{a}(\v{r}) = z^2 \, \ez \, . \) Show that on $S_C$
  \[
    \v{a} \cdot \diff \v{S}
    \,=\,
    -r^3 \sin(\theta_0)^2 \cos(\theta_0)^2 \, \diff r \, \diff \phi \,,
  \]
  and hence that
  \[
    \int_{S_C} \v{a} \cdot \diff \v{S}
    \,=\,
    -\frac{\pi b^4}{2} \sin(\theta_0)^2 \, \cos(\theta_0)^2 \,.
  \]

  \Question We consider a curved surface $S_I$ closing the top of the cone forms, part of a sphere of
  radius $b$ and centre $\v{0}$. $S_I$ can be parametrised by the angles $\theta$ and $\phi$ in spherical
  polars, so that the vector element of area on $S_I$ is
  \[
    \diff \v{S}
    \,=\,
    b^2 \sin(\theta) \, \diff \theta \, \diff \phi \, \v{e}_{\,r} \,,
  \]
  where $0 \le \theta \le \theta_0 \, , 0 \le \phi < 2\pi$.
  %
  Show that
  \[
    \int_{S_I} \v{a} \cdot \diff \v{S}
    \,=\,
    \frac{\pi b^4}{2} \left[1 - \cos(\theta_0)^4 \right] \, .
  \]

  \Question Use the divergence theorem to check the \emph{sum} of the answers to questions 3 \& 4.\\
  \emph{Hint: think carefully about the integral over the whole volume.}

\end{Exercise}