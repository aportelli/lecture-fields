% !TEX root = ../../fields.tex
%% Brian Lecture 8 %%%%%%%%%%%%%%%%%%%%%%%%%%%%%%%%%%%%%%%%%%%%%%%%%%%%%%%%%%%%%%%%%%%%%%%
\subsection{Level surfaces or equipotentials of a scalar field}

If $\phi(\v{r})$ is a non-constant scalar field, then the equation $\phi(\v{r}) = c$
where $c$ is a constant, defines a \emph{level surface} of the field. Different level
surfaces do not intersect, or $\phi$ would be multi-valued at the point of intersection.

Familiar examples in two dimensions are contours of constant height on a geographical
map, $h(x_1,x_2) = c$, which are of course level curves rather than level surfaces.
Isobars on a weather map are level curves of pressure $p(x_1,x_2)=c$.

In physics, when the field is a scalar potential (see later) the level surfaces are known
as \emph{equipotentials}.

\paragraph{Examples in three dimensions:}

\begin{enumerate}
  \item Let \( \; \phi(\v{r}) = r^2 \equiv x_1^2+x_2^2+x_3^2 = x^2 + y^2 + z^2
        %\equiv r_1^2 + r_2^2 + r_3^2
        \)

        The level surface $\phi(\v{r}) = r_0^2$ is a sphere of radius $r_0$ centred on the
        origin.

        As $r_0$ is varied, we obtain a family of level surfaces or equipotentials which are
        concentric spheres.

  \item The electrostatic potential at $\v{r}$ due to a point charge $q$ situated at the point
        $\v{a}$ is
        \[
          \phi(\v{r}) = \frac{q}{4\pi\epsilon_0} \; \frac{1}{|\v{r} -\v{a}|}
        \]
        where $|\v{r} -\v{a}|$ denotes the length of the vector $\v{r} -\v{a}\,$:
        \[
          |\v{r} -\v{a}|
          = \sqrt{(\v{r} -\v{a}) \cdot (\v{r} -\v{a})}
          = \sqrt{(x_1-a_1)^2 + (x_2-a_2)^2 + (x_3-a_3)^2}
        \]
        The equipotentials or level surfaces $\phi(\v{r}) = \mbox{constant}$ are concentric
        spheres centred on the point $\v{a}$, as shown in the figure.

        %\vspace*{-1ex}

        \centerline{
          \epsfxsize=0.35\textwidth
          \includegraphics{tikz_coulomb_pot.pdf}
        }
        If $\v{a}=0$, the equipotentials are centered on the origin.

  \item Let $\phi(\v{r}) = \vk\cdot\v{r}$.

        The level surfaces are planes $\vk\cdot\v{r}= \mbox{\emph{constant}}$, with $\vk$ normal
        to the planes.

  \item Let $\phi(\v{r}) = \exp(i \vk\cdot\v{r})$, which is a complex scalar field.

        Since $\vk\cdot\v{r} = \mbox{\emph{constant}}$ is the equation for a plane, the level
        surfaces are again planes.

\end{enumerate}
%\newpage
\subsection{Gradient of a scalar field}
\label{grad_scalar}

How do we describe mathematically the variation of a scalar field as a function of
position?

As an example, think of a 2-d contour map of the height $h=h(x_1,x_2)$ of a hill.
$h(x_1,x_2)$ is a scalar field. If we are on the hill and move in the $x_1{-}x_2$ plane
then the change in height will depend on the direction in which we move (unless the hill
is completely flat!) For example there will be a direction in which the height increases
most steeply: `straight up the hill'. We now introduce a formalism to describe how a
scalar field $\phir$ changes as a function of position $\v{r}$.

We begin by recalling Taylor's theorem and the definition of partial derivatives.

\paragraph{Taylor's theorem:} Recall that if $f(x)$ is an analytic function of a single variable $x$, Taylor's theorem
states that $f(x+\delta x)$ can be expanded in powers of $\delta x$ about $x$.
\[
  f(x+\delta x)
  =
  f(x)
  +
  \delta x \, \frac{df(x)}{dx}
  +
  \frac{(\delta x)^2}{2!} \, \frac{d^2f(x)}{dx^2}
  + \cdots +
  \frac{(\delta x)^n}{n!}  \, \frac{d^nf(x)}{dx^n}
  + \cdots
\]
If $\delta x$ is very small, we may approximate\ \ \( \displaystyle f(x+\delta x) = f(x)
+ \delta x \, \frac{df(x)}{dx} + O\left((\delta x)^2\right) \)

\paragraph{Partial derivatives:} If $f(x_1,x_2,x_3)$ is a function of the three \emph{independent} variables $x_1$, $x_2$
and $x_3$, then the \emph{partial derivative}
\[
  \frac{\partial f(x_1,x_2,x_3)}{\partial x_1}
\]
is obtained by differentiating $f(x_1,x_2,x_3)$ with respect to $x_1$, whilst keeping
$x_2$ and $x_3$ \emph{fixed}. Similarly for the partial derivatives with respect to $x_2$
and $x_3$.

\paragraph{Mathematical aside:} A scalar field $\phir = \phi(x_1, x_2, x_3)$ is said to be \emph{continuously
  differentiable} in a region $R$ if its partial derivatives
\[
  \parpar{\phir}{x_1}\,, \hspace*{4ex}
  \parpar{\phir}{x_2}\, \hspace*{3ex} {\rm and} \hspace*{3ex}
  \parpar{\phir}{x_3}
\]
\emph{exist}, and are \emph{continuous} at every point $\v{r}\in R$.
We will generally assume scalar fields are continuously
differentiable.

Let $\phi(\v{r})$ be a scalar field, and consider $2$ nearby points: $P$ with position
vector $\v{r}$, and $Q$ with position vector $\v{r}+\delta\v{r}$, where $\delta\v{r}$ has
components $(\delta x_1, \delta x_2, \delta x_3)$. Assume $P$ and $Q$ lie on
\emph{different} level surfaces as shown:

\bigskip

\centerline{
  \epsfxsize=0.5\textwidth
  \includegraphics{tikz_level_1.pdf}
}

Now use Taylor's theorem to first order in each of the 3 variables $x_1$, $x_2$ and $x_3$
to evaluate the change in $\phi$ as we move from $P$ to $Q$
\begin{eqnarray*}
  \delta \phi & \equiv & \phi(\v{r}+\delta \v{r}) -\phi(\v{r}) \\[0.5ex]
  & = &
  \phi(x_1+\delta x_1, \, x_2+\delta x_2, \, x_3+\delta x_3)
  - \phi(x_1, \, x_2, \, x_3) \\[0.5ex]
  %  & = &
  %    \phi(x_1, \, x_2+\delta x_2, \, x_3+\delta x_3)
  %    + \frac{\partial\phi(\v{r})}{\partial x_1}\,\delta x_1
  %    - \phi(x_1, \, x_2, \, x_3) + \ldots \\[0.5ex]
  %  & = &
  %    \phi(x_1, \, x_2, \, x_3+\delta x_3)
  %    + \frac{\partial\phi(\v{r})}{\partial x_1}\,\delta x_1
  %    + \frac{\partial\phi(\v{r})}{\partial x_2}\,\delta x_2
  %    - \phi(x_1, \, x_2, \, x_3) + \ldots \\[0.5ex]
  & = &
  \left[
    \phi(x_1, \, x_2, \, x_3)
    + \frac{\partial\phi(\v{r})}{\partial x_1}\,\delta x_1
    + \frac{\partial\phi(\v{r})}{\partial x_2}\,\delta x_2
    + \frac{\partial\phi(\v{r})}{\partial x_3} \, \delta x_3
    + O(\delta\x{i} \, \delta\x{j})
    \right]
  - \phi(x_1, \, x_2, \, x_3)
  \\[0.5ex]
  & = &
  \frac{\partial\phi(\v{r})}{\partial x_1} \, \delta x_1
  + \frac{\partial\phi(\v{r})}{\partial x_2} \, \delta x_2
  + \frac{\partial\phi(\v{r})}{\partial x_3} \, \delta x_3
  + O(\delta\x{i} \, \delta\x{j})
\end{eqnarray*}
where we assumed that the higher order partial derivatives exist.
Neglecting these higher order terms, we write
\[
  \bigbox{$\delta \phi = \grad \phi \cdot \delta\v{r}$}
\]
where the $3$ quantities
\[
  \boxed{
    \left(\grad \phi\right)_i
    ~\equiv~
    \frac{\partial\phi}{\partial \x{i}}
    ~\equiv~
    \partial_i \phi}
\]
form the Cartesian components of a \emph{vector field}
\[
  \bigbox{$ \ds
      \grad \phi(\v{r})
      \equiv
      \frac{\partial\phi}{\partial x_1} \, \eone
      +
      \frac{\partial\phi}{\partial x_2} \, \etwo
      +
      \frac{\partial\phi}{\partial x_3} \, \ethree
      ~=~
      \frac{\partial\phi}{\partial \x{i}} \, \ei
    $}
\]
where we used the summation convention in the last line, there is an implicit sum over
the dummy index $i$. (See later for a derivation of the transformation properties of
$\grad \phi$.)

% Recall that the \emph{partial derivative} $\partial\phi/\partial \x{1}$
% is the derivative of $\phi(\v{r})$ with respect to $x_1$, keeping
% $x_2$ and $x_3$ fixed, \emph{etc}.

In `$xyz$' notation
\[
  \grad \phi
  =
  \frac{\partial\phi}{\partial x}\, \v{e}_{\,x}
  + \frac{\partial\phi}{\partial y}\, \v{e}_{\,y}
  + \frac{\partial\phi}{\partial z}\, \v{e}_{\,z}
\]

The \emph{vector field} $\grad\phi(\v{r})$ is called the \emph{gradient} of
$\phi(\v{r})$, and is pronounced \emph{`grad phi'}.

Let's do few examples in `longhand' $(x_1$, $x_2$, $x_3)$ notation before switching to
suffix notation with the summation convention.

\paragraph{Example:} Calculate the gradient of the scalar field $\ds \phi(\v{r}) = r^2 = x_1^2 + x_2^2 +
  x_3^2\,.$

First, recall that the \emph{partial derivative} $\partial\phi/\partial \x{1}$ is the
derivative of $\phi(\v{r})=\phi(x_1,x_2,x_2)$ with respect to $x_1$, keeping $x_2$ and
$x_3$ \emph{fixed}, \emph{etc}.
\[
  \mbox{So} \quad
  \frac{\partial{x_1}}{\partial{x_1}} = 1
  \quad \mbox{and} \quad
  \frac{\partial{x_2}}{\partial{x_1}} = 0\,.
  \quad\mbox{Similarly}\quad
  \frac{\partial{x_1^2}}{\partial{x_1}} = 2 x_1
  \quad \mbox{and} \quad
  \frac{\partial{x_2^2}}{\partial{x_1}} = 0\,,
  \quad \mbox{etc.}
\]
The first component of $\grad r^2$ is then
\[
  \frac{\partial}{\partial x_1} (x_1^2 + x_2^2 + x_3^2)
  =
  2x_1 + 0 + 0
  %   \equiv
  %   (\grad r^2)_1
\]
Similarly for the $2^{\rm nd}$ and $3^{\rm rd}$ components (exercise), and hence
\[
  \grad r^2
  =
  2 x_1 \, \eone + 2 x_2 \, \etwo + 2 x_3 \, \ethree
  =
  2\v{r}
\]
The vector $\grad r^2 = 2\v{r}$ points radially outwards from the origin with
\emph{magnitude} $2r$. The level surfaces, $r^2=\mbox{constant}$, are spheres centred on
the origin.

\paragraph{Example:} Calculate the gradient of $\ds \phi(\v{r}) = \sin x_1 + 2x_1x_2^2$
\[
  \frac{\partial{\phi}}{\partial{x_1}} = \cos x_1 + 2x_2^2 \,,
  \qquad
  \frac{\partial{\phi}}{\partial{x_2}} = 0 + 4x_1 x_2 \,,
  \qquad
  \frac{\partial{\phi}}{\partial{x_3}} = 0 + 0
\]
\[
  \Rightarrow \quad
  \grad \phi
  =
  (\cos x_1 + 2x_2^2) \, \eone
  +
  4x_1 x_2 \, \etwo
\]

\paragraph{Example:} Calculate $\grad(\v{a}\cdot\v{r})$ where $\v{a}$ is a constant vector. The first
component is
\[
  \frac{\partial}{\partial x_1} (a_1 x_1 + a_2 x_2 + a_3 x_3)
  =
  a_1 + 0 + 0
\]
Similarly for the other two components. Hence
\[
  \grad(\v{a}\cdot\v{r})
  =
  a_1 \, \eone + a_2 \, \etwo + a_3 \, \ethree
  =
  \v{a}
\]
This is a very important result -- as we shall see.

\textbf{NB:} An important result for partial derivatives is
\[
  \frac{\partial{x_i}}{\partial{x_j}} = \delta_{ij}
  \quad \mbox{which holds for all $i,j=1,2,3.$}
\]

\subsection{Interpretation of the gradient}
In deriving the expression for $\delta\phi$ above, we assumed that the points $P$ and $Q$
lie on \emph{different} level surfaces. Now consider the situation where $P$ and $Q$ are
nearby points on the \emph{same} level surface. In that case, $\delta\phi=0$ and so
\[
  \delta\phi = \grad\phi \cdot \delta\v{r} = 0
\]

\begin{center}
  \epsfxsize=9cm
  \includegraphics{tikz_grad_normal.pdf}
\end{center}
The infinitesimal vector $\delta\v{r}$ lies in the level surface at
$\v{r}$, and the above equation holds for all such $\delta\v{r}$,
hence
\begin{center}
  \bigbox{
    \vspace*{-2ex}
    \parbox{85mm}{
      $\grad\phi(\v{r})$ is \emph{normal to the level surface at $\v{r}$}.
    }
  }
\end{center}

To construct a \emph{unit normal} $\vn(\v{r})$ to the level surface at $\v{r}$, we divide
$\grad\phi$ by its length
\[
  \vn(\v{r}) ~=~ \frac{\grad\phi}{|\grad\phi|}
  \qquad
  \left(\mbox{for} \quad |\grad\phi| \neq 0\right)
\]

\mnote{13L 24/02/09}

\subsection{Directional derivative}
Consider the change, $\delta \phi$, produced in $\phi(\v{r})$ by moving a distance
$\delta s$ in the direction of the unit vector $\v{\hat{s}}$, so that $\delta\v{r} =
  \delta s \, \v{\hat{s}}$. Then
\[
  \delta \phi
  =
  \grad\phi \cdot \delta\v{r}
  =
  (\grad\phi) \cdot \v{\hat{s}} \;\delta s
\]
As $\delta s \rightarrow 0$, the rate of change of $\phi$ as we move in the direction of
$\v{\hat{s}}$ is
\begin{equation}
  \frac{d\phi}{ds}
  =
  \v{\hat{s}} \cdot \grad\phi
  =
  |\grad\phi| \, \cos\theta
  \label{eq:dir-deriv}
\end{equation}
where $\theta$ is the angle between $\v{\hat{s}}$ and the normal to the
level surface at $\v{r}$.
\begin{center}
  \bigbox{
    $\v{\hat{s}} \cdot \grad\phi$ is called the \emph{directional
      derivative} of the scalar field $\phi$ in the direction of
    $\v{\hat{s}}$
  }
\end{center}
The directional derivative has its \emph{maximum} value when
$\v{\hat{s}}$ is \emph{parallel} to $\grad\phi$, and is \emph{zero}
when $\delta s \, \v{\hat{s}}$ lies in the level surface. Therefore
\begin{center}
  \bigbox{$\grad \phi$ points in the direction of the \emph{maximum}
    rate of increase in $\phi$}
\end{center}
Recall that this direction is normal to the level surface.  A familiar
example is that of contour lines on a map: the steepest direction is
perpendicular to the contour lines.

%        Equation~(\ref{eq:dir-deriv}) can be used as a \emph{coordinate
%free} \emph{definition} of $\grad\phi(\v{r})$.

\paragraph{Example:} Find the directional derivative of $\phi(\v{r})=xy(x+z)$ at the point $(1, 2, -1)$ in the
direction of $(\v{e}_{\,x} +\v{e}_{\,y})/\sqrt{2}$.
\[
  \grad \phi
  =
  (2xy + yz) \, \v{e}_{\,x} + x(x+z) \, \v{e}_{\,y} + xy \, \v{e}_{\,z}
  =
  2 \v{e}_{\,x} + 2 \v{e}_{\,z} \qquad \mbox{at} \quad (1,2,-1)
\]
Thus at this point
\[
  \frac{1}{\sqrt{2}}(\v{e}_{\, x} +\v{e}_{\,y})\cdot \grad \phi
  =
  \sqrt{2}
\]

\mnote{12L 15/02/08}

\paragraph{Physical example:}
Let $T(\v{r})$ be the temperature of the atmosphere at the point $\v{r}$. An object flies
through the atmosphere with velocity $\v{v}\,$. Obtain an expression for the rate of
change of temperature experienced by the object.

As the object moves from $\v{r}$ to $\v{r}+\delta\v{r}$ in time $\delta t$, it sees a
change in temperature
\[
  \delta T
  ~=~ \grad T \cdot \delta\v{r}
  ~=~
  \left( \grad T \cdot \frac{\delta\v{r}}{\delta t}\right) \delta t
\]
Taking the limit $\delta t \to 0$, we obtain
\[
  \frac{dT(\v{r})}{dt} =   \vv \cdot \grad T(\v{r})
\]
The RHS is referred to as a \emph{convective derivative}.
%% Brian Lecture 9 %%%%%%%%%%%%%%%%%%%%%%%%%%%%%%%%%%%%%%%%%%%%%%%%%%%%%%%%%%%%%%%%%%%%%%%
\subsection{Examples of the gradient in physical laws}

A fundamental idea in physics is that we can express many forces in the form
\begin{center}
  \bigbox{$  \v{F} = -\grad V $}
\end{center}
where $V$ is the potential energy.

Such a force is called a {\em conservative} since, as we shall prove later on, Newton's
law then implies that the total energy (kinetic plus potential) is conserved.

\paragraph{Gravitational force due to the Earth:} The potential energy of a particle of mass $m$ at a height $z$ above the Earth's surface
is $V = mgz$. The force due to gravity can be written as
\[
  \v{F} = -\grad V = -mg \, \v{e}_{\,z}
\]

\paragraph{Newton's Law of Gravitation:} Now consider the gravitational force on a mass $m$ at $\vr$ due to a mass $m_0$ at the
origin. We can write this as
\[
  \v{F}
  \,=\,
  -\frac{G m m_0}{r^2} \: \v{\hat{r}}
  \,=\,
  - \grad V
\]
where the potential energy $V = -Gm m_0/r$ and $\v{\hat{r}}$ is a unit vector in the
direction of $\v{r}$. We shall show below that $\grad(1/r)= - \v{r}/r^3$.

In these two examples we see that the force acts down the potential energy gradient.

\paragraph{Electrostatic potential:}
We will see that we may express the electrostatic force $\v{F}(\v{r})$ experienced by a
point charge $q$ as
\[
  \v{F}(\v{r}) = q \v{E}(\v{r})
\]
If the force is conservative we deduce that we can express the electric field in terms of
an electrostatic potential $\phi$:
\begin{center}
  \bigbox{$\ds  \v{E} = -\grad \phi $}
\end{center}
where $V = q \phi$. (A comment for later: it is important to
distinguish the potential energy $V$ from the electrostatic potential
$\phi$.)

\paragraph{Diffusion:} The idea of diffusion: for example in a gas the molecular motion effectively smoothes out
the density. This can be described by the current of particles $\v{j}(\v{r})$ being
proportional to the gradient of the density $n(\v{r})$ (Fick's Law), \emph{i.e.}~the
diffusion current flows down the concentration gradient
\[
  \v{j}(\v{r}) \,=\,  - D \, \grad n(\v{r})
\]
where $D$ is a constant.

\subsection{Examples on gradient}\label{sec:examples-on-gradient}
The previous example on directional derivatives used the \emph{`xyz'} notation. This gets
unwieldy for more complicated examples, and suffix notation is more convenient.

% At this point we have two choices. We can calculate the $i^{\rm th}$
% component $(\grad\phi)$ of the gradient, or we can evaluate
% $\grad\phi$ directly. We shall use both methods.

\begin{enumerate}
  \item Let $\phi(\vr) = r^2 = x_1^2 + x_2^2 + x_3^2$, then
        \[
          (\grad r^2)_i
          \,=\,
          {\frac{\partial}{\partial x_i}} (x_1^2 + x_2^2 +x_3^2 )
          \,=\,
          2x_i \qquad \mbox{or} \qquad \grad r^2 = 2\v{r}
        \]
        Or, using the shorthand $\partial/\partial\x{i}\equiv\partial_i$ and the summation
        convention to write $r^2=x_jx_j$
        \[
          (\del r^2)_i \,=\, \partial_i\, r^2
          \,=\, \partial_i\, (x_jx_j)
          \,=\, \delt{ij}\x{j} + \x{j}\delt{ij}
          \,=\, 2\x{i}
        \]
        where we used the important property of partial derivatives
        \begin{center}
          \bigbox{$
              %  \parbox{20mm}{
              \ds\frac{\partial\x{j}}{\partial\x{i}} \,=\, \delt{ij}  \,=\, \partial_i x_j
            $}
        \end{center}
        The level surfaces of $r^2$ are spheres centred on the origin, and the
        gradient of $r^2$ at $\vr$ points radially outward with magnitude $2r$.

  \item Let $\phi = \va\cdot\vr$ where $\va$ is a \emph{constant} vector.
        \[
          (\grad (\va \cdot \vr))_i
          \,=\, \partial_i (a_jx_j)
          \,=\, a_j \delta_{ij} \,=\, a_i
        \]
        This is not surprising, since the equipotentials $\va\cdot\vr = c$ are planes orthogonal
        to $\va$.

  \item Let $\phi(\vr) \,=\, r \,=\, \sqrt{x_1^2 + x_2^2 + x_3^2} \,=\, (\x{j}\x{j})^{1/2}$
        \begin{eqnarray*}
          (\grad r)_i
          & = &
          \partial_i \; (\x{j} \x{j})^{1/2} \\[0.5ex]
          & = &
          \frac{1}{2}\, (\x{j} \x{j})^{-1/2} \; \partial_i (\x{k} \x{k})
          \qquad \mbox{(chain rule)}\\[0.5ex]
          & = &
          \frac{1}{2r}\; 2\, \x{i}\\[0.5ex]
          & = &
          (\v{\hat{r}})_i
          \qquad \mbox{so} \qquad
          \bigbox{$\ds \grad r \,=\, \frac1r \: \v{r} \,=\, \v{\hat{r}}$}
        \end{eqnarray*}
        The gradient of the length of the position vector is the unit vector
        pointing radially outwards from the origin.  It is normal to the level
        surfaces which are spheres centered on the origin.
\end{enumerate}

\subsection{Identities for gradients}\label{sec:identitiesforgradients}

Thus far, we have calculated gradients from first principles -- we worked out each
example from scratch, and we calculated the $i^{\rm th}$ component of $\grad\phi$ in each
example. We shall now see what we can gain by considering more general cases.

We shall derive several \emph{identities} which hold for the gradient of any scalar
field, and which we may use to speed up the evaluation of the gradient of more
complicated scalar fields.

If $\phi(\vr)$ and $\psi(\vr)$ are real scalar fields, then:
\begin{enumerate}

  %%%
  %%%     Linearity
  %%%

  \item \textbf{Distributive law}
        \begin{center}
          \bigbox{$
              \grad \left(\phi + \psi\right)
              \,=\,
              \grad\phi + \grad\psi
            $}
        \end{center}

        \textbf{Proof:} %(for the $i^{\rm th}$ component)
        % \[
        %   \left(\grad  \left(\phi \plus \psi\right)\right)_i
        %   ~\equiv~
        %   \frac{\partial}{\partial\x{i}} \, \left( \phi \plus  \psi \right)
        %   \,=\,
        %   \frac{\partial\phi}{\partial\x{i}}\,
        %   +
        %   \frac{\partial\phi}{\partial\x{i}}\,
        %   ~\equiv~
        %   \left(\grad\phi\right)_i + \left(\grad\psi\right)_i
        % \]
        % or, using the shorthand $\partial/\partial\x{i}=\partial_i$
        \[
          \left(\grad\left(\phi \plus \psi\right)\right)_i
          ~\equiv~
          \partial_i(\phi + \psi)
          \,=\,
          \partial_i\phi + \partial_i\psi
          ~\equiv~
          \left(\grad\phi\right)_i + \left(\grad\psi\right)_i
        \]

        %%%
        %%%     Product of two fields
        %%%

  \item \textbf{Product rule}

        \begin{center}
          \bigbox{$
              \grad  \left(\phi \, \psi\right)
              \,=\,
              \left(\grad\phi\right) \psi
              +
              \phi\, \left(\grad\psi\right)
            $}
        \end{center}

        \textbf{Proof:}
        \begin{eqnarray*}
          \left(\grad(\phi\psi)\right)_i
          ~\equiv~
          \partial_i \, (\phi\psi)
          \,=\,
          (\partial_i \phi)\psi + \phi(\partial_i\psi)
          ~\equiv~
          \left(\grad\phi\right)_i\psi + \phi\left(\grad\psi\right)_i
          %       \grad \left(\phi(\vr) \; \psi(\vr)\right) & = &
          %            \ei\, \frac{\partial}{\partial\x{i}}\;
          %              \left( \phi(\vr) \;  \psi(\vr) \right)\\
          %        & = & \ei\, \left(
          %                \psi(\vr)\; \frac{\partial\phi(\vr)}{\partial\x{i}}\,
          %                 \;+\;
          %              \phi(\vr)\; \frac{\partial\psi(\vr)}{\partial\x{i}}
          %              \right)\\[0.5ex]
          %        & = & \psi(\vr)\;\, \grad\phi(\vr) \;+\;
          %              \phi(\vr)\;\, \grad\psi(\vr)
        \end{eqnarray*}

        %%%
        %%%     Chain rule
        %%%
  \item \textbf{Chain rule:} If $F(\phi(\vr))$ is a scalar field, then

        \begin{center}
          \bigbox{$\ds
              \grad F\left(\phi\right)
              \,=\,
              \frac{d F(\phi)}{d\phi}\;\, \grad\phi
            $}
        \end{center}

        \textbf{Proof:}
        \[
          \left(\grad  F(\phi)\right)_i
          \,=\,
          \frac{\partial}{\partial x_i} (F(\phi) )
          \,=\,
          {\frac{dF(\phi)}{d\phi}} \: \frac{\partial\phi}{\partial x_i}
          \,=\,
          {\frac{dF(\phi)}{d\phi}} \, \left(\grad\phi\right)_i
          %\grad  F(\phi(\vr))\ =\  
          %\ei\, \frac{\partial}{\partial\x{i}}\; F(\phi(\vr))
          %\ =\ \ei\, \frac{\partial F(\phi)}{\partial\phi}\;\,
          %                 \frac{\partial\phi(\vr)}{\partial\x{i}}
          %         \;=\; \frac{\partial F(\phi)}{\partial\phi}\;\,\grad\phi(\vr)
        \]
\end{enumerate}
where we used the ordinary chain rule to get the second-last
expression.

\textbf{Example of Chain Rule:} If $\phir = r$ we can use result (iii)
from section~(\ref{sec:examples-on-gradient}) to give
\[
  \grad F(r) \;=\;
  \frac{dF(r)}{dr} \,\grad r
  \,=\,
  \frac{F^\prime(r)}{r} \; \v{r}
\]
If $F\left(\phi(\vr)\right) = r^n$, we have $\phir = r$ as in the previous example, and
so
\begin{center}
  \bigbox{$\ds
      \grad(r^n)
      \,=\,
      \frac{\:d\,r^n\!}{dr} \: \left(\grad r\right)
      \,=\,
      \left(n\,r^{n-1}\right)\,  \frac{1}{r}\, \vr
      \,=\,
      \left(n\,r^{n-2}\right)\, \vr
    $}
\end{center}
In particular
\[
  \grad \left(\frac{1}{r}\right)
  \,=\,
  - \frac{\v{r}}{r^3}
\]
We can also do this directly in suffix notation
\[
  \left(\grad \left(\frac1r \right)\right)_i
  \,=\,
  \partial_i\, (x_j x_j)^{-\half}
  \,=\,
  - \frac12 \, (x_k x_k)^{-3/2}\, 2 \, \delta_{ij} \, x_j
  \,=\,
  - { \frac{x_i}{r^3} }
\]

\paragraph{Example:}
Calculate $\grad\phi$ when $\ds \phi(\v{r}) = r^n \left(\v{a}\cdot\v{r}\right)^m$.
\begin{eqnarray*}
  \grad \left\{ r^n \left(\v{a}\cdot\v{r}\right)^m \right\}
  & = &
  \left( \grad r^n \right) \left(\v{a}\cdot\v{r}\right)^m
  +
  r^n \left\{ \grad \left(\v{a}\cdot\v{r}\right)^m\right\}
  \qquad \qquad \qquad\!\! \mbox{(using the product rule)} \\[1ex]
  %
  & = &
  \left( \grad r^n \right) \left(\v{a}\cdot\v{r}\right)^m
  +
  r^n \,
  m \left(\v{a}\cdot\v{r}\right)^{m-1}
  \left\{\grad \left(\v{a}\cdot\v{r}\right)\right\}
  \qquad \mbox{(using the chain rule)} \\[1ex]
  %
  & = &
  n \, r^{n-2} \left(\v{a}\cdot\v{r}\right)^m \v{r}
  +
  m r^n
  \left(\v{a}\cdot\v{r}\right)^{m-1} \v{a}
\end{eqnarray*}
where, in the last line, we used
$\grad\left(\v{a}\cdot\v{r}\right)=\v{a}$ and $\grad r^n = n \, r^{n-2}
  \v{r}$.

This example demonstrates the ``toolkit'' approach to evaluating gradients of complicated
expressions. Here, we combined the product rule with \emph{known} gradients of ``simple''
scalar fields $\left(\v{a}\cdot\v{r}\right)$ and $r^n$, which you should \emph{know},
\emph{and} be able to work out from first principle. If you prefer, you can work out
$\grad\phi$ for $\ds \phi(\v{r}) = r^n \left(\v{a}\cdot\v{r}\right)^m$ from scratch using
index notation. \emph{Exercise:} Do it right now!

The case $n=-3$, $m=1$ is used in calculating the electric field due to an electric
dipole. (See later.)

\subsection{Transformation of the gradient}
We now prove that the gradient of a scalar field is indeed a vector field - thus far we
merely \emph{assumed} it's a vector!

Let the point $P$ have coordinates $\x{i}$ in the $\{\ei\}$ basis, and the \emph{same}
point $P$ have coordinates $\x{i}'$ in the $\{\ei\!'\}$ basis, \emph{i.e.}~we consider
the vector transformation law $x_i \to x_i' = \ell_{ij} \, x_j$.

$\phi(\vr)$ is a \emph{scalar field} if it depends only on the
physical point $P$ and not on the coordinates $\x{i}$ or $\x{i}'$ used
to specify $P$.  The \emph{value} of $\phi$ at $P$ is \emph{invariant}
under a change of basis, but the function may look different, \emph{i.e.}
\begin{center}
  \bigbox{$\phi'(x'_1,\, x'_2,\, x'_3) \,=\, \phi(x_1,\, x_2,\, x_3)
    $}
\end{center}
Similarly $\v{a}$ is a \emph{vector field} if its components
transform as
\begin{center}
  \bigbox{$
      a_i^\prime(x'_1,\, x'_2,\, x'_3)
      \,=\,
      \ell_{ip}\, a_p(x_1,\, x_2,\, x_3)
    $}
\end{center}

Now consider $\grad \phi$ in the new (primed) basis. Its components transform as
\[
  \partial^{\prime}_i\,\phi'(x_1',\, x_2',\, x_3')
  ~\equiv~
  \frac{\partial}{\partial \x{i}'} \; \phi'(x_1',\, x_2',\, x_3')
  \,=\,
  \frac{\partial\x{j}}{\partial \x{i}'} \;
  \frac{\partial}{\partial \x{j}}\,
  \phi(x_1,\, x_2,\, x_3)
\]
(using the chain rule). Now since $\x{j} = \ell_{kj}\,\x{k}'$ (inverse
transformation for vector components)
\[
  \frac{\partial\x{j}}{\partial \x{i}'}
  \,=\,
  \ell_{kj}\, \frac{\partial\x{k}'}{\partial \x{i}'}
  \,=\,
  \ell_{kj}\, \delt{ik}
  \,=\,
  \ell_{ij}\,.
\]
Hence
\[
  \partial^{\prime}_i\,\phi^{\prime}(x'_1,\, x'_2,\, x'_3)
  \,=\,
  \ell_{ij} \, \frac{\partial}{\partial \x{j}}\,
  \, \phi(x_1,\, x_2,\, x_3)
  ~\equiv~
  \ell_{ij}\, \partial_j \phi(x_1,\, x_2,\, x_3)
\]
which shows that the components of $\grad \phi$ transform in the same way as the
components of a vector. Thus $\grad\phi(\vr)$ transforms as a \emph{vector field} as
claimed.

\mnote{14L 27/02/09}

\subsection{The operator \emph{del}}
We can think of $\grad$ as a \emph{vector operator}, called
\emph{del}, which acts on the \emph{scalar field} $\phi(\vr)$ to
produce the \emph{vector field} $\grad \phi(\vr)$.

In Cartesians: \hspace*{0.15\textwidth}
%
\bigbox{$\ds
    \grad
    \,\equiv\,
    \eone\;   \frac{\partial}{\partial x_1}\,
    \,+\,
    \etwo\;   \frac{\partial}{\partial x_2}\,
    \,+\,
    \ethree\; \frac{\partial}{\partial x_3}
    \,\equiv\,
    \ei \;  \frac{\partial}{\partial x_i}
    \,\equiv\,
    \ei \;  \partial_i
  $}
\vspace*{1ex}

We call $\grad$ an \emph{operator} since it operates on something to its \emph{right}. It
is a vector operator because it has vector transformation properties,
\begin{center}
  \bigbox{$\partial^\prime_i \,=\, \ell_{ip} \: \partial_p$}
\end{center}
We have seen how $\grad$ acts on a scalar field to produce a vector
field.  We can make products of the vector operator $\grad$ with other
vector quantities to produce new operators and fields in the same way
we could make scalar and vector products of two vectors.

For example, the directional derivative of $\phi$ in the direction $\v{\hat{s}}$, was
given by $\v{\hat{s}} \cdot \grad \phi$. More generally, we can interpret
$\v{a}\cdot\grad$ as a \emph{scalar operator}
\[
  \v{a}\cdot\grad \,=\, a_i \, \partial_i
\]
\emph{i.e.}~$\v{a}\cdot\grad$ acts on a scalar field to its
\emph{right} to produce another scalar field
\[
  (\v{a}\cdot\grad)\; \phi
  \,=\,
  a_i \, \partial_i \phi
  \,=\,
  a_1\, \frac{\partial \phi}{\partial x_1} \,+\,
  a_2\, \frac{\partial \phi}{\partial x_2} \,+\,
  a_3\, \frac{\partial \phi}{\partial x_3}
\]

We can also act with this operator on a vector field $\v{b}(\v{r})$ to get another vector
field,
\[
  (\v{a}\cdot\grad) \, \v{b}
  \,=\,
  \eone\, (\v{a}\cdot\grad) \,  b_1 \,+\,
  \etwo\, (\v{a}\cdot\grad) \,  b_2 \,+\,
  \ethree \, (\v{a}\cdot\grad) \, b_3
  %         (\vA\cdot\grad)\; \vV(\vr)
  %                & = &
  %                    \A{i}\; \frac{\partial}{\partial \x{i}}\, \vV(\vr)
  %                \equals 
  %                    \A{i}\; \frac{\partial}{\partial \x{i}}\,
  %                                \left( \V{j}(\vr) \, \ej \right)
  %                                \label{eq:q-dot-grad}\\[0.5ex]
  %                & = &
  %                \eone\, (\vA\cdot\grad)\,  V_1(\vr) \,+\,
  %                \etwo\, (\vA\cdot\grad)\,  V_2(\vr)\, \,+\,
  %                \ethree\, (\vA\cdot\grad)\,  V_3(\vr) \nonumber
\]
or, equivalently, in components
\[
  \left((\v{a}\cdot\grad)\, \v{b}\right)_i
  \,=\,
  (\v{a}\cdot\grad)\,b_i
  \,=\,
  a_j \, \partial_j \, b_i
\]
The alternative expression $\v{a}\cdot\left(\grad\v{b}\right)$ is {\em undefined} because
$\grad\v{b}$ doesn't make sense.

(For this reason, the parentheses are sometimes omitted, and
$\v{a}\cdot\grad\, \v{b}$ is taken to mean $(\v{a}\cdot\grad)\;
  \v{b}$, but I wouldn't recommend doing this as it's likely to lead to
mistakes!)

\paragraph{NB}
Great care is required with the order in products since, in general, products involving
operators are not commutative. For example
\[
  \v{a}\cdot\grad ~\neq~ \Div\v{a}
\]
The quantity $\v{a}\cdot\grad$ is a scalar differential \emph{operator} whereas
$\Div\v{a} \equiv \partial_i \, a_i$ gives a scalar field called the \emph{divergence} of
$\v{a}$.

\paragraph{Example:} Show that $\left(\v{a}\cdot\grad\right) \v{r} = \v{a}$, where $\v{a}(\v{r})$ is a vector
field. This is left as an important tutorial exercise for the student.

\paragraph{Examples:} In section~(\ref{grad_scalar}) we showed that, for small displacements $\delta\v{r}$, we
have
\begin{equation}
  \phi(\v{r}+\delta \v{r})
  \,=\,
  \phi(\v{r}) +  \left(\grad \phi\right) \cdot \delta\v{r}
  + O\left((\delta r)^2\right)
  \,=\,
  \phi(\v{r}) + \delta\v{r} \cdot \grad \phi + O\left((\delta r)^2\right)
  \label{eq:grad-phi}
\end{equation}
If we set $\delta \v{r} = \v{a}$, where $\v{a}$ is an arbitrary (but
small) vector, we have\footnote{We can write this equation as
  $\phi(\v{r}+\v{a}) \,=\, \phi(\v{r}) + a \frac{d\phi}{da} +
    O\left(a^2\right)$ where $\frac{d\phi}{da}$ is the directional
  derivative in the direction of the unit vector $\v{\hat{a}}$, and
  $a=|\v{a}|$, but this is rarely done in practice - possibly because
  its aparent simplicity leads to errors!}
\begin{equation}
  \bigbox{$
      \phi(\v{r}+\v{a})
      \,=\,
      \phi(\v{r}) + \v{a} \cdot \grad \phi(\v{r}) + O\left(a^2\right)
    $}
  \label{eq:phi-expansion}
\end{equation}
As we shall see, this expression is is very useful when $\phi(\vr)$ is
the electrostatic potential.

We can expand a \emph{vector field} about some point $\v{r}$ in exactly the same way. For
example, let $\v{E}(\v{r})$ be the electric field at $\v{r}$. We can take the result of
equation~(\ref{eq:grad-phi}), and simply replace $\phi(\v{r})$ by the $i^{\rm th}$
component of the electric field, $E_i(\v{r})$:
\[
  E_i(\v{r}+\delta \v{r})
  \,=\,
  E_i(\v{r}) +  \left(\delta\v{r} \cdot \grad\right) E_i(\v{r})
  +
  O\left((\delta r)^2\right)
\]
Setting $\delta\v{r} = \v{a}$ gives
\begin{equation}
  \bigbox{$
      \v{E}(\v{r}+\v{a})
      \,=\,
      \v{E}(\v{r}) + \left(\v{a} \cdot \grad\right) \v{E}(\v{r})
      +
      O\left(a^2\right)
    $}
  \label{eq:E-expansion}
\end{equation}

Equations~(\ref{eq:phi-expansion}) and~(\ref{eq:E-expansion}) arise in the study of
dipoles in electrostatics.

\mnote{13L 19/02/08}

\vfill
