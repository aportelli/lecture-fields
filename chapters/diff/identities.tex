% !TEX root = ../../fields.tex
%% Brian Lecture 11 %%%%%%%%%%%%%%%%%%%%%%%%%%%%%%%%%%%%%%%%%%%%%%%%%%%%%%%%%%%%%%%%%%%%%%%

% There are many identities involving div, grad, and curl.  It is not
% a good idea to know these from memory -- you may make mistakes --
% but you have to be able to \emph{derive} them when necessary.
% BOTOX! I guess they shouldn't learn their multiplication tables
% either! :)
% 
% Most importantly, this means that you should be at ease with div, grad
% and curl. This only comes through practice and deriving the various
% identities gives you just that. In these derivations the advantages
% of suffix notation, the summation convention and $\eps{ijk}$ will
% become apparent.

There are many identities involving div, grad, and curl. It is not necessary to know
\emph{all} of these, but you should know and be able to use the product and chain rules
for gradients (see Section~(\ref{sec:identitiesforgradients}), together with the product
laws for div and curl given below. These are almost obvious anyway!

You should be \emph{familiar} with the rest and to be able to \emph{derive} and
\emph{use} them when necessary.

It is also extremely useful to \emph{know} and be able to derive the results for
ubiquitous quantities such as $\grad r$, $\grad r^n$, $\Div \vr$, $\curl \vr$, $(\va
  \cdot \grad)\vr$, $\grad(\va \cdot \vr)$, $\curl(\va \times \vr)$ where $\v{a}$ is a
constant vector. This is like learning and understanding multiplication tables, or
knowing the derivatives of elementary functions such as $\sin x$.

% \emph{A word of warning:} not all of my colleagues agree with the
% above statements!  Some like to derive everything from scratch using
% the suffix notation, which seems a waste of time to me.

Most importantly you should be at ease with div, grad and curl. This only comes through
practice and deriving the various identities gives you just that. In these derivations
the advantages of suffix notation, the summation convention and $\eps{ijk}$ will
(hopefully) become apparent.

In what follows $\phi(\vr)$, $\v{a}(\vr)$ and $\v{b}(\vr)$ are
continuously-differentiable scalar and vector fields.

\subsection{Distributive laws}
\begin{description}

  \item{1.\ } $\Div (\v{a} + \v{b}) ~=~ \Div \v{a} +
          \Div\v{b}$

  \item{2.\ } $\curl (\v{a} + \v{b}) ~=~ \curl \v{a} + \curl \v{b}$
\end{description}
The proofs of these are straightforward using suffix or \emph{`xyz'}
notation and follow from the fact that div and curl are linear operations.

\subsection{Product laws}
The results of taking the div or curl of \emph{products} of vector and scalar fields are
the most useful:

\begin{description}
  \item{3.\ }
        \(
        \Div (\phi\,\v{a})
        ~=~
        \left(\grad\phi\right) \cdot\v{a}
        \,+\,
        \phi \left(\Div \v{a}\right)
        \)
  \item{4.\ }
        \(
        \curl (\phi\,\v{a})
        ~=~
        (\grad\phi)\times\v{a}
        \,+\,
        \phi\left(\curl \v{a}\right)
        \)
\end{description}

%%%
%%% This is where I first use the macro \pdiff{\ds j} for the partial
%%% derivative wrt x_j
%%%

\paragraph{Proof of (4):}

\begin{eqnarray*}
  \left(\curl \left(\phi\,\v{a}\right) \right)_i
  & = &
  \eps{ijk}\: \partial_j \left(\phi\,a_{k}\right)  \\[0.5ex]
  & = &
  \eps{ijk}
  \left(
  \left(\partial_j\phi\right) a_{k}
  +
  \phi \left(\partial_j a_{k}\right)
  \right) \\[0.5ex]
  & = &
  \left(\grad\phi\times\v{a}\right)_i
  +
  \phi \left(\curl \v{a}\right)_i
\end{eqnarray*}

One can also obtain this using \emph{`xyz'} notation: \( \ds \qquad \curl
\left(\phi\,\v{a}\right) = \left|
\begin{array}{ccc}
  \ex                            & \ey         & \ez         \\[1ex]
  \ds\frac{\partial}{\partial x} &
  \ds\frac{\partial}{\partial y} &
  \ds\frac{\partial}{\partial z}                             \\
  \phi \, a_x                    & \phi \, a_y & \phi \, a_z
\end{array}
\right|
\)

The $x$ component is
\begin{eqnarray*}
  \frac{\partial ( \phi a_z )}{\partial y}
  -\frac{\partial ( \phi a_y )}{\partial z}
  &=&
  \left( \frac{\partial  \phi }{\partial y} \right)  a_z
  -\left( \frac{\partial \phi}{\partial z} \right) a_y
  + \phi \left( \frac{\partial   a_z }{\partial y}
  -\frac{\partial  a_y }{\partial z} \right)   \\[1ex]
  &=&
  \left(\grad \phi \times \v{a} \right)_x + \phi \left(\curl \v{a}\right)_x
\end{eqnarray*}
A similar proof holds for the $y$ and $z$ components, but suffix
notation is so much quicker\ldots

Although we used Cartesian coordinates in our proofs, the identities hold in all
coordinate systems (the concept of a vector is coordinate-independent).

%\newpage

\subsection{Products of two vector fields}
The following identities are useful but less obvious:
\begin{description}
  \item{5.\ }
        \(
        \grad\,\left(\v{a}\cdot\v{b}\right)
        ~=~
        \left(\v{a}\cdot\grad\right)\,\v{b} + \left(\v{b}\cdot\grad\right)\v{a} +
        \v{a}\times\left(\curl\v{b}\right)  + \v{b}\times\left(\curl\v{a}\right)
        \)
  \item{6.\ }
        \(
        \Div \left(\v{a}\times\v{b}\right)
        ~=~
        \v{b}\cdot\left(\curl\v{a}\right) -
        \v{a}\cdot\left(\curl\v{b}\right)
        \)
  \item{7.\ }
        \(
        \curl\left(\v{a}\times\v{b}\right)
        \,=\,
        \left(\Div\v{b}\right) \v{a}
        \,+\,
        \left(\v{b}\cdot\grad\right)\,\v{a}
        \,-\,
        \left(\Div\v{a}\right) \v{b}
        \,-\,
        \left(\v{a}\cdot\grad\right)\,\v{b}
        \)
\end{description}
%{\bf Proof of (7):}\vspace*{-1ex}
%\begin{eqnarray*}
%        \curl (\v{a}\times\v{b}\right)
%        & = & \ei\; \eps{ijk}\, \pdiff{\ds j}\, (\v{a}\times\v{b}\right)_{\ds k}
%        \; = \; \ei\; \eps{ijk}\, \pdiff{\ds j}\;
%                                (\eps{klm}\, \A{l}\, \B{m})\\
% 
%       & = & \ei\;
%                \left( \delt{il}\,\delt{jm} - \delt{im}\,\delt{jl} \right)\;
%                 \left( \A{l}\, \pdiff{j}\, \B{m} +
%                        \B{m}\, \pdiff{j}\, \A{l} \right)\\[1ex]
%        & = & \ei\; \left(
%                \A{i}\,\pdiff{j}\,\B{j} +
%                \B{j}\,\pdiff{j}\,\A{i} -
%                \A{j}\,\pdiff{j}\,\B{i} - 
%                \B{i}\,\pdiff{j}\,\A{j} \right) \\[1ex]
%                & = & \v{a}\,(\Div\v{b})+ (\v{b}\cdot\grad)\,\v{a}
%                         - (\v{a}\cdot\grad)\,\v{b} -  \v{b}\,(\Div\v{a}) 
%\end{eqnarray*}

\paragraph{Proof of (6):}
\begin{eqnarray*}
  \Div \left(\v{a}\times\v{b}\right)
  ~ = ~
  \partial_i\, \left(\eps{ijk}\, a_{j}\, b_{k}\right)
  & = &
  \eps{ijk}\, \left(\partial_i a_{j}\right) \, b_{k} +
  \eps{ijk}\, a_{j}\, \partial_i b_{k} \\[1ex]
  & = &
  b_{k} \, \eps{kij}\, \partial_i \, a_{j} \;\; -
  a_{j} \, \eps{jik}\, \partial_i \, b_{k} \\[1ex]
  & = &
  b_k \left(\curl\v{a}\right)_k
  -
  a_j \left(\curl\v{b}\right)_j
\end{eqnarray*}

\vspace*{-4ex} %%% removing acres of space!

\paragraph{Proof of (7):}

\begin{eqnarray*}
  \left(\curl \left(\v{a}\times\v{b}\right)\right)_i
  & = &
  \epsilon_{ijk} \, \partial_j \left(\v{a}\times\v{b}\right)_k    \\[0.5ex]
  & = &
  \epsilon_{ijk} \, \partial_j \left(\epsilon_{klm} \, a_l b_m\right)
  \\[0.5ex]
  & = &
  \left(\delta_{il} \, \delta_{jm} - \delta_{im} \, \delta_{jl}\right) \,
  \partial_j \left(a_l \, b_m\right)				  \\[0.5ex]
  & = &
  \partial_j \left(a_i \, b_j \, \right) - \partial_j \left(a_j \, b_i\right)
  \\[0.5ex]
  & = &
  \left(\partial_j \, a_i\right)\,b_j + a_i \, \left(\partial_j \, b_j\right) -
  \left(\partial_j \, a_j\right)\,b_i - a_j \, \left(\partial_j \, b_i\right)
  \\[0.5ex]
  & = &
  \left(\v{b}\cdot\v{\nabla}\right)\,a_i +
  \left(\v{\nabla}\cdot\v{b}\right)\,a_i -
  \left(\v{\nabla}\cdot\v{a}\right)\,b_i -
  \left(\v{a}\cdot\v{\nabla}\right)\,b_i
\end{eqnarray*}

Other results involving one $\del$ can be derived similarly.

Identities 5, 6 \& 7 may be used in explicit calculations, but you can also apply the
standard index notation rules (as we did in deriving them.)

% Do this in the next section when we do the general case!

\paragraph{A shortcut:} We may evaluate $\curl \left( \v{a} \times \v{b}\right)$ by careful use of the vector
triple product formula
\begin{equation}
  \v{c} \times (\v{a}\times\v{b})
  =
  (\v{c}\cdot\v{b})\, \v{a} - (\v{c}\cdot\v{a})\,\v{b}
  =
  (\v{b}\cdot\v{c})\, \v{a} - (\v{a}\cdot\v{c})\,\v{b}\,.
  \label{eq:bac-cab}
\end{equation}
which holds for all vectors and vector fields $\v{a}$, $\v{b}$,
$\v{c}$.

Here we wish to evaluate
\[
  \curl \left( \v{a} \times \v{b} \right)
\]
Note that the derivatives in curl ($\curl$) act on \emph{both} $\v{a}$ and $\v{b}$. If we
want to replace $\v{c}$ by $\del$ in equation~(\ref{eq:bac-cab}), we must include
\emph{both} orderings of the scalar products in equation~(\ref{eq:bac-cab}), which gives
\begin{equation}
  \curl (\v{a}\times\v{b})
  \,=\,
  (\Div\v{b})\, \v{a} \,-\, (\Div\v{a})\,\v{b}
  \quad + \quad
  (\v{b}\cdot\del)\, \v{a} \,-\, (\v{a}\cdot\del)\,\v{b}\,.
\end{equation}

% In this case $\v{\omega}$ is a \emph{constant}, so the only non-zero
% derivatives are those in which $\grad$ acts on $\v{r}$. Using the
% expression after the second equals sign in
% equation~(\ref{eq:bac-cab}), so that $\grad$ acts only on $\v{r}$, we
% have
% \[
%   \curl \left( \v{\omega} \times \vr\right)
%    ~=~
%    \left(\Div\v{r}\right) \v{\omega}
%    -
%    \left(\v{\omega}\cdot\grad\right) \v{r}
%    ~=~
%    3 \v{\omega} - \v{\omega}
%    ~=~
%    2 \v{\omega}
% \]

%
% The following is Roger's stuff, which we've already covered (bjp)
%

% \paragraph{Example:} If $\va$ is a \emph{constant} vector, and $\vr$ is the
% position vector, show that
% \[
%         \grad\, \left(\va\cdot\vr\right) = \va
% \]
% We have
% \begin{eqnarray}
%    \left(\grad\, \left(\va\cdot\vr\right)\right)_i
%                 ~=~ \partial_i \, \left(a_j  x_j\right)
%                 ~=~ a_j \, \delt{ij}
%                 ~=~ a_i   \nonumber
% %\label{eq:trivial}
% \end{eqnarray}
% %\begin{eqnarray}
% %   \left(\left(\va\cdot\grad\right)\,\vr\right)_i
% %                ~=~ a_j \partial_j \, x_i
% %                ~=~ a_j\, \delt{ji}
% %                ~=~ a_i   \nonumber
% %%\label{eq:trivial}
% %\end{eqnarray}
%and the result holds. 

%{\bf Example:} Show that $\Div\left(\v{\omega}\times\vr\right) =0$
%where $\v{\omega}$ is a \emph{constant} vector. We have already
% seen this.)
% \begin{eqnarray}
%    \Div\left(\v{\omega}\times\vr\right) = \partial_i \epsilon_{ijk}\,\omega_j\, x_k
%                              = \epsilon_{ijk}\,\omega_j\, \delta_{i,k}
%                              = \epsilon_{iji}\,\omega_j
%                              = 0  
%                                                             \nonumber
% \end{eqnarray}                       
% ($\epsilon$ is antisymmetric and vanishes if two indices are identical).

%Using (6) $\Div\left(\v{\omega}\times\vr\right) =
%                \vr\cdot\left(\curl\v{\omega}\right)-\v{\omega}\cdot\left(\curl\vr\right)
%=0-0$\\[1ex]

\paragraph{Example:}
Show that $\Div \left(r^{-3}\vr\right) = 0$, for $r\ne 0$ (where $r=|\vr|$ as usual).

\paragraph{Method~1:} Using identities and simple results: Using identity~(3), we have
\[
  \Div \left(r^{-3}\vr\right)
  ~=~
  \left(\grad r^{-3}\right) \cdot \vr
  ~+~
  r^{-3}\; \left(\Div \vr\right)
\]

But we've shown previously that $\grad r^n = n\,r^{n-2}\,\vr$ and $\Div\vr = 3$. Hence
\begin{eqnarray*}
  \Div \left(r^{-3}\vr\right)
  %  & = &
  %  \left(\grad r^{-3}\right) \cdot \vr
  %  ~+~
  %  r^{-3}\; \left(\Div \vr\right) \\[1ex]
  & = &
  \left(\frac{-3}{r^5}\, \vr\right) \cdot \vr
  \,+\,
  \frac{3}{r^3}\\[1ex]
  & = &
  \frac{-3}{r^5} r^2 \,+\, \frac{3}{r^3}
  ~=~ 0
  \qquad \mbox{(except at\, }r = 0)
\end{eqnarray*}

\paragraph{Method~2:} Direct calculation using index notation:
\begin{eqnarray*}
  \Div \left(r^{-3}\vr\right)
  =
  \partial_i \left(x_i/r^3\right)
  &=&
  \left(\partial_i x_i\right)/r^3 + x_i \, \partial_i \, r^{-3} \\
  &=&
  \delta_{ii}/r^3 + x_i \left(-3 x_i / r^5\right) \\
  &=&
  3/r^3 - 3/r^3
  ~=~
  0
  \qquad \mbox{(except at\, }r = 0)
\end{eqnarray*}

\subsection{Identities involving two $\nabla$s}
%\subsection{Identities involving 2 $\grad$'s}
\begin{description}
  \item{8.\ \ } $\curl \left(\grad\phi\right) ~=~ 0$  \hfill
        curl grad $\phi$ is always zero.
  \item{9.\ \ } $\Div\left(\curl\v{a}\right) ~=~ 0$ \hfill
        div curl $\v{a}$ is always
        zero.
  \item{10.\ }
        \(
        \curl\left(\curl\v{a}\right)
        ~=~
        \grad\left(\Div\v{a}\right) - \lap \v{a}
        \)
\end{description}
Proofs are obtained readily in Cartesian coordinates using suffix
notation. You should know the first two, and knowing the second is
useful -- but you can always derive it from scratch.

\paragraph{Proof of (8):}
\begin{eqnarray*}
  \left(\curl \left(\grad\,\phi\right)\right)_i
  & = &
  \eps{ijk}\, \partial_j\, \left(\grad\,\phi\right)_k
  ~=~
  \eps{ijk}\, \partial_j \, \partial_k \, \phi   \\
  & = &
  \eps{ijk}\, \partial_k \, \partial_j \, \phi
  \hspace*{0.30\textwidth}
  \left(\mbox{since}\quad \frac{\partial^2\phi}{\partial x_1 \, \partial x_2}
  ~=~
  \frac{\partial^2\phi}{\partial x_2 \, \partial x_1}\;
  \mbox{\emph{etc}}\right)\\[0.5ex]
  & = &
  \eps{ikj}\, \partial_j \, \partial_k \, \phi
  \hspace*{0.35\textwidth}
  \left(\mbox{interchanging labels} \;j\; {\rm and}\; k \right)\\[0.5ex]
  & = &
  -\eps{ijk}\, \partial_j\, \partial_k \, \phi
  \hspace*{0.345\textwidth}
  \left(ikj \rightarrow ijk\; \mbox{gives minus sign}\right)\\[0.5ex]
  & = &
  - \left(\curl \left(\grad\,\phi\right)\right)_i ~=~ 0
\end{eqnarray*}
since any vector equal to minus itself is must be zero.  The proof of
(9) is similar. It is important to understand how these two identities
stem from the anti-symmetry of $\eps{ijk}$.

%\paragraph{Proof of (10):}
%\(
%  \curl\left(\curl\v{a}\right)
%  ~=~
%  \grad\left(\Div\v{a}\right) - \lap\,\v{a}
%\)
%\begin{eqnarray*}
%        \curl\left(\curl\v{a}\right)
%        & = & \ei\; \eps{ijk}\; \pdiff{j}\; \left(\curl\v{a}\right)_{k}\\[1ex]
%        & = & \ei\; \eps{ijk}\; \pdiff{j}\; \eps{klm}\, \pdiff{l}\,\A{m}
%                                                        \\[1ex]
%        & = & \ei\; \left(\left( \delt{il}\,\delt{jm}
%                        - \delt{im}\,\delt{jl}\right)\;
%                \pdiff{j}\;  \pdiff{l}\; \A{m}\right)\\[1ex]
%        & = & \ei\; \left( \pdiff{j}\; \pdiff{i}\; A_{j} -
%              \pdiff{j}\; \pdiff{j}\; A_{i} \right)\\[1ex]
%        & = & \ei\; \left(\pdiff{i}\; \left(\Div\v{a}\right) -
%                                                \lap\A{i}\right)\\[1ex]
%        & = & \grad\left(\Div\v{a}\right) - \lap\,\v{a}
%\end{eqnarray*}
Identity~(10) can be proven using the identity for the product of two epsilon symbols --
tutorial. Again, the proof is far simpler than trying to use \emph{`xyz'} -- try both and
see for yourself. It is an extremely important result and is used frequently in
electromagnetism, fluid mechanics, and other `field theories'.

Identity~(10) is also used in curvilinear coordinate systems to \emph{define} the action
of the Laplacian on a vector field as
\[
  \lap\,\v{a}
  ~\equiv~
  \grad\left(\Div\v{a}\right)
  -
  \curl\left(\curl\v{a}\right)
\]
A mnemonic for the Laplacian acting on a vector field is \emph{GDMCC} -- \emph{Grad-Div
  Minus Curl-Curl} -- pronounced ``Gudumcc''.

Finally, when a scalar field $\phi$ depends only on the magnitude of the position vector
$r=|\vr|$, we have
\[
  \lap \, \phi(r)
  \,=\,
  \phi''(r) + \frac{2\phi'(r)}{r}
  \,=\,
  \frac{1}{r^2} \left( r^2\phi^\prime(r)\right)^\prime
\]
where the prime denotes differentiation with respect to $r$. Proof of this relation
utilises the chain rule
\[
  \grad\phi(r)
  \,=\,
  \frac{d\phi(r)}{d r} \, \grad r
  \,=\,
  \phi'(r) \, \frac{\,\v{r}\,}{r},
\]
and is left to the tutorial.

\paragraph{Example:} Evaluate $\curl\left\{\left( \v{c}\times\v{r} \right) /r^3\right\}$ where $\v{c}$ is a
constant vector.

\medskip

Start with the product rule \( \curl (\phi\,\v{a}) \,=\, (\grad\phi)\times\v{a} \,+\,
\phi\left(\curl \v{a}\right) \) where $\phi=1/r^3$ and $\v{a}=\v{c}\times\v{r}$.
\begin{eqnarray*}
  \curl\left( \frac{\v{c}\times\v{r}}{r^3} \right)
  & = &
  \grad\left(\frac{1}{r^3}\right) \times \left(\v{c}\times\v{r}\right)
  \,+\,
  \frac{1}{r^3} \, \curl \left(\v{c}\times\v{r}\right) \\[1ex]
  & = &
  -3 \, \frac{\,\v{r}\,}{r^5} \times \left(\v{c}\times\v{r}\right)
  \,+\,
  \frac{1}{r^3} \, 2\v{c} \\[1ex]
  & = &
  -3\,\frac{1}{r^5}
  \left( r^2 \v{c} - \left(\v{r}\cdot\v{c}\right)\v{r}\right)
  +
  \frac{2}{r^3} \, \v{c} \\[1ex]
  & = &
  \frac{3\,\left(\v{r}\cdot\v{c}\right)\v{r}}{r^5}
  \,-\,
  \frac{\,\v{c}\,}{r^3}
\end{eqnarray*}
where we used $\grad r^n = n r^{n-2} \v{r}$ with $n=-3$, and $\curl
  \left(\v{c}\times\v{r}\right) = 2 \v{c}$

Of course, we could do the entire calculation using suffix notation throughout - try it
for yourself!

\paragraph{Physical example:} The vector potential of a magnetic dipole\footnote{The vector potential of an electric
  dipole also has this form, but in the electric case the scalar potential is more useful.}
has the form $\left( \v{c}\times\v{r} \right) /r^3$.

\newpage

\subsection{Summary}

\paragraph{Elementary results:} We can calculate \emph{div}, \emph{grad}, \emph{curl} and the \emph{Laplacian} of many of
the scalar and vector fields that occur in physics using the following elementary
results, which you must know and be able to derive using index notation with the
summation convention:
\[
  \begin{array}{ll ll ll l}
    \grad r \,=\, \v{r}/r \,=\, \v{\hat{r}}
     & \quad &
    \grad r^n \,=\, n r^{n-2} \v{r}
     & \quad &
    \Div \v{r} \,=\, 3
     & \quad &
    \curl \v{r}\,=\,0 \\[1.5ex]
    %
    \grad(\v{a} \cdot \v{r}) \,=\, \v{a}
     & \quad &
    (\v{a} \cdot \grad) \, \v{r} \,=\, \v{a}
     & \quad &
    \curl(\v{a} \times \v{r}) \,=\, 2\v{a}
     & \quad &
    \Div(\v{a} \times \v{r}) \,=\, 0
  \end{array}
\]
where $\v{r}$ is the position vector, $r=\left|\v{r}\right|$ is its magnitude (length),
and $\v{a}$ is a \emph{constant} vector.

The identity $\left(\v{a} \cdot \grad\right)\v{r} = \v{a}$ holds also for vector
\emph{fields} $\v{a}(\v{r})$, because no derivatives act on $\v{a}$.

\paragraph{Identities for scalar fields:} You must know, and be able to derive, all of the following for scalar fields
$\phi(\v{r})$ and $\psi(\v{r})$ using index notation:

\begin{enumerate}

  \item Distributive law:\ \ $\grad \left(\phi + \psi\right) \,=\, \grad\phi + \grad\psi $

  \item Product rule:\ \ \hspace*{5.8ex}$ \grad \left(\phi \, \psi\right) \,=\,
          \left(\grad\phi\right) \psi + \phi\, \left(\grad\psi\right) $

  \item Chain rule:\ \ If $F(\phi(\vr))$ is a scalar field, then $\ds \grad F\left(\phi\right)
          \,=\, \frac{d F(\phi)}{d\phi}\; \grad\phi(\v{r}) $

        \emph{Important example:}\ \ $\grad f(r) \,=\, \left(f'(r)/r\right) \v{r}$
\end{enumerate}

\paragraph{Identities for vector fields:}
For vector fields $\v{a}(\v{r})$ and $\v{b}(\v{r})$, and scalar fields $\phi(\v{r})$:

\begin{description}
  \item{1.\ } $\Div (\v{a} + \v{b}) \,=\, \Div \v{a} + \Div\v{b}$

  \item{2.\ } $\curl (\v{a} + \v{b}) \,=\, \curl \v{a} + \curl \v{b}$

  \item{3.\ }
        \(
        \Div (\phi\,\v{a})
        \,=\,
        \left(\grad\phi\right) \cdot\v{a}
        \,+\,
        \phi \left(\Div \v{a}\right)
        \)
  \item{4.\ }
        \(
        \curl (\phi\,\v{a})
        \,=\,
        (\grad\phi)\times\v{a}
        \,+\,
        \phi\left(\curl \v{a}\right)
        \)

  \item{5.\ }
        \(
        \Div \left(\v{a}\times\v{b}\right)
        \,=\,
        \v{b}\cdot\left(\curl\v{a}\right) -
        \v{a}\cdot\left(\curl\v{b}\right)
        \)
  \item{6.\ }
        \(
        \curl\left(\v{a}\times\v{b}\right)
        \,=\,
        \left(\Div\v{b}\right) \v{a}
        +
        \left(\v{b}\cdot\grad\right) \v{a}
        -
        \left(\Div\v{a}\right) \v{b}
        -
        \left(\v{a}\cdot\grad\right) \v{b}
        \)
  \item{7.\ }
        \(
        \grad\,\left(\v{a}\cdot\v{b}\right)
        \,=\,
        \left(\v{a}\cdot\grad\right)\,\v{b} + \left(\v{b}\cdot\grad\right)\v{a} +
        \v{a}\times\left(\curl\v{b}\right)  + \v{b}\times\left(\curl\v{a}\right)
        \)
  \item{8.\ \ } $\curl \left(\grad\phi\right) \,=\, 0$  \hfill
        (\emph{curl grad}$\:\phi\;$ is \emph{always} zero)
  \item{9.\ \ } $\Div\left(\curl\v{a}\right) \,=\, 0$ \hfill
        (\emph{div curl}$\:\v{a}\;$ is \emph{always} zero)
  \item{10.\ }
        \(
        \curl\left(\curl\v{a}\right)
        \,=\,
        \grad\left(\Div\v{a}\right) - \lap \v{a}
        \)
\end{description}
You must \emph{know} identities 1-5 and 8-10. You must be familiar
with the general form of the others. You should be able to derive them
all, and be able to use them all in practical calculations.

%\vfill

