% !TEX root = ../../fields.tex
%% Brian Lecture 10 %%%%%%%%%%%%%%%%%%%%%%%%%%%%%%%%%%%%%%%%%%%%%%%%%%%%%%%%%%%%%%%%%%%%%%%
%\nsection{More on Vector Operators {\it (RHB 8.7.2, 8.7.3)}}

We now combine the vector operator $\grad$ \emph{(del)} with a vector field to define two
new operations \emph{div} and \emph{curl}. Then we define the \emph{Laplacian}.

\subsection{Divergence}

We define the \emph{divergence} of a vector field $\v{a}(\v{r})$ (pronounced \emph{`div
  a'}) by
\begin{center}
  \fbox{
    \parbox{45mm}{
      \[
        {\rm div}\: \v{a}(\vr) ~\equiv~ \Div \v{a}(\vr)
      \]
    }}
\end{center}

In Cartesian coordinates
\begin{center}
  \bigbox{
    \parbox{0.6\textwidth}{
      \begin{eqnarray*}
        \Div \v{a}
        &\equiv&
        \frac{\partial a_1}{\partial x_1}\,
        + \frac{\partial a_2}{\partial x_2}\,
        + \frac{\partial a_3}{\partial x_3}
        ~=~
        \frac{\partial a_x}{\partial x}\,
        + \frac{\partial a_y}{\partial y}\,
        + \frac{\partial a_z}{\partial z} \\[0.5ex]
        &\equiv&
        \frac{\partial}{\partial \x{i}} \; a_{i}
        ~\equiv~
        \partial_i \, a_i
      \end{eqnarray*}
      %\begin{eqnarray*}
      %        \Div \v{a}(\vr)
      %                =  \partia\ell_i \, a_i(\vr) 
      %                \equiv \frac{\partial}{\partial \x{i}} \; a_{i}(\vr) 
      %                &=&    \frac{\partial a_1(\vr)}{\partial x_1}\,
      %                      +  \frac{\partial a_2(\vr)}{\partial x_2}\,
      %                      +  \frac{\partial a_3(\vr)}{\partial x_3}\\[1ex]
      %        \;\mbox{or}\; &&   \frac{\partial a_x(\vr)}{\partial x}\,
      %                      +  \frac{\partial a_y(\vr)}{\partial y}\,
      %                      +  \frac{\partial a_z(\vr)}{\partial z}
      %\quad\mbox{in $xyz$ notation}
      %\end{eqnarray*}
      %\vspace*{-1.5ex}
    }}
\end{center}

It's easy to show that $\Div\v{a}$ is a scalar field when $\v{a}$ is a vector field:

Under a transformation $x_i^\prime = \ell_{ij}\;x_j$, we have $\partial^\prime_i =
  \ell_{ij} \, \partial_j$ so
\[
  (\grad \cdot\v{a})^\prime
  = \partial^\prime_i \, a_i^\prime
  = (\ell_{ij}\,\partial_j) \, (\ell_{ik}\,a_k)
  = \delta_{jk} \, \partial_j \, a_k
  = \partial_j \, a_j
  = \grad \cdot\v{a}
\]
%\begin{eqnarray*}
%%% 1
%        \left(\grad\cdot \v{a})\right)'
%                & = & \frac{\partial}{\partial \x{i}'}\;
%                         a'(x_1',\, x_2',\, x_3')\ =\
%                      \frac{\partial\x{j}}{\partial \x{i}'}\;
%                      \frac{\partial}{\partial \x{j}}\;
%                      \left(\ell_{ik} a_{k}(x_1,\, x_2,\, x_3)\right)\\
%%% 2
%                & = & \ell_{ij}\,\ell_{ik} \, \frac{\partial}{\partial \x{j}}\;
%                         a_{k}(x_1,\, x_2,\, x_3)\ =\ 
%                      \delt{jk}\, \frac{\partial}{\partial \x{j}}\;
%                      a_{k}(x_1,\, x_2,\, x_3)\\
%%% 3
%                & = &  \frac{\partial}{\partial \x{j}}\;
%                                 a_{j}(x_1,\, x_2,\, x_3)
%                         =  \grad\cdot \v{a}
%\end{eqnarray*}
Hence $\Div\v{a}$ is invariant under a change of basis and is thus a {\bf scalar field}.

\paragraph{Example:} If $\v{a}(\vr)=\vr$ then \quad \bigbox{ $\Div \vr ~=~ 3$} \quad which is a useful and
important result.

\bigskip

\mbox{Explicitly:} \hspace*{0.15\textwidth}
\parbox{0.5\textwidth}{
  \(
  \ds \Div \vr
  = \frac{\partial x_1}{\partial x_1} +
  \frac{\partial x_2}{\partial x_2} +
  \frac{\partial x_3}{\partial x_3}
  =
  1+1+1
  =
  3
  \)}

\medskip

In suffix notation:
\[
  \Div \vr = \frac{\partial\x{i}}{\partial \x{i}} = \delt{ii} = 3
\]

\paragraph{Example:} In \emph{`xyz'} notation, let $\v{a} = x^2 z \,\v{e}_{\,x} - 2 y^3 z^2 \,\v{e}_{\,y} + x
  y^2 z \,\v{e}_{\,z}$
\begin{eqnarray*}
  \Div \v{a}
  & = &
  \frac{\partial}{\partial x}\; (x^2z)
  - \frac{\partial}{\partial y}\; (2 y^3z^2)
  + \frac{\partial}{\partial z}\; (x y^2 z)\\
  & = &
  2 x z - 6 y^2 z^2 + x y^2
\end{eqnarray*}
Then, at the point $(1,1,1)$ for instance, $\Div \v{a} = 2-6+1=-3$.

\subsection{Curl}
We define the \emph{curl} of a vector field $\v{a}(\v{r})$ by
\begin{center}
  \fbox{
    \parbox{45mm}{
      \[
        \mbox{curl}\: \v{a}(\vr) \;\equiv\; \curl \v{a}(\vr)
      \]
    }}
\end{center}
If $\v{a}$ is a (true) vector field, then $\curl \v{a}$ is a
\emph{pseudovector} field.  Exercise: Show that if $\v{a}$ is a
pseudovector field then $\curl \v{a}$ is a vector field.

In Cartesian coordinates
\[
  \curl\v{a}
  ~=~
  \ei\; \left(\curl \v{a}\right)_{i}
  ~=~
  \ei\; \eps{ijk}\, {\partial \over \partial x_j} \: a_{k}
\]
Equivalently, the $i^{\rm th}$ component of $\curl\v{a}$ is
\begin{center}
  \fbox{
    \parbox{55mm}{
      \[
        \left(\curl \v{a}\right)_{i}
        ~=~
        \eps{ijk}\, \partial_j \, a_{k}
      \]
    }
  }
\end{center}
More explicitly, the components of $\curl\v{a}$ are:
\[
  \left(\curl\v{a}\right)_1
  ~=~
  \frac{\partial a_3}{\partial x_2} -
  \frac{\partial a_2}{\partial x_3}
  \qquad
  %
  \left(\curl\v{a}\right)_2
  ~=~
  \frac{\partial a_1}{\partial x_3} -
  \frac{\partial a_3}{\partial x_1}
  \qquad
  %
  \left(\curl\v{a}\right)_3
  ~=~
  \frac{\partial a_2}{\partial x_1} -
  \frac{\partial a_1}{\partial x_2}
\]
We can also write the curl in determinant form, as for the ordinary vector product:
\begin{center}
  \bigbox{
    \parbox{105mm}{
      \[
        \curl\v{a}
        ~=~
        \left|
        \begin{array}{ccc}
          \eone                            & \etwo & \ethree \\[1.25ex]
          \ds\frac{\partial}{\partial x_1} &
          \ds\frac{\partial}{\partial x_2} &
          \ds\frac{\partial}{\partial x_3}                   \\[1.25ex]
          a_1                              & a_2   & a_3
        \end{array}
        \right|
        %
        \quad \mbox{or} \quad
        %
        \left|
        \begin{array}{ccc}
          \ex                            & \ey & \ez \\[1.25ex]
          \ds\frac{\partial}{\partial x} &
          \ds\frac{\partial}{\partial y} &
          \ds\frac{\partial}{\partial z}             \\[1.25ex]
          a_x                            & a_y & a_z
        \end{array}
        \right|
      \]
    }}
\end{center}

\paragraph{Example:}
If $\v{a}(\vr)=\vr$ \quad then \quad \bigbox{$ \curl \vr \,=\, 0 $} \quad another useful
and important result

\bigskip

\mbox{Explicitly:} \hspace*{0.15\textwidth}
\parbox{0.5\textwidth}{
  \(
  (\curl\vr)_i
  ~=~
  \eps{ijk}\, \partial_j x_{k}
  ~=~
  \eps{ijk}\, \delt{jk}  =  \eps{ijj} = 0
  \)
}

\medskip

or, using the determinant formula, \quad \( \ds \curl\vr ~=~ \left|
\begin{array}{ccc}
  \eone         & \etwo & \ethree \\[1ex]
  \ds\partial_1 &
  \ds\partial_2 &
  \ds\partial_3                   \\[1ex]
  x_1           & x_2   & x_3
\end{array}
\right|
~\equiv~
0
\)

In terms of explicit components
\[
  \curl \v{r}
  ~=~
  \eone
  \left(
  \frac{\partial x_3}{\partial x_2} -
  \frac{\partial x_2}{\partial x_3}
  \right)
  ~+~
  \etwo
  \left(
  \frac{\partial x_1}{\partial x_3} -
  \frac{\partial x_3}{\partial x_1}
  \right)
  ~+~
  \ethree
  \left(
  \frac{\partial x_2}{\partial x_1} -
  \frac{\partial x_1}{\partial x_2}
  \right)
  ~=~
  0
\]

\paragraph{Example:}
Compute the curl of $\v{a} = x^2y \, \eone +y^2x \, \etwo + xyz \, \ethree$
\[
  \curl\v{a}
  ~=~
  \left|
  \begin{array}{ccc}
    \v{e}_{\,x}                    & \v{e}_{\,y} & \v{e}_{\,z} \\[1.5ex]
    \ds\frac{\partial}{\partial x} &
    \ds\frac{\partial}{\partial y} &
    \ds\frac{\partial}{\partial z}                             \\[2ex]
    x^2y                           & y^2x        & xyz
  \end{array}
  \right|
  =
  (xz-0) \, \v{e}_{\,x}  -  (yz-0) \, \v{e}_{\,y}  + (y^2-x^2) \, \v{e}_{\,z}
\]

\subsection{Physical interpretation of \emph{div} and \emph{curl}} A full interpretation of the divergence and curl of a vector field is
best left until after we have studied the Divergence Theorem and Stokes' Theorem
respectively. However, we can gain some intuitive understanding by looking at simple
examples where div and/or curl vanish.

\parbox{11.5cm}{First consider the radial field $\v{a} = \v{r}$. We have
  just shown that $\Div \v{r} = 3$ and $\curl \v{r} = 0$.  We may
  sketch the vector field $\v{a}(\v{r})$ by drawing vectors of the
  appropriate direction and magnitude at selected points. These give
  the tangents of `flow lines'.  Roughly speaking, in this example the
  divergence is positive because bigger arrows come out of any point
  than go into it.  So the field `diverges'. (Once the concept of flux
  of a vector field is understood this will make more sense.)  }\hfill
\parbox{4.2cm}{
  \epsfxsize=4.2cm
  \includegraphics{tikz_myradial.pdf}
}

\parbox{4.2cm}{
  \epsfxsize=4.2cm
  \includegraphics{tikz_v_for_rigid_body.pdf}
}\hfill
\parbox{10cm}{ Now consider the field $\v{v} = \v{\omega} \times\vr$
  where $\v{\omega}$ is a \emph{constant} vector.  One can think of
  $\vv$ as the velocity of a point in a rigid rotating body.  The
  sketch shows a cross-section of the field $\vv$ with $\v{\omega}$
  chosen to point out of the page.  We can calculate $\curl\vv$ as
  follows: }

\begin{eqnarray*}
  \left(\curl \left( \v{\omega} \times \vr\right)\right)_i
  & = &
  \eps{ijk}\, \partial_j
  \left( \v{\omega} \times \vr\right)_{k}
  ~=~
  \eps{ijk}\, \partial_j \, \eps{klm} \,  \omega_{l} \, \x{m}\\
  & = &
  \left( \delt{il}\,\delt{jm} - \delt{im}\,\delt{jl} \right) \,
  \omega_{l}\, \delt{jm}
  \qquad \left(\mbox{because}\;
  \frac{\partial\omega_{l}}{\partial\x{j}} = 0 \right)\\
  & =
  & \left(\omega_{i}\,\delt{jj}  - \delt{ij}\, \omega_{j} \right)
  ~=~
  2\,\omega_{i}
\end{eqnarray*}
Thus we obtain yet another useful and important result:
\[
  \bigbox{$
      \curl \left( \v{\omega} \times \vr\right) = 2\v{\omega}
    $}
\]
We also have \bigbox{$\grad\cdot(\v{\omega}\times\v{r}) = 0$}:
\[
  \grad\cdot(\v{\omega}\times\v{r})
  = \partial_i \, \epsilon_{ijk} \, \omega_j \, x_k
  = \epsilon_{ijk} \, \omega_j \, \delta_{ik}
  = \epsilon_{iji} \, \omega_j
  = 0
\]

To understand intuitively the non-zero curl imagine that the flow lines are those of a
rotating fluid with a ball centred on a flow line of the field. The centre of the ball
will follow the flow line. However the effect of the neighbouring flow lines is to make
the ball rotate. Therefore the field has non-zero `curl' and the axis of rotation gives
the direction of the curl. In the previous example ($\v{a}=\vr$) the ball would just move
away from origin without rotating therefore the field $\vr$ has zero curl. Colloquially,
the vector field $\v{a}(\v{r})=\vr\,,$ doesn't ``curl around'' any point.

% Do this in the next section when we do the general case!

\paragraph{Terminology:}
\begin{enumerate}
  \item If $\Div\,\v{a} = 0 \;\;\;$ in some region $R$, we say $\v{a}$ is \emph{solenoidal} in
        $R$.
  \item If $\curl\,\v{a} = 0\;$ in some region $R$, we say $\v{a}$ \emph{irrotational} in $R$.
\end{enumerate}

%\newpage

\subsection{The Laplacian operator $\lap$ }
Consider taking the \emph{divergence} of the \emph{gradient} of a scalar field
$\phi(\vr)$
\[
  \Div \left(\grad\,\phi\right)
  ~=~
  \frac{\partial}{\partial \x{i}}\;
  \frac{\partial}{\partial \x{i}}\, \phi
  ~=~
  \partial^2 \, \phi
  ~\equiv~
  \lap \phi
\]
$\lap$ is the \emph{Laplacian operator}, pronounced
\emph{`del-squared'}.  In Cartesian coordinates
\[
  \lap
  ~=~ \frac{\partial}{\partial \x{i}}\;
  \frac{\partial}{\partial \x{i}}
  ~\equiv~
  \partial_i \, \partial_i
  ~\equiv~
  \partial^2
\]
More explicitly
\begin{center}
  \bigbox{
    \parbox{100mm}{
      \[
        \lap\, \phi
        ~=~
        \frac{\partial^2 \phi}{\partial x_1^2} \plus
        \frac{\partial^2 \phi}{\partial x_2^2} \plus
        \frac{\partial^2 \phi}{\partial x_3^2}
        %
        \quad\mbox{or}\quad
        %
        \frac{\partial^2 \phi}{\partial x^2} \plus
        \frac{\partial^2 \phi}{\partial y^2} \plus
        \frac{\partial^2 \phi}{\partial x^2}
      \]
    }}
\end{center}
The Laplacian of a scalar field $\lap\,\phi$ is a scalar field,
\emph{i.e.}~the Laplacian is a \emph{scalar operator}. Under the
transformation $x_i^\prime = \ell_{ij} \, x_j$ we have
$\partial^\prime_i = \ell_{ij} \, \partial_j$ so
\[
  (\lap)^\prime
  ~=~ \partial_i^\prime \, \partial_i^\prime
  ~=~ \ell_{ij}\,\partial_j \, \ell_{ik} \, \partial_k
  ~=~ \delta_{jk}\,\partial_j \, \partial_k
  ~=~  \partial_j \,\partial_j
  ~=~ \lap
\]

\paragraph{Example:} Using indices
\[
  \lap \, r^2
  ~=~ \partial_i \, \partial_i \, \left(x_j x_j\right)
  ~=~ \partial_i \, (2 x_i)
  ~=~ 2 \, \delt{ii}
  ~=~
  6
\]
Or, directly, using simple results derived previously,
\[
  \lap \, r^2
  ~=~
  \Div \left(\grad r^2\right)
  ~=~ \Div \left(2\v{r}\right)
  ~=~ 2 \times 3
  ~=~
  6\;.
\]

In \emph{Cartesian coordinates only}, the effect of the Laplacian on a vector field
$\v{a}$ is \emph{defined} to be
\[
  \lap \v{a}
  ~=~
  \partial_i \, \partial_i \, \v{a}
  ~=~
  \frac{\partial^2}{\partial x_1^2}\;\, \v{a}
  ~+~
  \frac{\partial^2}{\partial x_2^2}\;\, \v{a}
  ~+~
  \frac{\partial^2}{\partial x_3^2}\;\,  \v{a}
\]

The Laplacian acts on a vector field to produce another vector field.

\mnote{14L 22/02/08}
\mnote{15L 03/03/09}
