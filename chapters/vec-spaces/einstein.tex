% !TEX root = ../../fields.tex
%% Brian Lecture 5 %%%%%%%%%%%%%%%%%%%%%%%%%%%%%%%%%%%%%%%%%%%%%%%%%%%%%%%%%%%%%%%%%%%%%%%
%\section{More About Suffix Notation}

%\nsection{More About Suffix Notation}

The novelty of writing out summations soon wears thin. The standard way to avoid this
tedium is to adopt the Einstein summation convention. By adhering \emph{strictly} to the
following conventions or ``rules'' the summation signs are suppressed completely.

\paragraph{Rules of the summation convention}
\begin{enumerate}

  \item Omit \emph{all} summation signs.

  \item If a suffix appears \emph{twice}, a summation is implied, \emph{e.g.}~$a_i b_i = a_1 b_1
          +a_2 b_2 +a_3 b_3\,$.

        Here $i$ is a \emph{dummy} or \emph{repeated} index.

  \item If a suffix appears only \emph{once} it can take any value \emph{e.g.}~$a_i = b_i$ holds
        for $i=1,\, 2,\, 3$.

        Here $i$ is a \emph{free} index. Note that there may be more than one free index.

        \textbf{Always} check that the free indices match on both sides of an
        equation.\\ For example, $a_{j}=b_{i}$ is WRONG.

  \item A given suffix must \textbf{not} appear more than \textbf{twice} in any term in an
        expression.

        \textbf{Always} check that there aren't more than two identical
        indices \emph{e.g.}~$a_i b_i c_i$ is simply WRONG.

\end{enumerate}

\paragraph{Examples}\mbox{}

\hspace{5ex} $\v{a} = a_{i} \, \ei$ \hfill ($i$ is a dummy index)

\hspace{5ex} $\v{a}\cdot \ej = a_{i} \, \ei \cdot \ej = a_{i} \delt{ij}
  = a_{j}$ \hfill ($i$ is a dummy index, but $j$ is a free index)

\hspace{5ex} $\v{a}\cdot \v{b} = (a_{i} \, \ei) \cdot (b_{j} \, \ej ) =
  a_{i} b_{j} \delt{ij}= a_{j} b_{j}$ \hfill ($i,\,j$ are both dummy
indices)

\hspace{5ex} $(\v{a}\cdot \v{b})(\v{a}\cdot
  \v{c})=a_{i}b_{i}a_{j}c_{j}$\hfill (again $i,\,j$ are dummy
indices)

\bigskip

Armed with the summation convention one can rewrite many of the equations from the
previous sections without summation signs, \emph{e.g.}~the sifting property of
$\delt{ij}$ now becomes
\begin{center}
  \bigbox{${ [\ldots]_{j} \, \delt{jk}  = [\ldots]_{k}}$}
\end{center}
The repeated index $j$ is implicitly summed over, so that, for
example, $\displaystyle \delt{ij} \delt{jk} = \delt{ik}$.

\begin{center}
  \fbox{\parbox{15cm}{
      \vspace*{1ex}
      From now on, except where indicated, the summation convention
      will be assumed.\\
      You should make sure that you are completely at ease with it.
      \vspace*{1ex}
    }}
\end{center}

\subsection{Levi-Civita symbol $\eps{ijk}$}

We have seen how $\delt{ij}$ can be used to express the orthonormality of basis vectors
succinctly.

We now seek to make a similar simplification for the vector products of basis vectors
(taken here to be right handed), \emph{i.e.}~we seek a simple, uniform way of writing the
equations
\[
  \begin{array}{rcl rcl rcl}
    \eone \times\etwo    & =      & \ethree
                         & \qquad
    \etwo\times\ethree   & =      & \eone
                         & \qquad
    \ethree\times\eone   & =      & \etwo   \\[1ex]
    %
    \eone\times\eone     & =      & 0
                         &
    \etwo\times\etwo     & =      & 0
                         &
    \ethree\times\ethree & =      & 0
  \end{array}
\]
To do so we define the Levi-Cevita or `epsilon symbol' $\eps{ijk}$ (pronounced
\emph{`epsilon i j k'}), where $i$, $j$ and $k$ can take on the values 1 to 3, such that
\begin{center}
  \fbox{
    \parbox[c]{27em}{
      \vspace*{-2ex}
      \begin{eqnarray*}
        \eps{ijk} & = & +1
        \mbox{ if $ijk$ is an \emph{even} permutation of 123} \\
        & = & -1
        \mbox{ if $ijk$ is an \emph{odd\,}\ \ permutation of 123} \\
        & = &
        \ \ 0 \mbox{ otherwise (\emph{i.e.}~2 or more indices are the same)}
      \end{eqnarray*}
      \vspace*{-3ex}
    }}
\end{center}
An \emph{even} permutation consists of an \emph{even} number of
transpositions of two indices;\\
An \emph{odd\,}\ \ permutation consists of
an \emph{odd\,}\ \ number of transpositions of two indices.
\begin{eqnarray*}
  \mbox{\textbf{Examples:}} \qquad\quad\;
  \epsilon_{123} & = & +1\\
  \epsilon_{213} & = & -1
  \mbox{ \{since (123) $\to$ (213) under \emph{one} transposition
      [1 $\leftrightarrow$ 2]\}}\\
  \epsilon_{312} & = & +1
  \mbox{ \{(123) $\to$ (132) $\to$ (312);
  2 transpositions; [2 $\leftrightarrow$ 3][1 $\leftrightarrow$ 3]\}}\\
  \epsilon_{113} & = & 0\,;\quad\epsilon_{111} \ = \ 0\,;\mbox{ \emph{etc.}}
\end{eqnarray*}

\begin{center}
  \bigbox{$\epsilon_{123} = \epsilon_{231} = \epsilon_{312} = +1 \qquad
      \epsilon_{213} = \epsilon_{321} = \epsilon_{132} = -1 \qquad
      \mbox{all others } = 0$}
\end{center}

%This symbol is also called the {\bf alternating}
%symbol, or even the {\bf epsilon} symbol.
%\{ It generalizes straightforwardly to any number of dimensions. \}\\

Note the symmetry of $\eps{ijk}$ under \emph{cyclic permutations}
\begin{equation}
  \eps{ijk} =  \eps{kij} =  \eps{jki} =
  -\eps{jik} = -\eps{ikj} = -\eps{kji}
  \label{eq:cyclic-perms}
\end{equation}
This holds for all values of $i$, $j$ and $k$. To understand it, note
that
\begin{enumerate}
  \item If any two of the free indices $i$, $j$, $k$ are the same, all terms vanish.

  \item If ($ijk$) is an even (odd) permutation of (123), then so are ($jki$) and ($kij$), but
        ($jik$), ($ikj$) and ($kji$) are odd (even) permutations of (123).
\end{enumerate}
Each of equations~(\ref{eq:cyclic-perms}) has three free indices so
they each represent $3^3 = 27$ equations.\\ \emph{E.g.} in $\eps{ijk} =
  \eps{kij}$, 3 equations say `$1=1$', 3 equations say `$-1=-1$', and 21
equations say `$0=0$'.

\subsection{Vector product}
The equations satisfied by the vector products of the (right-handed) orthonormal basis
vectors $\ei$ can now be written uniformly as
\begin{center}
  \bigbox{
    $\ei\times\ej =  \eps{ijk}\,\ek$
  }
  \qquad $\forall \, i,j$ = 1,2,3
\end{center}
where there is an implicit sum over the `dummy' or `repeated' index
$k$. For example,
\begin{eqnarray*}
  \eone\times\etwo
  &=&
  \eps{12k}\:\ek
  \,=\,
  \epsilon_{121}\;\eone + \epsilon_{122}\;\etwo + \epsilon_{123}\;\ethree
  \,=
  \phantom{-}\,\ethree
  \\[0.75ex]
  \etwo\times\eone
  &=&
  \eps{21k}\:\ek
  \,=\,
  \epsilon_{211}\;\eone + \epsilon_{212}\;\etwo + \epsilon_{213}\;\ethree
  \,=\,
  -\ethree
  \\[0.75ex]
  \eone\times\eone
  &=&
  \eps{11k}\:\ek
  \,=\,
  \epsilon_{111}\;\eone + \epsilon_{112}\;\etwo + \epsilon_{113}\;\ethree
  \,= \,
  \phantom{-}0
\end{eqnarray*}
Now consider
\[
  \v{a}\times\v{b}
  \,=\,
  a_{i}\, b_{j}\, \ei\times\ej
  \,=\,
  \eps{ijk}\,a_{i}b_{j}\,\ek
\]
but, by definition, we also have
\[
  \v{a}\times\v{b} \,=\,  (\v{a}\times\v{b})_{k}\,\ek
\]
therefore
\begin{center}
  \bigbox{
    $(\v{a}\times\v{b})_{k} \,=\, \eps{ijk}\,a_{i}b_{j}$
  }
\end{center}
Note that we are using the summation convention. For example, writing
out the sums
\begin{eqnarray*}
  (\v{a}\times\v{b})_{3}
  &=&
  \epsilon_{113}\,a_1 b_1 +
  \epsilon_{123}\,a_2 b_3 +
  \epsilon_{133}\,a_3 b_3 +
  \epsilon_{213}\,a_2 b_1 +
  \cdots \\
  &=&
  \epsilon_{123}\,a_1 b_2 + \epsilon_{213}\,a_2 b_1
  \qquad\mbox{(plus terms that are zero)}\\
  &=&
  a_1 b_2 - a_2 b_1
\end{eqnarray*}

We can use the cyclic symmetry of the $\epsilon$ symbol to find an alternative form for
the components of the vector product
\[
  (\v{a}\times\v{b})_{k}
  \,=\,
  \eps{ijk}\,a_{i}b_{j}
  \,=\,
  \eps{kij}\,a_{i}b_{j} \,,
\]
or \emph{relabelling} both the free index $k\to i,$ and the dummy indices $i\to j, \quad
  j \to k$
\begin{center}
  \bigbox{$(\v{a}\times\v{b})_{i} \,=\, \eps{ijk}\,a_{j}b_{k}$}
\end{center}
which is (probably) the most useful form.

\mnote{05L 22/01/08}
\mnote{05L 27/01/09}

\newpage

The scalar triple product can also be written using $\eps{ijk}$
\[
  (\v{a},\v{b},\v{c})
  \,=\,
  \v{a}\cdot \, (\v{b}\times\v{c})
  \,=\,
  a_{i}(\v{b}\times\v{c})_{i}
\]
giving \vspace*{-2ex}
\begin{center}
  \bigbox{$
      (\v{a},\v{b},\v{c}) \,=\,  \eps{ijk}\,a_{i}b_{j}c_{k}$}
\end{center}
As an exercise in index manipulation we can prove the
cyclic symmetry of the scalar product
\vspace*{-2ex}
\begin{eqnarray*}
  (\v{a},\, \v{b},\, \v{c})
  &=&
  % \phantom{-}
  \eps{ijk}\,a_{i}b_{j}c_{k}\\[0.5ex]
  %  &=&
  %  -\eps{ikj}\,a_{i}b_{j}c_{k}
  %  \qquad\qquad\mbox{(interchanging two indices of $\eps{ijk}$})\\[0.5ex]
  %
  %  &=&
  %  +\eps{kij}\,a_{i}b_{j}c_{k}
  %  \qquad\qquad \mbox{(interchanging two indices again)}\\[0.5ex]
  &=&
  \eps{kij}\,a_{i}b_{j}c_{k}
  \qquad\qquad \mbox{(cyclic permutation of the indices of $\eps{ijk}$)}\\[0.5ex]
  &=&
  %  \phantom{-}
  \eps{ijk}\,a_{j}b_{k}c_{i}\quad\quad\qquad
  \mbox{(relabelling indices}\; k\to i,\; i\to j,\; j \to k) \\[0.5ex]
  &=&
  %  \phantom{-}
  \eps{ijk}\,c_{i} a_{j}b_{k}\\[0.5ex]
  &=&
  %  \phantom{-}
  (\v{c},\, \v{a},\, \v{b})
\end{eqnarray*}

\subsection{Product of two Levi-Civita symbols}
We have already shown geometrically that
\[
  \v{a}\times(\v{b}\times\v{c})
  \,=\,
  (\v{a}\cdot\v{c})\v{b} - (\v{a}\cdot\v{b})\v{c}
\]
This can be derived independently using components. For example,
\begin{eqnarray*}
  [\v{a}\times(\v{b}\times\v{c})]_1
  &=& a_2\, (\v{b}\times\v{c})_3 - a_3\, (\v{b}\times\v{c})_2 \\
  &=& a_2\, (b_1c_2-b_2c_1) - a_3\, (b_3c_1-b_1c_3) \\
  &=& b_1\, (a_2c_2+a_3c_3) - c_1\, (a_2b_2+a_3b_3) \\
  &=& b_1\, (a_1c_1 + a_2c_2+a_3c_3) - c_1\, (a_1b_1 +  a_2b_2+a_3b_3) \\
  &=& b_1\, (\v{a}\cdot\v{c}) - c_1\, (\v{a}\cdot\v{b})
\end{eqnarray*}
From this equality we deduce that there must be a relation between two
$\epsilon$ symbols (because there are two cross products on the LHS)
and some number of $\delta$ symbols (because there are dot products on
the RHS). Consider
\begin{eqnarray*}
  [\v{a}\times(\v{b}\times\v{c})]_i
  &=& \epsilon_{ijk}a_j \, (\v{b}\times\v{c})_k \\
  &=& \epsilon_{ijk}a_j \, \epsilon_{klm} \, b_l \, c_m \\
  &=& \epsilon_{ijk} \, \epsilon_{klm} \, a_j \, b_l \, c_m
\end{eqnarray*}
Alternatively
\begin{eqnarray*}
  [(\v{a}\cdot\v{c})\v{b} - (\v{a}\cdot\v{b})\v{c}]_i
  &=&
  (\v{a}\cdot\v{c}) \, b_i - (\v{a}\cdot\v{b}) \,c_i \\
  &=& (a_j \, c_m \, \delta_{jm}) \, \delta_{il} \, b_l -
  (a_j \, b_l \, \delta_{jl})\, \delta_{im} \, c_m \\
  &=& (\delta_{il} \, \delta_{jm} -
  \delta_{im} \, \delta_{jl}) \,
  a_j \, b_l \, c_m \,.
\end{eqnarray*}
These equations must be equal for \emph{all} components $a_j$, $b_l$, $c_m$
independently, so we must have
\begin{center}
  \bigbox{
    $\epsilon_{ijk} \, \epsilon_{klm}
      = \delta_{il} \, \delta_{jm} - \delta_{im} \, \delta_{jl}$}
\end{center}
This is a \textbf{very} important result and must be learnt by heart.

\newpage

To verify it, one can check all possible cases. For example
\[\epsilon_{12k}\,\epsilon_{k12}
  \,=\,
  \epsilon_{121}\,\epsilon_{112} +
  \epsilon_{122}\,\epsilon_{212} +
  \epsilon_{123}\,\epsilon_{312}
  \,=\,
  1
  \,=\,
  \delta_{11}\delta_{22} - \delta_{12}\delta_{21}
\]
However as we have $3^4 = 81$ equations, $6$ saying `$1=1$', $6$ saying `$-1=-1$', and
$69$ saying $0=0$', this will take some time. More generally, note that the left hand
side of the boxed equation may be written out as \vspace*{-2ex}
\begin{itemize}
  \item $\epsilon_{ij1}\,\epsilon_{1lm} + \epsilon_{ij2}\,\epsilon_{2lm}
          + \epsilon_{ij3}\,\epsilon_{3lm}$
        where $i,j,l,m$ are free indices;
  \item for this to be non-zero we must have $i\neq j$ and $l\neq m$;
  \item only one term of the three in the sum can be non-zero;
  \item if $i=l$ and $j=m$ we have $+1$,\ \ if $i=m$ and $j=l$ we have $-1$.
\end{itemize}

%\subsection{Product of two Levi-Civita symbols}
%We state without formal proof the following identity (see 
%questions on Problem Sheet 3)
%\begin{center}
%\bigbox{
%\rule{0cm}{.5cm}
%$
%\eps{ijk}\,\eps{rsk} 
% = \delt{ir}\delt{js} - \delt{is}\delt{jr}.
%$ }
%\end{center}

%To verify this is true one can check all possible cases
%{\it e.g.} $\epsilon_{12k}\,\epsilon_{12k} 
%=\epsilon_{121}\,\epsilon_{121}+ \epsilon_{122}\,\epsilon_{122}+\epsilon_{123}\,\epsilon_{123}
%=1 = \delta_{11}\delta_{22}-\delta_{12}\delta_{21}$.
%More generally, note that the left hand side of the boxed equation
%may be written out as
%\vspace*{-2ex}
%\begin{itemize}
%\item $\epsilon_{ij1}\,\epsilon_{rs1}+ \epsilon_{ij2}\,\epsilon_{rs2}+\epsilon_{ij3}\,\epsilon_{rs3}$
%where $i,j,r,s$ are free indices;
%\item for this to be non-zero we must have $i\neq j$ and
%$r\neq s$
%\item only one term of the three in the sum can be non-zero ;
%\item if $i=r$ and $j=s$ we have  $+1\quad$ ; if $i=s$ and $j=r$ we
%  have  $-1$ .
%\end{itemize}

%The product identity furnishes an  algebraic
%proof for the `BAC-CAB' rule. Consider the $i^{\rm th}$ 
%component of $\v{a}\times(\v{b}\times\v{c})$:
%\begin{eqnarray*}
%\left[\v{a}\times(\v{b}\times\v{c})\right]_{\ds\,i}  & = &
%\eps{ijk}\,a_{j}(\v{b}\times\v{c})_{k}\\[1ex]
%& = & \eps{ijk}\,a_{j}\,\eps{krs}\,b_{r}c_{s}
%\ = \  \eps{ijk}\,\eps{rsk}\,a_{j}b_{r}c_{s}\\[1ex]
%& = & (\delt{ir}\,\delt{js} - \delt{is}\,\delt{jr})\,a_{j}b_{r}c_{s}\\[9pt]
%& = & (a_{j}b_{i}c_{j} - a_{j}b_{j}c_{i})\\[9pt]
%& = & b_{i}(\v{a}\cdot\v{c}) - c_{i}(\v{a}\cdot\v{b})\\[9pt]
%& = & \left[\v{b}(\v{a}\cdot\v{c}) - \v{c}(\v{a}\cdot\v{b})\right]_{\ds\,i}
%\end{eqnarray*}
%Since $i$ is a free index we have proven the identity for 
%all three components $i=1,2,3$.

\paragraph{Example:}
Simplify $(\v{a}\times\v{b}) \cdot (\v{c}\times\v{d})$ using suffix notation.
\begin{eqnarray*}
  (\v{a}\times\v{b}) \cdot (\v{c}\times\v{d})
  & = &
  (\v{a}\times\v{b})_i \: (\v{c}\times\v{d})_i
  \,=\,
  \epsilon_{ijk}\,  a_j\, b_k \: \epsilon_{ilm}\,  c_l\, d_m \\[0.5ex]
  & = &
  (\delta_{jl} \, \delta_{km} - \delta_{jm} \, \delta_{kl}) \,
  a_j \, b_k \, c_l \, d_m
  \,=\,
  a_j \, b_k \, c_j \, d_k - a_j \, b_k \, c_k \, d_j \\[0.5ex]
  & = &
  (\v{a} \cdot \v{c}) \, (\v{b} \cdot \v{d})
  \, - \,
  (\v{a} \cdot \v{d}) \, (\v{b} \cdot \v{c})
\end{eqnarray*}
where we used the cyclic property $\epsilon_{ijk} = \epsilon_{jki}$ to
obtain the first expression in the second line.

\subsection{Determinants using the Levi-Civita symbol}\label{subsec:det-eps}

The result for the scalar triple product gives another expression for the determinant
\begin{equation}
  \left| \begin{array}{ccc}
    a_1 & a_2 & a_3 \\
    b_1 & b_2 & b_3 \\
    c_1 & c_2 & c_3 \\
  \end{array} \right|
  =
  \left(\v{a}, \v{b},\v{c}\right)
  =
  \epsilon_{ijk} \, a_i \, b_j \, c_k \,.
  \label{eq:det}
\end{equation}

%We may label the elements of a $3\times3$ matrix $A$ by $a_{ij}$ (or
%alternatively by $A_{ij}$), where $i$ labels the row and $j$
%labels the column in which $a_{ij}$ appears:

Consider the $3\times3$ matrix $A$, with elements $a_{ij}$
\[
  A \ = \ \left(\begin{array}{ccc}
      a_{11} & a_{12} & a_{13} \\
      a_{21} & a_{22} & a_{23} \\
      a_{31} & a_{32} & a_{33}
    \end{array}\right)
\]
Relabelling the rows in the matrix in equation~(\ref{eq:det}): $a_i \to a_{1i}, \; b_j
  \to a_{2j}, \; c_k \to a_{3k}$ gives
\begin{center}
  \bigbox{
    $\det A = \epsilon_{ijk} \, a_{1i} \, a_{2j} \, a_{3k}$}
\end{center}
which may be taken as the \emph{definition} of a determinant.

An alternative expression is given by noting that previously we showed that
\begin{eqnarray}
  \left| \begin{array}{ccc}
    a_{11} & a_{12} & a_{13} \\
    a_{21} & a_{22} & a_{23} \\
    a_{31} & a_{32} & a_{33} \\
  \end{array} \right|
  =
  \left| \begin{array}{ccc}
    a_{11} & a_{21} & a_{31} \\
    a_{12} & a_{22} & a_{32} \\
    a_{13} & a_{23} & a_{33} \\
  \end{array} \right| \qquad \mbox{or} \quad \det A = \det A^T
  \nonumber
\end{eqnarray}
where $A^T_{ij} = a_{ji}$ so now $\det A = \det A^T = \epsilon_{ijk}
  A^T_{1i} \, A^T_{2j} \,A^T_{3k}$ which may be rewritten
\begin{center}
  \bigbox{
    $\det A = \epsilon_{ijk} \, a_{i1} \, a_{j2} \, a_{k3}$}
\end{center}

The other properties of determinants can be proved easily: namely interchanging two rows
or columns changes the sign of the determinant and adding a multiple of one row/column to
another row/column respectively does not change the value of the determinant (exercises
for the student).

For completeness, we quote here one further important result
\begin{center}
  \bigbox{
    $\det AB = \det A \, \det B$}
\end{center}
[The definition of matix multiplication in suffix notation is given in
Section~(\ref{transf_matrix}).] The proof of this result is
discussed in a tutorial sheet.

Another important result is
\[
  (AB)^T  \,=\, B^T A^T
\]
which follows because
\[
  {\left( AB \right)^T}_{ij}
  \,=\, (AB)_{ji}
  \,=\, a_{jk} \, b_{ki}
  \,=\, (B^T)_{ik} \, (A^T)_{kj}
  \,=\, (B^TA^T)_{ij}
\]

%\vfill

%\vfill