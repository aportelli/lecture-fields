% !TEX root = ../../fields.tex
%% Brian Lecture 1 %%%%%%%%%%%%%%%%%%%%%%%%%%%%%%%%%%%%%%%%%%%%%%%%%%%%%%%%%%%%%%%%%%%%%%%
\subsection{Physics terminology}
To get started, we recall two basic geometrical definitions:
\begin{itemize}
  \item \textbf{Scalar}: quantity specified by a single number;
  \item \textbf{Vector}: quantity specified by a magnitude, which is a scalar quantity,
        and a direction. The direction is given by $d-1$ angles in $d$ dimensions (\eg one
        needs only one angle to define a direction in a plane, two in a three-dimensional
        volume, etc.).
\end{itemize}
In this course, we will (re)define vectors by the way their components transform under
rotations of the coordinate basis, and scalars as numbers that are unchanged by such
transformations.

\noindent\emph{Examples:} velocity is a vector, speed is a scalar.

Vector are often noted using a bold font (\eg $\v{a}$), which is the notation we will
mainly use here. Another common notation is the over arrow symbol $\vec{a}$, which is
more convenient for handwriting. Such notation helps to identify vectors quickly in
equations, but is not fully necessary if objects are defined unambiguously. Therefore, it
is also common to find literature where no special notation is used with vectors.

\subsection{Geometrical approach}
A vector is \emph{represented} by a \emph{directed line segment} with a length and
direction proportional to the magnitude and direction of the vector (in appropriate
units). A vector can be considered as a class of equivalent directed line segments. For
example, in~\cref{fig:dls} both displacements from $P$ to $Q$ and from $R$ to $S$ are
represented by the same vector. The vector given by the displacement from $P$ to $Q$ is
noted $\smash{\overrightarrow{PQ}}$, and therefore in the case of~\cref{fig:dls} we have
$\smash{ \overrightarrow{PQ} = \overrightarrow{RS}}$.
\begin{figure}[t]
  \centering
  \includegraphics{tikz_dls.pdf}
  \caption{Representation of vectors as a displacement between two points.}
  \label{fig:dls}
\end{figure}

\noindent\textbf{Notation}: For a vector $\v{a}$ its magnitude (or
length) is noted $|\v{a}|$. A unit vector is a dimensionless vector with magnitude $1$.
%
\subsubsection*{Addition of vectors -- parallelogram law (geometrical)}
\begin{tabular}{cc}
  \parbox{6.5cm}{\includegraphics{tikz_parallelogram.pdf}} &
  \parbox{4cm}{
    \begin{eqnarray*}
      \v{a} + \v{b} & = & \v{b} + \v{a}\mbox{\hspace*{10ex} (commutative) }\\
      (\v{a} + \v{b}) + \v{c} & = & \v{a} + (\v{b} + \v{c})
      \mbox{\hspace*{5ex} (associative) }
    \end{eqnarray*}}
\end{tabular}

\subsubsection*{Multiplication by scalars} A vector $\v{a}$ may be
multiplied by a scalar $\alpha$ to give a new vector $\alpha \,
  \v{a}$, {\it e.g.} \\[3mm]
\hspace*{1cm}
\epsfysize=1.5cm
\includegraphics{tikz_scalar_mult.pdf}\\
Also, for scalars $\alpha$, $\beta$ and vectors $\v{a}$ and $\v{b}$
\begin{eqnarray*}
  | \alpha \v{a} | & = & |\alpha| |\v{a}| \\
  \alpha(\v{a} + \v{b}) & = & \alpha\v{a} + \alpha\v{b}
  \;\;\qquad\mbox{  (distributive)}\\
  \alpha(\beta\v{a}) & = & (\alpha\beta)\v{a}\;\;\;
  \quad\qquad\mbox{  (associative)}\\
  (\alpha + \beta)\v{a} & = & \alpha\v{a} + \beta\v{a}\;.
\end{eqnarray*}
\subsubsection*{Notes}
\begin{enumerate}
  \item The direction of $\v{a}$ can be represented by the unit vector $\v{\hat{a}} =
          \v{a}/|\v{a}|$. Clearly, $|\v{\hat{a}}| = 1$.
  \item If for two vectors $\v{a}$ and $\v{b}$ there exist a scalar $\alpha$ such that
        $\v{b}=\alpha\v{a}$, then $\v{a}$ and $\v{b}$ are said to be \emph{collinear} (\ie they
        lie on the same line) and we use the notation $\v{a}\parallel\v{b}$.
  \item In three dimensions, $\v{a}$, $\v{b}$ and $\v{c}$ are said to be \emph{coplanar} if there
        exists two scalars $\beta$ and $\gamma$ such that $\v{a}=\beta\v{b}+\gamma\v{c}$ (\ie
        they lie in the same plane).
  \item There is a unique vector with zero magnitude noted $\v{0}$. Its direction is undefined,
        and it is collinear to all vectors. The zero vector represents the displacement from a
        point to itself (\ie no displacement).
\end{enumerate}

\subsection{Scalar or dot product}\label{sec:scalarproduct}

The \emph{scalar product} (also known as the \emph{dot product} or \emph{inner product})
between two vectors is the scalar quantity noted $\v{a}\cdot\v{b}$ and defined by
\begin{center}
  \parbox{4.5cm}{
    \includegraphics{tikz_scalar_prod.pdf}
  }
  \hspace{0.5cm}
  \bigbox{ $\v{a}\cdot\v{b} = |\v{a}||\v{b}|\cos(\theta), \mbox{ where $\theta$
        is the angle between $\v{a}$ and $\v{b}$} $ }
\end{center}

\subsubsection*{Notes on scalar product}
\begin{enumerate}
  \item $\v{a}\cdot\v{b}=\v{b}\cdot\v{a} \quad \mbox{(commutative)}$
  \item $\v{a}\cdot(\v{b}+\v{c})=\v{a}\cdot\v{b}+ \v{a}\cdot\v{c} \qquad
          \mbox{(distributive)}$
  \item $\v{n}\cdot \v{a}= |\v{a}| \cos(\theta)$: scalar projection of $\v{a}$ onto
        $\v{n}$, where $\v{n}$ is a unit vector
  \item $(\v{n}\cdot \v{a})\, \v{n} = |\v{a}|\cos(\theta)\, \v{n}$: the vector projection
        of $\v{a}$ onto $\v{n}$
  \item $\v{a}\cdot\v{b}=0$ if and only if they form a right angle, then
        $\v{a}$ and $\v{b}$ are said to be \emph{orthogonal}, and we note $\v{a}\perp\v{b}$
  \item A vector $\v{a}$ can be decomposed with respect to some direction $\v{n}$ into a parallel
        component $\v{a}_\parallel = \left(\v{n}\cdot \v{a}\right) \v{n}$ and a perpendicular
        component $\v{a}_\perp = \v{a}-\v{a}_\parallel$. One can check easily that
        $\v{a}_\perp\cdot \v{n} =0$.
  \item $\boxed{|\v{a}|^2=\v{a}\cdot\v{a}}$\,, for a unit vector $\v{n}$ clearly $\v{n}\cdot\v{n}= 1$.
        The notation $\v{n}^2=\v{n}\cdot\v{n}$ is often used.
\end{enumerate}

\subsection{The vector or cross product}
In three dimensions, the \emph{cross product} (also known as vector product) between two
vectors is the vector noted $\v{a}\times\v{b}$ and defined by
\begin{center}
  \bigbox{$\v{a}\times\v{b}=|\v{a}||\v{b}|\sin(\theta)\;\v{n}$}\;,
\end{center}
where $\theta$ is the angle between $\v{a}$ and $\v{b}$ chosen such that $0\leq
  \theta<\pi$, and where $\v{n}$ is a unit vector orthogonal (or \emph{normal}) to the
plane of $\v{a}$ and $\v{b}$, in the direction given by the \emph{right-hand rule}, as
represented in \cref{fig:vector_prod}. The cross-product is defined in that way
\emph{only for three-dimensional vectors}. In $d$ dimensions, the cross-product can be
defined from $d-1$ vectors similarly, however this definition is out of scope for this
course.
\begin{figure}[t]
  \centering
  \includegraphics{tikz_vector_prod.pdf}
  \includegraphics[height=4.5cm]{right_hand.pdf}
  \caption{Cross-product and right-hand rule representation.}
  \label{fig:vector_prod}
\end{figure}
\subsubsection*{Notes on vector product}
\begin{enumerate}
  \item $\v{a}\times\v{b}=-\v{b}\times\v{a}$ \quad
        (anti-commutative)

  \item $\v{a}\times \v{b}= 0$ if $\v{a}$ and $\v{b}$ are
        collinear

  \item $\v{a}\times( \v{b} + \v{c} ) = \v{a}\times \v{b} +
          \v{a}\times\v{c} $

  \item $\v{a}\times( \alpha \v{b} ) = (\alpha \v{a})\times
          \v{b}=\alpha(\v{a}\times\v{b})$
\end{enumerate}

\subsection{The scalar triple product}
In three dimensions, the scalar triple product is defined as follows
\begin{eqnarray*}
  \stpr{\v{a}}{\v{b}}{\v{c}} & \equiv & \v{a}\cdot(\v{b}\times\v{c})
  %& = & \v{c}\cdot(\v{a}\times\v{b})\\
  %& = & \v{b}\cdot(\v{c}\times\v{a})\,.
\end{eqnarray*}

\subsubsection*{Notes on scalar triple product}
\begin{figure}[t]
  \centering
  \includegraphics{tikz_volume.pdf}
  \caption{Parallelepiped with volume given by the triple scalar product $\stpr{\v{a}}{\v{b}}{\v{c}}$.}
  \label{fig:volume}
\end{figure}
\begin{enumerate}
  \item
        If $\v{a}$, $\v{b}$ and $\v{c}$ are three concurrent edges
        of a parallelepiped, its volume is $\stpr{\v{a}}{\v{b}}{\v{c}}$.

          %To see this, note that\\
          {To see this, note that:
            \begin{eqnarray*}
              \mbox{area of the base}&=&\mbox{area of parallelogram $OBDC$}\\
              &=&
              |\v{b}||\v{c}|\sin(\theta) \,=\, |\v{b}\times\v{c}|\\
              \mbox{height}
              &=&
              |\v{a}| \cos(\phi) \,=\, \v{n}\cdot\v{a}\\
              \mbox{volume}
              &=&
              \mbox{area of base $\times$ height}\\
              &=&
              |\v{b}||\v{c}|\sin(\theta)\,\v{n} \cdot \v{a}\\
              &=&
              \v{a}\cdot( \v{b}\times \v{c} )
            \end{eqnarray*}
          }

        In this case $\v{n}$ is a unit vector parallel to $\v{b}\times\v{c}$, which is normal to
        the plane of $\v{b}$ and $\v{c}$.

  \item
        If we choose $\v{c}, \v{a}$ to define the base then a similar calculation
        gives volume = $\v{b}\cdot( \v{c}\times \v{a})$

        We deduce the following symmetry/antisymmetry properties (exercise):
        \[
          \stpr{\v{a}}{\v{b}}{\v{c}}
          \,=\,  \stpr{\v{b}}{\v{c}}{\v{a}}
          \,=\,  \stpr{\v{c}}{\v{a}}{\v{b}}
          \,=\, -\stpr{\v{a}}{\v{c}}{\v{b}}
          \,=\, -\stpr{\v{b}}{\v{a}}{\v{c}}
          \,=\, -\stpr{\v{c}}{\v{b}}{\v{a}}
        \]
  \item \bigbox{
          $\v{a}$, $\v{b}$ and $\v{c}$ are coplanar
          if and only if $\stpr{\v{a}}{\v{b}}{\v{c}} = 0$
        }\,.

\end{enumerate}

\subsection{The vector triple product}

There are \emph{several} ways of combining three vectors to form a new vector, \eg
$\v{a}\times(\v{b}\times\v{c})$; $(\v{a}\times\v{b})\times\v{c}$, etc. Note carefully
that \emph{brackets are important}, since the cross product is \emph{not} associative
\[
  \v{a}\times(\v{b}\times\v{c}) ~\neq~ (\v{a}\times\v{b})\times\v{c}\;.
\]
Expressions involving two (or more) vector products can be simplified by using the
identity
\begin{center}
  \bigbox{$\v{a}\times(\v{b}\times\v{c}) \ =
      \ (\v{a}\cdot\v{c})\, \v{b} - (\v{a}\cdot\v{b})\,\v{c}$}.
\end{center}
It's sometimes known as the ``bac-cab rule'', because the expression above can also be
written $\v{a}\times(\v{b}\times\v{c}) = \v{b} \, (\v{a}\cdot\v{c})- \v{c} \,
  (\v{a}\cdot\v{b})$.

\noindent\textbf{Proof:} First note that $\v{b}\times\v{c}$ is
orthogonal to the $(\v{b},\, \v{c})$ plane. Now $\v{a} \times
  (\v{b}\times\v{c})$ is orthogonal to $\v{b}\times\v{c}$, so it must lie
in the $(\v{b},\, \v{c})$ plane. Hence, we can write
\[
  \v{a}\times(\v{b}\times\v{c}) = \beta\,\v{b} + \gamma\,\v{c}
\]
with $\beta$, $\gamma$ scalars. Taking the scalar product with $\v{a}$ gives
\[
  \v{a} \cdot [\v{a}\times(\v{b}\times\v{c})]
  \,=\,
  0
  \,=\,
  \beta \, (\v{a}\cdot\v{b}) \,+\, \gamma \, (\v{a}\cdot\v{c}),
\]
and from this we deduce that $\beta = \alpha\, (\v{a}\cdot\v{c})$, $\gamma = -\alpha \,
  (\v{a}\cdot\v{b})$ for some \emph{constant} $\alpha$, to give
\[
  \v{a}\times(\v{b}\times\v{c})
  \,=\,
  \alpha
  \left[
    \left(\v{a}\cdot\v{c}\right)\v{b}
    \,-\,
    \left(\v{a}\cdot\v{b}\right)\v{c}
    \right]
\]
The constant $\alpha$ may be determined by considering the particular case when
$\v{b}\perp\v{c}$ and $\v{c}\parallel\v{a}$, to give
\[
  \v{a}\times(\v{b}\times\v{c})
  \,=\,
  |\v{a}||\v{c}|\,\v{b}
  \,=\,
  \alpha (|\v{a}||\v{c}|\v{b}-0),
\]
therefore $\alpha = 1$.

\subsection{Some examples of vector products in physics}
\begin{figure}[t]
  \centering
  \includegraphics{tikz_torque.pdf}\hspace{2cm}
  \includegraphics{tikz_rotate.pdf}
  \caption{Representation of the torque and angular velocity.}
  \label{fig:torque}
\end{figure}
\subsubsection{Torque}
The \emph{torque} or \emph{couple} or \emph{moment} of a force about the origin is
defined as $\v{G} = \v{r} \times \v{F}$ where $\v{r}$ is the position vector of the point
where the force is acting and $\v{F}$ is the force vector at that point. Thus torque
about the origin is a vector quantity. The magnitude of the torque about an axis through
the origin in direction $\v{n}$ is given by $\v{n} \cdot \v{G} = \v{n}\cdot (\v{r} \times
  \v{F})$. Note that this is a scalar quantity formed by a scalar triple product.

\subsubsection{Angular velocity}

Consider a point in a rigid body rotating with \emph{angular velocity} $\v{\omega}$,
where $\,|\v{\omega}| \equiv \omega$ is the angular speed of rotation measured in radians
per second, and the unit vector $\v{\hat{\omega}}$ lies along the axis of rotation, so
$\v{\omega} = \omega \, \v{\hat{\omega}}$. Let the position vector of the point with
respect to an origin $O$ on the axis of rotation be $\v{r}$. The velocity of the point is
given by
\[
  \boxed{\v{v} \,=\, \v{\omega}\times\v{r}}
\]
We can show this as follows: in time $\delta t$, the point rotates through angle
$\delta\phi$ from $\v{r}$ to $\v{r} + \delta\v{r}$. The distance $|\delta \v{r}|$ moved
in time $\delta t$, is
\[
  |\delta \v{r}|
  \,= \,
  |\v{r}|\sin(\theta)\delta\phi
  \,=\,
  \left|\v{\hat{\omega}} \times \v{r}  \right|\delta\phi
\]
Since $\delta\v{r}$ is parallel to $\v{\omega}\times\v{r}$, we deduce that
\[
  \delta\v{r}
  \,=\,
  \v{\hat{\omega}} \times \v{r} \; \delta\phi
\]
and hence
\[
  \frac{\delta\v{r}}{\delta t}
  \,=\,
  \v{\hat{\omega}} \times \v{r} \; \frac{\delta\phi}{\delta t}
\]
In the limit $\delta t \to 0$, we find
\[
  \v{v}
  \,=\,
  \frac{\diff\v{r}}{\diff t}
  \,=\,
  \v{\omega} \times \v{r}
  \quad \mbox{where} \quad
  \v{\omega}
  \,=\,
  \frac{\diff\phi}{\diff t} \, \v{\hat{\omega}}\,.
\]

\subsubsection{Angular momentum}

Now consider the \emph{angular momentum} of a rotating particle, this is defined by
$\v{L}= \v{r}\times (m\v{v})$ where $m$ is the mass of the particle. Using the above
expression for $\v{v}$ we obtain
\[
  \v{L}
  \,=\, m \v{r}\times(\v{\omega}\times \v{r})
  \,=\, m \left[ \v{r}^2 \, \v{\omega}  \,-\, (\v{r}\cdot \v{\omega}) \, \v{r} \right]
\]
where we have used the identity for the vector triple product. Note that only if $\v{r}$
is perpendicular to $\v{\omega}$ do we obtain $\v{L} = m \v{r}^2 \v{\omega}$, which means
that only then are $\v{L}$ and $\v{\omega}$ in the same direction. Also note that
$\v{L}=0$ if $\v{\omega}$ and $\v{r}$ are parallel.