% !TEX root = ../../fields.tex
%% Brian Lecture 1 %%%%%%%%%%%%%%%%%%%%%%%%%%%%%%%%%%%%%%%%%%%%%%%%%%%%%%%%%%%%%%%%%%%%%%%
\subsection{Physics terminology}
To get started, we recall two basic geometrical definitions:
\begin{itemize}
  \item \textbf{Scalar}: quantity specified by a single number;
  \item \textbf{Vector}: quantity specified by $D$ numbers in $D$ dimensions.
\end{itemize}
In this course, we will (re)define vectors by the way their components transform under
rotations of the coordinate basis, and scalars as numbers that are unchanged by such
transformations.\footnote{ These definitions can be generalised to define \emph{tensors}
  in a straightforward way-- see next year.}

\noindent\emph{Examples:} velocity is a vector, speed is a scalar.

\subsection{Geometrical approach}
A vector is \emph{represented} by a `directed line segment' with a length and direction
proportional to the magnitude and direction of the vector (in appropriate units). A
vector can be considered as a class of equivalent directed line segments. For example,
in~\cref{fig:dls} both displacements from P to Q and from R to S are represented by the
same vector.
\begin{figure}[t]
  \centering
  \includegraphics{tikz_dls.pdf}
  \caption{Representation of vectors as a displacement between two points.}
  \label{fig:dls}
\end{figure}

\noindent\textbf{Notation}: For a vector $\v{a}$ its magnitude (or
length) is noted $|\v{a}|$. A unit vector is often denoted by a hat $\v{\hat{a}} = \v{a}\ /\, |\v{a}|$ and represents a direction. Clearly, $|\v{\hat{a}}| = 1$.

The vector $\v{n}$ is generally taken to be a unit vector (usually without a hat).
$\v{n}$ is often the unit vector normal to a surface or plane.

Sometimes, we use an `overarrow' to write $ \overrightarrow{PQ} = \overrightarrow{RS}$.

\subsubsection*{Addition of vectors -- parallelogram law (geometrical)}

{\it i.e.}\\
\parbox{4cm}{
  \includegraphics{tikz_parallelogram.pdf}
}
\parbox{35em}{
  \begin{eqnarray*}
    \v{a} + \v{b} & = & \v{b} + \v{a}\mbox{\hspace*{10ex} (commutative) }\\
    (\v{a} + \v{b}) + \v{c} & = & \v{a} + (\v{b} + \v{c})
    \mbox{\hspace*{5ex} (associative) }
  \end{eqnarray*}
}

\subsubsection*{Multiplication by scalars} A vector $\v{a}$ may be
multiplied by a scalar $\alpha$ to give a new vector $\alpha \,
  \v{a}$, {\it e.g.} \\[3mm]
\hspace*{1cm}
\epsfysize=1.5cm
\includegraphics{tikz_scalar_mult.pdf}

Also, for scalars $\alpha$, $\beta$ and vectors $\v{a}$ and $\v{b}$
\begin{eqnarray*}
  | \alpha \v{a} | & = & |\alpha| |\v{a}| \\
  \alpha(\v{a} + \v{b}) & = & \alpha\v{a} + \alpha\v{b}
  \;\;\qquad\mbox{  (distributive)}\\
  \alpha(\beta\v{a}) & = & (\alpha\beta)\v{a}\;\;\;
  \quad\qquad\mbox{  (associative)}\\
  (\alpha + \beta)\v{a} & = & \alpha\v{a} + \beta\v{a}\;.
\end{eqnarray*}

\subsection{Scalar or dot product}\label{sec:scalarproduct}

The scalar product (also known as the dot product or inner product) between two vectors
is \emph{defined} to be
\begin{center}
  \bigbox{ $\v{a}\cdot\v{b} \equiv ab\cos\theta, \mbox{ where $\theta$
        is the angle between $\v{a}$ and $\v{b}$} $ }
\end{center}
\parbox{3cm}{
  \epsfxsize=3cm
  \includegraphics{tikz_scalar_prod.pdf}}
\hspace*{15mm}
\parbox{50em}{
\bigbox{$\v{a}\cdot\v{b}$ is a scalar -- {\it i.e.} a single number.}
}

\subsubsection*{Notes on scalar product}

\begin{enumerate}
  \item $\v{a}\cdot\v{b}=\v{b}\cdot\v{a} \quad
          \mbox{(commutative)}$

  \item $\v{a}\cdot(\v{b}+\v{c})=\v{a}\cdot\v{b}+ \v{a}\cdot\v{c} \qquad
          \mbox{(distributive)}$

  \item $\v{n}\cdot \v{a}= a \cos\theta~=$ the
        scalar projection of $\v{a}$ onto $\v{n}$, where $\v{n}$ is a unit
        vector

  \item $(\v{n}\cdot \v{a})\, \v{n} = a \cos\theta\,
          \v{n}~=$ the vector projection of $\v{a}$ onto $\v{n}$

  \item A vector may be resolved with respect to some direction $\v{n}$ into a parallel component
        $\v{a}_\parallel = \left(\v{n}\cdot \v{a}\right) \v{n}$ and a perpendicular component
        $\v{a}_\perp = \v{a}-\v{a}_\parallel$. You should check that $\v{a}_\perp\cdot \v{n} =0$

  \item \bigbox{$\v{a}\cdot\v{a} \equiv |\v{a}|^2 \equiv
            a^2$} which defines the magnitude (or norm) $|\v{a}|$ of a vector.

  \item For a unit vector, $\v{\hat{a}}\cdot\v{\hat{a}}= 1$.

\end{enumerate}

\subsection{The vector or `cross' product}

\begin{center}
  \bigbox{$\v{a}\times\v{b} \ \equiv \ ab\sin\theta\;\v{n}\;,$ where
    $\v{n}$ is in the `right-hand screw direction'}
\end{center}
$\v{n}$ is a unit vector normal to the plane of $\v{a}$ and
$\v{b}$, in the direction of a right-handed screw for rotation of
$\v{a}$ to $\v{b}$ (through $< \pi$ radians).

\parbox{3cm}{
  \epsfxsize=3cm
  \includegraphics{tikz_vector_prod.pdf}}
\hspace*{20mm}
\parbox[b]{60em}{
  \bigbox{$\v{a}\times\v{b}$ is a vector -- \emph{i.e.} it has
    a direction and a length.}
}

[It is sometimes called the \emph{wedge} product (although this has a
  more general meaning), and in this case is denoted by   $\v{a}\wedge\v{b}$. ]

\subsubsection*{Notes on vector product}
\begin{enumerate}
  \item $\v{a}\times\v{b}=-\v{b}\times\v{a}$ \quad
        (\emph{not} commutative)

  \item $\v{a}\times \v{b}= 0$\ if\ $\v{a},\v{b}$ are
        parallel

  \item $\v{a}\times( \v{b} + \v{c} ) = \v{a}\times \v{b} +
          \v{a}\times\v{c} $

  \item $\v{a}\times( \alpha \v{b} ) = \alpha \v{a}\times
          \v{b}$
\end{enumerate}

%An example from mechanics is
%\v{v}= \v{\omega} \times \v{r}$ where $\v{omega}$ is angular velocity

\subsection{The scalar triple product}
The scalar triple product is defined as follows
\begin{eqnarray*}
  (\v{a},\v{b},\v{c}) & \equiv & \v{a}\cdot(\v{b}\times\v{c})
  %& = & \v{c}\cdot(\v{a}\times\v{b})\\
  %& = & \v{b}\cdot(\v{c}\times\v{a})\,.
\end{eqnarray*}

\textbf{Notes}

\begin{enumerate}
  \item
        If $\v{a}$, $\v{b}$ and $\v{c}$ are three concurrent edges
        of a parallelepiped, the volume is $(\v{a},\v{b},\v{c})$.

        %To see this, note that\\
        \hspace*{-3ex}
        \parbox{10cm}
        {To see this, note that:
          \begin{eqnarray*}
            \mbox{area of the base}&=&\mbox{area of parallelogram $OBDC$}\\
            &=&
            b\;c\,\sin\theta \,=\, |\v{b}\times\v{c}|\\
            \mbox{height}
            &=&
            a \cos \phi \,=\, \v{n}\cdot\v{a}\\
            \mbox{volume}
            &=&
            \mbox{area of base $\times$ height}\\
            &=&
            b\, c\, \sin\theta\, \v{n} \cdot \v{a}\\
            &=&
            \v{a}\cdot( \v{b}\times \v{c} )
          \end{eqnarray*}\\
        }
        \parbox{6cm}{ \epsfxsize=6cm \includegraphics{tikz_volume.pdf}}

        \vspace*{-5ex}

        In this case $\v{n}$ is a unit vector parallel to $\v{b}\times\v{c}$, which is normal to
        the plane of $\v{b}$ and $\v{c}$.

  \item
        If we choose $\v{c}, \v{a}$ to define the base then a similar calculation
        gives volume = $\v{b}\cdot( \v{c}\times \v{a})$

        We deduce the following symmetry/antisymmetry properties (exercise):
        \[
          (\v{a}, \v{b}, \v{c} )
          \,=\,  (\v{b}, \v{c}, \v{a} )
          \,=\,  (\v{c}, \v{a}, \v{b} )
          \,=\, -(\v{a}, \v{c}, \v{b} )
          \,=\, -(\v{b}, \v{a}, \v{c} )
          \,=\, -(\v{c}, \v{b}, \v{a} )
        \]

  \item

        \bigbox{
          \parbox{0.8\textwidth}{
            \medskip
            If $\ds \v{a}, \v{b}$ and $\v{c}$ are \emph{coplanar}
            (\emph{i.e.}~all three vectors lie in the same plane)
            then $\ds V = (\v{a},\v{b},\v{c}) = 0$, and vice-versa.
            \medskip
          }}
\end{enumerate}

\subsection{The vector triple product}

There are \emph{several} ways of combining 3 vectors to form a new vector.\\ {\it e.g.}
$\v{a}\times(\v{b}\times\v{c})$; $(\v{a}\times\v{b})\times\v{c}$, etc. Note carefully
that \emph{brackets are important}, since the cross product is \emph{not} associative
\[
  \v{a}\times(\v{b}\times\v{c}) ~\neq~ (\v{a}\times\v{b})\times\v{c}\;.
\]
Expressions involving two (or more) vector products can be simplified by using the
identity
\begin{center}
  \bigbox{$\v{a}\times(\v{b}\times\v{c}) \ =
      \ (\v{a}\cdot\v{c})\, \v{b} - (\v{a}\cdot\v{b})\,\v{c}$} .
\end{center}
This is a result you \emph{must} know -- memorise it!

It's sometimes known as the `bac-cab rule', but you must write the vectors in front of
the scalar products to use it correctly: $\v{a}\times(\v{b}\times\v{c}) = \v{b} \,
  (\v{a}\cdot\v{c})- \v{c} \, (\v{a}\cdot\v{b})$

\parbox{13.5cm}{\textbf{Proof:} First note that $\v{b}\times\v{c}$ is
  $\perp$ to the $(\v{b},\, \v{c})$ plane. Now $\v{a} \times
    (\v{b}\times\v{c})$ is $\perp$ to $\v{b}\times\v{c}$, so it must lie
  in $(\v{b},\, \v{c})$ plane. Hence we can write
  \[
    \v{a}\times(\v{b}\times\v{c}) = \beta\,\v{b} + \gamma\,\v{c}
  \]
  with $\beta$, $\gamma$ scalars which must be linear in $\v{a}$ \& $\v{c}$ and in $\v{a}$
  \& $\v{b}$ respectively. (Exercise: think about why this must be the case.)

  \medskip

  Taking the scalar product with $\v{a}$ gives
  \[
    \v{a} \cdot (\v{a}\times(\v{b}\times\v{c}))
    \,=\,
    0
    \,=\,
    \beta \, (\v{a}\cdot\v{b}) \,+\, \gamma \, (\v{a}\cdot\v{c}),
  \]
  and from this we deduce that $\beta = \alpha\, (\v{a}\cdot\v{c})$, $\gamma = -\alpha \,
    (\v{a}\cdot\v{b})$ for some \emph{constant} $\alpha$, to give
  \[
    \v{a}\times(\v{b}\times\v{c})
    \,=\,
    \alpha
    \left[
      \left(\v{a}\cdot\v{c}\right)\v{b}
      \,-\,
      \left(\v{a}\cdot\v{b}\right)\v{c}
      \right]
  \]
  The constant $\alpha$ may be determined by considering the particular case when $\v{b} \,
    \perp \v{c}$ and $\v{c} \, \parallel \, \v{a}$ (\emph{e.g.}~$\v{b} = b\,\v{e}_{\,x}$,\ \
  $\v{c} = c\,\v{e}_{\,y}$,\ \ $\v{a} = a\,\v{e}_{\,y}$), to give
  %%% \footnote{$\v{e}_{\,x}$, $\v{e}_{\,y}$ and $\v{e}_{\,z}$
  %%% are unit vectors parallel to the $x$, $y$ and $z$ axes)}
  \[
    \v{a}\times(\v{b}\times\v{c})
    \,=\,
    ac\,\v{b}
    \,=\,
    \alpha \,(ac\v{b}-0),
  \]
  so $\alpha = 1$. (Exercise: work through this for yourself.)} \hspace*{-3ex}
\parbox{4.5cm}{ \epsfxsize=4cm \includegraphics{tikz_vector_triple_prod.pdf} }

Note that this is a \emph{coordinate-independent} derivation of the bac-cab rule. We
didn't use orthonormal basis vectors or take components in any basis (unless using the
explicit $\v{b} = b\,\v{e}_{\,x}$, etc, in the very last step) to determine the constant
$\alpha$.\footnote{By \emph{constant} $\alpha$, we mean $\alpha$ is a number that's
  independent of the vectors $\v{a}$, $\v{b}$, $\v{c}$ \& $\v{d}$.}

\mnote{01L 08/01/08}
\mnote{01L 13/01/09}

\subsection{Some examples of vector products in physics}

\textbf{Torque}

The \emph{torque} or \emph{couple} or \emph{moment} of a force about the origin is
defined as $\v{G} = \v{r} \times \v{F}$ where $\v{r}$ is the position vector of the point
where the force is acting and $\v{F}$ is the force vector at that point. Thus torque
about the origin is a vector quantity.

\parbox{3cm}{
  \epsfxsize=3cm
  \includegraphics{tikz_torque.pdf}}
\hspace*{1cm}
\parbox{12cm} {The magnitude of the torque about an axis through the
  origin in direction $\v{n}$ is given by $\v{n} \cdot \v{G} =
    \v{n}\cdot (\v{r} \times \v{F})$. Note that this is a scalar
  quantity formed by a scalar triple product.  }

\textbf{Angular velocity}

\parbox{12cm}{
  %
  Consider a point in a rigid body rotating with \emph{angular velocity} $\v{\omega}\,$,
  where $\,|\v{\omega}| \equiv \omega$ is the angular speed of rotation measured in radians
  per second, and the unit vector $\v{\hat{\omega}}$ lies along the axis of rotation, so
  $\v{\omega} = \omega \, \v{\hat{\omega}}$. Let the position vector of the point with
  respect to an origin $O$ on the axis of rotation be $\v{r}$.

  The velocity of the point is \qquad \bigbox{ \( \v{v} \,=\, \v{\omega}\times\v{r} \) }
  \medskip

  We can show this as follows:}
%
%From the first figure, we note that $\v{v}$ is parallel to $\v{\omega}\times\v{r}$.  }
%
\hspace{1cm}
%
\parbox{3cm}{
  \epsfxsize=3cm
  \includegraphics{tikz_rotate.pdf}
}\hspace*{1cm}

%\parbox{12cm}{

% You should convince yourself that the point's velocity is given by
% $\v{v}= \v{\omega}\times \v{r}$ by checking that this gives the right
% direction for $\v{v}$; that it is perpendicular to the plane of
% $\v{\omega}$ and $\v{r}$; that the magnitude $|\v{v}| = \omega r \sin
% \theta$ = $\omega \rho$, where $\rho$ is the radius of the circle in
% which the point is travelling. }

%

%\vspace*{-4ex}

\parbox{0.67\textwidth}{
  %

  \vspace*{0.5ex}

  In time $\delta t$, the point rotates through angle $\delta\phi$ from $\v{r}$ to $\v{r} +
    \delta\v{r}$.

  From the second figure, the distance $|\delta \v{r}|$ moved in time $\delta t$, is
  \[
    |\delta \v{r}|
    \,= \,
    r\sin\theta \: \delta\phi
    \,=\,
    \left|\v{\hat{\omega}} \times \v{r}  \right| \, \delta\phi
  \]
  Since $\delta\v{r}$ is parallel to $\v{\omega}\times\v{r}$, we deduce that
  \[
    \delta\v{r}
    \,=\,
    \v{\hat{\omega}} \times \v{r} \; \delta\phi
  \]
  and hence
  \[
    \frac{\delta\v{r}}{\delta t}
    \,=\,
    \v{\hat{\omega}} \times \v{r} \; \frac{\delta\phi}{\delta t}
  \]
  In the limit $\delta t \to 0$, we find
  \[
    \v{v}
    \,=\,
    \frac{d\v{r}}{dt}
    \,=\,
    \v{\omega} \times \v{r}
    \quad \mbox{where} \quad
    \v{\omega}
    \,=\,
    \frac{d\phi}{dt} \, \v{\hat{\omega}}
  \]
}
%
\hspace*{0.05\textwidth}
%
\parbox{0.3\textwidth}{
  \begin{center}
    \epsfxsize=0.25\textwidth
    \includegraphics{fig_small_rotation-eps-converted-to.pdf}
  \end{center}
}

\textbf{Angular momentum}

Now consider the \emph{angular momentum} of a rotating particle, this is defined by
$\v{L}= \v{r}\times (m\v{v})$ where $m$ is the mass of the particle.

Using the above expression for $\v{v}$ we obtain
\[
  \v{L}
  \,=\, m \v{r}\times(\v{\omega}\times \v{r})
  \,=\, m \left[ r^2 \, \v{\omega}  \,-\, (\v{r}\cdot \v{\omega}) \, \v{r} \right]
\]
where we have used the identity for the vector triple product. Note that only if $\v{r}$
is perpendicular to $\v{\omega}$ do we obtain $\v{L} = m r^2 \v{\omega}$, which means
that only then are $\v{L}$ and $\v{\omega}$ in the same direction. Also note that
$\v{L}=0$ if $\v{\omega}$ and $\v{r}$ are parallel.