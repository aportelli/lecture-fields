% !TEX root = ../../fields.tex
%% Brian Lecture 2 %%%%%%%%%%%%%%%%%%%%%%%%%%%%%%%%%%%%%%%%%%%%%%%%%%%%%%%%%%%%%%%%%%%%%%%
A \emph{position vector} is a vector of a point $P$ relative to some origin $O$ is the
vector $\overrightarrow{OP}$.

\subsection{The equation of a line}

Suppose that $P$ lies on a straight line which passes through a point $A$ which has a
position vector $\v{a}$ with respect to an origin $O$. Let $P$ have position vector
$\v{r}$ relative to $O$ and let $\v{u}$ be a vector through the origin in a direction
parallel to the line.
\begin{figure}[t]
  \centering
  \includegraphics{tikz_line.pdf}\hspace{1.5cm}
  \includegraphics{tikz_plane.pdf}
  \caption{Representation of the parametric equations of the line (left), and the plane (right).}
  \label{fig:line-plane}
\end{figure}
We may write
\[
  \v{r} = \v{a} + \lambda\v{u}
\]
which is the \emph{parametric equation of the line}, \ie as we vary the parameter
$\lambda$ from $-\infty$ to $\infty$, $\v{r}$ describes all points on the line,
\cf\cref{fig:line-plane}. Rearranging and using $\v{u}\times\v{u} = 0$, we can also write
this as
\[
  (\v{r} - \v{a})\times\v{u} \, = \, 0
\]
or \vspace*{-2ex}
\begin{center}
  \bigbox{ $\v{r}\times \v{u} = \v{c}$}
\end{center}
where $\v{c} = \v{a}\times \v{u}$ is normal to the plane containing
the line and origin. The magnitude $|\v{c}|$ is the perpendicular distance of the
line from the origin, in units of $\v{u}$.

\paragraph{Physical example:}
If the angular momentum $\vL$ of a particle and its velocity $\v{v}$ are known, we still
don't know the position exactly because the solution for $\v{r}$ of $\vL = m \vr \times
  \vv $ is a line $\vr = \vr_{0} + \lambda \vv$.

\medskip

\subsubsection{Notes:}
\begin{enumerate}

  \item $\v{r}\times \v{u} = \v{c}$ is an \emph{implicit
          equation} for a line

  \item $\v{r}\times \v{u} = 0$ is the equation of a line
        through the origin.

\end{enumerate}

\subsection{The equation of a plane}
For a given plane in three dimensions, let $\v{r}$ be the position vector of an arbitrary
point $P$ on the plane, $\v{a}$ the position vector of a fixed point $A$ in the plane,
$\v{u}$ and $\v{v}$ are parallel to the plane but non-collinear, \ie $\v{u}\times\v{v}
  \neq 0$. Then the vector $\overrightarrow{AP}$ in terms of $\v{u}$ and $\v{v}$ (they are
coplanar), so that:
\[
  \v{r} = \v{a} + \overrightarrow{AP}
  = \v{a} + \lambda\v{u} + \mu\v{v}
\]
for some $\lambda$ and $\mu$. This is the \emph{parametric equation of the plane}.

We define the \emph{unit normal} to the plane
\[
  \v{n} = \frac{\v{u}\times\v{v}}{|\v{u}\times\v{v}|}\;.
\]
Since $\v{u}\cdot\v{n} = \v{v}\cdot\v{n} = 0$, we have the implicit equation
\[
  (\v{r} - \v{a})\cdot\v{n} = 0\;.
\]
Alternatively, we can write this as
\begin{center}
  \bigbox{$\v{r}\cdot\v{n} ~=~ p$}
\end{center}
where $p = \v{a}\cdot\v{n}$ is the perpendicular distance of the plane
from the origin. In particular, $\v{r}\cdot\v{n} = 0$ is the equation for a plane
through the origin (with unit normal $\v{n}$).

\subsection{Examples of dealing with vector equations}

Before going through some worked examples let us state two simple rules that can help to
avoid many common mistakes
\begin{enumerate}
  \item
        %
        {\bfseries Always} check that the quantities on both sides of an equation
        are of the same type. For example, any equation of the form
        \emph{vector} = \emph{scalar} is clearly wrong, as is \emph{vector} =
        \emph{scalar} + \emph{vector}.

  \item {\bfseries Never} try to divide by a vector -- there is no such
        operation!
\end{enumerate}

\paragraph{Example 1:} Is the following set of equations with variable $\v{r}$ consistent?
\begin{eqnarray}
  \v{r}\times\v{b} \,=\, \v{c} \label{line} \\[0.5ex]
  \v{r} \,=\, \v{a}\times\v{c}
  \label{pt}
\end{eqnarray}
Geometrical interpretation: the first equation is the (implicit)
equation for a line whereas the second equation is the (explicit)
equation for a point. Thus the question is whether the point is on the
line. To do this, take the vector product of equation~(\ref{pt}) with $\v{b}$
%If we insert equation~(\ref{pt}) for $\v{r}$ into the LHS of
%equation~(\ref{line}) we find
\begin{equation}
  \v{r}\times\v{b}
  \,=\,
  (\v{a}\times\v{c})\times\v{b}
  \,=\,
  -\v{b}\times(\v{a}\times\v{c})
  \,=\,
  -\v{a}\,(\v{b}\cdot\v{c})+\v{c}\,(\v{a}\cdot\v{b})
  \label{linea}
\end{equation}
Now from (\ref{line}) we have that $\v{b}\cdot\v{c} = \v{b}\cdot
  (\v{r}\times\v{b})=0$ thus (\ref{linea}) becomes
\begin{equation}
  \v{r}\times\v{b}
  =
  \v{c}\, (\v{a}\cdot\v{b})
  \label{lineb}
\end{equation}
so that, on comparing (\ref{line}) and (\ref{lineb}), we require
\[
  \v{a}\cdot\v{b}=1
\]
for the equations to be consistent.

\paragraph{Example 2:}
Solve the following set of equations for $\v{r}$.
\begin{eqnarray}
  \v{r}\times\v{a} &=& \v{b}			\label{line2} \\
  \v{r}\times\v{c} &=& \v{d}			\label{line3}
\end{eqnarray}
Geometrical interpretation: both equations are equations for lines,
\emph{e.g.}~(\ref{line2}) is for a line parallel to $\v{a}$ where
$\v{b}$ is normal to the plane containing the line and the origin. The
problem is to find the intersection of two lines -- assuming the
equations are consistent and the lines do indeed have an intersection.
Extending what we did in Example~1, we can check consistency and indeed solve these
equations for $\v{r}$ by taking scalar and vector products of the equations with the
vectors they contain, namely $\v{a}$, $\v{b}$, $\v{c}$ and $\v{d}$.
Are these equations consistent? Take the scalar product of (\ref{line2}) with $\v{c}$,
and of (\ref{line3}) with $\v{a}$:
\begin{eqnarray}
  (\v{r}\times\v{a}) \cdot \v{c} &=& \v{b} \cdot \v{c}  \label{line22} \\
  (\v{r}\times\v{c}) \cdot \v{a} &=& \v{d} \cdot \v{a}  \label{line33}
\end{eqnarray}
Using the cyclic properties of the scalar triple product, we must have
$\v{b}\cdot\v{c} = - \v{d}\cdot\v{a}$ for consistency.
To solve (\ref{line2}) and (\ref{line3}), we take the vector product of
equation~(\ref{line2}) with $\v{d}$, which gives
\[
  \v{b}\times\v{d}
  \,=\,  (\v{r}\times\v{a})\times\v{d}
  \,=\, -\v{d}\times(\v{r}\times\v{a})
  \,=\, -\v{r}\, (\v{a}\cdot\v{d}) + \v{a}\, (\v{d}\cdot\v{r})
\]
From (\ref{line3}) we see that $\v{d}\cdot\v{r} = \v{r}\cdot(\v{r}\times\v{c})= 0$, so
the solution is
\[
  \v{r}
  \,=\,
  -\frac{\v{b}\times\v{d}}{\v{a}\cdot\v{d}}
  \qquad(\mbox{for\ \ }
  \v{a}\cdot\v{d}\neq 0)
\]
Alternatively, we could have taken the vector product of $\v{b}$ with
equation~(\ref{line3}) to obtain
\[
  \v{b}\times\v{d}
  \,=\,
  \v{b}\times(\v{r}\times\v{c})
  \,=\,
  \v{r}\, (\v{b}\cdot\v{c}) - \v{c}\, (\v{b}\cdot\v{r})\;.
\]
From equation~(\ref{line2}), we find $\v{b}\cdot\v{r} = 0$, hence
\[
  \v{r} \,=\, \frac{\v{b}\times\v{d}}{\v{b}\cdot\v{c}}
  \qquad(\mbox{for\ \ }
  \v{b}\cdot\v{c}\neq 0)
\]
in agreement with our first solution (when $\v{b}\cdot\v{c} = - \v{d}\cdot\v{a}$).

\mnote{02L 11/01/08}

What happens when $\v a \cdot \v d = \v b \cdot \v c= 0$? In this case the above approach
does not give an expression for $\v{r}$. However from (\ref{line33}) we see
$\v{a}\cdot\v{d} = 0$ implies that $\v{a}\cdot(\v{r}\times\v{c})=0$ so that $\v{a},\;
  \v{c},\; \v{r}$ are coplanar. We can therefore write $\v{r}$ as a linear combination of
$\v{a}, \v{c}$:
\begin{equation}
  \v{r} \,=\, \alpha\, \v{a} + \gamma\,\v{c} \; .
  \label{lincom}
\end{equation}
To determine the scalar $\alpha$ we can take the vector product with
$\v{c}$ to find
\begin{equation}
  \v{d}\,=\, \alpha\,\v{a}\times\v{c}
  \label{alpha}
\end{equation}
(because $\v{r}\times\v{c} = \v{d}$ from (\ref{line3}) and
$\v{c}\times\v{c} =0$). In order to extract $\alpha$ we need to
convert the vectors in (\ref{alpha}) into scalars. We do this by
taking, for example, a scalar product with $\v{b}$
\[
  \v{b}\cdot\v{d} \,=\, \alpha\,\v{b}\cdot(\v{a}\times\v{c})
\]
so that
\[
  \alpha \,=\, \frac{-\v{b}\cdot\v{d}}{\stpr{a}{b}{c}}\;.
\]
Similarly, one can determine $\gamma$ by taking the vector product of (\ref{lincom}) with
$\v{a}$:
\[
  \v{b} \,=\, \gamma\,\v{c}\times\v{a}
\]
then taking a scalar product with $\v{b}$ to obtain finally
\[
  \gamma \,=\, \frac{\v{b}\cdot\v{b}}{\stpr{a}{b}{c}} \;\;.
\]

\mnote{02L 16/01/09}

\paragraph{Example 3:} Solve for $\v{r}$ the vector equation
\begin{equation}
  \v{r} + (\v{n}\cdot\v{r})\;\v{n} +2\v{n}\times\v{r} + 2\v{b}
  \,=\, \v{0}
  \label{eq}
\end{equation}
where $\v{n}\cdot\v{n} = 1$. In order to unravel this equation we can try taking scalar
and vector products of the equation with the vectors involved. However straight away we
see that taking various products with $\v{r}$ will not help, since it will produce terms
that are quadratic in $\v{r}$. Instead, we want to eliminate $(\v{n}\cdot\v{r})$ and
$(\v{n}\times\v{r})$ so we try taking scalar and vector products with $\v{n}$. Taking the
scalar product of $\v{n}$ with both sides of equation~(\ref{eq}) one finds
\[
  \v{n}\cdot\v{r} + (\v{n}\cdot\v{r})(\v{n}\cdot\v{n})
  + 0 + 2\v{n}\cdot\v{b}
  \,=\, 0
\]
so that, since $(\v{n}\cdot\v{n}) = 1$, we have
\begin{equation}
  \v{n}\cdot\v{r} \,=\, -\v{n}\cdot\v{b}
  \label{ndotr}
\end{equation}
Taking the vector product of $\v{n}$ with equation~(\ref{eq}) gives
\[
  \v{n}\times\v{r} + 0
  + 2 \left[\, \v{n}(\v{n}\cdot\v{r}) - \v{r} \,\right]
  + 2\v{n}\times\v{b}
  \,=\, \v{0}
\]
so that
\begin{equation}
  \v{n}\times\v{r}
  \,=\,
  2\left[\, \v{n}(\v{b}\cdot\v{n}) + \v{r} \,\right] - 2\v{n}\times\v{b}
  \label{ncrossr}
\end{equation}
where we have used (\ref{ndotr}). Substituting (\ref{ndotr}) and
(\ref{ncrossr}) into (\ref{eq}) one (eventually) obtains
\begin{equation}
  \v{r}
  \,=\,
  \frac{1}{5}
  \left[\,
    - 3(\v{b}\cdot\v{n})\,\v{n} + 4(\v{n}\times\v{b})
    - 2\v{b}
    \,\right]
\end{equation}
