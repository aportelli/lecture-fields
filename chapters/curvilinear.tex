% !TEX root = ../fields.tex
As we have seen already, it is often convenient to work with coordinate systems other
than Cartesian coordinates $\{x_i\}$, \emph{i.e.}~($x_1$, $x_2$, $x_3$) or ($x$, $y$,
$z$).

\medskip
\noindent
\parbox{0.6\textwidth}{For example, spherical polar coordinates ($r$,
  $\theta$, $\phi$) with \\
  $0 \le r < \infty$,\ \ $0 \le \theta < \pi$,\ \ $0 \le \phi < 2\pi$\ \ are defined by:
  \begin{eqnarray*}
    x &=& r \sin(\theta) \cos(\phi) \qquad \\[0.5ex]
    y &=& r \sin(\theta) \sin(\phi) \qquad \\[0.5ex]
    z &=& r \cos(\theta)
  \end{eqnarray*}
}
\parbox{0.3\textwidth}{
  \setlength{\unitlength}{1.1pt}
  \begin{picture}(100,80)(-70,-30)
    \put(0,0){\vector(1,0){70}}
    \put(72,0){\makebox(0,0)[lc]{$y$}}
    \put(0,0){\vector(0,1){60}}
    \put(-2,60){\makebox(0,0)[rt]{$z$}}
    \put(0,0){\vector(-2,-1){50}}
    \put(-50,-22){\makebox(0,0)[rb]{$x$}}
    \put(0,0){\line(3,2){60}}
    \put(32,20){\makebox(0,0)[lt]{$r$}}
    \put(58,42){\makebox(0,0)[rb]{$P$}}
    \put(2,6){\makebox(0,0)[lb]{$\theta$}}
    \bezier{30}(0,15)(8,15)(12,8)
    \put(0,-2){\makebox(0,0)[ct]{$\phi$}}
    \bezier{40}(-12,-6)(0,-16)(15,-6)
    %\put(60,40){\vector(3,2){12}}
    %\put(74,48){\makebox(0,0)[lb]{$\er$}}
    %\put(60,40){\vector(1,-2){6}}
    %\put(66,26){\makebox(0,0)[lt]{$\etheta$}}
    %\put(60,40){\vector(4,1){12}}
    %\put(74,42){\makebox(0,0)[lt]{$\ephi$}}
    \bezier{30}(0,0)(30,-12)(60,-24)
    \bezier{30}(60,40)(60,10)(60,-24)
  \end{picture}
  \setlength{\unitlength}{1pt}
}

\medskip
\noindent We shall set up a formalism to deal with rather general coordinate systems, of which
spherical polars are a very important example. Suppose we make a transformation from the
Cartesian coordinates $(x_1, x_2, x_3)$ to the variables ($u_1$, $u_2$, $u_3$), which are
functions of the $\{\x{i}\}$
\begin{eqnarray*}
  u_1 & = & u_1(x_1,\, x_2,\, x_3)\\
  u_2 & = & u_2(x_1,\, x_2,\, x_3)\\
  u_3 & = & u_3(x_1,\, x_2,\, x_3)
\end{eqnarray*}
If the variables $\{u_{i}\}$ are single-valued functions of the
variables $\{\x{i}\}$, then we can make the inverse transformations,
\[
  \x{i} ~=~  \x{i}(u_1,\, u_2,\, u_3) \quad \mbox{for $i=1,2,3,$}
\]
except possibly at certain points. A point may be specified by its Cartesian coordinates
$\{\x{i}\}$, or its \emph{curvilinear coordinates}~$\{u_{i}\}$. We may define the
curvilinear coordinates by equations giving $\{x_i\}$ as functions of $\{u_i\}$, or
vice-versa.\footnote{Sometimes the curvilinear coordinates are called $(u,v,w)$, just as
  Cartesians are called $(x,y,z)$.}
\begin{itemize}

  \item For Cartesian coordinates, the surfaces $x_i = {\rm constant}$ $(i=1,2,3)$ are
        \emph{planes}, with (constant) normal vectors $\ei\,$ (the Cartesian basis vectors)
        \emph{intersecting at right angles}.

  \item In general curvilinear coordinates, the surfaces $u_i = {\rm constant}$ do \emph{not}
        have constant normal vectors, nor do they intersect at right angles. For example, in 2-D,
        we might have the situation illustrated in \cref{fig:curvy}.
        %
\end{itemize}
\begin{figure}[t]
  \centering
  \includegraphics[width=0.7\linewidth]{VCA11_curvy.pdf}
  \caption{Lines of constant coordinates in two-dimensional Cartesian and curvilinear coordinates.}
  \label{fig:curvy}
\end{figure}

From the definition of spherical polar coordinates ($r$, $\theta$, $\phi$), we have
\[
  r \,=\, \sqrt{x^2+y^2+z^2}\quad\quad
  \theta \,=\, \cos^{-1}\left\{\frac{z}{\sqrt{x^2+y^2+z^2}}\right\}
  \quad \quad
  \phi \,=\, \tan^{-1}\left(\frac{y}{x}\right)\;.
\]
The surfaces of constant $r,\,\theta$, and $\phi$ are
\begin{eqnarray*}
  r
  & = &
  \mbox{constant} \ \Rightarrow \
  \mbox{spheres centred at the origin}\\
  \theta
  & = &
  \mbox{constant} \ \Rightarrow
  \ \mbox{cones of semi-angle $\theta$ and axis along the $z$-axis}\\
  \phi
  & = &
  \mbox{constant} \ \Rightarrow
  \ \mbox{planes passing through the $z$-axis}
\end{eqnarray*}
Not all of these surfaces are planes, but they \emph{do} intersect at
right angles. The angle $\theta$ is undefined at the origin; $\phi$ is
undefined on the $z$-axis.

\section{Orthogonal curvilinear coordinates}
If the coordinate surfaces (surfaces of constant $u_{i}$), intersect at right angles, as
in the example of spherical polars, the curvilinear coordinates are said to be
\emph{orthogonal}.

\subsection{Scale factors and basis vectors}
Let point $P$ have position vector $\v{r} = \v{r}(u_1,u_2,u_3)$. If we change $u_1$ by
$\diff u_1$ (with $u_2$ and $u_3$ fixed), then $\v{r} \to \v{r}+\diff \v{r}$, with
\[
  \diff \v{r}
  \,=\,
  \frac{\partial\v{r}}{\partial u_1} \, \diff u_1
  ~\equiv~
  h_1 \, \eone \, \diff u_1
\]
where we have defined the \emph{scale factor} $h_1$ and the \emph{unit~vector} $\eone$ by
\[
  h_1
  \,=\,
  \left|\frac{\partial\v{r}}{\partial u_1}\right|
  \quad\mbox{and} \quad
  \eone
  \,=\,
  \frac{1}{h_1} \, \frac{\partial\v{r}}{\partial u_1}
\]
\begin{itemize}
  \item The \emph{scale factor} $h_1$ gives the \emph{length} $h_1\, \dd u_1$ of $\diff \v{r}$
        when we change $u_1\to u_1 + \diff u_1$.

  \item $\eone$ is a \emph{unit vector} in the direction of increasing
        $u_1$ (with \emph{fixed} $u_2$ and $u_3$.)
\end{itemize}
Similarly, we can define $h_i$ and $\ei$ for $i = 2$ and 3.
In general, if we change a single $u_{i}$, keeping the other two fixed, we have
\[
  \frac{\partial\v{r}}{\partial u_{i}}
  \,=\,
  h_i\,\ei
  \qquad i = 1, 2, 3
  \qquad
  \mbox{[no sum on $i$]}
\]
It is important to note that:
\begin{itemize}
  \item
        The unit vectors $\{\ei\}$ are in general \emph{not} constant vectors
        -- their directions depend on the position vector $\vr$, and hence on
        the curvilinear coordinates $\{u_i\}$. [They should perhaps be called
            $\{\v{e}_{\,u_i}\}$ or $\{\v{e}_{\,u}, \,\v{e}_{\,v}, \,\v{e}_{\,w}\}$ to
            avoid confusion with Cartesian basis vectors.]

  \item
        If the curvilinear unit vectors satisfy $\ei\cdot\ej = \delt{ij}$, the
        $\{u_{i}\}$ are said to be \emph{orthogonal curvilinear coordinates},
        and the three unit vectors $\{\ei\}$ form an orthonormal basis. We
        will always \emph{choose} the $u_i$ to give a right-handed basis.
\end{itemize}

\subsection{Examples of orthogonal curvilinear coordinates (OCCs)}
\paragraph{Cartesian coordinates:}
\[
  \vr
  \,=\,
  x\,\ex \,+\, y\,\ey \,+\, z\,\ez
  \qquad \Rightarrow \qquad
  h_{x} \; \ex
  \,=\,
  \frac{\partial\v{r}}{\partial x}
  \,=\,
  \ex \,,\;
  \mbox{\emph{etc.}}
\]
The scale factors are all \emph{unity}, and each individual unit vector points in the
\emph{same} direction everywhere.

\paragraph{Spherical polar coordinates:} $u_1=r$, $u_2=\theta$,
$u_3=\phi\,$ (in that order)
\[
  \begin{array}{lcl}
    \v{r}
    =
    r \, \sin(\theta) \, \cos(\phi)  \ex +
    r \, \sin(\theta) \, \sin(\phi)  \ey +
    r \, \cos(\theta)  \ez \\[2ex]
    %
    {\ds\frac{\partial\v{r}}{\partial r}}
    =
    \sin(\theta) \, \cos(\phi)  \ex + \sin(\theta) \, \sin(\phi)  \ey +
    \cos(\theta)  \ez
     & \Rightarrow &
    h_{r}
    =
    \left|{\ds\frac{\partial\v{r}}{\partial r}}\right|
    =
    1                      \\[2.5ex]
    %
    {\ds\frac{\partial\v{r}}{\partial\theta}}
    =
    r \, \cos(\theta) \, \cos(\phi)  \ex +
    r \, \cos(\theta) \, \sin(\phi)  \ey -
    r \, \sin(\theta)  \ez
     & \Rightarrow &
    h_{\theta}
    =
    \left|{\ds\frac{\partial\v{r}}{\partial \theta}}\right|
    =
    r                      \\[2.5ex]
    %
    {\ds\frac{\partial\v{r}}{\partial\phi}} =
    -r \, \sin(\theta) \, \sin(\phi)  \ex +
    r  \sin(\theta) \, \cos(\phi)  \ey
     & \Rightarrow &
    h_{\phi}
    =
    \left|{\ds\frac{\partial\v{r}}{\partial \phi}}\right|
    =
    r \, \sin(\theta)
  \end{array}
\]
Hence the unit vectors for spherical polars are
\begin{eqnarray*}
  \er
  & = &
  \sin(\theta)\cos(\phi)\,\ex + \sin(\theta)\sin(\phi)\,\ey + \cos(\theta)\,\ez
  \,=\,
  \v{r}\,/\,r \\[1ex]
  %
  \e{\theta}
  & = &
  \cos(\theta)\cos(\phi)\,\ex + \cos(\theta)\sin(\phi)\,\ey - \sin(\theta)\,\ez \\[1ex]
  \e{\phi}
  & = &
  -\sin(\phi)\,\ex + \cos(\phi)\,\ey
\end{eqnarray*}
These unit vectors are normal to the surfaces described above
(spheres, cones and planes).

\noindent
\parbox{0.53\textwidth}{
  They are orthogonal:
  \[
    \er\cdot\etheta
    \,=\,
    \er\cdot\ephi
    \,=\,
    \etheta\cdot\ephi
    \,=\,
    0
  \]

  \medskip

  And they form a right-handed orthonormal basis:
  \[
    \er\times\etheta = \ephi \,, \quad
    \etheta\times\ephi = \er \,, \quad
    \ephi\times\er = \etheta \,.
  \]

  \medskip

  See also tutorial sheet. }
%
%
\parbox{0.4\textwidth}
{
  \centerline{
    \setlength{\unitlength}{1.45pt}
    \begin{picture}(200,100)(-100,-30)
      \put(0,0){\vector(1,0){70}}
      \put(72,0){\makebox(0,0)[lc]{$y$}}
      \put(0,0){\vector(0,1){60}}
      \put(-2,60){\makebox(0,0)[rt]{$z$}}
      \put(0,0){\vector(-2,-1){50}}
      \put(-50,-22){\makebox(0,0)[rb]{$x$}}
      \put(0,0){\line(3,2){60}}
      \put(32,20){\makebox(0,0)[lt]{$r$}}
      \put(58,42){\makebox(0,0)[rb]{$P$}}
      \put(2,6){\makebox(0,0)[lb]{$\theta$}}
      \bezier{30}(0,15)(8,15)(12,8)
      \put(0,-2){\makebox(0,0)[ct]{$\phi$}}
      \bezier{40}(-12,-6)(0,-16)(15,-6)
      \put(60,40){\vector(3,2){12}}
      \put(74,48){\makebox(0,0)[lb]{$\er$}}
      \put(60,40){\vector(1,-2){6}}
      \put(66,26){\makebox(0,0)[lt]{$\etheta$}}
      \put(60,40){\vector(4,1){12}}
      \put(74,42){\makebox(0,0)[lt]{$\ephi$}}
      \bezier{30}(0,0)(30,-12)(60,-24)
      \bezier{30}(60,40)(60,10)(60,-24)
    \end{picture}
  }
}

\paragraph{Cylindrical coordinates:} $u_1=\rho$, $u_2=\phi$,
$u_3=z\,$ (in that order)
\[
  \v{r}
  \,=\,
  \rho \, \cos(\phi)\; \ex + \rho\, \sin(\phi) \; \ey + z \; \ez
\]
\[
  \Rightarrow \quad
  {\ds\frac{\partial\v{r}}{\partial \rho}}
  \,=\,
  \cos(\phi)\;\ex + \sin(\phi)\;\ey
  \qquad
  {\ds\frac{\partial\v{r}}{\partial\phi}}
  \,=\,
  -\rho \, \sin(\phi) \; \ex +\rho \, \cos(\phi) \; \ey
  \qquad
  {\ds\frac{\partial\v{r}}{\partial z}}
  \,=\,
  \ez
\]
The scale factors are then (tutorial) \( h_{\rho} = 1 \,,\: h_{\phi} = \rho \,,\: h_{z} =
1 \,,\: \) and the basis vectors are
\[
  \e{\rho}
  \,=\,
  \cos(\phi)\;\ex + \sin(\phi)\;\ey
  \qquad
  \e{\phi}
  \,=\,
  -\sin(\phi)\,\ex + \cos(\phi)\,\ey
  \qquad
  \e{z}
  \,=\,
  \ez
\]
These unit vectors are normal to surfaces which are (respectively): cylinders centred on
the $z$-axis ($\rho$ = constant), planes through the $z$-axis ($\phi$ = constant), planes
perpendicular to the $z$ axis ($z$ = constant), and they are clearly orthonormal.

\section{Elements of length, area and volume}

\subsection{Vector length}
If we change $u_1\to u_1+\diff u_1$, keeping $u_2$ and $u_3$ \emph{fixed}, then $\v{r}\to
  \v{r} + \diff \v{r}_{\,1}$ with $\diff \v{r}_{\,1} = h_1 \, \eone \, \diff u_1$. The
infinitesimal element of length along $\eone$ is $h_1\,\diff u_1$. Similarly, the
infinitesimal lengths along the \emph{curvilinear} basis vectors $\etwo$, $\ethree$, are
\( \,
%  h_1 \, \diff u_1
%  , \;
h_2 \, \diff u_2 , \; h_3 \, \diff u_3 \, \) respectively. If we change $u_i\to u_i+\diff
  u_i$, for all $i=1,2,3$, then%\mbox{and} \;
\[
  \diff \v{r}
  \,=\,
  h_1 \, \diff u_1 \; \eone
  \,+\,
  h_2 \, \diff u_2 \; \etwo
  \,+\,
  h_3 \, \diff u_3 \; \ethree
  \,=\,
  \sum_{i=1}^3   h_i \, \diff u_i \; \ei
\]
\textbf{Note:} We can't use the summation convention here -- because
this expression for $\diff \v{r}$ contains 3 identical indices!
Great care is needing when using the summation convention with OCCs.

\subsection{Arc length} % and metric tensor}
Defining $\dd s$ as the \emph{length} of the infinitesimal vector $\dd\v{r}$, we have
$(\dd s)^2 = \dd\v{r} \cdot \dd\v{r}$. In Cartesian coordinates:\ \ \ \( (\dd s)^2 \,=\,
(\dd x)^2 \,+\, (\dd y)^2 \,+\, (\dd z)^2 \). For an arbitrary set of curvilinear
coordinates (\emph{not} necessarily orthogonal), letting $u_{i} \to u_{i} + \dd u_{i}$,
$\forall \, i=1,2,3$, gives\footnote{The shorthand $\ds\sum_i\,$ means
  $\,\ds\sum_{i=1}^{3}\:$ etc.}
\begin{equation}
  (\dd s)^2
  \,=\,
  %  \dd\v{r} \cdot \dd\v{r}
  %  \,=\,
  \left(\sum_i h_{i} \, \ei \, \dd u_{i}\right)
  \cdot
  \left(\sum_j h_{j}\,\ej \, \dd u_{j}\right)
  \,=\,
  \sum_{ij} h_i \, h_j
  \left(\ei \cdot \ej \right) \diff u_{i} \, \diff u_{j}
  \label{eq:arclengthwithhs}
\end{equation}
For \emph{orthogonal} curvilinear coordinates, we have
$\ei{\,\cdot\,}\ej = \delt{ij}$, and hence
\begin{equation}
  (\dd s)^2
  \, = \,
  \sum_i h_{i}^2 \; (\dd u_{i})^2
  \, = \,
  h_1^2\; \dd u_1^2 \,+\, h_2^2\; \dd u_2^2 \,+\, h_3^2\; \dd u_3^2
  \label{eq:arclengthocc}
\end{equation}

\paragraph{Example:} For spherical polars, we showed that
\[
  h_{r} \,=\, 1
  \qquad
  h_{\theta} \,=\, r
  \qquad
  h_{\phi} \,=\, r \, \sin(\theta)
\]
In this case
\[
  (\diff s)^2
  \,=\,
  (\diff r)^2 +
  r^2 (\diff \theta)^2 +
  r^2  \sin^2\theta \; (\diff \phi)^2
\]

\subsection{Vector area}
If we let $u_1 \to u_1 + \diff u_1$, with $u_2$, $u_3$ fixed, then
\begin{center}
  $\v{r}\to \v{r}+\diff \v{r}_{\,1}$ with $\diff \v{r}_{\,1}
    = h_1\,\eone\, \diff u_1\,$
\end{center}
Similarly, if we let $u_2\to u_2+\diff u_2$, with $u_1$, $u_3$ fixed,
\begin{center}
  $\v{r}\to
    \v{r}+\diff \v{r}_{\,2}$ with $\diff \v{r}_{\,2} = h_2\,\etwo\,
    \diff u_2$
\end{center}
The \emph{vector area} of the infinitesimal parallelogram whose sides are the vectors
$\diff \v{ r\!}_{\,1}$ and $\diff \v{r}_{\,2}$ is
\[
  \diff \v{S}
  \,=\,
  (\diff \v{r}_{\,1})\times(\diff \v{ r\!}_{\,2})
  \,=\,
  \left(h_1 \, \diff u_1 \; \e1\right) \times
  \left(h_2 \, \diff u_2 \; \e2\right)
\]
For OCCs, this simplifies to
\[
  \diff \v{S}_{\,3}
  \,=\,
  h_1 \, h_2 \; \diff u_1 \, \diff u_2 \; \ethree\;,
\]
since $\eone\times\etwo = \ethree$ for orthogonal systems. $\diff \v{S\!}_{\,3}$ points
in the direction of $\ethree$, which is \emph{normal} to the surfaces $u_3 = $constant,
and the infinitesimal area is a rectangle. The vector areas $\diff \v{S}_{\,1}$ and
$\diff \v{S}_{\,2}$ are defined similarly.
\begin{figure}[t]
  \centering
  \includegraphics{tikz_curvy_area.pdf}
  \label{fig:curvy_area}
  \caption{Infinitesimal element of surface in curvilinear coordinates.}
\end{figure}

\paragraph{Example:} For \emph{spherical polars}, if we vary $\theta$ and $\phi$, keeping $r$ fixed, we easily
obtain a familiar result
\[
  \diff \v{S}_{\,r}
  \,=\,
  \left(h_{\theta} \, \diff \theta \: \etheta\right)
  \times
  \left(h_{\phi}   \, \diff \phi   \: \ephi \right)
  \,=\,
  h_{\theta} \, h_{\phi} \, \diff \theta \, \diff \phi \; \er
  \,=\,
  r^2  \sin(\theta) \, \diff \theta \, \diff \phi\,\er
\]
Similarly, if we vary $\phi$ and $r$, keeping $\theta$ fixed, we obtain the vector
element of area on a cone of semi-angle $\theta$, with its axis along the $z$ axis
\[
  \diff \v{S}_{\,\theta}
  \,=\,
  \left(h_{\phi}   \, \diff \phi   \: \ephi \right)
  \times
  \left(h_{r} \, \diff r \: \er \right)
  \,=\,
  h_{\phi} \, h_{r} \, \, \diff \phi \, \diff r \; \etheta
  \,=\,
  r \sin(\theta) \, \diff r \, \diff \phi \, \etheta
\]
Similarly for $\diff \v{S}_{\,\phi}$ (\cf exercises).

\subsection{Volume} The volume of the infinitesimal parallelepiped with edges $\diff \v{r\!}_{\,1}$,
$\diff\v{r\!}_{\,2}$ and $\diff \v{r\!}_{\,3}$ is
\[
  \diff V
  \, = \,
  \left|
  \left( \diff \v{r\!}_{\,1} \times \diff \v{r\!}_{\,2} \right)
  \cdot \diff \v{r\!}_{\,3}
  \right|
  \, = \,
  \left|
  \left(
  \left(h_1 \, \diff u_1\, \eone\right)   \times
  \left(h_2 \, \diff u_2\, \etwo\right)
  \right)
  \cdot
  \left(h_3 \, \diff u_3\, \ethree \right)
  \right| \\[1ex]
  %  & = & 
  %  h_1 \, h_2 \, h_3 \; \diff u_1 \, \diff u_2 \,\diff u_3
\]
For OCCs, we have $\left(\eone \times \etwo\right) \cdot \ethree=1$ (in a RH basis),
hence
\[
  \diff V
  \,=\,
  h_1 \, h_2 \, h_3 \; \diff u_1 \, \diff u_2 \,\diff u_3
\]
In this case, the infinitesimal volume is a cuboid. For \emph{spherical polars}, we have
\[ \diff V \,=\, h_r \, h_\theta \, h_\phi \, \diff r \, \diff \theta \, \diff \phi \,=\,
  r^2\sin(\theta) \, \diff r \, \diff \theta \, \diff \phi\,. \]

% For a general set of curvilinear coordinates
% \[
%   \diff V
%   \,=\,
%   \sqrt{g} \; \diff u_1 \, \diff u_2 \,\diff u_3
% \]
% where $g$ is the determinant of the metric tensor.

\section{Vector fields in curvilinear coordinates}

A vector field $\v{a}(\v{r})$ can be expressed in terms of \emph{curvilinear components}
$a_i$, defined by
\[
  \v{a}(\v{r})
  \,=\,
  \sum_{i=1}^3 a_{i}(u_1,u_2,u_3) \; \ei
\]
where $\ei$ is the $i^{\rm th}$ \emph{curvilinear} basis vector (which again should
really be called $\v{e}_{u_i}$ to avoid confusion with the Cartesian basis vectors.) For
\emph{orthogonal} curvilinear coordinates, the component $a_{i}$ can be obtained by
taking the scalar product of $\v{a}$ with the $i^{\rm th}$ curvilinear basis vector $\ei$
\[
  a_{i} \,=\, \v{a}(\v{r}) \cdot \ei
\]
NB: $a_{i}$ must be expressed in terms of $u_1$, $u_2$, $u_3$ (not $x,y,z$) when working
in the $\{u_{i}\}$ basis.

\paragraph{Example:} If $\v{a} = a\,\ex$ in Cartesians, then in spherical polars
\begin{align*}
  a_{r}
  \,=\,
  \v{a}\cdot\er
   & \,=\,
  \left(a \, \ex\right)
  \,\! \cdot
  \left(
  \sin(\theta)\cos(\phi) \, \ex
  \,+\,
  \sin(\theta)\sin(\phi) \, \ey
  \,+\,
  \cos(\theta) \, \ez
  \right)
  \\
   & \,=\,
  a \, \sin(\theta)\cos(\phi)
\end{align*}
Similarly, $a_\theta = \v{a}\cdot\etheta$ and $a_\phi = \v{a}\cdot\ephi$, and we obtain
$\v{a}$ in the spherical-polar basis (exercise)
\[
  \v{a}(r, \theta, \phi)
  \,=\,
  a
  \left(
  \sin(\theta)\cos(\phi) \, \er
  \,+\,
  \cos(\theta)\cos(\phi) \, \etheta
  \,-\,
  \sin(\phi) \:\ephi
  \right)
\]
In general, one chooses the set of coordinates which matches most closely the symmetries
of a given problem.

\section{Differential operator}
\subsection{Gradient}
The gradient of a scalar field\footnote{We use $f(\v{r})$ rather than $\phi(\v{r})$ in
  order to avoid confusion with the angle $\phi$ in spherical polars.} $f(\v{r})$ is
defined in terms of the change $\diff f(\v{r})$ in the field $f(\v{r})$ when $\v{r}\to
  \v{r}+\diff \v{r}$
\begin{equation}
  \diff  f(\v{r})
  \,=\,
  \grad f(\vr) \cdot \diff \v{r}
  \label{eq:def_grad}
\end{equation}
Now write $f(\v{r})$ in terms of orthogonal curvilinear
coordinates: $f(\v{r}) = f(u_1,\,u_2,\,u_3)$.
As usual, we denote the \emph{curvilinear} basis vectors by $\{\eone, \, \etwo,\,
  \ethree\}$.
Let $u_1\to u_1+\diff u_1$,\ \ $u_2\to u_2+\diff u_2$,\ \ and $u_3\to u_3+\diff u_3$.
Using the chain rule, we have
\begin{equation}
  \diff  f
  \,=\,
  \parpar{f}{u_1}\: \diff u_1 \,+\,
  \parpar{f}{u_2}\: \diff u_2 \,+\,
  \parpar{f}{u_3}\: \diff u_3
  \label{eq:occ-taylor}
\end{equation}
We need to rewrite the RHS of this equation in the form of
equation~(\ref{eq:def_grad}). Start with
\[
  \diff \v{r}
  \,=\,
  h_1 \, \diff u_1 \: \eone
  \,+\,
  h_2 \, \diff u_2 \: \etwo
  \,+\,
  h_3 \, \diff u_3 \: \ethree
\]
and use orthogonality of the \emph{curvilinear} basis vectors, $\ei\cdot\ej = \delt{ij}$,
to rewrite equation~(\ref{eq:occ-taylor}) as
\begin{eqnarray*}
  \diff f
  & = &
  \parpar{f}{u_1}\: \diff u_1 \,+\,
  \parpar{f}{u_2}\: \diff u_2 \,+\,
  \parpar{f}{u_3}\: \diff u_3 \\[1ex]
  %
  & = &
  \left(
  \parpar{f}{u_1}\: \eone  \,+\,
  \parpar{f}{u_2}\: \etwo  \,+\,
  \parpar{f}{u_3}\: \ethree
  \right)
  \cdot
  \left(
  \eone \, \diff u_1 \,+\, \etwo \, \diff u_2 \,+\, \ethree \, \diff u_3
  \right) \\[1ex]
  %
  & = &
  \left(
  \frac{1}{h_1}\; \parpar{f}{u_1}\; \eone  \,+\,
  \frac{1}{h_2}\; \parpar{f}{u_2}\; \etwo  \,+\,
  \frac{1}{h_3}\; \parpar{f}{u_3}\; \ethree
  \right)
  \cdot
  \left(
  h_1 \; \eone \; \diff u_1 \,+\,
  h_2 \; \etwo \; \diff u_2 \,+\,
  h_3 \; \ethree \; \diff u_3
  \right)\\[1ex]
  %
  & = &
  \left(
  \frac{1}{h_1}\; \parpar{f}{u_1}\; \eone  \,+\,
  \frac{1}{h_2}\; \parpar{f}{u_2}\; \etwo  \,+\,
  \frac{1}{h_3}\; \parpar{f}{u_3}\; \ethree
  \right)
  \cdot \diff \v{r}
\end{eqnarray*}
Comparing this result with equation~(\ref{eq:def_grad}), which holds
for all $\diff \v{r}$, gives us $\grad f$ in orthogonal
curvilinear coordinates
\begin{center}
  \bigbox{$
      \ds
      \grad f
      \,=\,
      \frac{1}{h_1}\; \parpar{f}{u_1}\; \eone  \,+\,
      \frac{1}{h_2}\; \parpar{f}{u_2}\; \etwo  \,+\,
      \frac{1}{h_3}\; \parpar{f}{u_3}\; \ethree
      \;
      \,=\,
      \sum_{i=1}^3 \, \frac{1}{h_{i}}\; \parpar{f}{u_{i}}\; \ei
    $}
\end{center}

For \emph{spherical polars}, $h_r=1$, $h_\theta=r$, $h_\phi=r\sin(\theta)$, and we have
\[
  \grad f(r,\theta,\phi)
  \,=\,
  \frac{\partial f}{\partial r} \; \er
  \,+\,
  \frac{1}{r} \: \frac{\partial f}{\partial\theta} \; \etheta
  \,+\,
  \frac{1}{r\sin(\theta)} \: \frac{\partial f}{\partial\phi} \; \ephi
\]

%%%
%%%	The following is not needed, and it's probably confusing...
%%%
% \vspace*{0.1ex}
% 
% 	We can use orthogonal curvilinear coordinates to 
% write $\del$ in operator form
% \[
% 	\del \,=\, \sum_{i=1}^3\,
%          \ei \; \frac{1}{h_{i}}\; \parpar{\ }{u_{i}}
% \]

\subsection{Divergence}
Let $\v{a}(\vr)$ be a vector field, which we write in orthogonal curvilinear coordinates
as
\[
  \v{a}(\v{r}) \,=\, \sum_{i=1}^3 a_{i}(u_1,u_2,u_3) \; \ei
\]
where $a_i$ are the components of $\v{a}$ in the curvilinear basis, $\ei$ is the $i^{\rm
      th}$ curvilinear basis vector, and $\v{a}$ is continuously differentiable. We obtain
$\Div\v{a}$ in orthogonal curvilinear coordinates using the integral definition of
divergence
\[
  \Div\v{a}
  \,=\,
  \lim_{\delta V\to0}\;
  \frac{1}{\delta V} \; \int_{\delta S}\v{a}\cdot\diff \v{S\!}\;,
\]
where $\delta S$ is the closed surface bounding $\delta V$. Let $P$ be the point with
curvilinear coordinates $(u_1,u_2,u_3)$. Choose $\delta V$ to be a small ``cuboid'' with
its three edges $\{\delta\v{r}_{\,i}\}$ along the basis vectors $\{\ei\}$ at $P$:
\begin{eqnarray*}
  \delta\v{r}_{\,1} & = & h_1 \; \delta u_1 \; \eone\\
  \delta\v{r}_{\,2} & = & h_2 \; \delta u_2 \; \etwo\\
  \delta\v{r}_{\,3} & = & h_3 \; \delta u_3 \; \ethree
\end{eqnarray*}
Using the illustration in \cref{fig:curvy_div},
the outward element of area on the face $ABCD$ is\ \ $\diff \v{S} = +h_2 \, h_3 \: \diff
  u_2 \, \diff u_3 \; \eone$ The outward element of area on the face $PQRS\,$ is\ \ $\diff
  \v{S} = -h_2 \, h_3 \: \diff u_2 \, \diff u_3 \; \eone$ The contributions to the surface
integral from the faces $ABCD$ and $PQRS$ are then
\begin{figure}[t]
  \centering
  \setlength{\unitlength}{1.3pt}
  \begin{picture}(160,100)(-20,-10)
    \put(0,0){\makebox(0,0)[rt]{$P$}}
    \put(60,0){\makebox(0,0)[rt]{$A$}}
    \bezier{80}(0,0)(30,10)(60,0)		%PA
    \put(60,0){\vector(3,-1){12}}
    \put(74,0){\makebox(0,0)[lt]{$u_1$}}

    \put(0,60){\makebox(0,0)[rb]{$S$}}
    \bezier{80}(0,0)(-10,30)(0,60)		%PS
    \put(0,60){\vector(1,3){4}}
    \put(0,74){\makebox(0,0)[lb]{$u_3$}}

    \put(60,60){\makebox(0,0)[lt]{$D$}}
    \bezier{80}(0,60)(30,70)(60,60)		%SD
    \bezier{80}(60,0)(50,30)(60,60)		%AD

    \put(100,20){\makebox(0,0)[lt]{$B$}}
    \bezier{80}(60,0)(80,15)(100,20)	%AB

    \put(100,80){\makebox(0,0)[lt]{$C$}}
    \bezier{80}(100,20)(90,50)(100,80)	%BC
    \bezier{80}(60,60)(80,75)(100,80)	%DC

    \put(40,82){\makebox(0,0)[rb]{$R$}}
    \bezier{80}(0,60)(20,75)(40,80)		%SR
    \bezier{80}(40,80)(70,90)(100,80)	%RC

    \put(40,20){\vector(3,1){12}}
    \put(50,26){\makebox(0,0)[rb]{$u_2$}}
    \put(40,18){\makebox(0,0)[lt]{$Q$}}
    \bezier{10}(40,20)(20,15)(0,0)		%QP
    \bezier{10}(40,20)(30,50)(40,80)	%QR
    \bezier{10}(40,20)(70,30)(100,20)	%QB
  \end{picture}
  \caption{Infinitesimal element of volume in curvilinear coordinates.}
  \label{fig:curvy_div}
\end{figure}
\begin{eqnarray}
  & \msp &
  \int_{u_3}^{u_3+\delta u_3}
  \int_{u_2}^{u_2+\delta u_2}
  \left\{
  \left[a_1\,h_2\,h_3 \right]_{\scriptstyle ABCD}
  \,-\,
  \left[a_1\,h_2\,h_3\right]_{\scriptstyle PQRS}
  \right\}
  \: \diff u_2^\prime  \, \diff u_3^\prime \nonumber \\[1ex]
  %
  & = &
  \int_{u_3}^{u_3+\delta u_3}
  \int_{u_2}^{u_2+\delta u_2}
  \left\{
  \left[a_1\,h_2\,h_3\right]_{(u_1+\delta u_1,u_2^\prime,u_3^\prime)}
  \,-\,
  \left[a_1\,h_2\,h_3\right]_{(u_1,u_2^\prime,u_3^\prime)}
  \right\}
  \: \diff u_2^\prime \, \diff u_3^\prime \nonumber \\[1ex]
  %
  & = &
  \int_{u_3}^{u_3+\delta u_3}
  \int_{u_2}^{u_2+\delta u_2}
  \left\{
  \delta u_1\,
  \left[
    \frac{\partial}{\partial u_1}\,(a_1\,h_2\,h_3)
    \right]_{(u_1,u_2^\prime,u_3^\prime)}
  \right\}
  \; \diff u_2^\prime \, \diff u_3^\prime
  \qquad \mbox{(by Taylor's theorem)} \nonumber \\[1ex]
  %
  & = &
  \delta u_1\,  \delta u_2 \, \delta u_3 \;
  \left[\frac{\partial}{\partial
      u_1}(a_1\,h_2\,h_3)\right]_{(u_1,u_2,u_3)}
  \label{eq:curvi_div}
  %  \left\{ \frac{\partial}{\partial u_1}\,
  %  (a_1(u_1,u_2,u_3)\,h_2\,h_3)
  %  \right\}
\end{eqnarray}
In the last step, we assumed that $\delta V$ is small enough that the
integrand is approximately constant over the range of integration. We
can then approximate the integrals over $u_2^\prime$ and $u_3^\prime$ by the
integrand evaluated at the point $P$,
\[
  \delta u_1\,
  \left[
    \frac{\partial}{\partial u_1}(a_1\,h_2\,h_3)
    \right]_{(u_1,u_2,u_3)}
\]
multiplied by the range of integration $\delta u_2 \: \delta u_3$. [This is a rather
    crude use of the mean value theorem!] The contributions of the other four faces to the
integral over $\delta S$ can be obtained similarly, or by cyclic permutations of the
indices $\{1,\,2,\,3\}$ in equation~(\ref{eq:curvi_div}). Finally, divide by the volume
of the cuboid $\delta V = h_1\,h_2\,h_3\,\delta u_1\,\delta u_2\,\delta u_3$, whereupon
all the factors of $\delta u_i$ cancel, and we obtain our final expression for
$\Div\v{a}$ in orthogonal curvilinear coordinates
\begin{center}
  \bigbox{
    \parbox{0.8\textwidth}
    {
      \[
        \Div\v{a}
        \:=\:
        \frac{1}{h_1 h_2 h_3} \;
        \left\{
        \frac{\partial}{\partial u_1}(a_1 h_2 h_3)
        \,+\,
        \frac{\partial}{\partial u_2}(a_2 h_3 h_1)
        \,+\,
        \frac{\partial}{\partial u_3}(a_3 h_1 h_2)
        \right\}
      \]
    }
  }
\end{center}
For Cartesian coordinates, the scale factors are all \emph{unity}, and
we recover the usual expression for $\Div\v{a}$ in Cartesians.
For \emph{spherical polars} we have
\begin{eqnarray*}
  \Div\v{a}\:\!(r,\theta,\phi)
  & = &
  \frac{1}{r^2\sin(\theta)}\left\{
  \frac{\partial}{\partial r}\left(r^2\sin(\theta)\,  a_r\right)
  \,+\,
  \frac{\partial}{\partial\theta}\left(r\sin(\theta) \, a_\theta\right)
  \,+\,
  \frac{\partial}{\partial\phi}\left(r \, a_\phi\right)\right\} \\[1ex]
  & = &
  \frac{1}{r^2} \frac{\partial}{\partial r}\left(r^2 \, a_r\right)
  \,+\,
  \frac{1}{r\sin(\theta)}
  \left\{
  \frac{\partial}{\partial\theta}\left(\sin(\theta) \, a_\theta\right)
  \,+\,
  \frac{\partial}{\partial\phi}\left( a_\phi\right)
  \right\}
\end{eqnarray*}
where $a_r$, $a_\theta$, and $a_\phi$ are the components of the vector
field $\v{a}$ in the basis $\{\er, \, \etheta, \, \ephi\}$.

\subsection{Curl} We obtain $\curl\v{a}$ in orthogonal curvilinear coordinates using the line integral
definition of curl. The component of $\curl\v{a}$ in the direction of the unit vector
$\vn$ is
\[
  \v{n} \cdot \left(\curl \v{a}\right)
  \,=\,
  \lim_{\delta S \to0} \: \frac{1}{\delta S} \,
  \oint_{\delta C} \v{a}\cdot\diff \v{r}
\]
where $\delta S$ is a small planar surface, with unit normal $\vn$, bounded by the closed
curve $\delta C$ as illustrated in \cref{fig:curvy_curl}. Let $\delta S$ be a small
rectangular surface parallel to the $(\etwo,\ethree)$ plane with one corner at
$\v{r}(u_1,u_2,u_3)$, and with edges
\[
  \delta\v{r}_{\,2}
  \,=\,
  h_2 \: \delta u_2 \;  \etwo
  \quad \mbox{and} \quad
  \delta\v{r}_{\,3}
  \,=\,
  h_3 \: \delta u_3 \; \ethree
\]
which lie along the basis vectors, so that $\v{n}=\eone$. The line integral around the
curve $\delta C$ is the sum of the line integrals along the lines $1\to4$ respectively,
\begin{figure}[htbp]
  \centering
  \includegraphics{tikz_curvy_curl.pdf}
  \caption{Infinitesimal circulation of vector field.}
  \label{fig:curvy_curl}
\end{figure}
\begin{eqnarray*}
  \oint_{\delta C} \v{a}\cdot\diff \v{r}
  & = &
  \phantom{-}
  \int_{u_2}^{u_2+\delta u_2} \left[a_2 \, h_2 \right]_{(u_1,u_2^\prime,u_3)} \;
  \diff u_2^\prime
  \quad\quad ~+~
  \int_{u_3}^{u_3+\delta u_3} \left[a_3 \, h_3\right]_{(u_1,\,u_2+\delta u_2,u_3^\prime)} \;
  \diff u_3^\prime
  \\[1ex]
  %
  &   &
  -
  \int_{u_2}^{u_2+\delta u_2} \left[a_2 \, h_2\right]_{(u_1,u_2^\prime,u_3+\delta u_3)} \;
  \diff u_2^\prime
  \:\, ~-~
  \int_{u_3}^{u_3+\delta u_3} \left[a_3 \, h_3\right]_{(u_1,\,u_2,u_3^\prime)} \,
  \diff u_3^\prime
\end{eqnarray*}
Using Taylor's theorem, we can write this as
\begin{eqnarray*}
  \oint_{\delta C} \v{a}\cdot\diff \v{r}
  &=&
  \,\phantom{-}\,
  \int_{u_3}^{u_3+\delta u_3}
  \left\{
  \delta u_2 \; \left[\parpar{\ }{u_2} (a_3\,h_3)\right]_{(u_1,\,u_2,u_3^\prime)}
  \right\} \; \diff u_3^\prime \\[1.5ex]
  &&
  \,-\,
  \int_{u_2}^{u_2+\delta u_2}
  \left\{
  \delta u_3 \; \left[\parpar{\ }{u_3} (a_2\,h_2)\right]_{(u_1,\,u_2^\prime,u_3)}
  \right\}
  \; \diff u_2^\prime
\end{eqnarray*}
In each case, we approximate the integrals over $u_3^\prime$ and $u_2^\prime$ by the
product of the integrand and the integration ranges $\delta u_3$ and
$\delta u_2$, respectively.  Hence,
\[
  \oint_{\delta C} \v{a}\cdot\diff \v{r}
  \,=\,
  \parpar{\ }{u_2} \; (a_3\,h_3) \; \delta u_2 \, \delta u_3
  \,-\,
  \parpar{\ }{u_3} \; (a_2\,h_2) \; \delta u_3 \, \delta u_2
\]
where all the $\{a_i\}$ and $\{h_i\}$ are evaluated at $\v{r}(u_1,u_2,u_3)$. Finally, we
divide by the area of the rectangle $\delta S = h_2\:\! h_3 \, \delta u_2 \:\! \delta
  u_3$, whereupon all the factors of $\delta u_i$ cancel, and we obtain
\[
  \eone \cdot \left(\curl\v{a}\right)
  \,=\,
  \left(\curl\v{a}\right)_1
  \,=\,
  \frac{1}{h_2\,h_3} \;
  \left\{
  \frac{\partial}{\partial u_2}(a_3h_3)
  \,-\,
  \frac{\partial}{\partial u_3}(a_2h_2)
  \right\}
\]
The components of $\curl\v{a}$ in the directions of the curvilinear basis vectors $\e2$
and $\e3$ may be obtained similarly, or by cyclic permutations of the indices. It is
convenient to write the final result in the form
\begin{center}
  \bigbox{
  \parbox{0.5\textwidth}
  {
  \[
    \curl\v{a}
    \,=\,
    \frac{1}{h_1\,h_2\,h_3} \,
    \left|
    {\renewcommand{\arraystretch}{1.5}
    \begin{array}{ccc}
      h_1\,\eone                         & h_2\,\etwo & h_3\,\ethree \\[1ex]
      {\ds\frac{\partial}{\partial u_1}} &
      {\ds\frac{\partial}{\partial u_2}} &
      {\ds\frac{\partial}{\partial u_3}}                             \\[1ex]
      h_1\,a_1                           & h_2\,a_2   & h_3\,a_3
    \end{array}}
    \right|
  \]
  }
  }
\end{center}
For \emph{spherical polars} we have
\[
  \curl\v{a}
  \,=\,
  \frac{1}{r^2\,\sin(\theta)}
  \left|
  {\renewcommand{\arraystretch}{1.5}
  \begin{array}{ccc}
    \er                                   & r\,\etheta   & r \sin(\theta)\,\ephi   \\[1ex]
    {\ds\frac{\partial}{\partial r}}      &
    {\ds\frac{\partial}{\partial \theta}} &
    {\ds\frac{\partial}{\partial \phi}}                                            \\[1ex]
    a_{r}                                 & r a_{\theta} & r \sin(\theta) a_{\phi}
  \end{array}}
  \right|
\]

\subsection{Laplacian of a scalar field}
The action of the Laplacian operator on a \emph{scalar field} $f(\v{r})$ is defined by
$\lap f = \Div(\grad f)$. Using the expression for $\Div \v{a}$, with $\v{a} = \grad f$,
derived above, we find
\begin{center}
  \bigbox{
    \parbox{0.96\textwidth}
    {
      \[
        \lap f
        \,=\,
        \frac{1}{h_1\,h_2\,h_3}
        \left\{
        \frac{\partial}{\partial u_1}\left(\frac{h_2\,h_3}{h_1}\;
        \frac{\partial f}{\partial u_1}\right) \,+\,
        \frac{\partial}{\partial u_2}\left(\frac{h_3\,h_1}{h_2}\;
        \frac{\partial f}{\partial u_2}\right) \,+\,
        \frac{\partial}{\partial u_3}\left(\frac{h_1\,h_2}{h_3}\;
        \frac{\partial f}{\partial u_3}\right)
        \right\}
      \]
    }
  }
\end{center}
In \emph{spherical polars}, we have
\begin{eqnarray*}
  \lap f(r,\theta,\phi)
  & = &
  \frac{1}{r^2\sin(\theta)}
  \left\{
  \frac{\partial}{\partial r}\left(r^2\sin(\theta)\,
  \frac{\partial f}{\partial r}\right) \,+\,
  \frac{\partial}{\partial\theta}\left(\sin(\theta)\,
  \frac{\partial f}{\partial\theta}\right) \,+\,
  \frac{\partial}{\partial\phi}\left(\frac{1}{\sin(\theta)}\,
  \frac{\partial f}{\partial\phi}\right)
  \right\}\\[1.5ex]
  & = &
  \frac{1}{r^2}\,\frac{\partial}{\partial r}
  \left(r^2\, \frac{\partial f}{\partial r}\right)
  \,+\,
  \frac{1}{r^2\sin^2\theta}
  \left\{ \sin(\theta)\,\frac{\partial}{\partial\theta}
  \left( \sin(\theta)\,\frac{\partial f}{\partial\theta}\right)
  \,+\,
  \frac{\partial^2f}{\partial\phi^2}\right\}
\end{eqnarray*}
The expression for the Laplacian of a scalar field is a very important identity, with applications in electromagnetism, quantum mechanics,
optics, elasticity, fluid mechanics, meteorology, general relativity, cosmology.

\subsection{Laplacian of a vector field}
The Laplacian of a \emph{vector field} $\v{a}(\v{r})$ in curvilinear coordinates is
\emph{defined} by means of the identity
\[
  \curl \left(\curl\v{a}\right)
  \,=\,
  \grad \left(\Div\v{a}\right) \,-\, \lap\,\v{a}
\]
in the form
\begin{center}
  \bigbox{$
      \ds
      \lap\,\v{a}
      \,=\,
      \grad\,\left(\Div\v{a}\right) \,-\,  \curl\left(\curl\v{a}\right)
    $}
\end{center}
The quantities on the right-hand side are evaluated using the
expressions for grad, div and curl derived above.

