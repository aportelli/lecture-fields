% !TEX root = ../../fields.tex
%% Brian Lecture 13 %%%%%%%%%%%%%%%%%%%%%%%%%%%%%%%%%%%%%%%%%%%%%%%%%%%%%%%%%%%%%%%%%%%%%%
\parbox{6cm}{
  \epsfxsize=6cm
  \includegraphics{tikz_surface.pdf}
}\hfill
\parbox{10cm}{ Let $S$ be a two-sided surface in three-dimensional
  space as shown.  If an infinitesimal element of surface with
  (scalar) area ${\rm d}S$ has unit normal $\vn$, then the infinitesimal
  \emph{vector element of area} is \emph{defined} by
  \begin{center}
    \bigbox{$
        \ds    \mbox{d} \v{S} \,=\, \vn\, \mbox{d}S
      $}
  \end{center}
}

\paragraph{Example:} If $S$ lies in the $(x,y)$ plane, then in Cartesian coordinates the infinitesimal
\emph{scalar} element of area is $\mbox{d}S = \mbox{d}x \, \mbox{d}y$, and the
infinitesimal \emph{vector} element of area is $\mbox{d}\v{S}=\v{e}_{\,z}\,\mbox{d}x \,
  \mbox{d}y$.

\paragraph{Geometrical interpretation:}  $\v{m} \cdot {\rm d}\v{S}$
gives the projected (scalar) element of area onto the plane with unit
normal $\v{m}$. See later for more details.

For \emph{closed} surfaces (\emph{e.g.}~a sphere) we always \emph{choose} $\vn$ to be the
\emph{outward} normal. For \emph{open} surfaces (\emph{e.g.}~the curved surface of a
hemisphere), the sense of $\vn$ is arbitrary -- except that it is chosen in the
\emph{same} sense for \emph{all} elements of the surface.
\begin{center}
  \epsfxsize=0.6\textwidth
  \includegraphics{tikz_surf_normal.pdf}
\end{center}
If $\v{a}(\vr)$ is a vector field defined on the surface $S$, we
define the (normal) \emph{surface integral} formally by the Riemann sum
\[
  \intS\v{a}\cdot {\rm d}\v{S} \,=\, \intS \v{a}\cdot\vn \, {\rm d}S
  \,=\,
  \raisebox{-3ex}{$
      \shortstack{lim \\ ${}_{m\to\infty}$ \\ ${}_{\delta S^{(i)}\to0}$}\;
    $}
  \sum_{i=0}^{m-1
  }\;
  \left(\v{a}\left(\vr^{(i)}\right)\cdot\vn^{(i)}\right)\:  \delta S^{(i)}
\]
where we divided the surface $S$ into $m$ small areas, the $i^{\rm th}$ area having
vector area $\delta\v{S}^{(i)}$. The quantity $\v{a}(\vr^{(i)})\cdot\vn^{(i)}$ is the
component of $\v{a}$ \emph{normal} to the surface at the point $\vr^{(i)}$. This is
similar to the definition of the Riemann sum for planar integrals,
% given in \emph{Linear Algebra   and Several Variable Calculus},
except that here the (scalar) area elements $\delta S^{(i)}$ are not constrained to lie
in a plane.

\begin{itemize}
  \item We use the notation $\intS\v{a}\cdot {\rm d}\v{S}$ for both \emph{open} and \emph{closed}
        surfaces. Sometimes the integral over a \emph{closed} surface is denoted by
        $\ointS\v{a}\cdot {\rm d}\v{S}$\ \ (\emph{not} used here).
  \item The integral over $S$ is an ordinary \emph{double integral} in each
        case.\footnote{Surface integrals are sometimes denoted by $\ds\int\!\!\!\int_S\v{a}\cdot
            {\rm d}\v{S}$\ \ (\emph{not} used here).}
\end{itemize}

%\newpage

\paragraph{Example:} Let $S$ be the surface of a unit cube. Note that $S$ is the sum of \emph{all six faces}
of the cube.

\parbox{0.58\textwidth}{On the \emph{front} face, parallel to the
  $(y,z)$ plane, at $x=1$,
  \[
    {\rm d}\v{S} \,=\, \vn\, {\rm d}S \,=\, \v{e}_{\,x}\, {\rm d}y\,{\rm d}z
  \]
  On the \emph{back} face at $x=0$ in the $(y,z)$ plane,
  \[
    {\rm d}\v{S} \,=\, \vn\, {\rm d}S \,=\, -\v{e}_{\,x}\, {\rm d}y\,{\rm d}z
  \]

  Similarly for the other four faces.

  \medskip

  In each case, the unit normal $\vn$ is an \emph{outward} normal because $S$ is a
  \emph{closed} surface.} \hfill \parbox{0.4\textwidth}{
  \begin{center}
    \epsfxsize=0.42\textwidth
    \includegraphics{tikz_cube_surf.pdf}
  \end{center}
}
%
If $\v{a}(\vr)$ is a vector field, the integral $\intS \v{a}\cdot {\rm d}\v{S}$ over the
front face, where ${\rm d}\v{S} = \v{e}_{\,x}\, {\rm d}y\,{\rm d}z$, is
\[
  \int_{z=0}^{z=1} \int_{y=0}^{y=1}
  \,  \v{a}\cdot \left(\v{e}_{\,x}\,{\rm d}y\,{\rm d}z\right)
  \,=\,
  \int_{z=0}^{z=1} {\rm d}z \int_{y=0}^{y=1}
  \v{a}\cdot\v{e}_{\,x}\,{\rm d}y
  \,=\,
  \int_{z=0}^{z=1} \,{\rm d}z \left. \int_{y=0}^{y=1}
  a_x\right|_{x=1} \,{\rm d}y
\]
where $\left. a_x\right|_{x=1} \equiv a_x(1,y,z)$. The integral over $y$ and $z$ is an
ordinary double integral over a square of side $1$.

Similarly, the integral over the back face, where ${\rm d}\v{S} = -\v{e}_{\,x}\, {\rm
      d}y\,{\rm d}z$ is
\[
  \int_{z=0}^{z=1} \int_{y=0}^{y=1}
  \, \v{a}\cdot \left(-\v{e}_{\,x}\,{\rm d}y\,{\rm d}z\right)
  \,=\,
  -\int_{z=0}^{z=1} {\rm d}z \int_{y=0}^{y=1}
  \v{a}\cdot\v{e}_{\,x}\,{\rm d}y \,=\, -\int_{z=0}^{z=1} {\rm d}z
  \left. \int_{y=0}^{y=1} a_x\right|_{x=0} \,{\rm d}y
\]
where $\left. a_x\right|_{x=0} \equiv a_x(0,y,z)$. Again, this is an ordinary double
integral over a square of side $1$.

The total integral over $S$ is the sum of integrals over all $6$ faces of the cube. See
tutorial for an explicit example.

\subsection{Parametric form of the surface integral}
We now introduce a parametric representation for surface integrals.

%\footnote{This is the extension to 3D of the representation
%  used for planar integrals in \emph{LA\&SVC} Section~(18.4).}

Suppose the points on a surface $S$ can be specified by two real parameters $u$ and $v$,
so that the position vector on the surface may be written as
\[
  \vr \,=\, \vr(u,v) \,=\, x(u,v)\,\ex \,+\,  y(u,v)\,\ey \,+\, z(u,v) \, \ez
\]
Then
\begin{itemize}
  \item the lines $\vr(u,v)$ for fixed $u$, variable $v$, and
  \item the lines $\vr(u,v)$ for fixed $v$, variable $u$
\end{itemize}
are \emph{parametric lines} and form a \emph{grid} on the surface $S$ as
shown.

\vspace*{-3ex}

\begin{center}
  \epsfxsize=6.5cm
  \includegraphics{tikz_grid.pdf}
\end{center}

\vspace*{-2ex}

If we change $u$ by ${\rm d}u$, and $v$ by ${\rm d}v$, then $\vr$ changes by ${\rm
      d}\v{r}$, where the infinitesimal vector ${\rm d}\v{r}$ lies in the surface, and is given
by
\[
  {\rm d}\v{r}
  \,=\,
  \parpar{\vr}{u}\; {\rm d}u + \parpar{\vr}{v}\; {\rm d}v
\]

\parbox{0.55\textwidth}{Along the curves $v=\mbox{\emph{constant}}$,
we have ${\rm d}v=0$, so ${\rm d}\v{r}$ is just
\[
  {\rm d}\v{r}_{\,u}
  \,=\,
  \parpar{\vr}{u}\; {\rm d}u
\]
The vector $\partial\vr/\partial u$ is tangent to the surface at $\vr$, and tangent to
the lines $v = \mbox{\emph{constant}}$.

\bigskip

Similarly, for $u=\mbox{\emph{constant}}$, we have
\[
  {\rm d}\v{r}_{\,v} \,=\,  \parpar{\vr}{v}\; {\rm d}v
\]
so $\partial\vr/\partial v$ is tangent to the surface at $\vr$, and tangent to the lines
$u = \mbox{\emph{constant}}$, as shown in the figure.}
%
\hfill
%
\parbox{0.42\textwidth}{
  \epsfxsize=0.42\textwidth
  \includegraphics{tikz_grid_uv.pdf}
}

\medskip

The infinitesimal vectors ${\rm d}\v{r}_{\,u}$ and ${\rm d}\v{r}_{\,v}$ lie \emph{in the
  surface} at $\vr$.

\smallskip

The vectors $\ds\parpar{\vr}{u}$ and $\ds\parpar{\vr}{v}$ lie in the \emph{tangent plane}
to the surface at $\vr$.

\smallskip

We can construct a \emph{unit vector} $\vn$, \emph{normal} to the surface at $\vr$
\vspace*{0.5ex}
\[
  \vn
  \,=\,
  \left(\parpar{\vr}{u}\times\parpar{\vr}{v}\right)
  \left/\,
  \left|\parpar{\vr}{u}\times\parpar{\vr}{v}\right|
  \right.
\]
Since the vector element of area, ${\rm d}\v{S}$, has magnitude equal to the area of the
infinitesimal parallelogram shown in the figure above, and it points in the direction of
$\vn$, we can write
\[
  {\rm d}\v{S}
  \,=\,
  {\rm d}\v{r}_{\,u} \times {\rm d}\v{r}_{\,v}
  \,=\,
  \left(\parpar{\vr}{u}\; {\rm d}u \right)  \times \left(\parpar{\vr}{v}\; {\rm d}v\right)
  \,=\,
  \left(\parpar{\vr}{u} \times \parpar{\vr}{v}\right)\; {\rm d}u\,{\rm d}v
  %               \qquad
  %               \left( \,=\,  \left(\v{t}_{\,u}\times \v{t}_{\,v}\right)
  %                        \; {\rm d}u\,{\rm d}v \right)
\]
\begin{center}
  \bigbox{
    $
      \ds {\rm d}\v{S}
      =
      \left(\parpar{\vr}{u} \times \parpar{\vr}{v}\right)\, {\rm d}u\,{\rm d}v$
  }
\end{center}
Finally, our integral is parameterised as
\begin{center}

  \bigbox{
  \parbox{0.5\textwidth}{
  \[
    \intS \v{a}\cdot {\rm d}\v{S}
    \,=\,
    \int_v \int_u
    \v{a}\cdot\left(\parpar{\vr}{u} \times
    \parpar{\vr}{v}\right)\; {\rm d}u\,{\rm d}v
  \]
  }
  }
\end{center}
We use two integral signs when writing surface integrals in
terms of \emph{explicit} parameters $u$ and $v$.  The limits for the
integrals over $u$ and $v$ must be chosen appropriately for the
surface.

%This was a little abstract and a little complicated\ldots

Fortunately, in most practical cases we don’t need the detailed form of this expression,
since we tend to use orthogonal curvilinear coordinates, where the vectors
$\partial{\vr}/\partial u$ and $\partial{\vr}/{\partial v}$ are orthogonal to each other.
It is normally clear from the geometry of the situation whether this is the case, as it
is for cylindrical coordinates and spherical polar coordinates - as we shall see.

\newpage
