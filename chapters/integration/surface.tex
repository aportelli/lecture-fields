% !TEX root = ../../fields.tex
%% Brian Lecture 13 %%%%%%%%%%%%%%%%%%%%%%%%%%%%%%%%%%%%%%%%%%%%%%%%%%%%%%%%%%%%%%%%%%%%%%
\begin{figure}[t]
  \centering
  \includegraphics{tikz_surface.pdf}
  \caption{Infinitesimal surface element and its unit normal vector.}
  \label{fig:surface}
\end{figure}
Let $S$ be a two-sided surface in three-dimensional
space as shown in \cref{fig:surface}.  If an infinitesimal element of surface with
(scalar) area $\diff S$ has unit normal $\vn$, then the infinitesimal
\emph{vector element of area} is \emph{defined} by
\begin{center}
  \bigbox{$
      \ds    \diff  \v{S} \,=\, \vn\, \diff S
    $}
\end{center}}

\paragraph{Example:} If $S$ lies in the $(x,y)$ plane, then in Cartesian coordinates the infinitesimal
\emph{scalar} element of area is $\diff S = \diff x \, \diff y$, and the infinitesimal
\emph{vector} element of area is $\diff \v{S}=\v{e}_{\,z}\,\diff x \, \diff y$.

\paragraph{Geometrical interpretation:}
$\v{m} \cdot \diff \v{S}$ gives the projected (scalar) element of area onto the plane
with unit normal $\v{m}$. See later for more details. For \emph{closed} surfaces
(\emph{e.g.}~a sphere) we always \emph{choose} $\vn$ to be the \emph{outward} normal. For
\emph{open} surfaces (\emph{e.g.}~the curved surface of a hemisphere), the sense of $\vn$
is arbitrary -- except that it is chosen in the \emph{same} sense for \emph{all} elements
of the surface. We give some two-dimensional examples in \cref{fig:surf_normal}.
\bigskip

If $\v{a}(\vr)$ is a vector field defined on the surface $S$, we define the (normal)
\emph{surface integral} formally by the Riemann sum
\[
  \intS\v{a}\cdot \diff \v{S} \,=\, \intS \v{a}\cdot\vn \, \diff S
  \,=\,
  \lim_{\substack{\delta S_i\to 0 \\ m\to\infty}}\,
  \sum_{i=0}^{m-1
  }\;
  \delta S_i\,\v{a}(\vr_i)\cdot\vn_i\:
\]
where we divided the surface $S$ into $m$ small areas, the $i$-th area having vector area
$\delta\v{S}_{i}$. The quantity $\v{a}(\vr_i)\cdot\vn_i$ is the component of $\v{a}$
\emph{normal} to the surface at the point $\vr_{i}$. This is similar to the definition of
the Riemann sum for planar integrals,
% given in \emph{Linear Algebra   and Several Variable Calculus},
except that here the (scalar) area elements $\delta S_{i}$ are not constrained to lie in
a plane.

\begin{itemize}
  \item We use the notation $\int_S\v{a}\cdot \diff \v{S}$ for both \emph{open} and \emph{closed}
        surfaces. Sometimes the integral over a \emph{closed} surface is denoted by
        $\oint_S\v{a}\cdot \diff \v{S}$.
  \item The integral over $S$ is an ordinary \emph{double integral} in each
        case.\footnote{Surface integrals are sometimes denoted by $\int\!\!\!\int_S\v{a}\cdot
            \diff \v{S}$.}
\end{itemize}
\begin{figure}[t]
  \centering
  \includegraphics{tikz_surf_normal.pdf}
  \caption{Examples of normal unit vectors}
  \label{fig:surf_normal}
\end{figure}
%\newpage

\paragraph{Example:} Let $S$ be the surface of a unit cube. Note that $S$ is the sum of \emph{all six faces}
of the cube. On the \emph{front} face, parallel to the $(y,z)$ plane, at $x=1$,
\[
  \diff \v{S} \,=\, \vn\, \diff S \,=\, \v{e}_{\,x}\, \diff y\,\diff z
\]
On the \emph{back} face at $x=0$ in the $(y,z)$ plane,
\[
  \diff \v{S} \,=\, \vn\, \diff S \,=\, -\v{e}_{\,x}\, \diff y\,\diff z
\]
Similarly for the other four faces, \cf \cref{fig:cube_surf}.

In each case, the unit normal $\vn$ is an \emph{outward} normal because $S$ is a
\emph{closed} surface.
%
If $\v{a}(\vr)$ is a vector field, the integral $\int_S \v{a}\cdot \diff \v{S}$ over the
front face, where $\diff \v{S} = \v{e}_{\,x}\, \diff y\,\diff z$, is
\[
  \int_{z=0}^{z=1} \int_{y=0}^{y=1}
  \,  \v{a}\cdot \left(\v{e}_{\,x}\,\diff y\,\diff z\right)
  \,=\,
  \int_{z=0}^{z=1} \diff z \int_{y=0}^{y=1}
  \v{a}\cdot\v{e}_{\,x}\,\diff y
  \,=\,
  \int_{z=0}^{z=1} \,\diff z \left. \int_{y=0}^{y=1}
  a_x\right|_{x=1} \,\diff y
\]
where $\left. a_x\right|_{x=1} \equiv a_x(1,y,z)$. The integral over $y$ and $z$ is an
ordinary double integral over a square of side $1$. Similarly, the integral over the back
face, where $\diff \v{S} = -\v{e}_{\,x}\, \diff y\,\diff z$ is
\[
  \int_{z=0}^{z=1} \int_{y=0}^{y=1}
  \, \v{a}\cdot \left(-\v{e}_{\,x}\,\diff y\,\diff z\right)
  \,=\,
  -\int_{z=0}^{z=1} \diff z \int_{y=0}^{y=1}
  \v{a}\cdot\v{e}_{\,x}\,\diff y \,=\, -\int_{z=0}^{z=1} \diff z
  \left. \int_{y=0}^{y=1} a_x\right|_{x=0} \,\diff y
\]
where $\left. a_x\right|_{x=0} \equiv a_x(0,y,z)$. Again, this is an ordinary double
integral over a square of side $1$. The total integral over $S$ is the sum of integrals
over all $6$ faces of the cube. See exercises for an explicit example.
\begin{figure}[t]
  \centering
  \includegraphics{tikz_cube_surf.pdf}
  \caption{Illustration of infinitesimal surface elements on a unit cube.}
  \label{fig:cube_surf}
\end{figure}

\subsection{Parametric form of the surface integral}
We now introduce a parametric representation for surface integrals. Suppose the points on
a surface $S$ can be specified by two real parameters $u$ and $v$, so that the position
vector on the surface may be written as
\[
  \vr \,=\, \vr(u,v) \,=\, x(u,v)\,\ex \,+\,  y(u,v)\,\ey \,+\, z(u,v) \, \ez
\]
Then
\begin{itemize}
  \item the lines $\vr(u,v)$ for fixed $u$, variable $v$, and
  \item the lines $\vr(u,v)$ for fixed $v$, variable $u$
\end{itemize}
are \emph{parametric lines} and form a \emph{grid} on the surface $S$ as shown in
\cref{fig:grid}. If we change $u$ by $\diff u$, and $v$ by $\diff v$, then $\vr$ changes
by $\diff \v{r}$, where the infinitesimal vector $\diff \v{r}$ lies in the surface, and is
given by
\[
  \diff \v{r}
  \,=\,
  \parpar{\vr}{u}\; \diff u + \parpar{\vr}{v}\; \diff v
\]
Along the curves $v=\mbox{\emph{constant}}$, we have $\diff v=0$, so $\diff \v{r}$ is
just
\[
  \diff \v{r}_{\,u}
  \,=\,
  \parpar{\vr}{u}\; \diff u
\]
The vector $\partial\vr/\partial u$ is tangent to the surface at $\vr$, and tangent to
the lines $v = \mbox{\emph{constant}}$. Similarly, for $u=\mbox{\emph{constant}}$, we
have
\[
  \diff \v{r}_{\,v} \,=\,  \parpar{\vr}{v}\; \diff v
\]
so $\partial\vr/\partial v$ is tangent to the surface at $\vr$, and tangent to the lines
$u = \mbox{\emph{constant}}$, as shown in \cref{fig:grid}. The infinitesimal vectors
$\diff \v{r}_{\,u}$ and $\diff \v{r}_{\,v}$ lie \emph{in the surface} at $\vr$. The
vectors $\smash{\parpar{\vr}{u}}$ and $\smash{\parpar{\vr}{v}}$ lie in the \emph{tangent
  plane} to the surface at $\vr$.

We can then construct a \emph{unit vector} $\vn$, \emph{normal} to the surface at $\vr$
\vspace*{0.5ex}
\[
  \vn
  \,=\,
  \left(\parpar{\vr}{u}\times\parpar{\vr}{v}\right)
  \left/\,
  \left|\parpar{\vr}{u}\times\parpar{\vr}{v}\right|\right.
\]
Since the vector element of area, $\diff \v{S}$, has magnitude equal to the area of the
infinitesimal parallelogram shown in \cref{fig:grid}, and it points in the direction of
$\vn$, we can write
\[
  \diff \v{S}
  \,=\,
  \diff \v{r}_{\,u} \times \diff \v{r}_{\,v}
  \,=\,
  \left(\parpar{\vr}{u}\; \diff u \right)  \times \left(\parpar{\vr}{v}\; \diff v\right)
  \,=\,
  \left(\parpar{\vr}{u} \times \parpar{\vr}{v}\right)\; \diff u\,\diff v
\]
\begin{center}
  \bigbox{
    $
      \ds \diff \v{S}
      =
      \left(\parpar{\vr}{u} \times \parpar{\vr}{v}\right)\, \diff u\,\diff v$
  }
\end{center}
Finally, our surface integral can be written as
\begin{center}
  \bigbox{
    \parbox{0.5\textwidth}{
      \[
        \intS \v{a}\cdot \diff \v{S}
        \,=\,
        \int_v \int_u
        \v{a}\cdot\left(\parpar{\vr}{u} \times
        \parpar{\vr}{v}\right)\; \diff u\,\diff v
      \]
    }
  }
\end{center}
We use two integral signs when writing surface integrals in
terms of \emph{explicit} parameters $u$ and $v$.  The limits for the
integrals over $u$ and $v$ must be chosen appropriately for the
surface.

Fortunately, in most practical cases we don’t need the detailed form of this expression,
since we tend to use orthogonal curvilinear coordinates, where the vectors
$\partial{\vr}/\partial u$ and $\partial{\vr}/{\partial v}$ are orthogonal to each other.
It is normally clear from the geometry of the situation whether this is the case, as it
is for cylindrical coordinates and spherical polar coordinates - as we shall see.
%
\begin{figure}[t]
  \centering
  \includegraphics{tikz_grid.pdf}\hspace{0.5cm}
  \raisebox{1cm}{\includegraphics{tikz_grid_uv.pdf}}
  \caption{<caption>}
  \label{fig:grid}
\end{figure}
