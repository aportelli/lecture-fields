% !TEX root = ../../fields.tex
%% Brian Lecture 18
%%%%%%%%%%%%%%%%%%%%%%%%%%%%%%%%%%%%%%%%%%%%%%%%%%%%%%%%%%%%%%%%%%%%%%%
A \emph{simply connected} region $R$ is a region where every closed curve in $R$ can be
shrunk continuously to a point while remaining entirely in $R$. The inside of a sphere is
simply connected while the region between two cylinders is {\bf not} simply connected:
it's doubly connected. In this course we'll be concerned (mainly) with simply connected
regions. A vector field $\v{a}(\v{r})$ is defined to be irrotational or
\emph{conservative} if its curl vanishes, \emph{i.e.} if
\[
  \curl \v{a} \,=\, 0
\]

\subsection{Path independence of line integrals for conservative fields}
Let $\curl\v{a} = 0$ everywhere in a simply-connected region, and consider two
(different) paths $C_1$ and $C_2$ from point $\v{r}_{\,0}$ to point $\v{r}$, say.
Applying Stokes' theorem to the \emph{open} surface $S$ bounded by the \emph{closed} path
$C_1{-}C_2$ gives
\[
  \int_S\, (\curl\v{a})\cdot \dd\v{S}
  \,=\,
  0
  \,=\,
  \int_{C_1}\, \v{a}(\v{r}^\prime)\cdot \dd\v{r}^\prime
  \,-\,
  \int_{C_2}\, \v{a}(\v{r}^\prime)\cdot \dd\v{r}^\prime
\]
where the minus sign occurs in the second integral on the RHS because both paths are
defined to go from $\v{r}_{\,0}$ to $\v{r}$. We use $\vr'$ as integration variable to
distinguish it from the integration \emph{limits} $\vr_{\,0}$ and $\vr$. Therefore, when
$\curl\v{a} = 0$ everywhere in $S$, we have
\[
  \int_{C_1}\, \v{a}(\v{r}^\prime)\cdot \diff \v{r}^\prime
  \,=\,
  \int_{C_2}\, \v{a}(\v{r}^\prime)\cdot \diff \v{r}^\prime
\]
This is true for \emph{any} $S$, and therefore for \emph{any} paths $C_1$ and $C_2$ from
$\v{r}_{\,0}$ to $\v{r}$.

Clearly, the converse is also true: if the line integral between two points is path
independent, then the line integral around any closed curve (connecting the two points)
is zero, and Stokes' theorem then gives $\curl \v{a} = 0$. We just reverse the steps of
the argument above. Therefore,
\[
  \bigbox{$ \ds
      \curl\v{a} = 0
      \quad \Leftrightarrow \quad
      \int_{\ul{r}_{\,0}}^{\ul{r}} \, \v{a}(\v{r}^\prime)\cdot \diff \v{r}^\prime
      \mbox{\ \ is path independent}
    $}
\]
\begin{figure}[t]
  \centering
  \includegraphics{tikz_irrot.pdf}
  \caption{Line integral between two points using different paths.}
  \label{fig:irrot}
\end{figure}

\subsection{Scalar potential for conservative vector fields}

Since the line integral of a conservative vector field between two fixed points
$\v{r}_{\,0}$ and $\v{r}$ is \emph{path independent}, it can be a function \emph{only} of
the \emph{end points} of the path. Hence, there must exist a function $\phi(\v{r})$ such
that
\begin{equation}
  \phi(\v{r}) - \phi(\v{r}_{\,0})
  \,=\,
  \int_{\ul{r}_{\,0}}^{\ul{r}}\, \v{a}(\v{r}^\prime)\cdot
  \diff \v{r}^\prime \; .
  \label{eq:pot-diff}
\end{equation}
The scalar field $\phi(\vr)$ is called the \emph{scalar potential} of
the \emph{vector field} $\v{a}(\vr)$.
It is useful to invert this equation (and to give a more conventional result) by
considering two neighbouring points $\v{r}$ and $\v{r}+\diff \v{r}$, for which
\begin{eqnarray*}
  \diff \phi
  &=&
  \phi(\v{r}+\diff \v{r}) - \phi(\v{r}) \\[1ex]
  &=&
  \left[\phi(\v{r}+\diff \v{r}) - \phi(\v{r}_{\,0})\right]
  -
  \left[\phi(\v{r}) - \phi(\v{r}_{\,0})\right] \\[1ex]
  &=&
  \int_{\ul{r}_{\,0}}^{\ul{r}+ {\rm d}\ul{r}}\,
  \v{a}(\v{r}^\prime) \cdot \diff \v{r}^\prime
  \,-\,
  \int_{\ul{r}_{\,0}}^{\ul{r}}\,
  \v{a}(\v{r}^\prime)\cdot \diff \v{r}^\prime
  \qquad \mbox{(using equation~(\ref{eq:pot-diff}))}\\[1ex]
  &=&
  \int_{\ul{r}}^{\ul{r}+{\rm d}\ul{r}}\,
  \v{a}(\v{r}^\prime) \cdot \diff \v{r}^\prime
  \hspace*{0.234\textwidth} \mbox{(along \emph{any} path from $\v{r}$ to
    $\v{r}+\diff \v{r}$)}\\[1ex]
  &=&
  \v{a}(\v{r}) \cdot \int_{\ul{r}}^{\ul{r}+{\rm d}\ul{r}}\,
  \diff \v{r}^\prime \,+\, O(|\diff \v{r}|^2)
  \hspace*{0.11\textwidth}
  \mbox{(\emph{choosing} the straight line from $\v{r}$ to
    $\v{r}+\diff \v{r}$)}
  \\[1ex]
  &=&
  \v{a}(\v{r})\cdot \left[ \left(\v{r} + \diff \v{r}\right) - \v{r} \right]
  \,=\,
  \v{a}(\v{r})\cdot \diff \v{r}
\end{eqnarray*}
because $\v{a}(\v{r})$ is approximately constant between $\v{r}$ and
$\v{r}+\diff \v{r}$, and the correction term $O(|\diff \v{r}|^2)$
can be ignored as $\diff \v{r}\to0$.
But $\diff \phi = \grad\phi\cdot \diff \v{r}$ (by definition), and so, since $\diff
  \v{r}$ is \emph{arbitrary}, we must have
\[
  \v{a}(\v{r}) \,=\, \grad\phi(\v{r})
\]
The converse is much easier to prove. If $\v{a} = \v{\nabla}\phi$, then $\curl\v{a} =
  \curl(\v{\nabla}\phi) \equiv 0$. Therefore,
\[
  \bigbox{$
      \curl\v{a} = 0
      \quad \Leftrightarrow \quad
      \v{a} = \v{\nabla}\phi$}
\]
To determine whether a vector field is conservative, one simply checks whether
$\curl\v{a} = 0$ (in the region of interest).

\paragraph{NB:}
The scalar potential $\phir$ is only determined up to a \emph{constant}. If $\psi = \phi
  + \mbox{\emph{constant}}$ then $\grad \psi = \grad \phi$, so $\psi$ is an equally good
potential. The freedom in the constant corresponds to the freedom in choosing $\vr_{\,0}$
when calculating the potential. So $\phi(\v{r}_{\,0})$ in equation~(\ref{eq:pot-diff}) is
just an irrelevant constant. Equivalently, the absolute value of a scalar potential has
no meaning, only \emph{potential differences} are significant.

\subsection{Finding scalar potentials}
\subsubsection*{Method~(1): Integration along a straight line}
We have shown that the scalar potential $\phir$ for a \emph{conservative} vector field
$\v{a}(\v{r})$ can be constructed from a line integral which is \emph{independent} of the
path of integration between the endpoints. A convenient way of evaluating such integrals
is to integrate along a \emph{straight line} from $\v{r}_0$ to $\v{r}$. Depending on the
convergence of the integral, there are two standard choices:

\begin{enumerate}
  \item $\v{r}_{\,0} = 0$:\ \ If $\phi(\v{r})$ is \emph{finite} or
        \emph{zero} at $\v{r}=0$, we parameterise the straight line
        by $\v{r}^\prime = \lambda \, \v{r}$\\[1ex] with $0 \le \lambda \le 1$.
        Thus, $\diff \v{r}^\prime = \diff \lambda \, \v{r}$, and hence
        \[
          \phi(\v{r})
          \,=\,
          \int_{0}^{\ul{r}}\, \v{a}(\v{r}^\prime) \cdot \diff \v{r}^\prime
          \,=\,
          \int_{\lambda=0}^{\lambda=1} \,
          \v{a}(\lambda\v{r}) \cdot \v{r} \, \diff \lambda \,,
        \]

  \item $|\v{r}_0|=\infty$:\ \ If $\phi(\v{r})$ is \emph{finite} or
        \emph{zero} as $|\v{r}|\to\infty$, we again parameterise the
        straight line \\[1ex] by $\v{r}^\prime = \lambda \, \v{r}$, but this
        time with $1 \le \lambda < \infty$. Again, we have
        $\diff \v{r}^\prime = \diff \lambda\, \v{r}$, and hence
        \[
          \phi(\v{r})
          \,=\,
          \int_{\infty}^{\ul{r}}\, \v{a}(\v{r}^\prime) \cdot \diff \v{r}'
          \,=\,
          \int_{\lambda=\infty}^{\lambda=1} \,
          \v{a}(\lambda\v{r}) \cdot \v{r} \, \diff \lambda \,,
        \]

\end{enumerate}

\paragraph{Example~1:}
Let $\v{a}(\v{r}) = (2xy+z^3)\v{e}_{\,x} + x^2\v{e}_{\,y} + 3xz^2\v{e}_{\,z}$. First
check that $\v{\nabla}\times\v{a} = 0$, so the field is conservative (exercise). Then
\begin{eqnarray*}
  \phi(\v{r})
  &=&
  \int_{0}^{\ul{r}}\, \v{a}(\v{r}^\prime) \cdot \diff \v{r}^\prime
  \,=\,
  \int_{\,0}^1\, \v{a}(\lambda\v{r})\cdot \v{r} \, \diff \lambda \\[0.7ex]
  &=&
  \int_{\,0}^1\,
  \left[
    \left(2\lambda^2xy + \lambda^3z^3\right)x
    + \left(\lambda^2x^2\right)y
    + \left(\lambda^3 \, 3x z^2\right)z
    \right] \diff \lambda \\[0.7ex]
  &=&
  \frac23 x^2 y + \frac14 xz^3 + \frac13 x^2y + \frac34 xz^3 \\[0.7ex]
  &=&
  x^2y + xz^3
\end{eqnarray*}
\textbf{NB:} Always check that your potential $\phir$ satisfies
$\v{a}(\v{r}) = \grad\phir$. (Exercise)

\paragraph{Example~2:}
Let $\ds \v{a}(\vr) = 2\,(\vc\cdot\v{r})\, \vr + r^2 \, \vc$ where $\v{c}$ is a constant
vector. %Check that $\v{a}$ is conservative:
\[
  \v{\nabla}\times\v{a}
  \,=\,
  2\,\left[
    \v{\nabla}\left(\v{c}\cdot\v{r}\right) \times \v{r}
    + \left(\v{c}\cdot\v{r}\right) \curl \v{r}
    \right]
  +
  \left(\grad r^2\right) \times \v{c}
  \,=\,
  2 \left[ \v{c} \times \v{r} + 0 \right] + 2 \, \v{r} \times \v{c}
  \,=\,
  0
\]
Then
\begin{eqnarray*}
  \phir
  & = & \int_{\,0}^{\ul{r}}\; \v{a}(\v{r}') \cdot \diff \v{r}'
  \,=\,
  \int_0^1\; \v{a}\left(\lambda\,\vr\right)
  \cdot \left(\diff \lambda\,\vr\right)
  \\[1ex]
  & = &
  \int_0^1
  \left(
  2\,\left(\vc\cdot\lambda\,\v{r}\right) \lambda\,\vr
  \,+\, \lambda^2 r^2 \, \vc
  \right)
  \cdot \vr \, \diff \lambda %\\[1ex]
  ~ = ~
  %  & = &
  \left(
  2\,(\vc\cdot\v{r}) \, \vr\cdot\vr \plus r^2 \, (\vc\cdot\v{r})
  \right)
  \int_0^1 \lambda^2 \, \diff \lambda \\[1.5ex]
  %        & = & \bigg[ 2\,(\va\cdot\v{r}) \, r^2 \plus
  %                                        r^2 \,(\va\cdot\v{r}) \bigg]
  %                \; \frac13\\[1ex]
  & = &
  r^2 \, (\vc\cdot\vr)
\end{eqnarray*}
Integration along a straight line is a straightforward and fairly
elegant method, and it's generally applicable.

\subsubsection*{Method~(2): Direct integration}

Since $\v{a} = \v{\nabla}\phi$, we have
\begin{eqnarray*}
  \frac{\partial\phi}{\partial x} = a_x(x,y,z) \qquad
  \frac{\partial\phi}{\partial y} = a_y(x,y,z) \qquad
  \frac{\partial\phi}{\partial z} = a_z(x,y,z)
\end{eqnarray*}
We can integrate these equations separately to give
\begin{eqnarray*}
  \phi(x,y,z) &=& \int^x a_x(x^\prime, y, z) \, \dd x^\prime + f(y,z) \\
  \phi(x,y,z) &=& \int^y a_y(x, y^\prime, z) \, \dd y^\prime + g(x,z) \\
  \phi(x,y,z) &=& \int^z a_z(x, y, z^\prime) \, \dd z^\prime + h(x,y)
\end{eqnarray*}
and then determine the ``constants'' of integration $f(y,z)$, $g(x,z)$
and $h(x,y)$ by consistency.

\paragraph{Example~1 (revisited):}

Let $\v{a} = (2xy+z^3)\v{e}_{\,x} + x^2\v{e}_{\,y} + 3xz^2\v{e}_{\,z}$. Then
\begin{eqnarray*}
  \phi &=& x^2y \,+\, xz^3 \,+\, f(y,z) \\
  \phi &=& x^2y \phantom{\,+\, xz^3}  \,+\, g(x,z) \\
  \phi &=& \phantom{x^2y \,+\,} xz^3 \,+\, h(x,y)
\end{eqnarray*}
These agree if we choose $f(y,z)=0$, $g(x,z)=xz^3$ and
$h(x,y)=x^2y$, hence
\[
  \phi(\v{r})
  \,=\,
  x^2y + xz^3
\]
as before. This method is straightforward, but it's rather clumsy for problems such as
Example~2, which is typical of many Physics applications.

\subsubsection*{Method~(3): Direct integration ``by inspection'' (guessing)}
Sometimes the result can be spotted directly.
For example, if $\v{a}(\v{r})=(\vc\cdot\v{r}) \vc$ where $\vc$ is a constant vector, then
\[
  \v{a}(\v{r})
  \,=\,
  (\vc\cdot\v{r}) \, \vc
  \,=\,
  (\vc\cdot\v{r}) \, \grad (\vc\cdot\v{r})
  \,=\,
  \grad \left(\frac12 \, (\vc\cdot\v{r})^2 \plus {\rm constant} \right)
\]

\paragraph{Example~2 (revisited)}\mbox{}
\[
  \v{a}(\v{r})
  \,=\,
  2\,(\v{c}\cdot\v{r})\, \vr \plus r^2 \, \v{c}
  \,=\,
  (\v{c}\cdot\v{r}) \grad r^2 \plus r^2 \, \grad (\v{c}\cdot\v{r})
  \,=\,
  \grad \bigg( (\v{c}\cdot\v{r}) \, r^2 \plus {\rm constant} \bigg)
\]
in agreement with what we had before if we choose the integration constant to be zero.

\subsection{Conservative forces: conservation of energy}
We now show how the name \emph{conservative field} arises in Physics. Let the vector
field $\vF(\vr)$ (assumed time-independent) be the total force acting on a particle of
mass $m$ at position $\v{r}$. We will show that for a conservative/irrotational force,
where we can write
\[
  \vF(\v{r}) = - \grad V(\v{r}) \,,
\]
the total energy is \emph{constant} in time. Note that the \emph{force} is \emph{minus}
the gradient of the (scalar) potential. The minus sign is conventional.

\paragraph{Proof:}
Let $\v{r}(t)$ be the position vector of a particle at time $t$. Denote the first and
second derivatives of $\v{r}$ with respect to time by $\v{\dot{r}}$ (velocity) and
$\v{\ddot{r}}$ (acceleration) respectively. The particle moves under the influence of
Newton's second law (N2):
\[
  m \v{\ddot r} = \vF(\vr)
\]
In time $\diff t$ the particle moves from $\v{r}$ to $\v{r}+\diff \v{r}$. From N2, we get
\[
  m \, \v{\ddot r}\cdot \diff \v{r}
  \,=\,
  \vF(\vr) \cdot \diff \v{r}
  \,=\,
  -\grad V(\vr) \cdot \diff \v{r}
\]
Integrating this expression along the path of the particle starting from $\v{r}_A$ at
time $t_A$, to $\v{r}_B$ at time $t_B$, gives
\begin{equation}
  m \int_{\ul{r}_A}^{\ul{r}_B} \v{\ddot r}\cdot \diff \v{r}
  \,=\,
  - \int_{\ul{r}_A}^{\ul{r}_B} \grad V(\vr) \cdot \diff \v{r}
  \label{eq:integrated-newton}
\end{equation}
We can evaluate the left-hand side of
equation~(\ref{eq:integrated-newton})
\[
  m \int_{\ul{r}_A}^{\ul{r}_B} \v{\ddot r} \cdot \diff \v{r}
  =
  m \int_{t_A}^{t_B} \v{\ddot r} \cdot \frac{\diff  \v{r}}{dt}\, \diff t
  =
  m \int_{t_A}^{t_B} \frac{1}{2} \, \frac{d}{dt} \,
  \left(\v{\dot r} \cdot \v{\dot r}\right) \diff t
  =
  \frac{1}{2} m \bigg[ \left|\v{\dot{r}}\right|^2 \bigg]_{t_A}^{t_B}
  =
  \frac{1}{2} m \left( v_B^2 - v_A^2 \right),
\]
where $v_A$ and $v_B$ are the magnitudes of the particle's velocity at points $A$ and $B$
respectively.

The right-hand side of equation~(\ref{eq:integrated-newton}) gives
\[
  - \int_{\ul{r}_A}^{\ul{r}_B} \grad V(\vr) \cdot \diff \v{r}
  \,=\,
  - \int_{\ul{r}_A}^{\ul{r}_B} dV = V_A - V_B
\]
where $V_A$ and $V_B$ are the values of the potential $V$ at $\v{r}_A$ and $\v{r}_B$,
respectively. Therefore,
\[
  \frac{1}{2} m \left( v_B^2 - v_A^2 \right)
  \,=~
  V_A - V_B
\]
Rearranging, we get
\[
  \frac{1}{2} m v_A^2 + V_A
  \,=\,
  \frac{1}{2} m v_B^2 + V_B
  \label{eq:energy_conserv_2}
\]
Hence the \emph{total energy}, defined as $\; \ds E \equiv \frac{1}{2} m v^2 + V\:\!$, is
\emph{conserved} -- it's \emph{constant in time}.

(Choosing $\vF = + \grad V$ would lead to $\; \ds E \equiv \frac{1}{2}
  m v^2 - V$, a less desirable convention.)

\paragraph{Examples:} \emph{Newtonian gravity} and the
\emph{electrostatic force} are both conservative.  \emph{Frictional
  forces} are not conservative: energy is dissipated and work is done
in traversing a closed path.  In general, time-dependent forces are
not conservative.

\subsection{Gravitation and Electrostatics} % (revisited)}
The foundation of Newtonian Gravity is \emph{Newton's Law of Gravitation}. The force
$\v{F}(\v{r})$ on a particle of mass $m_1$ at $\v{r}$ due to a particle of mass $m$
situated at the origin is given (in SI units) by
\[
  \v{F}(\v{r}) \,=\, -G\:\! m\:\! m_1 \, \frac{\v{r}}{r^3}
\]
where $G = 6.672\,59(85) \times 10^{-11} \, N m^2 \mbox{\emph{kg}}^2$ is Newton's
Gravitational Constant. The \emph{gravitational field} $\vG(\vr)$ due to the mass at the
origin is \emph{defined} by
\begin{equation}
  \v{F}(\v{r})
  ~\equiv~
  m_1 \, \v{G}(\v{r})
  \qquad \mbox{or} \qquad
  \v{G}(\v{r}) \,=\, -G\, m \: \frac{\v{r}}{r^3}
  \label{eq:grav-field-defn-2}
\end{equation}
where the test mass $m_1$ is so small that its gravitational field
can be ignored. The gravitational field is conservative because
\[
  \curl \left( \frac{\v{r}}{r^3}\right)
  \,=\,
  \grad\left(\frac{1}{r^3}\right) \times \v{r}
  \,+\,
  \frac{1}{r^3} \left(\curl \v{r}\right)
  \,=\,
  \left(-\frac{3\v{r}}{r^5}\right) \times \v{r} \,+\,  0
  \,=\,
  0
\]
%or, using indices,
%\[
%  \left(\v{\nabla}\times\left(\v{r}/r^3\right)\right)_i 
%  \,=\,
%  \epsilon_{ijk} \partial_j (x_k/r^3)
%  \,=\,
%  \epsilon_{ijk} \left(\delta_{jk}/r^3 - 3x_jx_k/r^5 \right)
%  \,=\,
%  0
%\]
The \emph{gravitational potential} defined by
\[
  \v{G} = -\grad \phi
\]
can be obtained from equation~(\ref{eq:grav-field-defn-2}) by spotting the direct
integration, $\grad(1/r) = -\v{r}/r^3$, giving
\[
  \phi   =-\frac{Gm}{r}
\]
Alternatively, we may evaluate it explicitly by a line integral. Choosing $\vr_{\,0}$ at
infinity gives
\begin{eqnarray*}
  \phir
  &=&
  -\int_{\ul{r}_0}^{\ul{r}} \v{G}(\vr') \cdot \diff \v{r}'
  \,=\,
  -\int_\infty^{1} \v{G}(\lambda\vr) \cdot \diff \lambda \v{r} \\[1ex]
  &=&
  (-)^2 \, \int_\infty^{1} \frac{Gm\, (\v{r}\cdot\vr)}{r^3} \:
  \frac{\diff \lambda}{\lambda^2}
  \,=\,
  -\frac{Gm}{r}
\end{eqnarray*}
\textbf{Notes:}
\begin{itemize}
  \item In this example, the vector field $\v{G}$ is \emph{singular} at the origin $\vr = 0$.
        This implies that we have to exclude the origin, so it's not possible to obtain the
        scalar potential at $\vr$ by integration along a path from the origin. Instead, we
        integrate from infinity, which in turn means that the gravitational potential at infinity
        is zero.
  \item Since $\vF = m_1 \, \v{G} = -\grad(m_1 \, \phi)$, the \emph{potential energy} of the mass
        $m_1$ is $V(\v{r})= m_1 \phi(\v{r})$. The distinction (a convention) between potential
        and potential energy is a common source of confusion.
\end{itemize}

\paragraph{Electrostatics:}
Coulomb's Law states that the force $\v{F}(\v{r})$ on a particle of charge $q_1$ situated
at $\v{r}$ in the electric field $\v{E}(\v{r})$ due to a particle of charge $q$ situated
at the origin is given (in SI units) by
\[
  \v{F}
  \,=\,
  q_1 \, \v{E}
  \,=\,
  \frac{q_1\,q}{4 \pi \epsilon_0}\: \frac{\v{r}}{r^3} \;,
\]
where $\epsilon_0 = 10^7/(4\pi c^2) = 8.854\,187\,817 \dots \times 10^{-12} \, C^2 N^{-1}
  m^{-2}$ is called the \emph{permittivity of free space}. Again the test charge $q_1$ is
taken as small, so as not to disturb the electric field. The \emph{electrostatic
  potential} may be obtained by inspection, or by integrating $E= -\grad \phi$ from
infinity to $\vr$,
\begin{equation}
  \phi(\v{r}) \,=\, \frac{q}{4 \pi \epsilon_0 r}
  \label{eq:phiforpointcharge}
\end{equation}
The \emph{potential energy} of a charge $q_1$ in the electric field is
$V(\v{r})= q_1 \phi(\v{r})$.
Note that electrostatics and gravitation are very similar mathematically, the only real
difference being that the gravitational force between two masses is always attractive,
whereas like charges repel.

\mnote{21L 18/03/08}
\mnote{22L 27/03/09}

%\newpage

% \subsection{The equations of Poisson and Laplace}
% In section~(\ref{sec:gausslaw}), we derived Gauss' Law and
% Maxwell's first equation (ME1) for the electrostatic field
% %$\v{E}(\v{r})$
% \[
%   \int_S \, \v{E}\cdot \diff \v{S}
%   \,=\,
%   \frac{Q}{\epsilon_0}
%   \qquad \mbox{and}\qquad
%   \Div \v{E}(\v{r})
%   \,=\,
%   \frac{\rho(\vr) }{\epsilon_0}
% \]
% where $\rho(\v{r})$ is the charge density at $\v{r}$, and $Q = \int_V \,
% \rho(\v{r})\, \diff V$ is the total charge in volume $V$.
% 
% Writing $\v{E}(\v{r})=-\grad\phi(\v{r})$ and using Maxwell's first
% equation gives \emph{Poisson's equation}
% \[
%   \lap \phi
%   \,=\,
%   -\frac{\rho}{\epsilon_0}
% \]
% If $\rho(\v{r})=0$ everywhere in some region, we have 
% \[
%   \lap \phi = 0
% \]
% which is \emph{Laplace's equation.}
% 
% These \emph{partial differential equations} are important in many
% branches of Physics and Mathematics. You will study (and solve) them
% next year.

%%%
%%% The following material on point charges was dropped in 2012/13.
%%% The students no longer know about delta functions after the
%%% abolition of MfP4: Waves.
%%%

%\subsubsection{Poisson's equation and ME1 for a point charge (not examinable)}
%Define the three-dimensional Dirac delta function
%$\delta^{(3)}(\v{r}-\v{r}_{\:\!0})$ as the product of three ordinary
%`one dimensional' delta functions
%\[
%  \delta^{(3)}(\v{r}-\v{r}_{\:\!0})
%  \,=\,
%  \delta(x-x_0)\, \delta(y-y_0)\, \delta(z-z_0)
%\]
%Clearly, $\delta^{(3)}(\v{r}-\v{r}_{\:\!0})$ is zero everywhere,
%\emph{except} at $\v{r}=\v{r}_{\:\!0}$ where it is formally
%`infinite'.\footnote{See \emph{MfP4: Waves} notes for details on the
%  definition and properties of the Dirac delta function.}
%
%The charge density for a point charge $q$ at $\v{r}_{\:\!0}$ is
%\[
%  \rho(\v{r}) \,=\, q \, \delta^{(3)}(\v{r}-\v{r}_{\:\!0})
%\]
%To see this, integrate over a small volume $V$ containing
%$\v{r}_{\:\!0}$
%\[
%  \int_V \rho(\v{r}) \diff V
%  \,=\,
%  q \int_V  \delta(x-x_0)\, \delta(y-y_0)\, \delta(z-z_0)\,
%  \diff x \, \diff y \, \diff z
%  \,=\,
%  q
%\]
%which is indeed the total charge in $V$. To get this, we used
%$\int_a^b \delta(x-x_0) = 1$ when $a < x_0 < b$, and zero otherwise.
%Similarly for the integrals over $y$ and $z$.
%
%For the case of a point charge $q$ at $\v{r}=\v{r}_{\:\!0}$, Maxwell's
%first equation becomes
%\[
%  \Div \v{E}(\v{r})
%  \,=\,
%  \frac{q}{\epsilon_0} \: \delta^{(3)}(\v{r}-\v{r}_{\:\!0})
%\]
%and Poisson's equation is
%\[
%  \lap \phi(\v{r})
%  \,=\,
%  -\frac{q}{\epsilon_0} \, \delta^{(3)}(\v{r}-\v{r}_{\:\!0})
%\]
%We know that $\phi(\vr) =q/(4\pi\epsilon_0 r)$ for a point charge at
%the origin (\emph{i.e.}~$\v{r}_{\:\!0}=0$) -- see
%  equation~(\ref{eq:phiforpointcharge}).  Substituting for $\phi$ into
%  Poisson's equation (and cancelling $q/\epsilon_0$) gives
%\begin{equation}
%  \lap \left( \frac{1}{r} \right) 
%  \,=\,
%  -4\pi \:\! \delta^{(3)}(\v{r})
%  \label{eq:lap1onr}
%\end{equation}
%Similarly, for a point charge $q$ at the origin, we have
%\[
% \v{E}(\v{r})
%  \,=\,
%  \frac{\,q}{4 \pi \epsilon_0}\: \frac{\v{r}}{r^3}
%\]
%and Maxwell's first equation becomes
%\begin{equation}
%  \Div \left( \frac{\v{r}}{r^3}\right)
%  \,=\,
%  4\pi \:\! \delta^{(3)}(\v{r})
%  \label{eq:divronrcubed}
%\end{equation}
%Note that equations~(\ref{eq:lap1onr}) and~(\ref{eq:divronrcubed})
%hold for \emph{all} $\v{r}$, including the \emph{singular} point
%$\v{r}=0$.

\subsection{Example on joint use of Divergence and Stokes' theorems}
\begin{figure}[t]
  \centering
  \includegraphics{tikz_stokes_closed.pdf}
  \caption{Cut of a closed surface in two parts.}
  \label{fig:stokes_closed}
\end{figure}
One can show that $\Div\curl\v{a}=0$ is independent of any coordinate system, as we discuss now.
Let $S$ be a closed surface, enclosing a volume $V$. Applying the divergence theorem to
$\curl\v{a}$, we obtain
\[
  \intV \Div \left(\curl\v{a}\right)\;\dd V
  \,=\,
  \intS \left(\curl\v{a}\right)\cdot\dd\v{S}
\]
Now divide $S$ into two surfaces $S_1$ and $S_2$ with a \emph{common} boundary $C$ as
shown in \cref{fig:stokes_closed}. Now use Stokes' theorem to write
\[
  \intS \left(\curl\v{a}\right)\cdot\dd\v{S}
  \,=\,
  \int_{S_1} \left(\curl\v{a}\right)\cdot\dd\v{S}
  \,+\, \int_{S_2} \left(\curl\v{a}\right)\cdot\dd\v{S}
  \,=\,
  \ointC \v{a}\cdot \diff \v{r} \,-\, \ointC \v{a}\cdot \diff \v{r}
  \,=\,
  0
\]
where the second line integral appears with a minus sign because it is traversed in the
\emph{opposite} direction. (Recall that Stokes' theorem applies to curves traversed in
the right-hand sense with respect to the outward normal of the surface.) Since this
result holds for arbitrary volumes, we must have
\[
  \Div\curl\v{a} \,=\, 0
\]

\subsection{Non-conservative forces}

As an example of a \emph{non-conservative force} let us mention the \emph{Lorentz Force}
on a moving point charge $q$ at $\vr$
\[
  \vF(\vr) \,=\, q \vE(\vr) \,+\, q \v{v} \times \vB(\vr)
\]
where $\vB(\vr)$ is the \emph{magnetic field} at $\vr$. Since the force involves the
particle velocity $\v{v}$ as well as position $\vr$ we cannot construct a scalar
potential for this force. It is often said that ``a magnetic force does no work'' because
\[
  \v{F} \cdot \dd \v{r}
  \,=\,
  \left(q\v{v} \times \vB(\vr)\right) \cdot\dd \v{r}
  \,=\,
  \left(q\v{v} \times \vB(\vr)\right) \cdot \v{v}\, \dd t
  \,=\,
  0 \,.
\]
In general, a force that depends on velocity $\vv$ is non-conservative because the $\vv$
dependence implies some dependence on how the particle arrives at $\vr$ in addition to
any explicit dependence on $\vr$. However, forces that depend only on $\vr$ are
non-conservative when $\curl \vF(\vr) \neq 0$. Generally, one may write \emph{any} vector
field $\va(\v{r})$ as
\[
  \va \,=\, -\del \phi \,+\, \curl \vA \,,
\]
which is known as \emph{Helmholtz's theorem} or the \emph{fundamental theorem of vector
  calculus}. It states that we can decompose a vector field into a curl-free (or
conservative) part $-\del \phi$ and a divergence-free part $\curl \vA$. We will admit
this result here.
