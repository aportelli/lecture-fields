% !TEX root = ../../fields.tex
%% Brian Lecture 15 %%%%%%%%%%%%%%%%%%%%%%%%%%%%%%%%%%%%%%%%%%%%%%%%%%%%%%%%%%%%%%%%%%%%%%%

Most of the material on the next two or three pages should be revision. I will summarise
it in lectures, but I will not cover all of it in detail. It should be sufficiently self
contained to learn from.

Volume integrals are conceptually simpler than line and surface integrals because the
element of volume ${\rm d}V$ is a \emph{scalar} quantity.

We define integrals over a volume $V$ in the standard way using the Riemann sum.

\subsection{Integrals over scalar fields}

Let $f(\v{r})$ be a scalar field defined in a volume $V$. Divide $V$ into $n$ small
volumes $\delta V^{(i)}$. Then
\[
  \intV f(\v{r})\, {\rm d}V
  \,=\,
  \raisebox{-3ex}
  {$
      \shortstack{lim \\ $ {}_{n\to\infty}$ \\ ${}_{\delta
              V^{(i)} \to 0}$}\;
    $}
  %%% \lim_{\stackrel{n\to\infty}{\it\small \delta V\to0}}\,
  \sum_{i=0}^{n-1}\;
  f(\v{r}^{(i)})\; \delta V^{(i)}
\]
where $f(\v{r}^{(i)})$ is the value of the function $f$ at some point $\v{r}^{(i)}$ in
the element of volume $\delta V^{(i)}\,$.

In Cartesian coordinates
\[
  \intV f(\v{r})\, {\rm d}V
  \,=\,
  \int_z \int_y \int_x f(x,y,z) \, {\rm d}x \, {\rm d}y \, {\rm d}z
\]
where the integrals over $x$, $y$ and $z$ have appropriate limits.
%
Note that $\int_V f(\v{r}) \, {\rm d}V$ is a \emph{triple} integral, but we use three
integral signs only when writing it in terms of explicit integration variables, namely
$x$, $y$, $z$ in this case.

\paragraph{Example:} Integrate $f(x,y,z) = x y z^2$ over the cuboid: $\left\{0 \le x < a, \: 0 \le y < b, \: 0
  \le z < c \right\}$.

We \emph{choose} to perform the integral over $x$ first, keeping $y$ and $z$ fixed; then
the integral over $y$, keeping $z$ fixed; and finally the integral over $z\,$:
\begin{eqnarray*}
  I
  &=&
  \intV f(\v{r}) \, {\rm d}V
  \,=\,
  \int_{z=0}^{z=c}  \int_{y=0}^{y=b}  \int_{x=0}^{x=a}
  x \:\! y \:\! z^2 \: {\rm d}x \, {\rm d}y \, {\rm d}z \\
  &=&
  \int_{z=0}^{z=c}  {\rm d}z \int_{y=0}^{y=b} {\rm d}y
  \; \frac{a^2}{2} \:\! y \:\! z^2
  \,=\,
  \frac{a^2}{2} \int_{z=0}^{z=c}  {\rm d}z \;
  \frac{b^2}{2} \:\! z^2 \,
  \,=\,
  \frac{a^2}{2}  \, \frac{b^2}{2} \, \frac{c^3}{3}
  \,=\,
  \frac {a^2 \, b^2 \, c^3}{12}
\end{eqnarray*}
The volume of the cuboid is simply
\[
  V
  \,=\,
  \intV {\rm d}V
  \,=\,
  \int_{z=0}^{z=c}  \int_{y=0}^{y=b}  \int_{x=0}^{x=a}
  \; {\rm d}x \, {\rm d}y \, {\rm d}z
  \,=\,
  abc
\]

The integrals may be performed in any order, but the limits on the integrals must be
chosen appropriately to cover the volume $V$.

For example, if we choose to perform the $z$ integral first, its limits may depend on $x$
and $y$. The limits on the second integral over $y$ may then depend on $x$ (but not on
$z$, because we have already integrated over $z$). The limits on the last integral over
$x$ can't depend on either $y$ or $z$ (because we have already integrated over them.)

\textbf{Example:}
\[
  \int_{0}^{1}   {\rm d}x
  \int_{0}^{x^2} {\rm d}y
  \int_{xy}^{1}  \,  2x^2 \:\! z \: {\rm d}z
  \,=\,
  \frac{28}{165} \quad \mbox{(exercise)}
\]

We now illustrate how to find the limits in an explicit example, albeit a rather
complicated (and slightly masochistic) one.

%\newpage

\parbox{0.5\textwidth}{
  \paragraph{Example:} Use Cartesian coordinates to evaluate
  \[
    \intV\,(x+y+z)\,{\rm d}V
  \]
  where $V$ is the positive octant of the unit sphere:
  \[
    x^{2}+y^{2}+z^{2}\leq 1, \qquad
    x\geq 0, \; y\geq 0, \; z\geq 0 \\[0.3ex]
  \]
  If we \emph{choose} to perform the $z$ integral first: } \hfill \parbox{0.45\textwidth}{
  \epsfxsize=0.45\textwidth \includegraphics{tikz_spherical_octant.pdf} }
\begin{enumerate}
  \item For \emph{fixed} $(x,y)$, we first integrate with respect to $z$, from the point
        $(x,y,0)$ (\emph{i.e.} $z=0$), to the point $(x,y,z)$ with $z=\sqrt{1-x^2-y^2}$. This is
        the integral up the strip shown.

  \item Then, for fixed $x$, we integrate wrt $y$ from the vertical strip at $y=0$ to the
        vertical strip at $y=\sqrt{1-x^2}$. This sums over all such strips in the planar quadrant
        shown.

  \item Finally, we integrate from $x=0$ to $x=1$, which sums over all planes from $x=0$ to $1$.
\end{enumerate}

\vspace*{-1ex}

\[
  \Rightarrow \quad
  \intV\,(x+y+z)\,{\rm d}V
  \,=\,
  \int_{0}^{1}   {\rm d}x
  \int_{0}^{\sqrt{1-x^2}} {\rm d}y
  \int_{0}^{\sqrt{1-x^2-y^2}}    (x+y+z) \; {\rm d}z
\]

\subsection{Parametric form of volume integrals}
The example above was very complicated -- because we used Cartesians. It's \emph{much}
easier using a \emph{parametric} representation. (We would use spherical polars in this
example.)

Suppose we can write the position vector $\v{r}$ in terms of three real parameters $u$,
$v$ and $w$, so that $\v{r}=\v{r}(u,v,w)$. If we make a small change in each of these
parameters, then $\v{r}$ changes by
\[
  {\rm d}\v{r}
  \,=\,
  \parpar{\v{r}}{u}\; du +
  \parpar{\v{r}}{v}\; dv +
  \parpar{\v{r}}{w}\; dw
\]
Along the curves $\{v={\rm constant},\, w={\rm constant}\}$, we have $dv = dw = 0$, so
${\rm d}\v{r}$ is simply
\[
  {\rm d}\v{r}_{\,u} \,=\,  \parpar{\v{r}}{u}\; du
\]
with ${\rm d}\v{r}_{\,v}$ and $ {\rm d}\v{r}_{\,w}$ having similar definitions.

\bigskip

\parbox{0.65\textwidth}{As illustrated in the figure, the vectors
  ${\rm d}\v{r}_{\,u}$, $ {\rm d}\v{r}_{\,v}$ and ${\rm d}\v{r}_{\,w}$
  form the sides of an infinitesimal parallelepiped of volume
  \begin{eqnarray*}
    {\rm d}V
    & = &
    \left|\,
    {\rm d}\v{r}_{\,u} \cdot
    \left(
    {\rm d}\v{r}_{\,v} \times {\rm d}\v{r}_{\,w}
    \right)
    \right|
  \end{eqnarray*}
  \begin{center}
    \bigbox{
      \parbox{0.48\textwidth}{
        \vspace{-1.5ex}
        \[
          \Rightarrow \quad
          {\rm d}V
          \,=\,
          \left|
          \parpar{\v{r}}{u} \,\cdot\,
          \left(
          \parpar{\v{r}}{v} \,\times\, \parpar{\v{r}}{w}
          \right)
          \right|
          \; \mbox{d}u\, \mbox{d}v\, \mbox{d}w
        \]
        \vspace{-1.8ex}
      }
    }
  \end{center}
}
\hfill
\parbox{0.35\textwidth}{
  \vspace*{-2ex}
  \epsfxsize=0.35\textwidth
  \includegraphics{tikz_dV.pdf}
}
%
We take the \emph{magnitude} of the scalar triple product so that the element of volume
${\rm d}V$ is always \emph{positive}.

\paragraph{Jacobians revisited:} We may write the volume element in terms of the Jacobian $J\,$, \\ \( {\rm d}V \,=\,
\left|J\right| {\rm d}u \, {\rm d}v \, {\rm d}w \,,\; \) where
\[
  J
  \,\equiv\,
  \frac{\partial(x,y,z)}{\partial(u,v,w)}
  \,\equiv\,
  \det \left(
  \begin{array}{ccc}
      \ds\frac{\partial x}{\partial u} &
      \ds\frac{\partial y}{\partial u} &
      \ds\frac{\partial z}{\partial u}
      \\[2ex]
      \ds\frac{\partial x}{\partial v} &
      \ds\frac{\partial y}{\partial v} &
      \ds\frac{\partial z}{\partial v}
      \\[2ex]
      \ds\frac{\partial x}{\partial w} &
      \ds\frac{\partial y}{\partial w} &
      \ds\frac{\partial z}{\partial w}
    \end{array}
  \right)
\]

% This is the three dimensional version of the $2\times2$ Jacobian
% derived in \emph{LA\&SVC} Section~18 and (presumably)
% in \emph{SVC\&DE}. It generalises to higher dimensions in a
% straightforward way.

% section (\ref{Elements-of-area-and-Jacobians}), and it

\paragraph{Example:} Consider a circular cylinder of radius $a$, height $c$. We can parameterise $\v{r}$ using
cylindrical coordinates. Within the cylinder, we have
\[
  \v{r}
  \,=\,
  \rho\cos\phi \: \eone + \rho\sin\phi \: \etwo + z \: \ethree
  \quad \mbox{with} \quad
  \left\{
  0\leq\rho\leq a,\:
  0\leq\phi\leq2\pi,\:
  0\leq z \leq c
  \right\}
\]
\parbox{0.57\textwidth}{
  Then
  \begin{eqnarray*}
    \parpar{\v{r}}{\rho}
    &=&
    \cos\phi\,\eone + \sin\phi\,\etwo \quad\\[1ex]
    %
    \parpar{\v{r}}{\phi}
    &=&
    -\rho\, \sin\phi\,\eone + \rho\, \cos\phi\,\etwo\\[1ex]
    %
    \parpar{\v{r}}{z}
    &=&
    \ethree
  \end{eqnarray*}
  And hence (easy exercise)
  \begin{eqnarray*}
    {\rm d}V
    & = &
    \left|
    \parpar{\v{r}}{\rho} \cdot
    \left(
    \parpar{\v{r}}{\phi} \times \parpar{\v{r}}{z}
    \right)
    \right|
    \: {\rm d}\rho\, {\rm d}\phi\, \mbox{d}z
    \,=\,
    \rho\, {\rm d}\rho\, {\rm d}\phi\, \mbox{d}z
    \phantom{Hence---------}
  \end{eqnarray*}
  The \emph{volume} of the cylinder is
  \[
    V
    \,=\,
    \intV {\rm d}V
    \,=\,
    \int^{z=c}_{z=0} \int^{\phi=2\pi}_{\phi=0} \int^{\rho=a}_{\rho=0}
    \rho\, {\rm d}\rho\, {\rm d}\phi\, dz
    \,=\,
    \pi\,a^2c.
  \]
}\hfill
\parbox{0.35\textwidth}{
  \epsfxsize=0.35\textwidth
  \includegraphics{tikz_cyl_dV.pdf}
}
\paragraph{Cylindrical basis:} the normalised vectors
\[
  \erho \,=\, \parpar{\v{r}}{\rho}\left/ \: \left|
  \parpar{\v{r}}{\rho}\right| \right.  \quad;\quad \ephi \,=\,
  \parpar{\v{r}}{\phi}\left/ \: \left| \parpar{\v{r}}{\phi} \right|
  \right. \quad;\quad \ez \,=\, \parpar{\v{r}}{z}\left/ \: \left|
  \parpar{\v{r}}{z} \right| \right.
\]
(shown in the figure) are the orthonormal basis vectors that we wrote
down earlier for cylindrical coordinates (exercise).

\paragraph{Exercise:} For spherical polars: \( \v{r} = r\sin \theta\cos\phi\,\eone + r\sin \theta \sin
\phi\,\etwo + r \cos \theta\, \ethree \,,\; \) show that
\begin{eqnarray*}
  {\rm d}V
  &=&
  \left|
  \parpar{\v{r}}{r} \, \cdot
  \left(
  \parpar{\v{r}}{\theta} \times \parpar{\v{r}}{\phi}
  \right)
  \right|
  \; {\rm d}r\, {\rm d}\theta\, {\rm d}\phi
  \,=\,
  r^2 \sin \theta \, {\rm d}r\, {\rm d}\theta\, {\rm d}\phi
\end{eqnarray*}

\subsection{Integrals over vector fields}
The integral of a vector field $\v{a}(\v{r})$ over a volume $V$ is
\[
  \intV \v{a}(\v{r})\; {\rm d}V
  \,=\,
  %  \sum_{i=1}^3 \;
  \ei \intV a_i(x,y,z)\; {\rm d}V
  \,=\,
  %  \sum_{i=1}^3  \;
  \ei \int_z\int_y\int_x a_i(x,y,z) \; \, {\rm d}x \, {\rm d}y \, {\rm d}z
\]
in Cartesian coordinates (using suffix notation and the summation convention).

The result of the integral is a \emph{vector}, and we must evaluate a triple integral for
\emph{each} of the $3$ components of the vector.

\paragraph{Example:}
Consider a solid hemisphere of radius $a$ centered on the $\ethree$ axis, with its bottom
face at $x_3=0$. If the mass density (mass/unit volume), a scalar field, is $\rho(r)=
  \rho_0/r$ where $\rho_0$ is a constant, what is the total mass, $M$, of the hemisphere?

It is most convenient to use spherical polar coordinates. Then ${\rm d}V = r^2\sin\theta
  \, {\rm d}r \, {\rm d}\theta \, {\rm d}\phi$ and
\begin{eqnarray}
  M
  \,=\,
  \int_V \rho(r) \, {\rm d}V
  \,=\,
  \int_{0}^{a}  \rho(r) \, r^2 \, {\rm d}r
  \int_{0}^{\pi/2} \sin \theta \, {\rm d}\theta
  \int_{0}^{2\pi} {\rm d}\phi
  \,=\,
  2\pi\rho_0 \int_{0}^{a} r \, {\rm d}r
  \,=\,
  \pi\rho_0 \, a^{2}\nonumber
\end{eqnarray}

\smallskip

Now consider the \emph{centre of mass vector} $\v{R}=\left(X \, \eone + Y \, \etwo +Z \,
  \ethree\right)$, defined by
\[
  M \vR
  \,\equiv\,
  \int_{V} \; \rho(r) \, \v{r} \: {\rm d}V
\]
We integrate each component of the vector field $\,\rho(r) \:\! \v{r}\,$ in turn using
\[
  \vr
  \,=\,
  r \sin\theta \cos \phi \, \eone
  + r \sin\theta \sin \phi \, \etwo
  + r \cos \theta \, \ethree
  \quad \mbox{and} \quad
  \mbox{d}V
  \,=\,
  r^2 \sin \theta \, {\rm d}r\, {\rm d}\theta\, {\rm d}\phi
\]
which gives
\begin{eqnarray*}
  MX
  &=&
  \int_{0}^{a} \rho(r) \, r^3 \, {\rm d}r
  \int_{0}^{\pi/2} \sin^2\theta \, {\rm d}\theta
  \int_{0}^{2\pi} \cos \phi\, {\rm d}\phi
  \,=\, 0 \quad \mbox{(since the $\phi$ integral gives 0)}\\[1ex]
  %
  MY &=& \int_{0}^{a} \rho(r) \, r^3 \, {\rm d}r \int_{0}^{\pi/2} \sin^2\theta \, {\rm
      d}\theta \int_{0}^{2\pi} \sin \phi \, {\rm d}\phi \,=\, 0 \quad \, \mbox{(since the
    $\phi$ integral gives 0)} \\[1ex]
  %
  MZ &=& \int_{0}^{a} \rho(r) \, r^3 \, {\rm d}r \int_{0}^{\pi/2} \sin\theta \cos\theta \,
  {\rm d}\theta \int_{0}^{2\pi} {\rm d}\phi \,=\, 2 \pi \rho_0 \int_{0}^{a} r^{2}{\rm d}r
  \left[\frac12 \sin^2\theta\right]_0^{\pi/2} \\[1ex]
  %  \int_{0}^{\pi/2} \frac{\sin 2\theta}{2} \, {\rm d}\theta
  %
  &=&
  % \frac{2\pi\rho_0 a^3}{3}\left[\frac{-\cos 2\theta}{4}\right]_0^{\pi/2}
  2\pi\rho_0 \, \frac{ a^3}{3} \, \frac12
  \,=\,
  \frac{\pi\rho_0 \, a^{3}}{3}
\end{eqnarray*}

Hence
\[
  %  \qquad \Rightarrow \quad
  \vR
  \,=\,
  \frac{\pi\rho_0 \, a^3/3}{\pi\rho_0 \, a^2} \: \ethree
  \,=\,
  \frac{a}{3} \: \ethree
  \qquad \mbox{(independent of $\rho_0$)}
\]

Note that the integrals over $\phi$ in the first two components of $M\v{R}$ are zero.
Watch out for integrals that are zero -- spotting them can save you a lot of un-necessary
work!

\newpage

\subsection{Summary of polar coordinate systems}
\label{polar_coordinate_systems}
To conclude this section, we give a brief summary of polar coordinate
systems.

In the figures below, ${\rm d}S$ indicates an area element and ${\rm d}V$ a volume
element. We also sketch geometrical ``derivations'' of the infinitesimal elements of area
and volume.

% Note that different conventions, \emph{e.g.}~for the angles $\phi$ and
% $\theta$, are sometimes used.

\paragraph{Plane polar coordinates:} $(\rho, \, \phi)$

\begin{figure}[ht]
  %
  \parbox{\textwidth}{
    \centerline{\epsfxsize=0.4\textwidth\includegraphics{tikz_plane-polar-summary.pdf}}
    %
  }
\end{figure}

\vspace*{-2ex}

\paragraph{Cylindrical coordinates:} $(\rho,\, \phi,\, z)$

\begin{figure}[ht]
  %
  \centerline{\epsfxsize=0.5\textwidth\includegraphics{cylin_polar.pdf}}
  %
\end{figure}

\paragraph{Spherical polar coordinates:} $(r,\, \theta,\, \phi)$

\begin{figure}[ht]
  %
  \centerline{\epsfxsize=0.65\textwidth\includegraphics{sphere_polar.pdf}}
  %
\end{figure}
