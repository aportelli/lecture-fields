% !TEX root = ../../fields.tex
%% Brian Lecture 15 %%%%%%%%%%%%%%%%%%%%%%%%%%%%%%%%%%%%%%%%%%%%%%%%%%%%%%%%%%%%%%%%%%%%%%%
Volume integrals are conceptually simpler than line and surface integrals because the
element of volume $\diff V$ is a \emph{scalar} quantity. We define integrals over a
volume $V$ in the standard way using the Riemann sum (\ie splitting $V$ into small
cuboids and considering the limit where their size goes to zero).

\subsection{Integrals over scalar fields}
Let $f(\v{r})$ be a scalar field defined in a volume $V$. Divide $V$ into $n$ small
volumes $\delta V^{(i)}$. Then
\[
  \intV f(\v{r})\, \diff V
  \,=\,
  \raisebox{-3ex}
  {$
      \shortstack{lim \\ $ {}_{n\to\infty}$ \\ ${}_{\delta
              V^{(i)} \to 0}$}\;
    $}
  \sum_{i=0}^{n-1}\;
  f(\v{r}_{i})\; \delta V_i
\]
where $f(\v{r}^{(i)})$ is the value of the function $f$ at some point $\v{r}_i$ in the
element of volume $\delta V_i$. In Cartesian coordinates
\[
  \intV f(\v{r})\, \diff V
  \,=\,
  \int_z \int_y \int_x f(x,y,z) \, \diff x \, \diff y \, \diff z
\]
where the integrals over $x$, $y$ and $z$ have appropriate domains. Note that $\int_V
  f(\v{r}) \, \diff V$ is a \emph{triple} integral, but we use three integral signs only
when writing it in terms of explicit integration variables, namely $x$, $y$, $z$ in this
case.

\paragraph{Example:} Integrate $f(x,y,z) = x y z^2$ over the cuboid: $\left\{0 \le x < a, \: 0 \le y < b, \: 0
  \le z < c \right\}$. We \emph{choose} to perform the integral over $x$ first, keeping $y$
and $z$ fixed; then the integral over $y$, keeping $z$ fixed; and finally the integral
over $z\,$:
\begin{eqnarray*}
  I
  &=&
  \intV f(\v{r}) \, \diff V
  \,=\,
  \int_{z=0}^{z=c}  \int_{y=0}^{y=b}  \int_{x=0}^{x=a}
  x \:\! y \:\! z^2 \: \diff x \, \diff y \, \diff z \\
  &=&
  \int_{z=0}^{z=c}  \diff z \int_{y=0}^{y=b} \diff y
  \; \frac{a^2}{2} \:\! y \:\! z^2
  \,=\,
  \frac{a^2}{2} \int_{z=0}^{z=c}  \diff z \;
  \frac{b^2}{2} \:\! z^2 \,
  \,=\,
  \frac{a^2}{2}  \, \frac{b^2}{2} \, \frac{c^3}{3}
  \,=\,
  \frac {a^2 \, b^2 \, c^3}{12}
\end{eqnarray*}
The volume of the cuboid is simply
\[
  V
  \,=\,
  \intV \diff V
  \,=\,
  \int_{z=0}^{z=c}  \int_{y=0}^{y=b}  \int_{x=0}^{x=a}
  \; \diff x \, \diff y \, \diff z
  \,=\,
  abc
\]
The integrals may be performed in any order, but the limits on the integrals must be
chosen appropriately to cover the volume $V$. For example, if we choose to perform the
$z$ integral first, its limits may depend on $x$ and $y$. The limits on the second
integral over $y$ may then depend on $x$ (but not on $z$, because we have already
integrated over $z$). The limits on the last integral over $x$ can't depend on either $y$
or $z$ (because we have already integrated over them.)

\paragraph{Example:}
\[
  \int_{0}^{1}   \diff x
  \int_{0}^{x^2} \diff y
  \int_{xy}^{1}  \,  2x^2 \:\! z \: \diff z
  \,=\,
  \frac{28}{165} \quad \mbox{(exercise)}
\]

\begin{figure}[htbp]
  \centering
  \includegraphics{tikz_spherical_octant.pdf}
  \caption{Parameterisation of the positive octant of the unit sphere.}
  \label{fig:spherical_octant}
\end{figure}
\paragraph{Example:} Use Cartesian coordinates to evaluate
\[
  \intV\,(x+y+z)\,\diff V
\]
where $V$ is the positive octant of the unit sphere:
\[
  x^{2}+y^{2}+z^{2}\leq 1, \qquad
  x\geq 0, \; y\geq 0, \; z\geq 0 \\[0.3ex]
\]
If we \emph{choose} to perform the $z$ integral first (\cf\cref{fig:spherical_octant} for
illustration):
\begin{enumerate}
  \item For \emph{fixed} $(x,y)$, we first integrate with respect to $z$, from the point
        $(x,y,0)$ (\emph{i.e.} $z=0$), to the point $(x,y,z)$ with $z=\sqrt{1-x^2-y^2}$. This is
        the integral up the strip shown.

  \item Then, for fixed $x$, we integrate in $y$ from the vertical strip at $y=0$ to the vertical
        strip at $y=\sqrt{1-x^2}$. This sums over all such strips in the planar quadrant shown.

  \item Finally, we integrate from $x=0$ to $x=1$, which sums over all planes from $x=0$ to $1$.
\end{enumerate}

\vspace*{-1ex}

\[
  \Rightarrow \quad
  \intV\,(x+y+z)\,\diff V
  \,=\,
  \int_{0}^{1}   \diff x
  \int_{0}^{\sqrt{1-x^2}} \diff y
  \int_{0}^{\sqrt{1-x^2-y^2}}    (x+y+z) \; \diff z
\]

\subsection{Parametric form of volume integrals}
The example above was quite involved -- because we used Cartesian coordinates. It is
considerably simpler using a \emph{parametric} representation (\eg spherical coordinates
in this example). However, to do this we need to understand how the volume element $\diff
  V$ is expressed in another coordinate system. Suppose we can write the position vector
$\v{r}$ in terms of three real parameters $u$, $v$ and $w$, so that $\v{r}=\v{r}(u,v,w)$.
If we make a small change in each of these parameters, then $\v{r}$ changes by
\[
  \diff \v{r}
  \,=\,
  \parpar{\v{r}}{u}\; \diff u +
  \parpar{\v{r}}{v}\; \diff v +
  \parpar{\v{r}}{w}\; \diff w
\]
Along the curves $\{v={\rm constant},\, w={\rm constant}\}$, we have $\diff v = \diff w =
  0$, so $\diff \v{r}$ is simply
\[
  \diff \v{r}_{\,u} \,=\,  \parpar{\v{r}}{u}\; \diff u
\]
\noindent\parbox{0.65\textwidth}{with $\diff \v{r}_{\,v}$ and $ \diff \v{r}_{\,w}$
  having similar definitions. As illustrated in the figure, the vectors $\diff
    \v{r}_{\,u}$, $ \diff \v{r}_{\,v}$ and $\diff \v{r}_{\,w}$ form the sides of an
  infinitesimal parallelepiped of volume
  \begin{eqnarray*}
    \diff V
    & = &
    \left|\,
    \diff \v{r}_{\,u} \cdot
    \left(
    \diff \v{r}_{\,v} \times \diff \v{r}_{\,w}
    \right)
    \right|
  \end{eqnarray*}
  \begin{center}
    \bigbox{
      \parbox{0.5\textwidth}{
        \vspace{-1.5ex}
        \[
          \Rightarrow \quad
          \diff V
          \,=\,
          \left|
          \parpar{\v{r}}{u} \,\cdot\,
          \left(
          \parpar{\v{r}}{v} \,\times\, \parpar{\v{r}}{w}
          \right)
          \right|
          \; \diff u\, \diff v\, \diff w
        \]
        \vspace{-1.8ex}
      }
    }
  \end{center}
}
\hfill
\parbox{0.34\textwidth}{
  \includegraphics{tikz_dV.pdf}
}\\
%
We take the \emph{magnitude} of the scalar triple product so that the element of volume
$\diff V$ is always \emph{positive}.

\paragraph{Jacobians revisited:} We may write the volume element in terms of the Jacobian $J$, \\ \( \diff V \,=\,
\left|J\right| \diff u \, \diff v \, \diff w \,,\; \) where
\[
  J
  \,\equiv\,
  \frac{\partial(x,y,z)}{\partial(u,v,w)}
  \,\equiv\,
  \det \left(
  \begin{array}{ccc}
      \ds\frac{\partial x}{\partial u} &
      \ds\frac{\partial y}{\partial u} &
      \ds\frac{\partial z}{\partial u}
      \\[2ex]
      \ds\frac{\partial x}{\partial v} &
      \ds\frac{\partial y}{\partial v} &
      \ds\frac{\partial z}{\partial v}
      \\[2ex]
      \ds\frac{\partial x}{\partial w} &
      \ds\frac{\partial y}{\partial w} &
      \ds\frac{\partial z}{\partial w}
    \end{array}
  \right)
\]
\begin{figure}[t]
  \centering
  \includegraphics{tikz_cyl_dV.pdf}
  \caption{Cylindrical element of volume.}
  \label{fig:cyl_dV}
\end{figure}
\paragraph{Example:} Consider a circular cylinder of radius $a$, height $c$. We can parameterise $\v{r}$ using
cylindrical coordinates. Within the cylinder, we have
\[
  \v{r}
  \,=\,
  \rho\cos\phi \: \eone + \rho\sin\phi \: \etwo + z \: \ethree
  \quad \mbox{with} \quad
  \left\{
  0\leq\rho\leq a,\:
  0\leq\phi\leq2\pi,\:
  0\leq z \leq c
  \right\}
\]

Then
\begin{eqnarray*}
  \parpar{\v{r}}{\rho}
  &=&
  \cos\phi\,\eone + \sin\phi\,\etwo \quad\\[1ex]
  %
  \parpar{\v{r}}{\phi}
  &=&
  -\rho\, \sin\phi\,\eone + \rho\, \cos\phi\,\etwo\\[1ex]
  %
  \parpar{\v{r}}{z}
  &=&
  \ethree
\end{eqnarray*}
And hence
\begin{eqnarray*}
  \diff V
  & = &
  \left|
  \parpar{\v{r}}{\rho} \cdot
  \left(
  \parpar{\v{r}}{\phi} \times \parpar{\v{r}}{z}
  \right)
  \right|
  \: \diff \rho\, \diff \phi\, \diff z
  \,=\,
  \rho\, \diff \rho\, \diff \phi\, \diff z\,,
\end{eqnarray*}
which is left as an exercise to the reader.
The volume of the cylinder is therefore
\[
  V
  \,=\,
  \intV \diff V
  \,=\,
  \int^{z=c}_{z=0} \int^{\phi=2\pi}_{\phi=0} \int^{\rho=a}_{\rho=0}
  \rho\, \diff \rho\, \diff \phi\, dz
  \,=\,
  \pi\,a^2c.
\]
\paragraph{Cylindrical basis:} the normalised vector
\[
  \erho \,=\, \parpar{\v{r}}{\rho}\left/ \: \left|
  \parpar{\v{r}}{\rho}\right| \right.  \quad;\quad \ephi \,=\,
  \parpar{\v{r}}{\phi}\left/ \: \left| \parpar{\v{r}}{\phi} \right|
  \right. \quad;\quad \ez \,=\, \parpar{\v{r}}{z}\left/ \: \left|
  \parpar{\v{r}}{z} \right| \right.
\]
(shown in the figure) are the orthonormal basis vectors that we wrote
down earlier for cylindrical coordinates (exercise).

\paragraph{Exercise:} For spherical polars: \( \v{r} = r\sin \theta\cos\phi\,\eone + r\sin \theta \sin
\phi\,\etwo + r \cos \theta\, \ethree \,,\; \) show that
\begin{eqnarray*}
  \diff V
  &=&
  \left|
  \parpar{\v{r}}{r} \, \cdot
  \left(
  \parpar{\v{r}}{\theta} \times \parpar{\v{r}}{\phi}
  \right)
  \right|
  \; \diff r\, \diff \theta\, \diff \phi
  \,=\,
  r^2 \sin \theta \, \diff r\, \diff \theta\, \diff \phi
\end{eqnarray*}

\subsection{Integrals over vector fields}
The integral of a vector field $\v{a}(\v{r})$ over a volume $V$ is
\[
  \intV \v{a}(\v{r})\; \diff V
  \,=\,
  %  \sum_{i=1}^3 \;
  \ei \intV a_i(x,y,z)\; \diff V
  \,=\,
  %  \sum_{i=1}^3  \;
  \ei \int_z\int_y\int_x a_i(x,y,z) \; \, \diff x \, \diff y \, \diff z
\]
in Cartesian coordinates (using suffix notation and the summation convention). The result
of the integral is a vector, and we must evaluate a triple integral for each of the $3$
components of the vector.

\paragraph{Example:}
Consider a solid hemisphere of radius $a$ centred on the $\ethree$ axis, with its bottom
face at $x_3=0$. If the mass density (mass/unit volume), a scalar field, is $\rho(r)=
  \rho_0/r$ where $\rho_0$ is a constant, what is the total mass, $M$, of the hemisphere?
It is most convenient to use spherical polar coordinates. Then $\diff V = r^2\sin\theta
  \, \diff r \, \diff \theta \, \diff \phi$ and
\begin{eqnarray}
  M
  \,=\,
  \int_V \rho(r) \, \diff V
  \,=\,
  \int_{0}^{a}  \rho(r) \, r^2 \, \diff r
  \int_{0}^{\pi/2} \sin \theta \, \diff \theta
  \int_{0}^{2\pi} \diff \phi
  \,=\,
  2\pi\rho_0 \int_{0}^{a} r \, \diff r
  \,=\,
  \pi\rho_0 \, a^{2}\nonumber
\end{eqnarray}
Now consider the \emph{centre of mass vector} $\v{R}=\left(X \, \eone + Y \, \etwo +Z \,
  \ethree\right)$, defined by
\[
  M \vR
  \,\equiv\,
  \int_{V} \; \rho(r) \, \v{r} \: \diff V
\]
We integrate each component of the vector field $\,\rho(r) \:\! \v{r}\,$ in turn using
\[
  \vr
  \,=\,
  r \sin\theta \cos \phi \, \eone
  + r \sin\theta \sin \phi \, \etwo
  + r \cos \theta \, \ethree
  \quad \mbox{and} \quad
  \diff V
  \,=\,
  r^2 \sin \theta \, \diff r\, \diff \theta\, \diff \phi
\]
which gives
\begin{eqnarray*}
  MX
  &=&
  \int_{0}^{a} \rho(r) \, r^3 \, \diff r
  \int_{0}^{\pi/2} \sin^2\theta \, \diff \theta
  \int_{0}^{2\pi} \cos \phi\, \diff \phi
  \,=\, 0 \quad \mbox{(since the $\phi$ integral gives 0)}\\[1ex]
  %
  MY &=& \int_{0}^{a} \rho(r) \, r^3 \, \diff r \int_{0}^{\pi/2} \sin^2\theta \, {\rm
      d}\theta \int_{0}^{2\pi} \sin \phi \, \diff \phi \,=\, 0 \quad \, \mbox{(since the $\phi$
    integral gives 0)} \\[1ex]
  MZ &=& \int_{0}^{a} \rho(r) \, r^3 \, \diff r \int_{0}^{\pi/2} \sin\theta \cos\theta \,
  \diff \theta \int_{0}^{2\pi} \diff \phi \,=\, 2 \pi \rho_0 \int_{0}^{a} r^{2}\diff r
  \left[\frac12 \sin^2\theta\right]_0^{\pi/2} \\[1ex]
  &=&
  2\pi\rho_0 \, \frac{ a^3}{3} \, \frac12
  \,=\,
  \frac{\pi\rho_0 \, a^{3}}{3}
\end{eqnarray*}
Hence
\[
  \vR
  \,=\,
  \frac{\pi\rho_0 \, a^3/3}{\pi\rho_0 \, a^2} \: \ethree
  \,=\,
  \frac{a}{3} \: \ethree
  \qquad \mbox{(independent of $\rho_0$)}
\]
Note that the integrals over $\phi$ in the first two components of $M\v{R}$ are zero.

\subsection{Summary of polar coordinate systems}
\label{polar_coordinate_systems}
To conclude this section, we give a brief summary of polar coordinate
systems. In the figures below, $\diff S$ indicates an area element and $\diff V$ a volume element.
We also sketch geometrical ``derivations'' of the infinitesimal elements of area and
volume.

\paragraph{Plane polar coordinates:} $(\rho, \, \phi)$
\smallskip\\
\includegraphics[width=0.5\linewidth]{tikz_plane-polar-summary.pdf}

\paragraph{Cylindrical coordinates:} $(\rho,\, \phi,\, z)$
\smallskip\\
\includegraphics[width=0.7\linewidth]{cylin_polar.pdf}

\paragraph{Spherical polar coordinates:} $(r,\, \theta,\, \phi)$
\smallskip\\
\includegraphics[width=0.7\linewidth]{sphere_polar.pdf}
