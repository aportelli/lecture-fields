% !TEX root = ../../fields.tex
%% Brian Lecture 16 %%%%%%%%%%%%%%%%%%%%%%%%%%%%%%%%%%%%%%%%%%%%%%%%%%%%%%%%%%%%%%%%%%%%%%%
\subsection{Integral definition of divergence}
\begin{figure}[t]
  \centering
  \includegraphics{tikz_cube_div.pdf}
  \caption{Small element of volume around point $P$ considered in the integral definition of divergence.}
  \label{fig:cube_div}
\end{figure}
Let $\v{a}$ be a vector field in the region $R$, and let $P$ be a point in $R$, then the
divergence of $\v{a}$ at $P$ may be \emph{defined} by
\begin{center}
  \bigbox{
    \parbox{60mm}{
      \[
        \mathrm{div}(\v{a})
        \,=\,
        \lim_{\delta V\to0}\; \frac{1}{\delta V}\,
        \int_{\delta S} \v{a}\,\cdot\, \diff \v{S}
      \]}
  }
\end{center}
where $\delta S$ is a \emph{closed} surface in $R$ which encloses the
volume $\delta V$.  The limit must be taken so that the point $P$ is
within $\delta V$. It can be shown that the limit is independent of
the \emph{shape} of $\delta V$.
This definition of $\mathrm{div}(\v{a})$ is also \emph{basis independent}.

We now show that our original definition of $\Div \v{a}$ is recovered in Cartesian
coordinates. Let $P$ be a point with Cartesian coordinates $(x_0, \, y_0, \, z_0)$
situated at the \emph{centre} of a \emph{small} rectangular block of size $\delta_x
  \times \delta_y \times \delta_z$, with volume $\delta V= \delta_x\, \delta_y\, \delta_z$,
as represented in~\cref{fig:cube_div}.
\begin{itemize}
  \item On the \emph{front} face of the block, parallel to the $(y,z)$ plane at
        $x=x_0+\delta_x/2$, we have \emph{outward} normal $\vn=\v{e}_{\,x}$ and so $ \diff \v{S}
          = \v{e}_{\,x}\, \diff y\, \diff z$

  \item On the \emph{back} face of the block, parallel to the $(y,z)$ plane at
        $x=x_0-\delta_x/2$, we have \emph{outward} normal $\vn=-\v{e}_{\,x}$ and so $\diff \v{S}
          = -\v{e}_{\,x}\, \diff y\, \diff z$
\end{itemize}
Hence, $\v{a}\cdot \diff \v{S} = \pm \, a_x\, \diff y\,\diff z$ on these two faces.
Let us denote the union of the two faces orthogonal to the $x$ axis by $\delta S_x$.

The contribution of these two surfaces to the integral $\int_{\delta S}\v{a}\cdot \diff
  \v{S}$ is given by
\begin{eqnarray*}
  \int_{\delta S_x}\v{a}\cdot \diff \v{S}
  & = &
  \int_z\!\int_y
  \bigg\{
  a_x(x_0+\delta_x/2,y,z) - a_x(x_0-\delta_x/2,y,z)
  \bigg\}
  \: \diff y \, \diff z \\[1.5ex]
  %
  & = &
  \int_z \int_y
  \left\{\:
  \left[
    a_x(x_0,y,z) \,+\, \frac{\delta_x}{2}\;\parpar{a_x}{x}\bigg|_{(x_0,y,z)}
    + \; O(\delta_x^2)
    \right]
  \right. \\[1ex]
  &   &
  \hspace*{9ex}
  \left. - \;
  \left[
    a_x(x_0,y,z) \,-\, \frac{\delta_x}{2}\: \parpar{a_x}{x}\bigg|_{(x_0,y,z)}
    + \; O(\delta_x^2)\,
    \right]
  \: \right\}
  \; \diff y \, \diff z \\[1.5ex]
  & = &
  \int_z \int_y \,\delta_x \, \parpar{a_x}{x}\bigg|_{(x_0,y,z)}
  \diff y \, \diff z
\end{eqnarray*}
where we Taylor-expanded $a_x(x,y,z)$ about the point $(x_0,y,z)$, and
dropped the $O(\delta_x^2)$ terms which will vanish when we divide by
$\delta V$ at the end. So
\[
  \frac{1}{\delta V}\; \int_{\delta S_x} \v{a}\cdot \diff \v{S}
  \,=\,
  \frac{1}{\delta_y\, \delta_z}
  \int_z\int_y \; \parpar{a_x}{x}\bigg|_{(x_0,y,z)} \: \diff y \, \diff z
\]
As we take the limit $\delta_x,\delta_y,\delta_z \to 0$, the integrand tends to a
constant in $\delta V$, so the \\[1ex] integral over $y$
and $z$ tends to $\delta_y \:\! \delta_z
  \ds\left. \parpar{a_x}{x}\right|_{(x_0,y_0,z_0)}$, and we obtain
\[
  \lim_{\delta V\to0} \;
  \frac{1}{\delta V}\; \int_{\delta S_x} \v{a}\cdot \diff \v{S}
  \,=\,
  \parpar{a_x}{x}\bigg|_{(x_0,y_0,z_0)}
\]
Adding similar contributions from the other $4$ faces, we find
\[
  {\mathrm{div}}(\v{a})
  \,=\,
  \parpar{a_x}{x} + \parpar{a_y}{y} + \parpar{a_z}{z}
  \,=\,
  \Div\v{a}
\]
in agreement with our original definition in Cartesian coordinates. Thus, we can continue
to use the notation $\v{\nabla}\cdot\v{a}$ for \emph{both} forms for the divergence of a
vector field $\v{a}(\v{r})$. The integral definition
\[
  \mathrm{div}(\v{a})
  \,=\,
  \Div \v{a}
  \,=\,
  \lim_{\delta V\to0}\; \frac{1}{\delta V}\,
  \int_{\delta S} \v{a}\,\cdot\, \diff \v{S}
\]
provides a precise intuitive understanding of divergence as the (net) flux per unit
volume leaving a small volume around a point $\vr$, \cf\cref{fig:net_flux} for
illustration.
\begin{figure}[t]
  \centering
  \includegraphics{tikz_net_flux.pdf}
  \caption{Illustration of the flux of a vector field $\v{a}$ through an infinitesimal volume $\diff V$ depending on the sign of its divergence.}
  \label{fig:net_flux}
\end{figure}
%
\subsection{The divergence theorem (Gauss' theorem)}

Let $\v{a}$ be a vector field in a compact volume $V$, and let $S$ be the closed surface
bounding $V$, then
\begin{center}
  \bigbox{$
      \intV \Div \v{a} \; \diff V
      \,=\,
      \intS \v{a}\cdot \diff \v{S} \label{eq:div-th}
    $}
\end{center}

\paragraph{Proof:}
We derive the divergence theorem using the integral definition of $\v{\nabla}\cdot\v{a}$
\[
  \v{\nabla} \cdot \v{a}
  \,=\,
  \lim_{\delta V\to0}\; \frac{1}{\delta V} \,  \int_{\delta S} \,
  \v{a} \cdot \diff \v{S}.
\]
Since this \emph{definition} of $\v{\nabla}\cdot\v{a}$ is valid for volumes of arbitrary
shape, we can build a smooth surface $S$ from a large number, $N$, of small blocks of
volume $\delta V_{i}$ and area $\delta S_{i}$. For small, but not infinitesimal, $\delta
  S_{i}$ we may write
\[
  \v{\nabla}\cdot \v{a}\:\!(\vr_{i})
  \,=\,
  \frac{1}{\delta V_{i}}\, \int_{\delta S_{i}} \, \v{a}\cdot \diff \v{S}
  + \epsilon_{i}
\]
where $\epsilon_{i}\to0$ as $\delta V_{i}\to0$. Now multiply both sides by $\delta V_{i}$
and sum over all $i$
\begin{equation}
  \sum_{i=1}^{N}\; \v{\nabla}\cdot\v{a}\:\!(\vr_{i}) \; \delta V_{i}
  \,=\,
  \sum_{i=1}^{N}\; \int_{\delta S_{i}} \v{a}\cdot \diff \v{S}
  \,+\,
  \sum_{i=1}^{N}\; \epsilon_{i}\, \delta V_{i}
  \label{eq:sumoverblocks}
\end{equation}
On the RHS the contributions from surface elements which are \emph{interior} to
$S$ \emph{cancel}. This is because where two blocks touch, the outward
normals are in \emph{opposite} directions, implying that the
contributions to the respective integrals cancel.

To illustrate this, consider two adjacent blocks, $1$~and~$2$. The volume of block 1 is
$\delta V_1$ and its surface is $\delta S_1$. Similarly, for block~2. Denote the shaded
surface common to the two blocks by $\delta{\cal S}$, then
\begin{eqnarray*}
  &&
  \phantom{-} \int_{\delta{\cal S}} \v{a} \cdot \diff \v{S}
  \quad \mbox{regarded as part of block 1}		\\[1ex]
  &=&
  - \int_{\delta{\cal S}} \v{a} \cdot \diff \v{S}
  \quad \mbox{regarded as part of block 2}
\end{eqnarray*}
because their outward normals $\diff \v{S}_1$ and $\diff \v{S}_2$ are equal and opposite.
Therefore,
\begin{eqnarray*}
  \int_{\delta S_1} \v{a} \cdot \diff \v{S}
  \,+\,
  \int_{\delta S_2} \v{a} \cdot \diff \v{S}
  \,=\,
  \int_{\delta S_{1+2}} \v{a} \cdot \diff \v{S} \quad
  \mbox{(because contributions from $\delta{\cal S}$ cancel)}
\end{eqnarray*}
where we have denoted the \emph{exterior} surface of the
\emph{compound block} by $\delta S_{1+2}$.
\begin{figure}[t]
  \centering
  \includegraphics{tikz_two_cube_div.pdf}
  \caption{Illustration of normal cancellation between two adjacent elementary volumes in
    the proof of the divergence theorem.}
  \label{fig:two_cube_div}
\end{figure}
Thus, the contributions from all \emph{interior} surface elements cancel \emph{pairwise}.
Using this result, and letting $N\to\infty$ in equation~(\ref{eq:sumoverblocks}), the
$O(\epsilon^{(i)} \, \delta V^{(i)})$ terms go to zero, and we get
\[
  \intV \Div \v{a} \: \diff V
  \,=\,
  \intS \v{a}\cdot \diff \v{S}\; .
\]
For an alternative proof of the divergence theorem, see \emph{Bourne \& Kendall}, Chapter
6.\\ \emph{Note:} The divergence theorem is the multidimensional generalisation of
\[
  \int^b_a\; f'(x) \, \diff x
  \,=\,
  f(b) - f(a)\,.
\]
%
\subsection{Examples of the use of the divergence theorem}
\paragraph{Volume of a body:}
This is simply given by \( \ds \; V = \int_V \diff V \) Recalling that $\Div \v{r} =3$ we
can write
\[
  V
  \,=\,
  \frac{1}{3} \int_V \Div \v{r}\; \diff V
  \,=\,
  \frac{1}{3} \int_S \v{r}\cdot  \diff \v{S}
\]
where we used the divergence theorem in the last step.

\paragraph{Example:}
Consider the hemisphere $x^2+y^2+z^2 \leq R^2$ centred on $\ethree$ with its bottom face
at $z=0$. Recalling that the divergence theorem holds for a \emph{closed} surface, the
volume of the hemisphere is
\[
  V
  \,=\,
  \frac{1}{3}
  \left[
    \int_{S_C} \vr \cdot  \diff \v{S}
    +
    \int_{S_B} \vr \cdot  \diff \v{S}
    \right]
\]
where $S_C$ is the curved surface of the hemisphere and $S_B$ is its bottom. On $S_B$, we
have $\diff \v{S} = -\ez \,{\rm d}S$ and $z=0$, so $\vr\cdot \diff \v{S} = -z\, dS =0$.
Therefore, the only contribution comes from the (open) curved surface $S_C$,
\[
  V
  \,=\,
  \frac{1}{3} \int_{S_C} \vr \cdot \diff \v{S}
\]
We can evaluate this surface integral using spherical polars. For a hemisphere of radius
$R$ we showed previously that \( \: \diff \v{S} \,=\, R^2\, \sin(\theta) \, \diff \theta
\, \diff \phi\, \er \: . \) On the hemisphere, $\vr \cdot \diff \v{S} = R \: \er \cdot
  \diff \v{S} = R^3\, \sin(\theta)\, \diff \theta\, \diff \phi\,,$ therefore
\[
  \int_S \vr\cdot  \diff \v{S}
  \,=\,
  R^3 \int_{0}^{\pi/2}\sin(\theta)\,
  \diff \theta \int_{0}^{2\pi} \diff \phi
  \,=\,
  2\pi R^3
\]
which gives the anticipated result
\[
  V
  \,=\,
  \frac{2\pi R^3}{3}\,.
\]

\subsection{The continuity equation}
Consider a fluid with mass density $\rho(\v{r},t)$ and velocity field $\v{v}(\vr,t)$. We
have seen previously that the \emph{volume flux} (volume per unit time) flowing across a
surface $S$ is $\int_S \v{v} \cdot \diff \v{S}$. The corresponding \emph{mass flux} (mass
per unit time) flowing across the surface is
\begin{equation}
  \int_S \left(\rho\vv\,\right) \cdot \diff \v{S}
  \,\equiv\,
  \int_S \v{J}\cdot \diff \v{S}
  \label{eq:mass-flux}
\end{equation}
where $\v{J}(\vr,t) \equiv \rho(\vr,t)\, \v{v}(\vr,t)$ is called the
\emph{mass current density}.
Now consider a volume $V$ bounded by the \emph{closed} surface $S$ containing \emph{no
  sources or sinks} of fluid. Conservation of mass means that the \emph{outward} mass flux
through the surface $S$ must be equal to the rate of \emph{decrease} of mass contained in
the volume $V$,
\begin{equation}
  \int_S \v{J}\cdot \diff \v{S}
  =
  -\parpar{M}{t} \;.
  \label{eq:mass-decrease}
\end{equation}
where $M$ is the total mass in $V$, which may be written as $M = \int_V \rho\, \diff V$.
Substituting this into equation~(\ref{eq:mass-decrease}), we get
\[
  \frac{\partial}{\partial t} \int_V \rho\, \diff V +\int_S \v{J}\cdot
  \diff \v{S}
  =
  0\;.
\]
Using the divergence theorem to rewrite the second term as a volume integral, we obtain
\[
  \int_V \left[ \parpar{\rho}{t} + \Div \v{J} \right]\, \diff V
  \,=\,
  0
\]
Since this holds for arbitrary volumes $V$, we must have that
\begin{center}
  \bigbox{$\ds \parpar{\rho}{t} + \Div \v{J}= 0$}
\end{center}
This is known as the \emph{continuity equation}.

%\begin{center}
%\bigbox{
%\parbox{30mm}{  
%\[
%  \parpar{\rho}{t} + \Div \v{J}= 0
%\]
%}}
%\end{center}

The continuity equation appears in many contexts because it holds for \emph{any conserved
  quantity}. Here we considered mass density $\rho$ and mass current density $\v{J}$ of a
fluid; but equally it could have been number density of molecules in a gas and current
density of molecules; electric charge density and electric-current density; thermal
energy density and heat-current density; or more abstract quantities such as probability
density and probability-current density in quantum mechanics (tutorial). In case of fluid
flow, the continuity equation tells us that
\begin{eqnarray*}
  \mbox{if} \quad \Div \v{J} \,>\, 0 \quad \mbox{then} \quad
  \parpar{\rho}{t} <0 \quad
  \mbox{and the mass density at $\vr$ \emph{decreases}}   \\[1ex]
  %
  \mbox{if}  \quad \Div \v{J} \,<\, 0\quad \mbox{then} \quad
  \parpar{\rho}{t} >0 \quad
  \mbox{and the mass density at $\v{r}$ \emph{increases}}
\end{eqnarray*}

\paragraph{Static case:} Now consider \emph{time-independent} behaviour where the mass density is constant in
time: $\partial\rho/\partial t=0$. The continuity equation tells us that for the mass
density to be constant in time when the total mass is conserved, we must have $\, \Div
  \v{J} =0$ so that the flux into a point equals the flux out. Further, if $\rho(\v{r},t)$
is a constant (\ie time-independent and independent of $\v{r}$), then
\[
  \Div \v{J}
  \,=\,
  \rho \: \Div \v{v}
  \,=\,
  0
  \qquad
  \Rightarrow
  \qquad
  \Div \v{v} = 0
\]
Fluid flows with $\rho = $ \emph{constant}, and hence $\, \Div \v{v} = 0$, are said to be
\emph{incompressible flows}.

\subsection{Sources and sinks}
\label{sec:sourcesandsinks}

\paragraph{Sources and sinks:} If we have a \emph{source} or a \emph{sink} of a vector field, the divergence is
\emph{non-zero} at that point. The quantity
\[
  \frac{1}{V}\, \intS \v{a}\cdot \diff \v{S}
\]
tells us whether there are sources or sinks of the vector field $\v{a}$ within $V$. If
$V$ contains
\begin{itemize}
  \item a net \emph{source}, then $\intS \v{a}\cdot \diff \v{S} \,=\, \int_V \Div \v{a}\; \diff V
          ~>~ 0$;

  \item a net \emph{sink}, then $\intS \v{a}\cdot \diff \v{S} \,=\, \int_V \Div \v{a}\; \diff V
          ~<~ 0$.
\end{itemize}
When $S$ contains neither sources nor sinks, then
$\intS \v{a}\cdot \diff \v{S} = 0$.
Therefore,
\[
  \Div \v{a}
  \,=\,
  \lim_{V\to0}\; \frac{1}{V}\, \intS \v{a} \cdot \diff \v{S}
\]
is a measure of the \emph{density} of sources or sinks,
\[
  \Div\v{a}
  \,=\,
  \mbox{net \emph{outward} flux per unit volume at $\vr$.}
\]
These ideas generalise to \emph{electric} and \emph{magnetic} fields.
%
\subsection{Corollaries of the divergence theorem}
We may deduce several immediate consequences of the divergence theorem
\[
  \int_V \v{\nabla}\cdot\v{a}\: \diff V
  \,=\,
  \int_S \v{a}\cdot \diff \v{S}
\]

\begin{enumerate}

  \item
        Let $\v{a}=\v{c}$ where $\v{c}$ is an arbitrary \emph{constant}
        vector, then $\Div\v{c}=0$, and hence
        \[
          \int_S\v{c}\cdot \diff \v{S}
          =
          \v{c}\cdot \int_S\diff \v{S}
          =
          0
        \]
        Since this holds for \emph{all} constant vectors~$\v{c}$, we must have
        \[
          \int_S \diff \v{S} = 0
        \]
        for \emph{any closed surface} $S$ (as claimed previously).

  \item Let $\v{a}(\v{r})=f(\v{r}) \, \v{c}$ where $f(\v{r})$ is a scalar field and $\v{c}$ is a
        constant vector, then
        \[
          \int_V \left(\grad f\right) \diff V
          \,=\,
          \int_S f  \, \diff \v{S}\,,
        \]
        \cf exercises.

  \item Consider the vector field $\v{a}\times\v{c}$, where $\v{a}(\v{r})$ is an arbitrary vector
        field and $\v{c}$ is a \emph{constant} vector. We have, using a standard identity,
        \[
          \Div (\v{a}\times\v{c})
          \,=\,
          \v{c} \cdot (\curl \v{a})
          -
          \v{a} \cdot (\curl \v{c})
          \,=\,
          \v{c} \cdot (\curl\v{a}) - 0
        \]
        Now apply the divergence theorem to $\v{a}\times\v{c}\,$
        \[
          \v{c} \, \cdot \int_V (\v{\nabla}\times\v{a})\, \diff V
          \,=\,
          \int_V \v{\nabla}\cdot(\v{a}\times\v{c})\,\diff V
          \,=\,
          \int_S  \diff \v{S} \cdot (\v{a}\times\v{c})
          \,=\,
          \v{c} \, \cdot \int_S \diff \v{S}\times\v{a}
        \]
        This holds for all constant vectors $\v{c}$, hence
        \[
          \int_V \v{\nabla}\times\v{a}\: \diff V
          \,=\,
          \int_S \diff \v{S}\times\v{a}
        \]

  \item In suffix notation, the divergence theorem becomes
        \[
          \int_V \partial_i\, a_i \,\dd V \,=\, \int_S a_i\, \dd S_i
        \]
\end{enumerate}

