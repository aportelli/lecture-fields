% !TEX root = ../../fields.tex
%% Brian Lecture 16 %%%%%%%%%%%%%%%%%%%%%%%%%%%%%%%%%%%%%%%%%%%%%%%%%%%%%%%%%%%%%%%%%%%%%%%

\subsection{Integral definition of divergence}

Let $\v{a}$ be a vector field in the region $R$, and let $P$ be a point in $R$, then the
divergence of $\v{a}$ at $P$ may be \emph{defined} by
\begin{center}
  \bigbox{
    \parbox{50mm}{
      \[
        \rm{div}\, \v{a}
        \,=\,
        \lim_{\delta V\to0}\; \frac{1}{\delta V}\,
        \int_{\delta S} \v{a}\,\cdot\, \mbox{d}\v{S}
      \]}
  }
\end{center}
where $\delta S$ is a \emph{closed} surface in $R$ which encloses the
volume $\delta V$.  The limit must be taken so that the point $P$ is
within $\delta V$. It can be shown that the limit is independent of
the \emph{shape} of $\delta V$.

This definition of $\rm{div}\,\v{a}$ is also \emph{basis independent}.
%We shall not give a proof of this result.

We now show that our original definition of $\Div \v{a}$ is recovered in Cartesian
co-ordinates.

\parbox{0.6\textwidth}{Let $P$ be a point with Cartesian coordinates
  $(x_0, \, y_0, \, z_0)$ situated at the \emph{centre} of a
  \emph{small} rectangular block of size $\delta_x \times \delta_y
    \times \delta_z$, with volume $\delta V= \delta_x\, \delta_y\,
    \delta_z$.
  \begin{itemize}
    \item On the \emph{front} face of the block, parallel to the $y{-}z$ plane at
          $x=x_0+\delta_x/2$, we have \emph{outward} normal $\vn=\v{e}_{\,x}$ and so $
            \mbox{d}\v{S} = \v{e}_{\,x}\, \mbox{d}y\, \mbox{d}z$

    \item On the \emph{back} face of the block, parallel to the $y{-}z$ plane at
          $x=x_0-\delta_x/2$, we have \emph{outward} normal $\vn=-\v{e}_{\,x}$ and so
          $\mbox{d}\v{S} = -\v{e}_{\,x}\, \mbox{d}y\, \mbox{d}z$
  \end{itemize}
}
%
\hfill
%
\parbox{0.37\textwidth}{
  \epsfxsize=0.37\textwidth
  \includegraphics{tikz_cube_div.pdf}
}

Hence $\v{a}\cdot \mbox{d}\v{S} = \pm \, a_x\, \mbox{d}y\,\mbox{d}z$ on these two faces.
Let us denote the union of the two faces orthogonal to the $x$ axis by $\delta S_x$.

The contribution of these two surfaces to the integral $\int_{\delta S}\v{a}\cdot
  \mbox{d}\v{S}$ is given by
\begin{eqnarray*}
  \int_{\delta S_x}\v{a}\cdot \mbox{d}\v{S}
  & = &
  \int_z\!\int_y
  \bigg\{
  a_x(x_0+\delta_x/2,y,z) - a_x(x_0-\delta_x/2,y,z)
  \bigg\}
  \: \mbox{d}y \, \mbox{d}z \\[1.5ex]
  %
  & = &
  \int_z \int_y
  \left\{\:
  \left[
    a_x(x_0,y,z) \,+\, \frac{\delta_x}{2}\;\parpar{a_x}{x}\bigg|_{(x_0,y,z)}
    + \; O(\delta_x^2)
    \right]
  \right. \\[1ex]
  &   &
  \hspace*{9ex}
  \left. - \;
  \left[
    a_x(x_0,y,z) \,-\, \frac{\delta_x}{2}\: \parpar{a_x}{x}\bigg|_{(x_0,y,z)}
    + \; O(\delta_x^2)\,
    \right]
  \: \right\}
  \; \mbox{d}y \, \mbox{d}z \\[1.5ex]
  & = &
  \int_z \int_y \,\delta_x \, \parpar{a_x}{x}\bigg|_{(x_0,y,z)}
  \mbox{d}y \, \mbox{d}z
\end{eqnarray*}
where we Taylor-expanded $a_x(x,y,z)$ about the point $(x_0,y,z)$, and
dropped the $O(\delta_x^2)$ terms which will vanish when we divide by
$\delta V$ at the end.

So
\[
  \frac{1}{\delta V}\; \int_{\delta S_x} \v{a}\cdot \mbox{d}\v{S}
  \,=\,
  \frac{1}{\delta_y\, \delta_z}
  \int_z\int_y \; \parpar{a_x}{x}\bigg|_{(x_0,y,z)} \: \mbox{d}y \, \mbox{d}z
\]
As we take the limit $\delta_x,\delta_y,\delta_z \to 0$, the integrand tends to a
constant in $\delta V$, so the \\[1ex] integral over $y$
and $z$ tends to $\delta_y \:\! \delta_z
  \ds\left. \parpar{a_x}{x}\right|_{(x_0,y_0,z_0)}\,$ and we obtain
\[
  \lim_{\delta V\to0} \;
  \frac{1}{\delta V}\; \int_{\delta S_x} \v{a}\cdot \mbox{d}\v{S}
  \,=\,
  \parpar{a_x}{x}\bigg|_{(x_0,y_0,z_0)}
\]
Adding similar contributions from the other $4$ faces, we find
\[
  {\rm div}\,\v{a}
  \,=\,
  \parpar{a_x}{x} + \parpar{a_y}{y} + \parpar{a_z}{z}
  \,=\,
  \Div\v{a}
\]
in agreement with our original definition in Cartesian co-ordinates. Thus we can continue
to use the notation $\v{\nabla}\cdot\v{a}$ for \emph{both} forms for the divergence of a
vector field $\v{a}(\v{r})$.

The integral definition
\[
  \rm{div}\, \v{a}
  \,=\,
  \Div \v{a}
  \,=\,
  \lim_{\delta V\to0}\; \frac{1}{\delta V}\,
  \int_{\delta S} \v{a}\,\cdot\, \mbox{d}\v{S}
\]
provides a precise intuitive understanding of divergence as the (net) flux per unit
volume leaving a small volume around a point $\vr$. In pictures, for an infinitesimal
volume $\mbox{d}V$,
\begin{center}
  \epsfxsize=0.9\textwidth
  \includegraphics{tikz_net_flux.pdf}
\end{center}
\vspace*{-4ex}

\mnote{18L 07/03/08}

%\newpage

\subsection{The divergence theorem (Gauss' theorem)}

Let $\v{a}$ be a vector field in a volume $V$, and let $S$ be the closed surface bounding
$V$, then
\begin{center}
  \bigbox{$
      \intV \Div \v{a} \; \mbox{d}V
      \,=\,
      \intS \v{a}\cdot \mbox{d}\v{S} \label{eq:div-th}
    $}
\end{center}

\paragraph{Proof:}
We derive the divergence theorem using the integral definition of $\v{\nabla}\cdot\v{a}$
\[
  \v{\nabla} \cdot \v{a}
  \,=\,
  \lim_{\delta V\to0}\; \frac{1}{\delta V} \,  \int_{\delta S} \,
  \v{a} \cdot \mbox{d}\v{S}.
\]
Since this \emph{definition} of $\v{\nabla}\cdot\v{a}$ is valid for volumes of arbitrary
shape, we can build a smooth surface $S$ from a large number, $N$, of small blocks of
volume $\delta V^{(i)}$ and area $\delta S^{(i)}$. For small, but not infinitesimal,
$\delta S^{(i)}$ we may write
\[
  \v{\nabla}\cdot \v{a}\:\!(\vr^{(i)})
  \,=\,
  \frac{1}{\delta V^{(i)}}\, \int_{\delta S^{(i)}} \, \v{a}\cdot \mbox{d}\v{S}
  + \epsilon^{(i)}
\]
where $\epsilon^{(i)}\to0$ as $\delta V^{(i)}\to0$. Now multiply both sides by $\delta
  V^{(i)}$ and sum over all $i$
\begin{equation}
  \sum_{i=1}^{N}\; \v{\nabla}\cdot\v{a}\:\!(\vr^{(i)}) \; \delta V^{(i)}
  \,=\,
  \sum_{i=1}^{N}\; \int_{\delta S^{(i)}} \v{a}\cdot \mbox{d}\v{S}
  \,+\,
  \sum_{i=1}^{N}\; \epsilon^{(i)}\, \delta V^{(i)}
  \label{eq:sumoverblocks}
\end{equation}
On the RHS the contributions from surface elements \emph{interior} to
$S$ \emph{cancel}. This is because where two blocks touch, the outward
normals are in \emph{opposite} directions, implying that the
contributions to the respective integrals cancel.

\parbox{0.61\textwidth}{To illustrate this, consider two
  adjacent blocks, $1$~and~$2$.

  \medskip

  The volume of block 1 is $\delta V_1$ and its surface is $\delta S_1$. Similarly for
  block~2. Denote the shaded surface common to the two blocks by $\delta{\cal S}$, then
  \begin{eqnarray*}
    &&
    \phantom{-} \int_{\delta{\cal S}} \v{a} \cdot \mbox{d}\v{S}
    \quad \mbox{regarded as part of block 1}		\\[1ex]
    &=&
    - \int_{\delta{\cal S}} \v{a} \cdot \mbox{d}\v{S}
    \quad \mbox{regarded as part of block 2}
  \end{eqnarray*}

} \hfill
\parbox{0.36\textwidth}{ \epsfxsize=0.33\textwidth
  \includegraphics{tikz_two_cube_div.pdf} }
%
because their outward normals $\mbox{d}\v{S}_1$ and $\mbox{d}\v{S}_2$ are equal and
opposite. Therefore
\begin{eqnarray*}
  \int_{\delta S_1} \v{a} \cdot \mbox{d}\v{S}
  \,+\,
  \int_{\delta S_2} \v{a} \cdot \mbox{d}\v{S}
  \,=\,
  \int_{\delta S_{1+2}} \v{a} \cdot \mbox{d}\v{S} \quad
  \mbox{(because contributions from $\delta{\cal S}$ cancel)}
\end{eqnarray*}
where we have denoted the \emph{exterior} surface of the
\emph{compound block} by $\delta S_{1+2}$.

%\begin{eqnarray*}
%  \int_{\delta V_{1+2}} \Div\v{a} \;\mbox{d}V
%  & = &
%  \int_{\delta V_1} \Div\v{a}  \;\mbox{d}V \plus 
%  \int_{\delta V_2} \Div\v{a}  \;\mbox{d}V %\\[1ex]
%  ~ = ~
%  & = &
%  \int_{\delta S_1} \v{a} \cdot \mbox{d}\v{S} \plus
%  \int_{\delta S_2} \v{a} \cdot \mbox{d}\v{S}	\\[1ex]
%  & = &
%  \int_{\delta S_{1+2}} \v{a} \cdot \mbox{d}\v{S} \quad 
%  \mbox{(because contributions from $\delta{\cal S}$ cancel)}
%\end{eqnarray*}
%where we have denoted the \emph{exterior} surface of the
%\emph{compound block} by $\delta S_{1+2}$, and its volume by $\delta
%V_{1+2} = \delta V_1+\delta V_2$.

Thus the contributions from all \emph{interior} surface elements cancel \emph{pairwise}.
Using this result, and letting $N\to\infty$ in equation~(\ref{eq:sumoverblocks}), the
$O(\epsilon^{(i)} \, \delta V^{(i)})$ terms go to zero, and we get
\[
  \intV \Div \v{a} \: \mbox{d}V
  \,=\,
  \intS \v{a}\cdot \mbox{d}\v{S}\; .
\]
For an alternative proof of the divergence theorem, see \emph{Bourne \& Kendall}, Chapter
6.

\emph{Note:} The divergence theorem is the generalisation to 3-D of
\[
  \int^b_a\; \frac{\mbox{d}f(x)}{\mbox{d}x} \, \mbox{d}x
  \,=\,
  f(b) - f(a)
\]

%\newpage

\subsection{Examples of the use of the divergence theorem}

\paragraph{Volume of a body:} This is simply given by \( \ds \; V = \int_V \mbox{d}V \)

Recalling that $\Div \v{r} =3$ we can write
\[
  V
  \,=\,
  \frac{1}{3} \int_V \Div \v{r}\; \mbox{d}V
  \,=\,
  \frac{1}{3} \int_S \v{r}\cdot  \mbox{d}\v{S}
\]
where we used the divergence theorem in the last step.

\paragraph{Example:} Consider the hemisphere $x^2+y^2+z^2 \leq R^2$ centered on $\ethree$ with its bottom face
at $z=0$. Recalling that the divergence theorem holds for a \emph{closed} surface, the
volume of the hemisphere is
\[
  V
  \,=\,
  \frac{1}{3}
  \left[
    \int_{S_C} \vr \cdot  \mbox{d}\v{S}
    +
    \int_{S_B} \vr \cdot  \mbox{d}\v{S}
    \right]
\]
where $S_C$ is the curved surface of the hemisphere and $S_B$ is its bottom. On $S_B$, we
have $\mbox{d}\v{S} = -\ez \,{\rm d}S$ and $z=0$, so $\vr\cdot \mbox{d}\v{S} = -z\, dS
  =0$. Therefore the only contribution comes from the (open) curved surface $S_C$,
\[
  V
  \,=\,
  \frac{1}{3} \int_{S_C} \vr \cdot \mbox{d}\v{S}
\]
We can evaluate this surface integral using spherical polars. For a hemisphere of radius
$R$ we showed previously that \( \: \mbox{d}\v{S} \,=\, R^2\, \sin\theta \,
\mbox{d}\theta \, \mbox{d}\phi\, \er \: . \)

On the hemisphere, $\vr \cdot \mbox{d}\v{S} = R \: \er \cdot \mbox{d}\v{S} = R^3\,
  \sin\theta\, \mbox{d}\theta\, \mbox{d}\phi\,,$ therefore
\[
  \int_S \vr\cdot  \mbox{d}\v{S}
  \,=\,
  R^3 \int_{0}^{\pi/2}\sin \theta\,
  \mbox{d}\theta \int_{0}^{2\pi} \mbox{d}\phi
  \,=\,
  2\pi R^3
\]
which gives the anticipated result
\[
  V
  \,=\,
  \frac{2\pi R^3}{3}
\]

\subsection{The continuity equation}
Consider a fluid with mass density $\rho(\v{r},t)$ and velocity field $\v{v}(\vr,t)$. We
have seen previously that the \emph{volume flux} (volume per unit time) flowing across a
surface $S$ is $\int_S \v{v} \cdot \mbox{d}\v{S}$. The corresponding \emph{mass flux}
(mass per unit time) flowing across the surface is
\begin{equation}
  \int_S \left(\rho\vv\,\right) \cdot \mbox{d}\v{S}
  \,\equiv\,
  \int_S \v{J}\cdot \mbox{d}\v{S}
  \label{eq:mass-flux}
\end{equation}
where $\v{J}(\vr,t) \equiv \rho(\vr,t)\, \v{v}(\vr,t)$ is called the
\emph{mass current density}.

Now consider a volume $V$ bounded by the \emph{closed} surface $S$ containing \emph{no
  sources or sinks} of fluid. Conservation of mass means that the \emph{outward} mass flux
through the surface $S$ must be equal to the rate of \emph{decrease} of mass contained in
the volume $V$,
\begin{equation}
  \int_S \v{J}\cdot \mbox{d}\v{S}
  =
  -\parpar{M}{t} \;.
  \label{eq:mass-decrease}
\end{equation}

where $M$ is the total mass in $V$, which may be written as $M = \int_V \rho\,
  \mbox{d}V$. Substituting this into equation~(\ref{eq:mass-decrease}), we get
\[
  \frac{\partial}{\partial t} \int_V \rho\, \mbox{d}V +\int_S \v{J}\cdot
  \mbox{d}\v{S}
  =
  0\;.
\]
Using the divergence theorem to rewrite the second term as a volume integral, we obtain
\[
  \int_V \left[ \parpar{\rho}{t} + \Div \v{J} \right]\, \mbox{d}V
  \,=\,
  0
\]
Since this holds for arbitrary volumes $V$, we must have that
\begin{center}
  \bigbox{$\ds \parpar{\rho}{t} + \Div \v{J}= 0$}
\end{center}
This is known as the \emph{continuity equation}.

%\begin{center}
%\bigbox{
%\parbox{30mm}{  
%\[
%  \parpar{\rho}{t} + \Div \v{J}= 0
%\]
%}}
%\end{center}

The continuity equation appears in many different contexts because it holds for \emph{any
  conserved quantity}. Here we considered mass density $\rho$ and mass current density
$\v{J}$ of a fluid; but equally it could have been number density of molecules in a gas
and current density of molecules; electric charge density and electric-current density;
thermal energy density and heat-current density; or more abstract quantities such as
probability density and probability-current density in quantum mechanics (tutorial).

In case of fluid flow, the continuity equation tells us that
\begin{eqnarray*}
  \mbox{if} \quad \Div \v{J} \,>\, 0 \quad \mbox{then} \quad
  \parpar{\rho}{t} <0 \quad
  \mbox{and the mass density at $\vr$ \emph{decreases}}   \\[1ex]
  %
  \mbox{if}  \quad \Div \v{J} \,<\, 0\quad \mbox{then} \quad
  \parpar{\rho}{t} >0 \quad
  \mbox{and the mass density at $\v{r}$ \emph{increases}}
\end{eqnarray*}

\paragraph{Static case:} Now consider \emph{time independent} behaviour where the mass density is
time-independent: $\partial\rho/\partial t=0$. The continuity equation tells us that for
the mass density to be constant in time when the total mass is conserved, we must have
$\, \Div \v{J} =0$ so that the flux into a point equals the flux out.

Further, if $\rho(\v{r},t)$ is a constant (\emph{i.e.}~time-independent and independent
of $\v{r}$), then
\[
  \Div \v{J}
  \,=\,
  \rho \: \Div \v{v}
  \,=\,
  0
  \qquad
  \Rightarrow
  \qquad
  \Div \v{v} = 0
\]
Fluid flows with $\rho = $ \emph{constant}, and hence $\, \Div \v{v} = 0$, are said to be
\emph{incompressible flows}.

\subsection{Sources and sinks}
\label{sec:sourcesandsinks}

\paragraph{Sources and sinks:} If we have a \emph{source} or a \emph{sink} of a vector field, the divergence is
\emph{non-zero} at that point. The quantity
\[
  \frac{1}{V}\, \intS \v{a}\cdot \mbox{d}\v{S}
\]
tells us whether there are sources or sinks of the vector field $\v{a}$ within $V$. If
$V$ contains

\begin{itemize}
  \item a net \emph{source}, then $\intS \v{a}\cdot \mbox{d}\v{S} \,=\, \int_V \Div \v{a}\;
          \mbox{d}V ~>~ 0$

  \item a net \emph{sink}, then $\hspace{2ex} \intS \v{a}\cdot \mbox{d}\v{S} \,=\, \int_V \Div
          \v{a}\; \mbox{d}V ~<~ 0$
\end{itemize}

\vspace*{-2ex}

When $S$ contains neither sources nor sinks, or sources and sinks in equal measure, then\\[1.5ex]
$\intS \v{a}\cdot \mbox{d}\v{S} = 0$.

\medskip

Therefore
\[
  \Div \v{a}
  \,=\,
  \lim_{V\to0}\; \frac{1}{V}\, \intS \v{a} \cdot \mbox{d}\v{S}
\]
is a measure of the \emph{density} of sources or sinks,
\[
  \Div\v{a}
  \,=\,
  \mbox{net \emph{outward} flux per unit volume at $\vr$.}
\]
These ideas generalise to \emph{electric} and \emph{magnetic} fields.

%\newpage

%\subsection{Electrostatics - Gauss' law and Maxwell's first equation}
%\label{sec:gausslaw}
%
%As an example, we return to electrostatics.
%
%%You will have learned that electric field lines can only start and
%%stop at charges.
%
%The electric field at $\v{r}$ due to a point charge $q$ at the \emph{origin}
%is
%\[
%  \v{E}(\v{r}) \,=\, \frac{q}{4\pi \epsilon_0} \, \frac{\v{r}}{r^3}
%\]
%Then, for $r \ne 0$,
%\[
%  \v{\nabla}\cdot\v{E}
%  \,=\,
%  \frac{q}{4\pi\epsilon_0} \;\Div \left( \frac{\v{r}}{r^3}\right)
%  \,=\,
%  \frac{q}{4\pi\epsilon_0}
%  \left(\,
%    \grad\left(\frac{1}{r^3}\right) \cdot \v{r}
%    +
%    \frac{\Div \v{r}}{r^3} 
%  \,\right)
%  \,=\,
%  \frac{q}{4\pi\epsilon_0}
%  \left(
%    -\frac{3\v{r}}{r^5} \,\cdot \v{r} +  \frac{3}{r^3} 
%  \right)
%  \,=\,
%  0 	
%\]
%%\begin{eqnarray}
%%   \v{\nabla}\cdot\v{E}
%%                    = {q \over 4\pi\epsilon}
%%                     \partial_i \left(x_i \over r^3\right)
%%                   = {q \over 4\pi\epsilon}
%%                     \left( {\delta_{ii}\over r^3} 
%%                            - {3\over 2}r^{-5}\,2x_ix_i \right)
%%                   = 0 \qquad r \not= 0
%%\end{eqnarray}
%
%In section~(\ref{flux_concept}), we showed that
%\begin{equation}
%  \int_{\rm sphere} \v{E}\cdot \mbox{d}\v{S}
%  \,=\,
%  \frac{q}{4\pi\epsilon_0} \, \int_{\rm sphere}
%     \frac{\v{r}\cdot \mbox{d}\v{S}}{r^3}
%  \,=\,
%  \frac{q}{4\pi\epsilon_0} \, 4\pi 
%  \,=\, \frac{q}{\epsilon_0}
%  \label{eq:gauss-sphere}
%\end{equation}
%where the integral is over the surface of a sphere centred on the
%origin. The key result was $\int_{\rm sphere} (\v{r}\cdot
%\mbox{d}\v{S})/r^3=4\pi$, independent of the radius of the sphere.
%
%\parbox{0.53\textwidth}{Now consider an \emph{arbitrary} closed
%  surface $S$ which encloses the charge at the origin. Define the
%  volume $V$ to be the region \emph{between} the surfaces $S$ and
%  $S_1$, where $S_1$ is a small sphere, radius $\delta$, centred on
%  the origin. The volume $V$ is then bounded by the closed surface
%  $S{+}S_1$.
%
%\medskip
%
%(Ignore the spheres $S_2$ and $S_3$ in the figure for now.)}
%\hspace*{0.05\textwidth}
%\parbox{0.42\textwidth}{
%  \epsfxsize=0.42\textwidth
%  \includegraphics{tikz_gauss-law.pdf}
%}
%
%\medskip
%
%Since the volume $V$ does \emph{not} contain the origin, $\v{r}=0$,
%then $\Div \v{E}=0\,$ \emph{everywhere} in $V$, and the divergence
%theorem tells us that
%\begin{equation}
%   \int_{S+S_1} \, \v{E}\cdot \mbox{d}\v{S}
%   \,=\,
%   \int_S \, \v{E}\cdot \mbox{d}\v{S}
%   \,+\,
%   \int_{S_1} \, \v{E}\cdot \mbox{d}\v{S}
%   \,=\,
%   \int_V \, \v{\nabla}\cdot\v{E} \; \mbox{d}V 
%   \,=\,
%   0
%  \label{eq:gauss-general}
%\end{equation}
%Since the \emph{outward normal} from $V$ on the sphere $S_1$ points
%\emph{towards} the origin, equation~(\ref{eq:gauss-sphere}) gives
%\[
%   \int_{\rm S_1} \, \v{E}\cdot \mbox{d}\v{S}
%   \,=\,
%   -\frac{q}{\epsilon_0}
%\]
%independent of $\delta$, and we may safely take the limit
%$\delta\to0$. Equation~(\ref{eq:gauss-general}) then becomes
%\begin{equation}
%   \int_S \, \v{E}\cdot \mbox{d}\v{S}
%   \,=\,
%   \frac{q}{\epsilon_0}
%\end{equation}
%This holds for \emph{any} closed surface $S$ which encloses the
%charge at the origin.
%
%If, instead of charge $q$ at the origin, we have charge $q_i$ at
%position $\v{r}_{\, i}\,$ inside $S$, we can change integration
%variable from $\v{r}$ to $\v{\rho} = \vr - \v{r}_{\, i}$ when
%integrating over the sphere $S_i$ (with outward normal pointing
%\emph{towards} its centre), and we get
%\begin{equation}
%  \int_{S_i} \v{E}_{\,i} \cdot \mbox{d}\v{S}
%  \,=\,
%  \frac{q_i}{4\pi\epsilon_0}
%  \int_{S_i}
%  \frac{(\v{r}-\v{r}_{\,i})\cdot \mbox{d}\v{S}} {|\v{r}-\v{r}_{\,i}|^3}
%  \,=\,
%  \frac{q_i}{4\pi\epsilon_0}
%  \int_{S_i}
%  \frac{\v{\rho}\cdot \mbox{d}\v{S}}{\rho^3}
%  \,=\,
%  -\frac{q_i}{\epsilon_0}
%  \label{eq:gauss-for-qi}
%\end{equation}
%Equations~(\ref{eq:gauss-for-qi}) and (\ref{eq:gauss-general}) then give
%\begin{equation}
%   \int_S \, \v{E}_{\,i}\cdot \mbox{d}\v{S}
%   \,=\,
%   \frac{q_i}{\epsilon_0}
%  \label{eq:gauss-for-qi2}
%\end{equation}
%Now let's replace the single charge by a set of $N$ charges $q_i$ at
%positions $\v{r}_{\,i}\,$.
%% (with $\v{r}_{\,1}=0$, as in the figure). % Not needed!
%Experiment tells us that the total electric field $\v{E}$ is the sum
%of the electric fields $\v{E}_{\,i}$ due to the individual
%charges. Therefore, using equation~(\ref{eq:gauss-for-qi2}), we get
%\[
%  \int_S \, \v{E}\cdot \mbox{d}\v{S}
%  \,=\,
%  \int_S \, \left(\sum_{i=1}^N \v{E}_{\, i}\right) \cdot \mbox{d}\v{S}
%  \,=\,
%  \sum_{i=1}^N \, \int_S \v{E}_{\, i} \cdot \mbox{d}\v{S}
%  \,=\,
%  \sum_{i=1}^N \frac{q_i}{\epsilon_0}
%  \,=\,
%  \frac{Q}{\epsilon_0}
%\]
%where $Q=\sum_{i=1}^N q_i$ is \emph{the total charge} enclosed by
%$S$. This is \emph{Gauss' Law} of electrostatics.
%
%\smallskip
%
%Generalising further, if we have a \emph{charge density} $\rho(\v{r})$
%(charge/unit volume), then the total charge in a volume $V$ is
%\[
%  Q
%  =
% \int_V \, \rho(\v{r})\, \mbox{d}V
%\]
%Applying the divergence theorem and Gauss' Law (respectively), we get
%\[
%  \int_V \, \v{\nabla}\cdot\v{E} \: \mbox{d}V
%  \,=\,
%  \int_S \, \v{E}\cdot \mbox{d}\v{S}
%  \,=\,
%  \frac{Q}{\epsilon_0}
%  \,=\,
%  \frac{1}{\epsilon_0} \int_V \, \rho(\v{r})\, \mbox{d}V 
%\]
%
%%\newpage
%
%Since this holds for arbitrary volumes $V$, we must have
%\begin{center}
%\bigbox{
%  \parbox{40mm}{
%\vspace*{-1.8ex}
%\[
%  \Div \v{E}(\v{r})
%  \,=\,
%  \frac{\rho(\vr) }{\epsilon_0}
%\]
%\vspace*{-2ex}
%}}
%\end{center}
%which holds for all $\v{r}\,$. This is \emph{Maxwell's first equation}
%of electromagnetism.\footnote{We proved it for static electric
%  fields, but it also holds in electrodynamics -- see Junior Honours
%  EM course.}
%
%Evidently, it states that the divergence of the electric field at any
%point $\v{r}$ is equal to the charge density at that point divided by
%(in SI units) the constant $\epsilon_0$.
%
%So a positive charge is a \emph{source} of electric field
%(\emph{i.e.}~it creates a positive flux) and a negative charge is a
%\emph{sink} (\emph{i.e.}~it absorbs flux, or, equivalently, creates a
%negative flux).

%xxxxxxxxxxxxxxxxxxxxxxxxxxxxxxxxxxxxxxxxxxxxxxxxxxxxxxxxxx

%FoMP/MP2h(?) version

%The electric field due to a charge $q$ at the origin is
%\[
%\v{E} = \frac{q}{4\pi \epsilon_0}\frac{\v{\hat{r}}}{r^2}.
%\]
%It is easy to verify that $\Div \v{E}=0$ except at the origin where
%the field is singular.

%The flux integral for this type of field across a sphere (of any
%radius) around the origin was evaluated in the last lecture and we
%find the flux out of the sphere as:
%\[
%\int_S \v{E}\cdot  \mbox{d}\v{S} = \frac{q}{\epsilon_0}
%\]
%Now since $\Div \v{E}=0$ away from the origin the results holds for
%\emph{any surface enclosing the origin}. Moreover if we have several
%charges enclosed by $S$ then
%\[
%\int_S \v{E}\cdot  \mbox{d}\v{S} = \sum_i \frac{q_i}{\epsilon_0}.
%\]

%This recovers \emph{Gauss' Law} of electrostatics.

%We can go further and consider a \emph{charge density} of $\rho(\v{r})$
%per unit volume.  Then
%\[
%\int_S \v{E}\cdot  \mbox{d}\v{S} =  \int_V \frac{\rho (\vr) }{\epsilon_0}\mbox{d}V\;.
%\]
%We can rewrite the lhs using the divergence theorem
%\[
%\int_V \Div \v{E}\, \mbox{d}V = \int_V \frac{\rho(\vr) }{\epsilon_0}\mbox{d}V\;.
%\]

\subsection{Corollaries of the divergence theorem}
We may deduce several immediate consequences of the divergence theorem
\[
  \int_V \v{\nabla}\cdot\v{a}\: \mbox{d}V
  \,=\,
  \int_S \v{a}\cdot \mbox{d}\v{S}
\]

\begin{enumerate}

  \item
        Let $\v{a}=\v{c}$ where $\v{c}$ is an arbitrary \emph{constant}
        vector, then $\Div\v{c}=0$, and hence
        \[
          \int_S\v{c}\cdot \mbox{d}\v{S}
          =
          \v{c}\cdot \int_S\mbox{d}\v{S}
          =
          0
        \]
        Since this holds for \emph{all} constant vectors~$\v{c}$, we must have
        \[
          \int_S \mbox{d}\v{S} = 0
        \]
        for \emph{any closed surface} $S$ (as claimed previously.)

  \item Let $\v{a}(\v{r})=f(\v{r}) \, \v{c}$ where $f(\v{r})$ is a scalar field and $\v{c}$ is a
        constant vector, then (tutorial)
        \[
          \int_V \left(\grad f\right) \mbox{d}V
          \,=\,
          \int_S f  \, \mbox{d}\v{S}
        \]

  \item Consider the vector field $\v{a}\times\v{c}\,$, where $\v{a}(\v{r})$ is an arbitrary
        vector field and $\v{c}$ is a \emph{constant} vector. We have, using a standard identity,
        \[
          \Div (\v{a}\times\v{c})
          \,=\,
          \v{c} \cdot (\curl \v{a})
          -
          \v{a} \cdot (\curl \v{c})
          \,=\,
          \v{c} \cdot (\curl\v{a}) - 0
        \]
        Now apply the divergence theorem to $\v{a}\times\v{c}\,$
        \[
          \v{c} \, \cdot \int_V (\v{\nabla}\times\v{a})\, \mbox{d}V
          \,=\,
          \int_V \v{\nabla}\cdot(\v{a}\times\v{c})\,\mbox{d}V
          \,=\,
          \int_S  \mbox{d}\v{S} \cdot (\v{a}\times\v{c})
          \,=\,
          \v{c} \, \cdot \int_S \mbox{d}\v{S}\times\v{a}
        \]
        This holds for all constant vectors $\v{c}\,$, hence
        \[
          \int_V \v{\nabla}\times\v{a}\: \mbox{d}V
          \,=\,
          \int_S \mbox{d}\v{S}\times\v{a}
        \]

  \item In suffix notation, the divergence theorem becomes
        \[
          \int_V \partial_i\, a_i \,\dd V \,=\, \int_S a_i\, \dd S_i
        \]

        % For a second-rank tensor $T$, we regard one index ($j$ in this case)
        % as a `spectator' index, so
        % \[
        %   \int_V \, \partial_i\, T_{ij} \, \mbox{d}V \,=\, \int_S \, T_{ij}\,dS_i
        % \]
        % This is the \emph{generalised divergence theorem}.  In particular with
        % $T_{ij} = -\epsilon_{ijk}a_k$ we recover the result in (ii) above.
        % 

        %\item \textbf{Green's theorem in the plane}
        % 
        %\medskip
        %
        %\parbox{0.6\textwidth}{Let $V$ be the volume inside the cylinder $0 <
        %  z < 1$, and define the vector field $\v{a}(\v{r})$ as
        %\[
        %  \v{a}
        %  \,=\,
        %  P(x,y) \; \ex + Q(x,y) \; \ey
        %\]
        %Then
        %\begin{eqnarray*}
        %  \int_V \v{\nabla}\cdot\v{a} \, \mbox{d}V
        %  &=&
        %  \int_V \left( \frac{\partial P}{\partial x} +
        %  \frac{\partial Q}{\partial y}
        %  \right) \, \mbox{d}x \, \mbox{d}y \, \mbox{d}z \\[1ex]
        %  &=&
        % \int_{A}
        % \left(
        % \frac{\partial P}{\partial x}
        % +
        % \frac{\partial Q}{\partial y}
        % \right) \, 
        % \mbox{d}x \, \mbox{d}y
        %\end{eqnarray*}
        %}
        %\hfill
        %\parbox{0.35\textwidth}{ \epsfxsize=0.35\textwidth
        %  \includegraphics{tikz_green_cylinder.pdf} } where we performed the (trivial)
        %integral over $z$ to get unity, and defined $A=S_B$ to be the bottom
        %surface of the cylinder in the $x{-}y$ plane.
        %
        %Let $S = S_C + S_T + S_B$ be the closed surface bounding $V$. On the
        %top and bottom surfaces, $S_T$ and $S_B$, we have $\mbox{d}\v{S} = \pm
        %\mbox{d}S \, \ez$ and therefore $\v{a}\cdot\mbox{d}\v{S}=0\,$.  On the
        %curved surface, $S_C$,
        %\[
        %  \mbox{d}\v{S}
        %  \,=\,
        %  \left(
        %    \mbox{d}x \, \ex + \mbox{d}y \, \ey
        %  \right)
        %  \times \ez \, \mbox{d}z
        %  \,=\,
        %  (\mbox{d}y \, \ex - \mbox{d}x \, \ey)  \, \mbox{d}z
        %\]
        %Hence
        %\[
        %  \int_S \v{a}\cdot \mbox{d}\v{S}
        %  \,=\,
        %  \int_{S_C} \v{a}\cdot \mbox{d}\v{S}
        %  \,=\,
        %  \int \!\!\! \int \left( P \, \mbox{d}y - Q \, \mbox{d}x \right) \mbox{d}z
        %\]
        %The (trivial) integral over $z$ again gives unity. Using the
        %divergence theorem, we get
        %\begin{center}
        %\bigbox{
        %\parbox{80mm}{
        %\[
        %  \int_{A}
        %  \left(
        %    \frac{\partial P}{\partial x} +
        %    \frac{\partial Q}{\partial y }
        %  \right) \,
        %  \mbox{d}x \, \mbox{d}y
        %  \,=\,
        %  \oint_C \left( P \, \mbox{d}y - Q\mbox \, {d}x \right)
        %\]
        %}}
        %\end{center}
        %which is \emph{Green's theorem in the plane} (sometimes called the
        %\emph{two dimensional divergence theorem}) relating the integral over
        %a planar surface $A$ to the line integral around the closed curve $C$
        %enclosing $A$. The theorem applies to \emph{any} surface in the
        %$x{-}y$ plane, because the proof above doesn't rely on the base of the
        %cylinder being circular.
        %

\end{enumerate}

