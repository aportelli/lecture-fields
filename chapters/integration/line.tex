% !TEX root = ../../fields.tex
%% Brian Lecture 12 %%%%%%%%%%%%%%%%%%%%%%%%%%%%%%%%%%%%%%%%%%%%%%%%%%%%%%%%%%%%%%%%%%%%%%%
Having completed our study of differential calculus of scalar and vector fields using
Cartesian coordinates, suffix notation and the Einstein summation convention, we now
embark on integral calculus of scalar and vector fields, which we will (eventually!) use
to do some physics.

We begin with some revision of the standard definition of the integral of a function of a
single variable. We then discuss line integrals, surface integrals and volume integrals.
We shall assume familiarity with integrals over plane areas (double integrals) using
Cartesian and plane polar coordinates, as covered in the specialist mathematics course
\emph{Several Variable Calculus and Differential Equations}, but we will revise material
when appropriate.

\subsection{Revision: defining integrals in \texorpdfstring{$\thickR^1$}{R1}}

We start with some formal definitions and discuss some limits, but we shall not be
rigorous!

Let $f(x)$ be a continuous real-valued function defined on the interval $a \le x \le b$.
Begin by subdividing this interval into $n$ subintervals:
\[
  a = x_0 < x_1 < x_2 \, \ldots \, < x_n = b \,.
\]
In interval $j$, pick an arbitrary point $x_j^*$ with $x_j \le x_j^* \le x_{j+1}$, as
illustrated in the figure.

\medskip

\centerline{\epsfxsize=0.5\textwidth \includegraphics{tikz_riemann.pdf}}

Define the Riemann sum
\[
  S_n = \sum_{j=0}^{n-1} f(x_j^*) \left(x_{j+1}-x_j\right)
\]
It can be shown that $S_n \to$ \emph{unique limit} as we let $n\to\infty$ and
$\left(x_{j+1}-x_j\right)\to0$ for all $j$, with
$\sum_{j=0}^{n-1}\left(x_{j+1}-x_j\right)=(b-a)$ kept fixed. In this limit
\[
  S_n \to \int_a^b f(x) \, {\rm d}x % = \mbox{\emph{unique limit}}
\]
One can then prove the usual properties of integrals. Note that we also define
\[
  \int_b^a f(x) \, {\rm d}x
  =
  - \int_a^b f(x) \, {\rm d}x
\]

\subsection{Motivation and formal definition of line integrals}

%
% 2010/11 version
%
% Following the development of planar/double integrals of (scalar)
% functions of two variables, we return to three dimensions and develop
% integration of scalar and vector fields over surfaces and volumes. We
% shall revert back to using $(x_1,x_2,x_3)$ as our default notation for
% Cartesian coordinates, although we shall often use $(x,y,z)$ in
% specific examples.

%\medskip

\parbox{0.63\textwidth}{\paragraph{Motivation:} As a physical example, consider a particle constrained to move along a wire under the
influence of a force $\v{F}(\v{r})$.

\medskip

Only the component of the force \emph{along} the wire does work.

\medskip

Thus the work, $\mbox{d}W$, done by the force in moving the particle along the wire from
$\vr$ to $\vr + {\rm d} \vr$ is
\[
  \mbox{d}W
  \,=\,
  \left(\left|\vF\right| \cos\theta\right) \, \left|\mbox{d}\vr\right|
  \,=\,
  \vF\cdot \mbox{d}\vr
\]
$\left|\vF\right| \cos\theta$ is the component of $\v{F}$ along
$\mbox{d}\vr\,$, where $\theta$ is the angle between
$\v{F}(\v{r})$ and $\mbox{d}\v{r}$.
}
%
\hfill
%
\parbox{0.35\textwidth}{
  \epsfxsize=0.36\textwidth
  \includegraphics{tikz_line_int.pdf}
}

\medskip

The \emph{total} work done in moving the particle along a wire which follows some curve
$C$ from point $P$ to point $Q$ is then
\[
  W_C= \int_{P}^{Q} \mbox{d}W = \int_C \vF(\vr)\cdot \mbox{d} \v{r}
\]
This is a \emph{line} integral along the curve $C$.

\paragraph{Formal definition:} Let $\v{a}(\vr)$ be a vector field defined in the region $R$, and let $C$ be a curve in
$R$ between two points $P$ and $Q$. Let $\v{r}$ be the position vector at some point on
the curve, and let $\dd\v{r}$ be an infinitesimal vector \emph{along} the curve at
$\vr\,$.

The \emph{magnitude} of ${\rm d}\v{r}$ is the infinitesimal \emph{arc length}: $\mbox{d}s
  = \sqrt{{\rm d}\v{r}\cdot {\rm d}\v{r}}$

If $\v{t}$ is the \emph{unit vector} tangent to the curve at $\vr$ ($\v{t}$ points in the
direction of ${\rm d}\v{r}$ at $\v{r}$), then
\[
  \dd \vr \,=\, \vt \: \dd s
  \qquad\mbox{or}\qquad
  \frac{\dd\vr}{\dd s} \,=\, \vt
\]
The line integral is defined formally as a Riemann sum by dividing the curve into $n$
intervals
\[
  \intC\v{a}\cdot \mbox{d}\v{r}
  \,=\,
  \intC \, \v{a}\cdot\vt \; \mbox{d}s
  \,\equiv \,
  \raisebox{-3ex}
  {$
      \shortstack{lim \\ $ \delta s^{(i)}\to0$ \\ $n\to\infty$ }\;\,
    $}
  %             \,=\, \lim_{\stackrel{n\to\infty}{\it\small ds\to 0}}\;
  \sum_{i=0}^{n-1}\;
  \bigg(\v{a}\left(\v{r}^{(i)}\right)\cdot\vt^{(i)}\bigg)
  \, \delta s^{(i)}
\]
the $i^{\rm th}$ interval having length $\delta s^{(i)}$, unit tangent vector
$\vt^{(i)}$, \emph{etc}. It can be shown that the limit is unique for sufficiently smooth
vector fields $\v{a}(\v{r})$.

%$\v{r^*}$ is an arbitrary point between $\vr$ and $\vr+{\rm d}\v{r}$.

In Cartesian coordinates, we have
\begin{equation}
  \intC\v{a}\cdot {\rm d}\v{r}
  \,=\,
  \intC \left(a_1 \mbox{d}x_1 + a_2 \mbox{d}x_2 + a_3 \mbox{d}x_3 \right)
  \,=\,
  \intC a_{i} \, {\rm d}x_i
  %  \intC \sum_{i=1}^{3} a_{i} \, {\rm d}x_i
\end{equation}
where we used the summation convention in the last expression.

\textbf{NB:} In general,
\emph{the line integral depends on the path joining $P$ and $Q$}.  For
example, the $a_1$ component is $a_1(x_1 , x_2 , x_3)$ and all three
coordinates will generally change at once along the path. Therefore,
you can \emph{not} compute $\int a_1 \,\mbox{d}x_1$ as an ordinary
integral over $x_1$ holding $x_2$ and $x_3$ constant. That would only
be correct if $x_2$ and $x_3$ were \emph{constant} along the path,
\emph{i.e.} if the path were parallel to the $x_1$ axis. This is a
common source of mistakes.

\subsection{Parametric representation of line integrals}
The definition above was rather formal. What follows is more useful in practice.

Often a curve in 3d can be parameterised by a single real parameter, \emph{e.g.}~if the
curve were the trajectory of a particle then a natural parameter would be the time $t$.
Sometimes the parameter of a line integral is chosen to be the arc-length $s$ along the
curve $C$.

If we parameterise the curve by a real parameter $\lambda$ (varying from $\lambda_P$ to
$\lambda_Q$) then we can write the three coordinates $x_i$ as functions of $\lambda$:
\[
  x_i \,=\, x_i(\lambda), \quad \mbox{with\ } \lambda_P \le \lambda \le \lambda_Q
\]
%In longhand notation:
%\[
%  x_1 \,=\, x_1(\lambda)\,, \quad
%  x_2 \,=\, x_2(\lambda)\,, \quad
%  x_3 \,=\, x_3(\lambda)\,,
%\]

\vspace*{-3ex}

so that
\[
  \intC\v{a}\cdot {\rm d}\v{r}
  \,=\,
  \int_{\lambda_P}^{\lambda_Q}
  \left( \v{a} \,\cdot\, \frac{{\rm d}\v{r}}{{\rm d}\lambda} \right)\,
  {\rm d}\lambda
  \,=\,
  \int_{\lambda_P}^{\lambda_Q}
  a_i\, \frac{{\rm d}x_i}{{\rm d}\lambda} \, {\rm d}\lambda
  \,=\,
  \int_{\lambda_P}^{\lambda_Q}
  \left( a_1\, \frac{{\rm d}x_1}{{\rm d}\lambda} +
  a_2\, \frac{{\rm d}x_2}{{\rm d}\lambda} +
  a_3\, \frac{{\rm d}x_3}{{\rm d}\lambda}
  \right) {\rm d}\lambda
\]

If necessary, the curve $C$ may be subdivided into sections, each with a different
parameterisation (piecewise smooth curve).

\paragraph{Example:}
Let $\v{a}(\v{r}) = -ky\:\v{e}_{\,x} + kx \:\v{e}_{\,y}\,$, where $k$ is a positive
constant.

Evaluate $\int_C \v{a}\cdot {\rm d}\v{r}$ between the points with Cartesian coordinates
$(1,0,0)$ and $(0,1,0)$ along the curve $C$:\ \ $x=\cos\lambda$, $y=\sin\lambda$, $z=0$
where $0 \le \lambda \le \pi/2$.

Convince yourself that $C$ is one quarter of a unit circle. Sketch the curve $C$ and the
field $\v{a}(\v{r})$ on $C$.

On the curve $C$, we have

\vspace*{-5ex}

\begin{eqnarray*}
  \v{a}(\vr)
  & = &
  -ky \; \v{e}_{\,x} + kx \; \v{e}_{\,y}\,
  \,=\,
  -k\sin\lambda \; \v{e}_{\,x} + k\cos\lambda \; \v{e}_{\,y}  \\[0.75ex]
  %
  \vr
  & = &
  \cos\lambda\; \v{e}_{\,x} + \sin\lambda\; \v{e}_{\,y} \\[0.5ex]
  %
  \frac{{\rm d}\v{r}}{{\rm d}\lambda}
  & = &
  \left(-\sin\lambda\; \v{e}_{\,x}
  +
  \cos\lambda\; \v{e}_{\,y}\right)
\end{eqnarray*}
Therefore
\[
  \int_C \v{a}\cdot {\rm d}\v{r}
  \,=\,
  \int_{0}^{\pi/2}
  \left( \v{a} \,\cdot\, \frac{{\rm d}\v{r}}{{\rm d}\lambda} \right)\,
  {\rm d}\lambda
  \,=\,
  \int_{0}^{\pi/2} k \left(\sin^2\lambda+\cos^2\lambda\right)
  {\rm d}\lambda
  \,=\,
  \int_{0}^{\pi/2} k \, {\rm d}\lambda
  \,=\,
  \frac{k\pi}{2}
\]
Note: the field $\v{a}$ is parallel to $\frac{\mbox{d}\v{r}}{\mbox{d}\lambda}$ and hence
to $\mbox{d}\v{r}$, so the calculation is simple here.

\paragraph{Example:}
Let $\v{a}(\v{r})=(3x^2+6y)\:\v{e}_{\,x} - 14yz\:\v{e}_{\,y} + 20xz^2\:\v{e}_{\,z}$.

Evaluate $\int_C \v{a}\cdot {\rm d}\v{r}$ between the points with Cartesian coordinates
$(0,0,0)$ and $(1,1,1)$, along the two paths $C$:
\begin{enumerate}
  \item $(0,0,0) \rightarrow (1,0,0) \rightarrow (1,1,0) \rightarrow
          (1,1,1)$

        (These are $3$ contiguous straight lines parallel to the $x$, $y$ \&
        $z$ axes respectively.)

  \item $x=\lambda$, $y=\lambda^2$, $z=\lambda^3\,,\,$ from $\lambda=0$ to
        $\lambda=1\,$.
\end{enumerate}

\begin{figure}[ht]
  %
  \centerline{\epsfxsize=5.8cm\includegraphics{tikz_line_paths.pdf} }
  %
\end{figure}

\begin{enumerate}
  \item
        \begin{enumerate}

          \item Along the line from $(0,0,0)$ to $(1,0,0)$, we have $y=z=0$, so ${\rm d}y={\rm d}z=0,$
                hence ${\rm d}\v{r} = \v{e}_{\,x}\,{\rm d}x$ and $\v{a}=3x^2\,\v{e}_{\,x}$ (here the
                parameter is just $x$ itself), and
                \[
                  \int^{(1,0,0)}_{(0,0,0)} \v{a}\cdot {\rm d}\v{r}
                  \,=\,  \int^{x=1}_{x=0}3x^2\,{\rm d}x
                  \,=\,  \left[x^3\right]_0^1 \,=\, 1
                \]

          \item Along the line from $(1,0,0)$ to $(1,1,0)$, we have $x=1$, ${\rm d}x=0$, $z={\rm d}z=0$,
                so ${\rm d}\v{r} = \v{e}_{\,y}\,{\rm d}y$ (here the parameter is $y$), and
                \begin{eqnarray*}
                  \v{a}
                  &=&
                  \left(\left. 3x^2+6y\,\right)\right|_{x=1}\, \v{e}_{\,x}
                  \,=\,
                  (3+6y)\,\v{e}_{\,x} \\[1ex]
                  %
                  \Rightarrow \quad
                  \int^{(1,1,0)}_{(1,0,0)} \v{a}\cdot {\rm d}\v{r}
                  &=&
                  \int^{y=1}_{y=0}\left(3+6y\right)
                  \; \v{e}_{\,x} \!\cdot \v{e}_{\,y} \; {\rm d}y
                  \,=\, 0
                \end{eqnarray*}

          \item Along the line from $(1,1,0)$ to $(1,1,1)$, we have $x=y=1$, ${\rm d}x={\rm d}y=0,$ and
                hence ${\rm d}\v{r} = \v{e}_{\,z}\,{\rm d}z$ and
                $\v{a}=9\,\v{e}_{\,x}-14z\,\v{e}_{\,y}+20z^2\, \v{e}_{\,z}\,$. Therefore
                \[
                  \int^{(1,1,1)}_{(1,1,0)} \v{a}\cdot {\rm d}\v{r}
                  \,=\,
                  \int^{z=1}_{z=0} 20 z^2 \, {\rm d}z
                  \,=\,
                  \left[\frac{20}{3} z^3 \right]_0^1
                  \,=\,
                  \frac{20}{3}
                \]

                Adding up the $3$ contributions, we get
                \[
                  \intC\v{a}\cdot {\rm d}\v{r}
                  \,=\, 1 + 0 + \frac{20}{3}
                  \,=\,
                  \frac{23}{3}  \qquad \mbox{along path (i)}
                \]

        \end{enumerate}
        %\vspace*{-2ex}
  \item To integrate $\v{a}=(3x^2+6y)\,\v{e}_{\,x} - 14yz\v{e}_{\,y}\, + 20xz^2\v{e}_{\,z}$ along
        path~(ii), we parameterise
        \begin{eqnarray*}
          \vr
          & = &
          \lambda\,\v{e}_{\,x}
          +
          \lambda^2\,\v{e}_{\,y}
          +
          \lambda^3\, \v{e}_{\,z} \\[1ex]
          %
          \frac{{\rm d}\vr}{{\rm d}\lambda}
          &=&
          \v{e}_{\,x} + 2\lambda\, \v{e}_{\,y} + 3\lambda^2\, \v{e}_{\,z}
          \\[1ex]
          %
          \v{a}
          & = &
          \left(3\lambda^2+6\lambda^2\right)\,
          \v{e}_{\,x} - 14\lambda^5\,\v{e}_{\,y} + 20\lambda^7\,\v{e}_{\,z}
          \quad\mbox{so that} \\[1ex]
          %\end{eqnarray*}
          %\begin{eqnarray*}
          \intC \left(\v{a}\cdot\frac{{\rm d}\vr}{{\rm d}\lambda}\right)\,
          {\rm d}\lambda
          &=&
          \int^{\lambda=1}_{\lambda=0}
          \left(9\lambda^2- 28\lambda^6+60\lambda^9\right)\, {\rm d}\lambda
          \,=\,
          \big[3\lambda^3-4\lambda^7+6\lambda^{10}\big]^1_0
          \,=\,
          5 \\[0.8ex]
          %
          \mbox{Hence}\qquad
          \intC\v{a}\cdot {\rm d}\v{r}
          &=&
          5  \qquad \mbox{along path (ii)}
        \end{eqnarray*}
\end{enumerate}
In this case, the integral of $\v{a}$ from $(0,0,0)$ to $(1,1,1)$
depends on the path taken.

\paragraph{Notes:}

\begin{enumerate}

  \item The line integral $\intC\v{a}\cdot {\rm d}\v{r}$ is a \emph{scalar} quantity. Another
        \emph{scalar} line integral is $\intC f\, {\rm d}s$ where $f(\vr)$ is a scalar field and
        ${\rm d}s = \sqrt{{\rm d}\v{r}\cdot {\rm d}\v{r}}$ is the infinitesimal arc-length
        introduced earlier.

        %{\it e.g.} the length of a curve is simply given by $\int_C {\rm d}s$.

  \item A line integral around a \emph{simple} (doesn't intersect itself) \emph{closed} curve $C$
        is denoted by the symbol $\ointC$
        \[
          \mbox{Example:}\quad \ointC \v{a}\cdot {\rm d}\v{r}
          \,\equiv\, \mbox{the \emph{circulation} of $\v{a}$ around $C$}
        \]

\end{enumerate}

\paragraph{Example:}
Let $f(\vr) = ax^2 + by^2$. Evaluate $\ointC f\,{\rm d}s$ around the unit circle $C$
centred on the origin in the $x{-}y$ plane:
\[
  x=\cos\phi, \, y=\sin\phi, \, z=0; \quad 0\leq\phi\leq2\pi.
\]
\begin{eqnarray*}
  \mbox{On the curve $C$:} \hspace*{0.09\textwidth}
  f(\vr)
  & = &
  ax^2 + by^2
  \,=\,
  a\cos^2\phi + b\sin^2\phi \hspace*{0.29\textwidth}\\[0.5ex]
  \vr
  & = &
  \cos\phi\, \v{e}_{\,x} + \sin\phi\, \v{e}_{\,y} \\[0.25ex]
  {\rm d}\v{r}
  & = &
  \left(-\sin\phi\, \v{e}_{\,x} + \cos\phi\, \v{e}_{\,y}\right)\, {\rm d}\phi\\
  \Rightarrow \quad {\rm d}s & = & \sqrt{{\rm d}\v{r}\cdot {\rm d}\v{r}}
  \,=\,
  \left(\sin^2\phi+\cos^2\phi\right)^{1/2}\,{\rm d}\phi
  \,=\,
  {\rm d}\phi
\end{eqnarray*}
Therefore, in this example,
\[
  \ointC f\,{\rm d}s
  \,=\,
  \int^{2\pi}_0 \left(a\cos^2\phi + b\sin^2\phi \right) {\rm d}\phi
  \,=\, \pi\left(a+ b\right)
\]
The \emph{length} $s$ of a curve $C$ is given by $s = \intC {\rm d}s$. In this example
$\ds s = \int^{2\pi}_0 {\rm d}\phi = 2\pi$.

%\medskip

We can also define \emph{vector} line integrals, \emph{e.g.}
\begin{enumerate}

  \item
        \(
        \intC \v{a}\; {\rm d}s
        \,=\,
        \ei  \intC a_{i}\, {\rm d}s
        \,=\,
        \intC
        \left(
        \eone \: a_1 \,+\, \etwo \: a_2 \,+\, \ethree \: a_3 \,
        \right)
        {\rm d}s
        \)
        \ \ in Cartesian coordinates.

  \item
        \(
        \intC \v{a}\times {\rm d}\v{r}
        \,=\,
        \ei \, \epsilon_{ijk} \,   \intC a_j \, {\rm d}x_k
        %  \big[\,
        %    \eone   \left(a_2 \, {\rm d}x_3 - a_3 \, {\rm d}x_2 \right)
        %    \,+\,
        %    \etwo   \left(a_3 \, {\rm d}x_1 - a_1 \, {\rm d}x_3 \right)
        %    \,+\,
        %    \ethree \left(a_1 \, {\rm d}x_2 - a_2 \, {\rm d}x_1 \right)
        %  \big]
        \)
        \ \ in Cartesian coordinates, which is again a vector

        The parametric form is simply \( \intC \v{a}\times {\rm d}\v{r} \,=\, \intC \left( \v{a}
        \times \frac{\mbox{d}\v{r}}{\mbox{d}\lambda} \right) \, \mbox{d}\lambda \)

  \item
        \(
        \intC f\, {\rm d}\v{r}
        \,=\,
        \ei \intC f \; {\rm d}x_i
        \,=\,
        \intC f \;
        \left(
        \eone   \, {\rm d}x_1
        \,+\,
        \etwo   \, {\rm d}x_2
        \,+\,
        \ethree \, {\rm d}x_3
        \right)
        \)
        in Cartesian coordinates. In parametric form, this becomes
        \(
        \intC f\, {\rm d}\v{r}
        \,=\,
        \intC f \: \frac{\mbox{d}\v{r}}{\mbox{d}\lambda} \;\, \mbox{d}\lambda
        \)

\end{enumerate}

Note that the result of performing the integral is a vector in each case.

\mnote{16L 29/02/08}

\newpage
\subsection{Current loop in a magnetic field}
Consider an electric current of magnitude $I$ flowing along a thin wire in the shape of a
closed path $C$.

The magnetic force on an element ${\rm d}\v{r}$ of the wire at $\v{r}$ due to an external
magnetic field $\v{B}(\v{r})$ is given by the Lorentz force
\[
  {\rm d}\v{F}(\v{r})
  \,=\,
  I\, {\rm d}\v{r} \times \v{B}(\v{r})\,.
\]
The \emph{total} force $\v{F}$ on the wire is the vector sum of the forces on the
individual elements, which is given by the line integral of ${\rm d}\v{F}$ around the
closed curve $C$.
\[
  \v{F}
  \,=\,
  \ointC {\rm d}\v{F}
  \,=\,
  \ointC I \, {\rm d}\v{r} \times \v{B}(\v{r})
  ~=
  -I \ointC \v{B}(\v{r}) \times  {\rm d}\v{r}
\]
\parbox{0.55\textwidth}{\textbf{Example:} For the (unrealistic!) case
  where the external magnetic field is $\vB(\vr) =
    B_0 \big(x\,\v{e}_{\,x} + y\,\v{e}_{\,y}\big)$, evaluate the total
  force on a circular current loop of radius $a$ which lies in the
  $x{-}y$ plane and is centred on the origin.

  \medskip

  We parameterise the curve by the angle $\phi$ (as in plane polars), so that on the curve
  $C$, we have } \hfill \parbox{0.42\textwidth}{ \epsfxsize=0.42\textwidth
  \includegraphics{tikz_loop-in-B-field.pdf} }

\vspace*{-5ex}

\begin{eqnarray*}
  \v{r}
  & = &
  a\cos\phi\,\v{e}_{\,x} + a\sin\phi\,\v{e}_{\,y} \\[1ex]
  %
  {\rm d}\v{r}
  & = &
  \big(-a\sin\phi\, \v{e}_{\,x} + a\cos\phi\,
  \v{e}_{\,y}\big)\, {\rm d}\phi\\[1ex]
  %
  \vB
  & = &
  B_0 \big( a\cos\phi\,\v{e}_{\,x} + a\sin\phi\,\v{e}_{\,y} \big)
  \\[1ex]
  %
  \Rightarrow\quad
  \ointC\vB\times {\rm d}\v{r}
  &=&
  B_0 \,
  \int^{2\pi}_0  \big(a^2\cos^2\phi + a^2\sin^2\phi \big)\,
  \v{e}_{\,z}\; {\rm d}\phi
  \,=\,
  B_0 \,  \v{e}_{\,z}\; a^2 \int^{2\pi}_0 {\rm d}\phi
  \,=\,
  2\pi a^2 B_0 \, \v{e}_{\,z} \\[1ex]
  \mbox{So} \quad
  \v{F}
  & = &
  -2\pi a^2 B_0 I \, \v{e}_{\,z}
  \quad \mbox{which is in a vertically downward direction.}
\end{eqnarray*}

\mnote{17L 10/03/09}

%\vfill
%\begin{flushright}
%\tiny{K.C.~Bowler, March 2005}
%\end{flushright}
