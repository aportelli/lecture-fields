% !TEX root = ../../fields.tex
%% Brian Lecture 17 %%%%%%%%%%%%%%%%%%%%%%%%%%%%%%%%%%%%%%%%%%%%%%%%%%%%%%%%%%%%%%%%%%%%%%%
\subsection{Line integral definition of curl}

\parbox{0.62\textwidth}{Consider a small planar surface with unit
  normal $\vn$ and (scalar) area $\delta S$, bounded by a
  \emph{closed} curve $\delta C$, and containing the point $P$. Let
  $\v{a}$ be a vector field defined on this surface.

  \medskip

  The component of $\curl\v{a}$ in the direction of $\vn$ is \emph{defined} to be} \hfill
\parbox{0.35\textwidth}{ \epsfxsize=0.35\textwidth \includegraphics{tikz_loop.pdf} }

\begin{center}
  \vspace*{-1ex}
  \bigbox{
    \parbox{60mm}{
      \vspace*{-1.5ex}
      \[
        % \v{n} \cdot \left({\rm curl\ }\v{a}\right)
        %  \,=\,
        \vn \, \cdot\, \left(\curl\v{a}\right)
        \,=\,
        \lim_{\delta S\to0} \; \frac{1}{\delta S}\;
        \oint_{\delta C} \v{a} \,\cdot\, \mbox{d}\v{r}
      \]
      \vspace*{-2ex}
    }
  }
\end{center}

\paragraph{NB:} The integral around $\delta C$ is taken in the right-hand sense with respect to the
normal $\vn$ to the surface -- as shown in the figure above.

This definition of curl is \emph{independent of the choice of basis}.
%respectively, but you are not required to prove this.

\paragraph{Cartesian form of curl:}
The usual Cartesian form for curl can be recovered from this general definition by
considering small rectangles parallel to the $x{-}y$, $y{-}z$, and $z{-}x$ planes
respectively.

\parbox{0.5\textwidth}{Let $P$ be a point with Cartesian coordinates
  $(x_0,y_0,z_0)$ situated at the \emph{centre} of a small rectangle
  $\delta C=ABCD$ of size $\delta_x \times \delta_y$, area $\delta S =
    \delta_x\, \delta_y$, parallel to the $x{-}y$ plane.

  \bigskip\bigskip
}
\hfill
\parbox{0.47\textwidth}{\epsfxsize=0.47\textwidth \includegraphics{tikz_curl.pdf} }

The line integral around $\delta C$ is given by the sum of four terms
\begin{eqnarray*}
  \ointdC \v{a}\cdot \mbox{d}\v{r}
  \,=\,
  \int_A^B  \v{a}\cdot \mbox{d}\v{r}
  \,+\,
  \int_B^C  \v{a}\cdot \mbox{d}\v{r}
  \,+\,
  \int_C^D  \v{a}\cdot \mbox{d}\v{r}
  \,+\,
  \int_D^A  \v{a}\cdot \mbox{d}\v{r} \\[1ex]
  \,=\,
  \int_A^B  \v{a}\cdot \mbox{d}\v{r}
  \,-\,
  \int_C^B  \v{a}\cdot \mbox{d}\v{r}
  \,-\,
  \int_D^C  \v{a}\cdot \mbox{d}\v{r}
  \,+\,
  \int_D^A  \v{a}\cdot \mbox{d}\v{r}
\end{eqnarray*}
Since $\vr = x\,\ex+y\,\ey+z\,\ez$ we have
$\mbox{d}\v{r}=\ex\,\mbox{d}x$ along $D\rightarrow A$ and
$C\rightarrow B$, and $ \mbox{d}\v{r}=\ey\,\mbox{d}y$\\[0.5ex] along
$A\rightarrow B$ and $D\rightarrow C$.  Therefore
\[
  \ointdC \v{a}\cdot \mbox{d}\v{r}
  \,=\,
  \int_A^B  a_y\, \mbox{d}y
  \,-\,
  \int_C^B  a_x\, \mbox{d}x
  \,-\,
  \int_D^C  a_y\, \mbox{d}y
  \,+\,
  \int_D^A  a_x\, \mbox{d}x
\]
For small $\delta_x$ and $\delta_y$, we can Taylor expand the integrands about $y=y_0$,
\begin{eqnarray*}
  \int_D^A a_x\, \mbox{d}x
  & = &
  \int_D^A a_x(x,y_0-\delta_y/2,z_0) \, \mbox{d}x \\[1ex]
  %
  & = &
  \int_{x_0-\delta_x/2}^{x_0+\delta_x/2}
  \left[
    a_x(x,y_0,z_0)
    \,-\,
    \frac{\delta_y}{2}\; \parpar{a_x}{y}\bigg|_{(x,y_0,z_0)}
    \,+\,
    O(\delta_y^2)
    \right]\, \mbox{d}x \\[1.5ex]
  %
  \int_C^B a_x\, \mbox{d}x
  & = &
  \int_C^B a_x(x,y_0+\delta_y/2,z_0) \, \mbox{d}x\\[1ex]
  %
  & = &
  \int_{x_0-\delta_x/2}^{x_0+\delta_y/2}
  \left[
    a_x(x,y_0,z_0)
    \,+\,
    \frac{\delta_y}{2}\; \parpar{a_x}{y}\bigg|_{(x,y_0,z_0)}
    \,+\,
    O(\delta_y^2)
    \right]\, \mbox{d}x
\end{eqnarray*}
so
\begin{eqnarray*}
  \frac{1}{\delta S}
  \left[
    \int_D^A
    \v{a}\cdot \mbox{d}\v{r} \,+\, \int_B^C  \v{a}\cdot \mbox{d}\v{r}\right]
  & = &
  \frac{1}{\delta_x\, \delta_y}
  \left[
    \int_D^A a_x\, \mbox{d}x \,-\, \int_C^B a_x\, \mbox{d}x
    \right]\\[1ex]
  %
  & = &
  \frac{1}{\delta_x \delta_y}\;  \int_{x_0-\delta_x/2}^{x_0+\delta_x/2}
  \left[
    -\delta_y \parpar{a_x}{y}\bigg|_{(x,y_0,z_0)} \,+\, O(\delta_y^2)
    \right] \, \mbox{d}x \\[1ex]
  %
  & \to &
  -\parpar{a_x}{y}\bigg|_{(x_0,y_0,z_0)}
  \quad \mbox{as}\quad \delta_x,\, \delta_y \to 0
\end{eqnarray*}
In the last step, as we take the limit $\delta_x \to 0$, the
integrand tends to a constant in the region of integration:
\[
  \parpar{a_x}{y}\bigg|_{(x,y_0,z_0)}
  ~\to~
  \parpar{a_x}{y}\bigg|_{(x_0,y_0,z_0)}
\]
and the integral over $x$ is then trivial (as the integrand is a constant).

A similar analysis of the line integrals along $A\rightarrow B$ and $C\rightarrow D$
gives (exercise)
\[
  \frac{1}{\delta S}
  \left[
    \int_A^B \v{a}\cdot \mbox{d}\v{r} \,+\, \int_C^D \v{a}\cdot \mbox{d}\v{r}
    \right]
  ~\rightarrow~
  \parpar{a_y}{x}\bigg|_{(x_0,y_0,z_0)}
  \quad \mbox{as}\quad \delta_x,\, \delta_y \to 0
\]

Adding the results gives, for our line integral definition of curl,
\[
  \ez \cdot \left(\curl\v{a}\right)
  \,=\,
  \left(\curl\v{a}\right)_z
  \,=\,
  \left[ \parpar{a_y}{x} - \parpar{a_x}{y} \right]_{\:(x_0,\,y_0,\,z_0)}
\]
in agreement with our original definition in Cartesian coordinates.

The other components of $\v{\nabla}\times\v{a}$ can be obtained from similar rectangles
parallel to the $y{-}z$ and $x{-}z$ planes, respectively.

It can be shown that $\v{\nabla}\times\v{a}$, when defined in this way, is independent of
the \emph{shape} of the infinitesimal area $\delta S$.

%\mnote{20L 20/03/09}

\newpage

\subsection{Physical/geometrical interpretation of curl}
Consider a force field $\v{F}(\v{r}),$ and let $\delta C$ be a small rectangular contour
which encloses an area $\delta S$ in the $x{-}y$ plane -- as in the line-integral
definition of curl above.

The \emph{work done} on a (point) test particle in moving it around the closed curve
$\delta C$ is
\[
  \oint_{\delta C} \v{F}\cdot \mbox{d}\v{r}
  \,=\,
  \mbox{\emph{circulation} of $\v{F}(\v{r})$ about $\delta C$}
\]
From the integral definition of curl, we know that for small $\delta S$
\[
  \oint_{\delta C} \v{F}\cdot \mbox{d}\v{r}
  ~\approx~
  \left( \v{\nabla}\times\v{F}\right)_z \, \delta S
\]
Therefore $\,\left(\v{\nabla}\times\v{F}\right)_z \, \neq \, 0 \;$ is equivalent to
saying that a non-zero amount of work is done in moving the test particle around a small
closed path in the $x{-}y$ plane.

Alternatively one can think of the non-zero circulation of $\v{F}$ as causing a small
test particle to \emph{rotate} about its centre, with the axis of rotation in the
direction of $\curl\v{F}$.

More generally, $\vn\cdot\left(\v{\nabla}\times\v{a}\right)$ is a measure of the net
\emph{circulation} (per unit area) of the vector field $\v{a}$ about an
\emph{infinitesimal} area $\mbox{d}S$ with normal $\vn$.

\begin{center}
  \epsfxsize=0.9\textwidth
  \includegraphics{tikz_net_circulation.pdf}
\end{center}

%\vspace*{-3ex}

%xxxxxxxxxxxxxxxxxxxxxxxxxxxxxxxxxxxxxxxxxxxxxxxxxxxxxxxxx
%\subsection{Physical Interpretation of Curl}
%We illustrate the meaning of curl in physics by means of a
%simple example (see figure).
%
%Consider a force $\v{F}= a y \v{e}_x$ so that $\displaystyle
%\parpar{F_1}{y}\neq0$ and let $C$ be a small rectangular contour in
%the ($\v{e}_x{-}\v{e}_y$) plane, enclosing an area $\delta S$.
%\begin{center}
%\epsfxsize=10cm
%%\epsfxsize=12cm
%\includegraphics{tikz_small_loop.pdf}
%\end{center}
%The \emph{ work done} on a test particle in moving it around the closed
%curve $C$ is
%\begin{center}
%\vspace*{-1ex}
%\bigbox{
%\parbox{80mm}{
%\vspace*{-1.5ex}
%\[
%\ointC \v{F}\cdot \mbox{d}\v{r}\quad
%                        \mbox{= \emph{circulation} of $\v{F}$ about $C$}
%\]
%\vspace*{-2.5ex}
%}}
%\end{center}
%(Remember that when the line integral of $\v{F}$ around a closed
%curve is non-zero the force is non-conservative which implies
%curl $\v{F}$ is non-zero.)
%The integral is given by the sum of four terms
%\begin{eqnarray*}
%        \ointC \v{F}\cdot \mbox{d}\v{r}
%                & = &   \int_A^B \v{F}\cdot \mbox{d}\v{r} \,+\,
%                        \int_B^C \v{F}\cdot \mbox{d}\v{r} \,+\,
%                        \int_C^D \v{F}\cdot \mbox{d}\v{r} \,+\,
%                        \int_D^A \v{F}\cdot \mbox{d}\v{r} \\[1ex]
%                & = &   \int_A^B F_x\,\mbox{d}x \,-\, \int_d^C F_x\,\mbox{d}x
%\end{eqnarray*}
%The $2$nd and $4$th terms vanish because $\v{F}\cdot \mbox{d}\v{r}=0$
%along the vertical lines $bc$ and $ad$.\\[0.1ex]
%
%In the example (see figure), $F_x= ay$ $\ds \quad
%\parpar{F_x}{y}\neq0, \quad$ therefore $\ds \quad \int_A^B F_x\,\mbox{d}x
%\;\neq\; \int_D^C F_x\,\mbox{d}x$.  Hence
%\[
%        \ointC\v{F}\cdot \mbox{d}\v{r}\;\neq\;0.
%\]
%But, from the integral definition of curl, we know that
%\[
%        \ointC\v{F}\cdot \mbox{d}\v{r} \approx \left(
%        \v{\nabla}\times\v{F}\right)_z
%        \, \delta S
%\]
%Therefore $\,\left(\v{\nabla}\times\v{F}\right)_z \, \neq \, 0 \; $ (in
%our example it is equal to $-a$) is equivalent to saying that a
%non-zero amount of work is done in moving the test particle around the
%small closed path.  Alternatively one can think of the non-zero
%circulation of $\v{F}$ causing a small test particle to \emph{rotate}
%about its centre with axis of rotation in the $-\v{e}_z$ direction
%(using r.h. thumb rule).
%
%Thus, in general, $\vn\cdot\left(\v{\nabla}\times\v{F}\right)$ is a
%measure of the \emph{circulation} of the vector field $\v{F}$ about an
%\emph{infinitesimal} area with normal $\vn$.
%(It can be shown that $\v{\nabla}\times\v{F}$, when defined in this way,
%is independent of the shape of the infinitesimal area $\delta S$.)  
%xxxxxxxxxxxxxxxxxxxxxxxxxxxxxxxxxxxxxxxxxxxxxxxxxxxxxxxxx

\subsection{Stokes' theorem}

\parbox{0.58\textwidth}{Let $S$ be an \emph{open} surface, bounded by
  a simple \emph{closed} curve $C$, and let $\v{a}$ be a vector field
  defined on $S$, then
  \begin{center}
    \bigbox{
      \parbox{0.4\textwidth}{
        \[
          \intS \left(\curl\v{a}\right) \cdot \mbox{d}\v{S}
          \,=\,
          \ointC \v{a}\cdot \mbox{d}\v{r}
        \]
      }}
  \end{center}
  where $C$ is traversed in a right-hand sense about $\mbox{d}\v{S}$.
  As usual, $\mbox{d}\v{S} = \vn \, \mbox{d}S$ where $\vn$ is the unit
  normal to $S$.
}
\hfill
\parbox{0.38\textwidth}{
  \epsfxsize=0.38\textwidth
  \includegraphics{tikz_stokes.pdf}
}

%\emph{Proof:} {\it (D 6.1; RHB 9.9)}

\newpage

\paragraph{Proof:}
Divide the surface $S$ into $N$ adjacent small surfaces. Let $\delta \v{S}^{(i)} = \delta
  S^{(i)}\, \v{n}^{(i)}$ be the vector element of area at $\vr^{(i)}$, enclosed by the
curve $\delta C^{(i)}$.

\centerline{\epsfxsize=0.4\textwidth \includegraphics{tikz_mesh.pdf}}

Start with the integral definition of curl
\[
  \vn \cdot \left(\curl\v{a}\right)
  \,=\,
  \lim_{\delta S\to0} \; \frac{1}{\delta S}\;
  \oint_{\delta C} \v{a}\cdot \mbox{d}\v{r} \,,
\]
For a small but not infinitesimal open surface $\delta S^{(i)}$
\[
  \bigg( \curl \v{a}\left(\vr^{(i)}\right)\bigg)
  \cdot \vn^{(i)}
  \,=\,
  \frac{1}{\delta S^{(i)}} \oint_{\delta C^{(i)}} \v{a}\cdot \mbox{d}\v{r}
  \,+\,
  \epsilon^{(i)}
\]
where $\epsilon^{(i)}\to0$ as $\delta S^{(i)}\to0$.

Multiply by $\delta S^{(i)}$ (before taking the limit), and sum over all $i$ to get
\[
  \sum_{i=1}^{N}\:
  \bigg( \curl \v{a}\left(\vr^{(i)}\right)\bigg)
  \cdot \vn^{(i)}\: \delta S^{(i)}
  \,=\,
  \sum_{i=1}^{N}\: \oint_{\delta C^{(i)}} \v{a}\cdot \mbox{d}\v{r}
  %  \,+\, \ldots
  \,+\, \sum_{i=1}^{N}\; \epsilon^{(i)}\, \delta S^{(i)}
\]
Since each small closed curve $\delta C^{(i)}$ is traversed in the \emph{same} sense,
then, from the diagram, all contributions to ${\ds \sum_{i=1}^{N}\; \oint_{\delta
        C^{(i)}}} \v{a}\cdot \mbox{d}\v{r}$ \emph{cancel}, except on those curves where part of
$\delta C^{(i)}$ lies on\\[0.7ex] the curve $C$.  For example, the line integrals along
the \emph{common} section of the two small closed curves $\delta
  C^{(1)}$ and $\delta C^{(2)}$ in the figure \emph{cancel exactly}.
Therefore
\[
  \sum_{i=1}^{N}\;\oint_{\delta C^{(i)}} \v{a}\cdot \mbox{d}\v{r}
  \,=\,
  \ointC \v{a}\cdot \mbox{d}\v{r}
\]
Hence, as $N\to\infty$,
\[
  \ointC \v{a}\cdot \mbox{d}\v{r}
  \,=\,
  \intS \left(\curl\v{a}\right) \cdot \mbox{d}\v{S}
  \,=\,
  \intS \vn\cdot\left(\curl\v{a}\right) \; \mbox{d}S
\]

\paragraph{Mathematical note:} For those worried about the `error term', note that, for finite $N$, we can establish an
upper bound \vspace*{-1ex}
\[
  \left| \sum_{i=1}^{N}\; \epsilon^{(i)}\, \delta S^{(i)} \right|
  \;\le\; S \;
  \raisebox{-1.6ex}{$ \shortstack{max \\ \scriptsize{$i$} } $}
  \left\{ \left|\epsilon^{(i)} \right|\right\}
\]
The RHS tends to zero in the limit $N\to\infty$, because $S$ is finite and
$\epsilon^{(i)}\to0,$ $\,\forall i$. A similar analysis works in the proof of the
divergence theorem.\footnote{The case of an infinite surface $S$ (or infinite $V$ in the
  case of the divergence theorem) requires more care.}

\subsection{Examples of the use of Stokes' theorem}
\label{example_stokes}

\paragraph{Hemisphere:} Given the vector field $\v{a} = 4y \, \v{e}_{\,x} + x \, \ey + 2z \, \ez\,$, verify
Stokes' theorem for \\[0.5ex] the
(open) hemispherical surface $x^2 + y^2 + z^2 = R^2$ with $z > 0$.

In this case, we have $\curl\v{a} = -3\,\ez\,$, and we have shown previously that
$\mbox{d}\v{S} = R^2\sin\theta \, \mbox{d}\theta \, \mbox{d}\phi \, \er$ on the surface
of a (hemi)sphere of radius $R$. Direct integration then gives
\begin{eqnarray*}
  \int_{S_C} \curl \v{a}\cdot \mbox{d}\v{S}
  &=&
  \int_{S_C} (-3 \ez) \cdot
  R^2 \, \sin\theta \, \mbox{d}\theta \, \mbox{d}\phi \, \er \\[1ex]
  &=&
  -3 R^2 \int_0^{\pi/2} \, \sin\theta  \cos\theta\, \mbox{d}\theta
  \; \int_0^{2\pi} \mbox{d}\phi
  \,=\,
  -3 \pi R^2
\end{eqnarray*}

%\end{document}

%\newpage

We can check our result using Stokes' theorem. The closed curve $C$ bounding the
hemisphere is a circle of radius $R$ in the $x{-}y$ plane. Parameterising this by $x =
  R\cos\phi$, $y = R\sin\phi$, $z=0$, gives $\mbox{d}x = -R\sin\phi\,\mbox{d}\phi$,\ \
$\mbox{d}y = R\cos\phi\,\mbox{d}\phi$, and $a_x = 4y = 4R\sin\phi$, $a_y = x =
  R\cos\phi$, $a_z=2z=0$. Hence
\begin{eqnarray*}
  \oint_C \v{a}\cdot \mbox{d}\v{r}
  &=&
  \oint_C (4y \, \mbox{d}x + x \, \mbox{d}y) \\[1ex]
  &=&
  \int_0^{2\pi}
  \left( -4R^2\sin^2\phi + R^2 \cos^2\phi \right)\,\mbox{d}\phi
  \,=\,
  -3\pi R^2
\end{eqnarray*}

\paragraph{Planar areas:}
Consider a planar surface $S$ parallel to the $x{-}y$ plane, bounded by a closed curve
$C$, and let the vector field $\v{a}(\v{r})$ be
\[
  \v{a}
  \,=\,
  \frac{1}{2} \left[ -y \, \ex + x \, \ey \right]
\]
In this case $\curl \v{a} = \v{e}_{\,z}$, and the vector element of area normal to the
$x{-}y$ plane is $\mbox{d}\v{S} = \mbox{d}S\,\ez$. Hence
\[
  \int_S \curl\v{a} \cdot \mbox{d}\v{S}
  \,=\,
  \int_S \v{e}_{\,z}\cdot \mbox{d}\v{S} = \int_S \mbox{d}S
  \,=\,
  S
\]
We can then use Stokes' theorem to find the area of the surface
\begin{eqnarray*}
  S
  &=&
  \ointC \v{a}\cdot  \mbox{d}\v{r}
  \,=\,
  \frac{1}{2} \ointC
  (-y \, \ex + x \ey)
  \cdot
  (\mbox{d}x \, \v{e}_{\,x} + \mbox{d}y \, \v{e}_{\,y} )
\end{eqnarray*}
which gives
\[
  S = \frac{1}{2} \ointC (x\,\mbox{d}y - y\,\mbox{d}x)
\]

\textbf{Example:} Find the area inside the curve
\[
  x^{2/3}+y^{2/3} =1 \, .
\]
The curve can be parameterised by $x= \cos^3 \phi$, $y= \sin^3 \phi$, for $0\leq \phi
  \leq 2\pi$, so that
\[
  \frac{\mbox{d}x}{\mbox{d}\phi} = -3\cos^2\phi\, \sin\phi\;,
  \quad
  \frac{\mbox{d}y}{\mbox{d}\phi} = 3\sin^2\phi\, \cos\phi
\]
which gives
\begin{eqnarray*}
  S
  &=&
  \frac{1}{2} \ointC (x\,\mbox{d}y - y\,\mbox{d}x)
  \,=\,
  \frac{1}{2} \ointC
  \left(
  x\,\frac{\mbox{d}y}{\mbox{d}\phi} -
  y\,\frac{\mbox{d}x}{\mbox{d}\phi}
  \right)
  \,\mbox{d}\phi
  \\[1ex]
  &=&
  \frac{1}{2} \int_{0}^{2\pi}
  \left(
  3\cos^4\phi \, \sin^2\phi + 3 \sin^4\phi \, \cos^2\phi
  \right) \mbox{d}\phi \\[1ex]
  &=&
  \frac{3}{2} \int_{0}^{2\pi} \sin^2\phi \, \cos^2\phi \, \mbox{d}\phi
  \,=\,
  \frac{3}{8} \int_{0}^{2\pi} \sin^2 2 \phi \,  \mbox{d}\phi
  \,=\,
  \frac{3\pi}{8}
\end{eqnarray*}

%\newpage

\subsection{Corollaries of Stokes' theorem}
We may deduce several immediate consequences of Stokes' theorem,
\[
  \int_S \, \left(\v{\nabla}\times\v{a}\right) \cdot \mbox{d}\v{S}
  \,=\,
  \oint_C \, \v{a}\cdot \mbox{d}\v{r}
\]
where $C$ is the boundary (traversed in the anticlockwise direction) of the open surface
$S$.

\begin{enumerate}

  \item If $\v{a} = \v{c}$, where $\v{c}$ is a constant vector, then $\curl \v{a}=0$. Therefore
        $\ds \v{c}\cdot \oint_C\, \mbox{d}\v{r} = 0$, and because $\v{c}$ is arbitrary, we have
        \[
          \oint_C \, \mbox{d}\v{r} \,=\, 0
        \]

        %\item Take $\v{a} = -Q(x,y) \, \ex + P(x,y) \, \ey$, and $S$ to lie in
        %  the $x{-}y$ plane with area $A$. Then
        %\begin{eqnarray*}
        %  \curl \v{a} 
        %  &=&
        %  \left(
        %       \frac{\partial P}{\partial x} + \frac{\partial Q}{\partial y}
        %  \right) \ez
        %  \quad\mbox{and}\quad
        %  \mbox{d}\v{S} \,=\, \mbox{d}x \, \mbox{d}y \: \ez \\[1ex]
        %%
        %  \v{a}\cdot \mbox{d}\v{r}
        %  &=& -Q \, \mbox{d}x + P \, \mbox{d}y
        %  \nonumber
        %\end{eqnarray*}
        %so 
        %\begin{eqnarray*}
        %  \int_A
        %  \left(
        %    \frac{\partial P}{\partial x} 
        %    +
        %    \frac{\partial Q}{\partial y}
        %  \right)
        %  \, \mbox{d}x \, \mbox{d}y
        %  \,=\,
        %  \oint_C \left( -Q \, \mbox{d}x + P \, \mbox{d}y \right)
        %\end{eqnarray*}
        %which is again Green's theorem in the plane, sometimes known as
        %Stokes' theorem in the plane.
        %
        %Taking $P = x/2$ and $Q = y/2$ and gives the planar area result of the
        %previous section. Indeed, Green's theorem is a generalisation of this
        %result.

  \item Applying Stokes' theorem to $\v{a} = \phi \, \v{c}$ where $\v{c}$ is a constant vector,
        we have
        \[
          \curl\left(\phi \, \v{c}\right)
          \,=\,
          \left(\grad \phi\right) \times \v{c}
          +
          \phi\left( \curl \v{c} \right)
          \,=\,
          \left(\grad \phi\right) \times \v{c}
          +
          0
        \]
        Hence
        \[
          \left( \curl \left(\phi \, \v{c}\right) \right) \cdot \mbox{d}\v{S}
          \,=\,
          \left(\left(\grad \phi\right) \times \v{c} \right)\cdot \mbox{d}\v{S}
          \,=\,
          \v{c} \cdot (\mbox{d}\v{S}\times \v{\nabla}\phi )
        \]
        which gives
        \[
          \int_S\, \left( \v{\nabla}\times(\phi \, \v{c})\right)\cdot \mbox{d}\v{S}
          \,=\,
          \v{c}\cdot \int_S\, \mbox{d}\v{S} \times \v{\nabla}\phi
          \,=\,
          \v{c}\cdot \oint_C\, \phi \, \mbox{d}\v{r}
        \]
        This holds for all constant vectors $\v{c}$, so
        \[
          \oint_C \, \phi \: \mbox{d}\v{r}
          \,=\,
          \int_S \, \mbox{d}\v{S} \times \v{\nabla}\phi
          \,=\,
          -\int_S \, \v{\nabla}\phi \times  \mbox{d}\v{S}
        \]
        Such results are hard to remember, but as we have seen, they can be derived quite easily.

  \item In terms of indices, Stokes' theorem is
        \[
          \epsilon_{ijk} \int_S \, \frac{\partial a_k}{\partial x_j} \, \mbox{d}S_i
          \,=\,
          \oint_C \, a_k \, \mbox{d}x_k
        \]

        %For a second-rank tensor $T$, we again regard one index as a
        %`spectator' index, so
        %\[
        %  \epsilon_{ijk} \int_S \,\frac {\partial T_{kl}}{\partial x_j} \, \mbox{d}S_i
        %  \,=\,
        %  \oint_C \, T_{kl} \, \mbox{d}x_k
        %\]
        %This is the \emph{generalised Stokes' theorem}. In particular, with
        %$T_{kl} = \phi \, \delta_{kl}$ we recover the result in (ii) above.

\end{enumerate}
