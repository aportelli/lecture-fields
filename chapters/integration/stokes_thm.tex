% !TEX root = ../../fields.tex
%% Brian Lecture 17 %%%%%%%%%%%%%%%%%%%%%%%%%%%%%%%%%%%%%%%%%%%%%%%%%%%%%%%%%%%%%%%%%%%%%%%
\subsection{Line integral definition of curl}
\begin{figure}[htbp]
  \centering
  \includegraphics{tikz_loop.pdf}
  \caption{Illustration of the line integral definition of the curl.}
  \label{fig:loop}
\end{figure}
Consider a small planar surface with unit normal $\vn$ and (scalar) area $\delta S$,
bounded by a \emph{closed} curve $\delta C$, and containing the point $P$. Let $\v{a}$ be
a vector field defined on this surface. I can be shown, as we discuss below, that the
component of $\curl\v{a}$ in the direction of $\vn$ is given by
\begin{center}
  \vspace*{-1ex}
  \bigbox{
    \parbox{70mm}{
      \vspace*{-1.5ex}
      \[
        % \v{n} \cdot \left({\rm curl\ }\v{a}\right)
        %  \,=\,
        \vn \, \cdot\, \left(\curl\v{a}\right)
        \,=\,
        \lim_{\delta S\to0} \; \frac{1}{\delta S}\;
        \oint_{\delta C} \v{a} \,\cdot\, \diff \v{r}
      \]
      \vspace*{-2ex}
    }
  }
\end{center}
This form of curl is manifestly \emph{independent of the choice of basis}. See
\cref{fig:loop} for illustration. Please note that the integral around $\delta C$ is taken
in the right-hand sense with respect to the normal $\vn$ to the surface.

\paragraph{Cartesian form of curl:}
The usual Cartesian form for curl can be recovered from this general definition by
considering small rectangles parallel to the $(x,y)$, $(y,z)$, and $(z,x)$ planes
respectively. Let $P$ be a point with Cartesian coordinates $(x_0,y_0,z_0)$ situated at
the \emph{centre} of a small rectangle $\delta C=ABCD$ of size $\delta_x \times
  \delta_y$, area $\delta S = \delta_x\, \delta_y$, parallel to the $(x,y)$ plane, as
illustrated in \cref{fig:curl}. The line integral around $\delta C$ is given by the sum
of four terms
\begin{eqnarray*}
  \ointdC \v{a}\cdot \diff \v{r}
  \,=\,
  \int_A^B  \v{a}\cdot \diff \v{r}
  \,+\,
  \int_B^C  \v{a}\cdot \diff \v{r}
  \,+\,
  \int_C^D  \v{a}\cdot \diff \v{r}
  \,+\,
  \int_D^A  \v{a}\cdot \diff \v{r} \\[1ex]
  \,=\,
  \int_A^B  \v{a}\cdot \diff \v{r}
  \,-\,
  \int_C^B  \v{a}\cdot \diff \v{r}
  \,-\,
  \int_D^C  \v{a}\cdot \diff \v{r}
  \,+\,
  \int_D^A  \v{a}\cdot \diff \v{r}
\end{eqnarray*}
Since $\vr = x\,\ex+y\,\ey+z\,\ez$ we have
$\diff \v{r}=\ex\,\diff x$ along $D\rightarrow A$ and
$C\rightarrow B$, and $ \diff \v{r}=\ey\,\diff y$\\[0.5ex] along
$A\rightarrow B$ and $D\rightarrow C$. Therefore,
\[
  \ointdC \v{a}\cdot \diff \v{r}
  \,=\,
  \int_A^B  a_y\, \diff y
  \,-\,
  \int_C^B  a_x\, \diff x
  \,-\,
  \int_D^C  a_y\, \diff y
  \,+\,
  \int_D^A  a_x\, \diff x
\]
For small $\delta_x$ and $\delta_y$, we can Taylor expand the integrands about $y=y_0$,
\begin{eqnarray*}
  \int_D^A a_x\, \diff x
  & = &
  \int_D^A a_x(x,y_0-\delta_y/2,z_0) \, \diff x \\[1ex]
  %
  & = &
  \int_{x_0-\delta_x/2}^{x_0+\delta_x/2}
  \left[
    a_x(x,y_0,z_0)
    \,-\,
    \frac{\delta_y}{2}\; \parpar{a_x}{y}\bigg|_{(x,y_0,z_0)}
    \,+\,
    O(\delta_y^2)
    \right]\, \diff x \\[1.5ex]
  %
  \int_C^B a_x\, \diff x
  & = &
  \int_C^B a_x(x,y_0+\delta_y/2,z_0) \, \diff x\\[1ex]
  %
  & = &
  \int_{x_0-\delta_x/2}^{x_0+\delta_y/2}
  \left[
    a_x(x,y_0,z_0)
    \,+\,
    \frac{\delta_y}{2}\; \parpar{a_x}{y}\bigg|_{(x,y_0,z_0)}
    \,+\,
    O(\delta_y^2)
    \right]\, \diff x
\end{eqnarray*}
\begin{figure}[t]
  \centering
  \includegraphics{tikz_curl.pdf}
  \caption{Line integral version of the curl on a small rectangle in the $(x,y)$ plane.}
  \label{fig:curl}
\end{figure}
So
\begin{eqnarray*}
  \frac{1}{\delta S}
  \left[
    \int_D^A
    \v{a}\cdot \diff \v{r} \,+\, \int_B^C  \v{a}\cdot \diff \v{r}\right]
  & = &
  \frac{1}{\delta_x\, \delta_y}
  \left[
    \int_D^A a_x\, \diff x \,-\, \int_C^B a_x\, \diff x
    \right]\\[1ex]
  %
  & = &
  \frac{1}{\delta_x \delta_y}\;  \int_{x_0-\delta_x/2}^{x_0+\delta_x/2}
  \left[
    -\delta_y \parpar{a_x}{y}\bigg|_{(x,y_0,z_0)} \,+\, O(\delta_y^2)
    \right] \, \diff x \\[1ex]
  %
  & \to &
  -\parpar{a_x}{y}\bigg|_{(x_0,y_0,z_0)}
  \quad \mbox{as}\quad \delta_x,\, \delta_y \to 0
\end{eqnarray*}
In the last step, as we take the limit $\delta_x \to 0$, the
integrand tends to a constant in the region of integration:
\[
  \parpar{a_x}{y}\bigg|_{(x,y_0,z_0)}
  ~\to~
  \parpar{a_x}{y}\bigg|_{(x_0,y_0,z_0)}
\]
and the integral over $x$ is then trivial (as the integrand is a constant). A similar
analysis of the line integrals along $A\rightarrow B$ and $C\rightarrow D$ gives
(exercise)
\[
  \frac{1}{\delta S}
  \left[
    \int_A^B \v{a}\cdot \diff \v{r} \,+\, \int_C^D \v{a}\cdot \diff \v{r}
    \right]
  ~\rightarrow~
  \parpar{a_y}{x}\bigg|_{(x_0,y_0,z_0)}
  \quad \mbox{as}\quad \delta_x,\, \delta_y \to 0
\]
Adding the results gives, for our line integral definition of curl,
\[
  \ez \cdot \left(\curl\v{a}\right)
  \,=\,
  \left(\curl\v{a}\right)_z
  \,=\,
  \left[ \parpar{a_y}{x} - \parpar{a_x}{y} \right]_{\:(x_0,\,y_0,\,z_0)}
\]
in agreement with our original definition in Cartesian coordinates. The other components
of $\v{\nabla}\times\v{a}$ can be obtained from similar rectangles parallel to the
$(y,z)$ and $(x,z) $ planes, respectively. It can be shown that $\v{\nabla}\times\v{a}$,
when defined in this way, is independent of the \emph{shape} of the infinitesimal area
$\delta S$.

\subsection{Geometrical interpretation of curl}
Consider a force field $\v{F}(\v{r}),$ and let $\delta C$ be a small rectangular contour
which encloses an area $\delta S$ in the $(x,y)$ plane -- as in the line-integral
definition of curl above. The \emph{work done} on a (point) test particle in moving it
around the closed curve $\delta C$ is
\[
  \oint_{\delta C} \v{F}\cdot \diff \v{r}
  \,=\,
  \mbox{\emph{circulation} of $\v{F}(\v{r})$ about $\delta C$}
\]
From the integral definition of curl, we know that for small $\delta S$
\[
  \oint_{\delta C} \v{F}\cdot \diff \v{r}
  ~\approx~
  \left( \v{\nabla}\times\v{F}\right)_z \, \delta S
\]
Therefore $\,\left(\v{\nabla}\times\v{F}\right)_z \, \neq \, 0 \;$ is equivalent to
saying that a non-zero amount of work is done in moving the test particle around a small
closed path in the $(x,y)$ plane. Alternatively one can think of the non-zero circulation
of $\v{F}$ as causing a small test particle to \emph{rotate} about its centre, with the
axis of rotation in the direction of $\curl\v{F}$. More generally,
$\vn\cdot\left(\v{\nabla}\times\v{a}\right)$ is a measure of the net \emph{circulation}
(per unit area) of the vector field $\v{a}$ about an \emph{infinitesimal} area $\diff S$
with normal $\vn$. An illustration of this interpretation can be found in
\cref{fig:net_circulation}.
\begin{figure}[t]
  \centering
  \includegraphics{tikz_net_circulation.pdf}
  \caption{Illustration of the circulation of a vector field around an infinitesimal surface element.}
  \label{fig:net_circulation}
\end{figure}

\subsection{Stokes' theorem}
Let $S$ be an \emph{open} surface, bounded by a simple \emph{closed} curve $C$, and let
$\v{a}$ be a vector field defined on $S$, then
\begin{center}
  \bigbox{
    \parbox{0.4\textwidth}{
      \[
        \intS \left(\curl\v{a}\right) \cdot \diff \v{S}
        \,=\,
        \ointC \v{a}\cdot \diff \v{r}
      \]
    }}
\end{center}
where $C$ is traversed in a right-hand sense about $\diff \v{S}$.
As usual, $\diff \v{S} = \vn \, \diff S$ where $\vn$ is the unit
normal to $S$. See \cref{fig:stokes} for illustration.

\paragraph{Proof:}
Divide the surface $S$ into $N$ adjacent small surfaces. Let $\delta \v{S}_{i} = \delta
  S_{i}\, \v{n}_{i}$ be the vector element of area at $\vr_{i}$, enclosed by the curve
$\delta C_{i}$, where no implicit summation on repeated indices is assumed. Start with
the integral definition of curl
\[
  \vn \cdot \left(\curl\v{a}\right)
  \,=\,
  \lim_{\delta S\to0} \; \frac{1}{\delta S}\;
  \oint_{\delta C} \v{a}\cdot \diff \v{r} \,,
\]
For a small but not infinitesimal open surface $\delta S_{i}$
\[
  [\curl \v{a}\left(\vr_{i}\right)]
  \cdot \vn_{i}
  \,=\,
  \frac{1}{\delta S_{i}} \oint_{\delta C_{i}} \v{a}\cdot \diff \v{r}
  \,+\,
  \epsilon_{i}
\]
where $\epsilon_{i}\to0$ as $\delta S_{i}\to0$.
\begin{figure}[t]
  \centering
  \includegraphics{tikz_stokes.pdf}
  \caption{Geometrical illustration of the context of Stokes' theorem.}
  \label{fig:stokes}
\end{figure}
Multiply by $\delta S_{i}$ (before taking the limit), and sum over all $i$ to get
\[
  \sum_{i=1}^{N}\:
  [\curl \v{a}\left(\vr_{i}\right)]
  \cdot \vn_{i}\: \delta S_{i}
  \,=\,
  \sum_{i=1}^{N}\: \oint_{\delta C_{i}} \v{a}\cdot \diff \v{r}
  %  \,+\, \ldots
  \,+\, \sum_{i=1}^{N}\; \epsilon_{i}\, \delta S_{i}
\]
Since each small closed curve $\delta C_{i}$ is traversed in the \emph{same} direction,
then, from the diagram, all contributions to $\smash{\sum_{i=1}^{N}\; \oint_{\delta
      C_{i}} \v{a}\cdot \diff \v{r}}$ \emph{cancel}, except on those curves where part of
$\delta C_{i}$ lies on the curve $C$. For example, the line integrals along the
\emph{common} section of the two small closed curves $\delta C_1$ and $\delta C_2$ in
\cref{fig:mesh} \emph{cancel exactly}. Therefore,
\[
  \sum_{i=1}^{N}\;\oint_{\delta C_{i}} \v{a}\cdot \diff \v{r}
  \,=\,
  \ointC \v{a}\cdot \diff \v{r}
\]
Hence, as $N\to\infty$,
\[
  \ointC \v{a}\cdot \diff \v{r}
  \,=\,
  \intS \left(\curl\v{a}\right) \cdot \diff \v{S}
  \,=\,
  \intS \vn\cdot\left(\curl\v{a}\right) \; \diff S
\]

\paragraph{Mathematical note:} For those worried about the error term, note that, for finite $N$, we can establish an
upper bound \vspace*{-1ex}
\[
  \left| \sum_{i=1}^{N}\; \epsilon_{i}\, \delta S_{i} \right|
  \;\le\; S \;
  \raisebox{-1.6ex}{$ \shortstack{max \\ \scriptsize{$i$} } $}
  \left\{ \left|\epsilon_{i} \right|\right\}
\]
The RHS tends to zero in the limit $N\to\infty$, because $S$ is finite and
$\epsilon_{i}\to0$, for all $i$. A similar analysis works in the proof of the divergence
theorem.\footnote{The case of an infinite surface $S$ (or infinite $V$ in the case of the
  divergence theorem) requires more care.}

\subsection{Examples of the use of Stokes' theorem}
\label{example_stokes}
\begin{figure}[t]
  \centering
  \includegraphics{tikz_mesh.pdf}
  \caption{Illustration of the cancellation of small contour contributions in Stokes' theorem proof.}
  \label{fig:mesh}
\end{figure}
\paragraph{Hemisphere:} Given the vector field $\v{a} = 4y \, \v{e}_{\,x} + x \, \ey + 2z \, \ez$, verify Stokes'
theorem for the (open) hemispherical surface $x^2 + y^2 + z^2 = R^2$ with $z > 0$. In
this case, we have $\curl\v{a} = -3\,\ez$, and we have shown previously that $\diff \v{S}
  = R^2\sin(\theta) \, \diff \theta \, \diff \phi \, \er$ on the surface of a hemisphere of
radius $R$. Direct integration then gives
\begin{eqnarray*}
  \int_{S_C} \curl \v{a}\cdot \diff \v{S}
  &=&
  \int_{S_C} (-3 \ez) \cdot
  R^2 \, \sin(\theta) \, \diff \theta \, \diff \phi \, \er \\[1ex]
  &=&
  -3 R^2 \int_0^{\pi/2} \, \sin(\theta)  \cos(\theta)\, \diff \theta
  \; \int_0^{2\pi} \diff \phi
  \,=\,
  -3 \pi R^2
\end{eqnarray*}
We can check our result using Stokes' theorem. The closed curve $C$ bounding the
hemisphere is a circle of radius $R$ in the $(x,y)$ plane. Parameterising this by $x =
  R\cos(\phi)$, $y = R\sin(\phi)$, $z=0$, gives $\diff x = -R\sin(\phi)\,\diff \phi$,\ \ $\diff y
  = R\cos(\phi)\,\diff \phi$, and $a_x = 4y = 4R\sin(\phi)$, $a_y = x = R\cos(\phi)$, $a_z=2z=0$.
Hence
\begin{eqnarray*}
  \oint_C \v{a}\cdot \diff \v{r}
  &=&
  \oint_C (4y \, \diff x + x \, \diff y) \\[1ex]
  &=&
  \int_0^{2\pi}
  [-4R^2\sin(\phi)^2 + R^2 \cos(\phi)^2]\,\diff \phi
  \,=\,
  -3\pi R^2
\end{eqnarray*}

\paragraph{Planar areas:}
Consider a planar surface $S$ parallel to the $(x,y)$ plane, bounded by a closed curve
$C$, and let the vector field $\v{a}(\v{r})$ be
\[
  \v{a}
  \,=\,
  \frac{1}{2} \left[ -y \, \ex + x \, \ey \right]
\]
In this case $\curl \v{a} = \v{e}_{\,z}$, and the vector element of area normal to the
$(x,y)$ plane is $\diff \v{S} = \diff S\,\ez$. Hence,
\[
  \int_S \curl\v{a} \cdot \diff \v{S}
  \,=\,
  \int_S \v{e}_{\,z}\cdot \diff \v{S} = \int_S \diff S
  \,=\,
  S
\]
We can then use Stokes' theorem to find the area of the surface
\begin{eqnarray*}
  S
  &=&
  \ointC \v{a}\cdot  \diff \v{r}
  \,=\,
  \frac{1}{2} \ointC
  (-y \, \ex + x \ey)
  \cdot
  (\diff x \, \v{e}_{\,x} + \diff y \, \v{e}_{\,y} )
\end{eqnarray*}
which gives
\[
  S = \frac{1}{2} \ointC (x\,\diff y - y\,\diff x)
\]

\paragraph{Example:} Find the area inside the curve
\[
  x^{2/3}+y^{2/3} =1 \, .
\]
The curve can be parameterised by $x= \cos(\phi)^3$, $y= \sin(\phi)^3$, for $0\leq \phi
  \leq 2\pi$, so that
\[
  \frac{\diff x}{\diff \phi} = -3\cos(\phi)^2\, \sin(\phi)\;,
  \quad
  \frac{\diff y}{\diff \phi} = 3\sin(\phi)^2\, \cos(\phi)
\]
which gives
\begin{eqnarray*}
  S
  &=&
  \frac{1}{2} \ointC (x\,\diff y - y\,\diff x)
  \,=\,
  \frac{1}{2} \ointC
  \left(
  x\,\frac{\diff y}{\diff \phi} -
  y\,\frac{\diff x}{\diff \phi}
  \right)
  \,\diff \phi
  \\[1ex]
  &=&
  \frac{1}{2} \int_{0}^{2\pi}
  [
    3\cos(\phi)^4 \, \sin(\phi)^2 + 3 \sin(\phi)^4 \, \cos(\phi)^2
  ]\diff \phi \\[1ex]
  &=&
  \frac{3}{2} \int_{0}^{2\pi} \sin(\phi)^2 \, \cos(\phi)^2 \, \diff \phi
  \,=\,
  \frac{3}{8} \int_{0}^{2\pi} \sin (2 \phi)^2 \,  \diff \phi
  \,=\,
  \frac{3\pi}{8}
\end{eqnarray*}

\subsection{Corollaries of Stokes' theorem}
We may deduce several immediate consequences of Stokes' theorem,
\[
  \int_S \, \left(\v{\nabla}\times\v{a}\right) \cdot \diff \v{S}
  \,=\,
  \oint_C \, \v{a}\cdot \diff \v{r}
\]
where $C$ is the boundary (traversed in the anticlockwise direction) of the open surface
$S$.

\begin{enumerate}

  \item If $\v{a} = \v{c}$, where $\v{c}$ is a constant vector, then $\curl \v{a}=0$. Therefore,
        $\ds \v{c}\cdot \oint_C\, \diff \v{r} = 0$, and because $\v{c}$ is arbitrary, we have
        \[
          \oint_C \, \diff \v{r} \,=\, 0
        \]

  \item Applying Stokes' theorem to $\v{a} = \phi \, \v{c}$ where $\v{c}$ is a constant vector,
        we have
        \[
          \curl\left(\phi \, \v{c}\right)
          \,=\,
          \left(\grad \phi\right) \times \v{c}
          +
          \phi\left( \curl \v{c} \right)
          \,=\,
          \left(\grad \phi\right) \times \v{c}
          +
          0
        \]
        Hence
        \[
          \left( \curl \left(\phi \, \v{c}\right) \right) \cdot \diff \v{S}
          \,=\,
          \left(\left(\grad \phi\right) \times \v{c} \right)\cdot \diff \v{S}
          \,=\,
          \v{c} \cdot (\diff \v{S}\times \v{\nabla}\phi )
        \]
        which gives
        \[
          \int_S\, \left( \v{\nabla}\times(\phi \, \v{c})\right)\cdot \diff \v{S}
          \,=\,
          \v{c}\cdot \int_S\, \diff \v{S} \times \v{\nabla}\phi
          \,=\,
          \v{c}\cdot \oint_C\, \phi \, \diff \v{r}
        \]
        This holds for all constant vectors $\v{c}$, so
        \[
          \oint_C \, \phi \: \diff \v{r}
          \,=\,
          \int_S \, \diff \v{S} \times \v{\nabla}\phi
          \,=\,
          -\int_S \, \v{\nabla}\phi \times  \diff \v{S}
        \]
        Such results are hard to remember, but as we have seen, they can be derived quite easily.

  \item Using Einstein summation convention, Stokes' theorem can ve written
        \[
          \epsilon_{ijk} \int_S \, \frac{\partial a_k}{\partial x_j} \, \diff S_i
          \,=\,
          \oint_C \, a_k \, \diff x_k
        \]
\end{enumerate}
